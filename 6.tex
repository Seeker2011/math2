\chapter{三角比与角边关系}

人们为了要确定空间各点之间的相互位置,就得做一番
测量,测量是几何学的起源,也是几何学最直接的实践。

测量学的最基本原理,就是相似形的性质及三角形的边
角关系。例如,我们在第三章末用相似形性质测量两点间的,
距离,物体的高度、测绘具有多边形形状的地段的平面图
等。我们知道,在两个直角三角形中,只要有一个锐角对应
相等,它们就相似了,这就是说,一个直角三角形的各边之。
间的比是被它的一个锐角的大小所决定,例如在图6.1中,
一些含有$30^{\circ}$角的直角三角形,$30^{\circ}$角所对的直角边与斜边的
比都是1:2.

\begin{figure}[htp]
    \centering
\begin{tikzpicture}
\begin{scope}
\tkzDefPoint(30:2){B_1}
\tkzDefPoints{0/0/A_1, 1.732/0/C_1}
\tkzDrawPolygon(A_1,B_1,C_1)
\tkzMarkAngle[mark=none, size=.5](C_1,A_1,B_1)
\tkzLabelPoints[right](C_1,B_1)
\tkzLabelPoints[left](A_1)
\tkzMarkRightAngle(B_1,C_1,A_1)
\end{scope}
\begin{scope}[xshift=3.5cm]
    \tkzDefPoint(30:3){B_2}
    \tkzDefPoints{0/0/A_2, 2.6/0/C_2}
    \tkzDrawPolygon(A_2,B_2,C_2)
    \tkzMarkAngle[mark=none, size=.5](C_2,A_2,B_2)
    \tkzLabelPoints[right](C_2,B_2)
    \tkzLabelPoints[left](A_2)
    \tkzMarkRightAngle(B_2,C_2,A_2)
\end{scope}
\begin{scope}[xshift=11cm]
    \tkzDefPoint(150:4){B_3}
    \tkzDefPoints{0/0/A_3, -3.464/0/C_3}
    \tkzDrawPolygon(A_3,B_3,C_3)
    \tkzMarkAngle[mark=none, size=.6](B_3,A_3,C_3)
    \tkzLabelPoints[below](C_3,A_3)
    \tkzLabelPoints[left](B_3) 
    \tkzMarkRightAngle(A_3,C_3,B_3)
\end{scope}

\end{tikzpicture}
    \caption{}
\end{figure}

这一章,我们首先向同学介绍的就是直角三角形中,边
与边的比与它所含锐角之间的关系。这些边与边的比值叫做
\textbf{锐角三角比},它们是进行测量计算时的常用数据,也是从数
量方面研究几何学的基本工具。






















































