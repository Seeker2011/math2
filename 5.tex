\chapter{轨迹与作图}
在前面的几章中,我们学习了直线形和圆的有关性质。
学习的途径主要是根据图形的定义和已知性质去推演图形的
其它性质。这一章,我们将把图形看成点的集合(点集),
研究如何根据点所具有的某种性质来求出点集在平面上的形
状和位置。

\section{轨迹}
\subsection{轨迹的概念}
我们知道,物体在运动中都要经过一定的路线。例如,
人在雪地里行走会留下明显的足迹,飞机飞行有一定的航
线,地球运行也有它的轨道等等。一般,我们常把物体按某
种规律运动的路线叫做物体运动的\textbf{轨迹}。在几何中,我们用
点表示物体在空间的位置。这样,一个点在空间按某种规律
运动的路线,我们就把它叫做这个点运动的轨迹,这个点就
叫做\textbf{动点}。例如,我们用圆规画圆时,圆规的一个脚尖固定
不动,而另一个脚上装上的铅笔尖端就可看作一个动点。它
和固定的脚尖保持一定的距离运动,所画出的图形就是这个
动点的轨迹。我们知道,圆是“同一平面上和某定点的距离
等于定长的点的集合”。由此可见,按某种规律运动的点的
轨迹,也就是具有某种性质的点的集合。

\begin{blk}{定义}
具有性质$\alpha$的所有点构成的集合,叫做具有性
质$\alpha$的点的轨迹。
\end{blk}

设$X=\{\text{具有性质$\alpha$的点}\}$。由上述定义,当我们要证明
某图形$A$是具有某种性质$\alpha$的点的轨迹时,也就是要证明
集合$A=X$. 要证明$A=X$, 就必须从以下两方面进行证
明:
\begin{enumerate}
\item $P$点$\in A\Rightarrow P$点具有性质$\alpha$ $(P\in X)$.
\item $P$点具有性质$\alpha$ $(P\in X)\Rightarrow P$点$\in A$.
\end{enumerate}

按上述两个方面证明,这是缺一不可的。如果我们只证
了第一条,实际上只是说明$A$是$X$的一个子集,并不能断
定$A=X$; 如果只证了第二条,也只是说$X$是$A$的一个子
集,同样不能断定$A=X$, 只有当我们证明了第一条:$A\subseteq
X$, 又证明了第二条:$X\subseteq A$, 我们才能断定$A=X$.

第一条证明了$A\subseteq X$, 这就是说在图形$A$上的点,都具
有性质$\alpha$. 没有一点是鱼目混珠的,通常把证这一条叫做证
\textbf{轨迹的纯粹性}。第二条证明了$X\subseteq A$, 这就是说,具有性质
$\alpha$的点都在图形$A$上,没有一点被遗漏掉。通常又把证这一
条叫做证\textbf{轨迹的完备性}。

由于原命题与逆否命题等价,所以也可以分别去证上述
两条的逆否命题,即要证轨迹的纯粹性也可证:
\[P\text{点不具有性质}\alpha\Rightarrow P\notin A\]
要证轨迹的完备性时,也可证:
\[P\text{点}\notin A\Rightarrow P\text{点不具有性质}\alpha\]

\begin{ex}
\begin{enumerate}
\item 叙
述两个集合相等的定义。
\item 在证轨迹命题时,为什么即要证轨迹的纯粹性,又要证
轨迹的完备性?
\item 如果我们证明了
$\overline{AB}$的垂直平分线上的任一点到$A$、
$B$两
点的距离相等,能否就说与$A$、$B$两点距离相等的点
的轨迹是$\overline{AB}$的垂直平分线?
\end{enumerate}
\end{ex}


\subsection{基本轨迹}
这一小节,我们来学习六个平面上的点的基本轨迹,我
们只证了1和4, 其它四个由同学们自证。












