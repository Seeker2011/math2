\documentclass[b5paper, openany]{ctexbook}


\usepackage[margin=2.5cm]{geometry}


\usepackage{pifont}
\usepackage[perpage,symbol*]{footmisc}
\DefineFNsymbols{circled}{{\ding{192}}{\ding{193}}{\ding{194}}
{\ding{195}}{\ding{196}}{\ding{197}}{\ding{198}}{\ding{199}}{\ding{200}}{\ding{201}}}
\setfnsymbol{circled}

\usepackage{ulem}

\usepackage{amsmath,amsfonts,mathrsfs,amssymb}
\usepackage{graphicx}

\usepackage[font=bf,labelfont=bf,labelsep=quad]{caption}

\usepackage{tikz}


\usepackage{ntheorem}
\theoremseparator{\;}



\usepackage{blkarray}
\usepackage{bm}
\usepackage[colorlinks=true, linkcolor=black]{hyperref}

\usepackage{enumerate}


\theoremstyle{plain}
\theoremheaderfont{\normalfont\bfseries} 
\theorembodyfont{\normalfont}


\usepackage[framemethod=tikz]{mdframed}


\newtheorem{example}{\bf 例}[chapter]
\newenvironment{solution}{\noindent {\bf 解:}}{}
\newenvironment{analyze}{\noindent {\bf 分析:}}{}
\newenvironment{rmk}{\noindent {\bf 注意:}}{}
\newenvironment{note}{\noindent {\bf 说明:}}{}



\renewcommand{\proofname}{\bf 证明:}
\newenvironment{proof}{{\noindent \bf 证明:}}{}%{\hfill $\square$\par}

\newcommand{\E}{\mathbb{E}}
\renewcommand{\Pr}{\mathbb{P}}
\newcommand{\EP}{\mathbb{E}^{\mathbb{P}}}
\newcommand{\EQ}{\mathbb{E}^{\mathbb{Q}}}
\newcommand{\dif}{\,{\rm d}}
\newcommand{\Var}{{\rm Var}}
\newcommand{\Cov}{{\rm Cov}}
\newcommand{\x}{\times}


 \usepackage{tcolorbox}
 \tcbuselibrary{breakable}
 \tcbuselibrary{most}



\newtcolorbox{ex}[1][]
  {colback = white, colframe = cyan!75!black, fonttitle = \bfseries,
    colbacktitle = cyan!85!black, enhanced,
    attach boxed title to top center={yshift=-2mm},breakable, 
    title=练习, #1}

\newtcolorbox{blk}[2][]
  {colback = white, colframe = magenta!75!black, fonttitle = \bfseries,
    colbacktitle = magenta!85!black, enhanced,
    attach boxed title to top left={xshift=5mm, yshift=-2mm},breakable, 
    title=#2, #1}


\setcounter{tocdepth}{2}

\setcounter{secnumdepth}{3}



\ctexset {
section = {
	name = {第,节},
 	number = \chinese{section}},
subsection = {
	name = {,、\hspace{-1em}},
	number = \chinese{subsection}
},
subsubsection = {
	name = {(,)\hspace{-1em}},
	number = \chinese{subsubsection},
}
}



\renewcommand{\contentsname}{目~~录}

\newcommand{\poly}{\polynomial[reciprocal]}



\usepackage{mathtools}

\setlength{\abovecaptionskip}{0.cm}
\setlength{\belowcaptionskip}{-0.cm}

\usetikzlibrary{decorations.pathmorphing, patterns}
\usetikzlibrary{calc, patterns, decorations.markings}
\usetikzlibrary{positioning, snakes}


\usepackage{yhmath}
\usepackage{longdivision}
\usepackage{polynom}
\usepackage{polynomial}
\usepackage{cancel}

\renewcommand{\frac}{\dfrac}
\newcommand{\oc}{$^{\circ}{\rm C}$}

\usepackage{tkz-base}
\usepackage{tkz-euclide}

\usepackage{multicol}
\usepackage{cases}
\begin{document}













\title{中学数学实验教材\\第二册}



\author{中学数学实验教材编写组编}
\date{1982年7月}

\maketitle




\frontmatter

\chapter{前~~言}

这一套中学数学实验教材,内容的选取原则是精简实
用,教材的处理力求深入浅出,顺理成章,尽量作到使人人
能懂,到处有用。

    本教材适用于重点中学,侧重在满足学生将来从事理工
方面学习和工作的需要。

    本教材的教学目的是:使学生切实学好从事现代生产、
特别是学习现代科学技术所必需的数学基础知识;通过对数
学理论、应用、思想和方法的学习,培养学生运算能力,思
维能力,空间想象力,从而逐步培养运用数学的思想和方法
去分析和解决实际问题的能力;通过数学的教学和学习,培
养学生良好的学习习惯,严谨的治学态度和科学的思想方
法,逐步形成辩证唯物主义世界观。

   根据上述教学目的,本教材精选了传统数学那些普遍实
用的最基础的部分,这就是在理论上、应用上和思想方法上
都是基本的、长远起作用的通性、通法。比如,代数中的数
系运算律,式的运算,解代数方程,待定系数法;几何中的
图形的基本概念和主要性质,向量,解析几何;分析中的函
数,极限,连续,微分,积分;概率统计以及逻辑、推理论
证等知识。对于那些理论和应用上虽有一定作用,但发展余
地不大,或没有普遍意义和实用价值,或不必要的重复和过
于繁琐的内容,如立体几何中的空间作图,几何体的体积、
表面积计算,几何难题,因式分解,对数计算等作了较大的
精简或删减。

    全套教材共分六册。第一册是代数。在总结小学所学自
然数、小数、分数基础上,明确提出运算律,把数扩充到有
理数和实数系。灵活运用运算律解一元一次、二次方程,二
元、三元一次方程组,然后进一步系统化,引进多项式运
算,综合除法,辗转相除,余式定理及其推论,学到根式、
分式、部分分式。第二册是几何。由直观几何形象分析归纳
出几何基本概念和基本性质,通过集合术语、简易逻辑转入
欧氏推理几何,处理直线形,圆、基本轨迹与作图,三角比
与解三角形等基本内容。第三册是函数。数形结合引入坐
标,研究多项式函数,指数、对数、三角函数,不等式等。
第四册是代数。把数扩充到复数系,进一步加强多项式理论,
方程式论,讲线性方程组理论,概率(离散的)统计的初步
知识。第五册是几何。引进向量,用向量和初等几何方法综
合处理几何问题,坐标化处理直线、圆、锥线,坐标变换与
二次曲线讨论,然后讲立体几何,并引进空间向量研究空间
解析几何初步知识。第六册是微积分初步。突出逼近法,讲
实数完备性,函数,极限,连续,变率与微分,求和与积分。

本教材基本上采取代数、几何、分析分科,初中、高中
循环排列的安排体系。教学可按初一、初二代数、几何双科
并进,初三学分析,高一、高二代数(包括概率统计)、几
何双科并进,高三学微积分的程序来安排。

    本教材的处理力求符合历史发展和认识发展的规律,深
入浅出,顺理成章。突出由算术到代数,由实验几何到论证
几何,由综合几何到解析几何,由常量数学到变量数学等四
个重大转折,着力采取措施引导学生合乎规律地实现这些转
折,为此,强调数系运算律,集合逻辑,向量和逼近法分别
在实现这四个转折中的作用。这样既遵循历史发展的规律,
又突出了几个转折关头,缩短了认识过程,有利于学生掌握
数学思想发展的脉络,提高数学教学的思想性。

这一套中学数学实验教材是教育部委托北京师范大学、
中国科学院数学研究所、人民教育出版社、北京师范学院、
北京景山学校等单位组成的领导小组组织“中学数学实验教
材编写组”,根据美国加州大学伯克利分校数学系项武义教
授的《关于中学实验数学教材的设想》编写的。第一版印出
后,由教育部实验研究组和有关省市实验研究组指导在北
京景山学校、北京师院附中、上海大同中学、天津南开中
学、天津十六中学、广东省实验中学、华南师院附中、长春
市实验中学等校试教过两遍,在这个基础上编写组吸收了实
验学校老师们的经验和意见,修改成这一版《中学数学实验
教材》,正式出版,内部发行,供中学选作实验教材,教师
参考书或学生课外读物。在编写和修订过程中,项武义教授
曾数次详细修改过原稿,提出过许多宝贵意见。

    本教材虽然试用过两遍,但是实验基础仍然很不够,这
次修改出版,目的是通过更大范围的实验研究,逐步形成另
一套现代化而又适合我国国情的中学数学教科书。在实验过
程中,我们热忱希望大家多提意见,以便进一步把它修改好。

\begin{flushright}
    中学数学实验教材编写组\\
    一九八一年三月
\end{flushright}








\tableofcontents


\mainmatter

%  \chapter{实验几何}

几何学是研究“空间”的形体和性质的科学。“空间”
就是我们和万物以至星象天体共存的所在。在日常生活中,
我们举目四望所见到的地方,都是空间的一部分。同学们在
小学数学课中学过的柱体、锥体、球体等等,它们都各自占
有空间的一部分,并且构成不同的形体。各种形体的种种性
质,如各部分的长度、角度、面积,以及体积等等。都是在
我们的生活和生产实践中所不可缺少的知识。
\begin{figure}[htp]
	\centering
\begin{tikzpicture}[scale=.7]
\begin{scope}
	\draw (1.5,0) ellipse [x radius=1.5, y radius=.5];
	\shade (0,0) rectangle (3,4);
	\shade (1.5,4)[draw] ellipse [x radius=1.5, y radius=.5];
	\draw (0,0)--(0,4);
	\draw (3,0)--(3,4);
		\node at (1.5,-1.5){圆柱};

\end{scope}
\begin{scope}[xshift=5.5cm]	
	\shade (1.5,0)[draw] ellipse [x radius=1.5, y radius=.5];
		\fill[gray!50] (0,0)-- (3,0)--(1.5,4)--(0,0);
		\draw (0,0)--(1.5,4)-- (3,0);
	

	\node at (1.5,-1.5){圆锥};
\end{scope}
\begin{scope}[xshift=12cm]
		\shade [draw, ball color =gray!20] (0,2) circle(2);
	\node at (0,-1.5){球};
\end{scope}
\end{tikzpicture}	
	\caption{}
\end{figure}

自古以来,人们经过实践、观察、分析,已总结出一
系列的有关空间方面的知识,例如,从中国、埃及、巴比
伦、玛雅等古文明中,可以看出对空间的知识都已掌握得
相当丰富了。对于空间知识有系统的研究,从西方的古文
明中可知,起始于古埃及和巴比仑,而在古希腊得到蓬勃
的发展,获得较辉煌的成就。大体说来,古希腊在空间知
识方面的成就,由欧几里得集其大成于他所著的《几
何原
本》\footnote{欧几里得(Euclid约公元前300年左右)所著此书原名Elements, 
	我国明代数学家徐光启(公元1562---1633)把书中部分几何内容
	译成中文定名为“几何原本”。“几何学”这个中文的名称即来源于
	此。}这部书中。在这部书里,欧儿里得把当时所知道的几何
知识经过整理,建立起一个初步完整的理论体系,使这部书反
映出几何学是一门偏重于推理、论证的高度理论性的科学。
但是,和任何其它科学一样,几何学的理论基础也是建
立在实验所得的一些基本事实之上的。在这一章里,我们就
通过实验、观察、归纳来研究所得到的知识,为以后进一
步学习论证几何作准备。

\section{点、直线和平面}
点、直线和平面是空间最简单的,也是最基本的图形。
同学们在日常生活中,对它们早已有直观的认识了。在这一
节里,我们再对它们的本质和相互关系作进一步的分析,确
立点、直线和平面这三个基本的几何概念,并总结点、直线
和平面之间相互关系方面的一些基本性质。

\subsection{点和直线}
在空间,最原始的,也是最基本的概念就是“位置”。
通常,我们就用“点”来标记“位置”。例如在一张地图
上,我们就以小圆点来标记各地的位置(见图1.2).你
可能发现,在地图上北京用“$\bigstar$”,南京用“$\bigcirc$”印制
的,这只是为了把首都和地方城市区别开来。其实,北京、
南京的。“位置”与地图上印制的图形“$\bigstar$”或“$\bigcirc$”的形状
和大小是没有关系的。这样,仅仅考虑“位置”,的图形就是
点。在天象图上也是以小圆点来标记各星体的位置的(见图1.3)

在几何学的讨论中,我们用不同的大写字母$A,B,C,\ldots$
表示不同的点,如图1.4中的五个点,就在点旁分别标记
以$A$、$B$、$C$、$D$、$E$, 并分别读作点$A$、点$B$、点$C$、点
$D$、点$E$。



在日常生活中,我们经常需要从一个地方走到另一个地
方。例如,同学们早起上学,就得由自己的家所在的位置走
到学校所在的位置。因此,在空间第二个原始的基本概念就
要算是“通路”了。所谓“通路”,就是从一个位置移到
另一个位置的路线。通常在地图上,我们用线来标记各地之
间的种种通路,如铁路、公路等。在几何学的讨论中,“线”
就是表示通路的。它的直观含义就是:一个“动点”由一
个位置移动到另一个位置所走过的“路线”。如图1.5所
示,设$A$、$B$两点分别表示空间的两个位置,那么连结$A$、$B$
两点的可能通路是很多很多的。

在通常情况下,大家都希望所要走的通路愈短愈好,所
以很自然的问题就是:

“在所有连结$A$、$B$两点的各种通路中,哪一条通路最
短?”

光线的存在,直截了当地显示给我们下述空间的基本性
质:

“连结A、B两点的最短通路唯一存在,它就是连结$A$、
$B$两点的\textbf{直线段}”(在均匀介质中,光走直线\footnote{由光学实验,我们知道光线其实走着最省时间的通路,而并不
	是走着最短的通路,再者,光的速度是随着“介质”而定的,例如在
	真空中走的最快,在空气中速度则稍慢(愈稀薄则其速度愈近于真空
	者),在水中则速度更慢,因为通常我们总是在均匀介质中观察光
	线,所以光线的速度是个不变的常数。这样,最省时间的通路也就是
	最短的通路。这就是我们常见常用的事实:光线在均匀介质中走直
	线。})。

如图1.6所示,由$A$点射向$B$点的光线可以由$A$向$B$的方
向无限延伸;而由$B$点射向$A$点的光线也可以由$B$向$A$的方向
无限延伸,所以对于空间任意两点$A$、$B$, 不但存在着唯一
的最短通路“直线段$AB$”,而且也唯一地确定了一条把直线
段$AB$两端无限延长的直线,这条直线就叫做由$A$、$B$两点所
确定的\textbf{直线},通常称为“直线$AB$”,而直线段$AB$是直线$AB$
介于A、B两点之间的那一段。

归纳上面的讨论,我们可以作出如下的总结:










%  \chapter{集合与简易逻辑}

 \section{集合}
\subsection{集合的概念}
我们在第一册中曾用到过集合的概念,如自然数所构成
的集合$\{1,2,3,\ldots\}$, 方程$x^2=4$的解所构成的集合$\{-2,2\}$
等等。那么,什么是集合呢?集合是数学中一个很根本的也
是很原始的概念,通常我们把一些确定的、彼此不同的“事
物”作为一个整体来考虑时,这个整体便说是一个\textbf{集合}。这
些事物叫做该集合的\textbf{元素}。例如:
某中学初二(一)班全体学生;小于100的全体质数;
一个生产队的全体社员;一个工厂的全部机器等等,都可分别构成一个集合。可见集合的概念是很简单
的。

对于集合这个概念,我们要注意以下几点:

第一,一个集合完全被它所含的元素所确定。至于集合
的元素之间是否具有某种相互关系,怎样排列,以及这些元
素所构成的集合是具有某种功效,单从集合的观点来看,
是一样的。例如,“一堆还没有组装的手表另件”和用“这
些另件组装好了的手表”是同一个集合。因为两者包含同样
的元素。因此,集合这个概念的要素是:\textbf{一个集合完全被它
所含的元素所确定}。

第二,集合是指构成集合的全体元素,而不是个别元
素。作为整体的集合和集合中的每个元素都是不同的。

例如,集合$A=\{a,b,c,d\}$, $A$代表的是字母$a$、$b$、
$c$、$d$的全体,而不是代表其中的个别字母,因此,作为字母
$a$、$b$、
$c$、$d$的整体的集合$A$与$A$中的个别元素如$a,b,c,d$不
能混为一谈。

第三,集合中所含的元素必须是“确定”的,是可以判
断的。例如,由“比较小的实数”的全体就不能构成一个集
合,因为到底什么叫做比较小的实数,没有判断的标准。但
是“比80小的实数”是完全可以确定的,这就有了检验一个
实数是否是这个集合的元素的标准。

任一几何图形,我们可以看作由点构成的,也就是可看
作点的集合。例如:

\textbf{圆}是同一平面上与一定点的距离等于定长的所有点的集
合(图2.1(1))。

\textbf{圆面}是同一平面上与一定点距离小于或等于定长的所有
点的集合。(图2.1(2))。

\begin{figure}[htp]
	\centering
	\begin{tikzpicture}[scale=.7]
		\begin{scope}
			\draw (0,0) circle (2);
\draw[ultra thick] (0,0)node[below]{$O$}--node[left]{$r$}(45:2);		
\node at (0,-3){(1)圆};	
		\end{scope}
		\begin{scope}[xshift=6cm]
			\draw[pattern=north west lines] (0,0) circle (2);
\draw[ultra thick] (0,0)node[below, fill=white]{$O$}--(45:2);	
			\node at (60:1.5) [left, rotate=45, fill=white]{$r$};
			
			\node at (0,-3){(2)圆面};
		\end{scope}
	\end{tikzpicture}	
	\caption{}
\end{figure}

我们通常用大写字母$A,B,C,\ldots$等表示某一个集合,
用小写字母$a,b,c,\ldots$表示集合的元素。如果$a$是集合$A$
的一个元素,我们就记为$a\in A$,
读作$a$属于$A$, 或说$a$是$A$中的一个元素。例如,$2\in\{2,3\}$,
表示$2$是集合$\{2,3\}$中的一个元素。

如果$a$不是集合$A$的元素,记作
$a\notin A$
读作$a$不属于$A$。

应该注意的是:几何图形中的元素“点”我们仍用大写
字母$A,B,C,\ldots$表示,这一点
请同学们务必注意,不要混
淆。如$X$点在直线$AB$上,也
可以说$X$点属于直线$AB$, 可
写成$X\in\text{直线}AB$. $Y$点不在直
线$AB$上,也可以说$Y$点不属于直线$AB$, 可写成$Y\notin\text{直线}
AB$(图2.2)。

\begin{figure}[htp]
	\centering
	\begin{tikzpicture}[scale=1]
\draw(0,0)--(6,0);
\draw (1,0)[fill=black]circle (1.5pt) node[below]{$A$};
\draw  (4.5,0)[fill=black]circle (1.5pt) node[below]{$B$};
\draw  (4,-.5)[fill=black]circle (1.5pt) node[below]{$Y$};
\draw  (3,0)[fill=black]circle (1.5pt) node[above]{$X$};
	\end{tikzpicture}	
	\caption{}
\end{figure}

\begin{ex}
\begin{enumerate}
\item 若$S$是
所有平方数的集合,试判定100至200之间哪些数
属于$S$.
\item 若$B$是所有英语元音字母所构成的集合,$A$是所有英语
辅音字母所构成的集合,试判定$a,b,c,d,e$这五个字
母分别属于哪一集合,又不属于哪一集合。
\end{enumerate}
\end{ex}

\subsection{集合的描述法}
决定一个集合的要素,就是它所含的元素,所以要描述
一个集合,也就是要描述它所含的是哪些元素。下面介绍两
种常用的集合描述法。

\subsubsection{列举法}
如果一个集合$A$只含有很少几个元素,那么可以直截了
当地把这个集合含有的所有元素逐一列举出来,并用大括号
$\{\quad \}$把它们括起来,这种描述法叫做\textbf{列举法}。

例如$\{0,1\}$是由0,1这两个元素所构成的集合;$\{+,-,\x,\div\}$表示由$+$、$-$、$\x$、$\div$四个运算符号所构成的
集合。用列举法描述集合时,描述方法与元素在括号内的排
列顺序无关,即$\{3,7,10\}$、$\{10,3,7\}$与$\{7,3,10\}$
都表示同一个集合。

\subsubsection{特征性质描述法}
当集合的元素稍多一些时,如小于100的质数所构成的
集合:$$\{2,3,5,7,11,13,17,19,23,29,31,37,
41,43,47,53,59,61,67,71,73,79,83,89,
97\}$$
逐一列举已是很麻烦的了,而对于含有无穷多个元素的
集合,例如全体整数所构成的集合,逐一列举它的元素更是
不可能的,这时我们可用某集合所含的元素的“特征性质”
去描述这个集合,这种方法叫做\textbf{特征性质描述法}。如:

\begin{enumerate}
\item 集合元素为
$\pm 2,\pm 4,\pm 6,\pm 8,\ldots,\pm 2n\ldots$的集
合,可描述为\{偶数}或\{能被2整除的数}。
\item 集合元素为$\pm 1,\pm 3,\pm 5,\pm 7,\ldots,\pm (2n+1)\ldots$的集合,可描述为\{奇数\}或\{被2除余1的数\}。
\item $\{-\sqrt{2},\sqrt{2}\}$,可描述为$\{\text{平方为2的数}\}$。
\item 圆面上不在圆上的点叫做圆内的点。在平面$P$上以$O$为
圆心,5厘米长为半径的圆内的点所成的集合,可描述为\{在
平面$P$上和点$O$的距离小于5厘米的点\}。	
\end{enumerate}

集合的特征性质描述法,常常采用下面更一般的形式:
\[A=\{x|\alpha\}\]
其中$x$表示集合$A$的任一元素,$x|\alpha$表示元素$x$具有特征性
质$\alpha$, 而$A=\{x|\alpha\}$则表示由所有具有性质$\alpha$的元素所构
成的集合$A$. 这样一来,上述各例又可表示如下:
\begin{enumerate}
    \item $A=\{x|x\text{能被2整除}\}$
    \item $B=\{x|x\text{被2除余1}\}$
    \item $C= \{x|x^2=2\}$
    \item $D=\{X|\overline{OX}<5{\rm cm},\; \text{且$O$是平面$P$上定点,$X\in$平面$P$} \}$
\end{enumerate}

有时候“任一元素$x$”也可用某种形式写出来,例如上面的集合$A$、$B$可写为:
\[\begin{split}
    A&=\{2n|n\text{为任意整数}\}=\{2n|n\in\mathbb{Z}\}\\
    B&=\{2n+1|n\text{为任意整数}\}=\{2n+1|n\in\mathbb{Z}\}
\end{split}\]

\begin{ex}
\begin{enumerate}
    \item 用列举法表示下列集合:
\begin{enumerate}
    \item 头五个质数的全体构成的集合。
    \item 12的所有因数构成的集合。
    \item 自然数里头五个平方数的全体构成的集合。
    \item 20与30间的奇数的全体构成的集合。
    \item 小于20的全体偶数构成的集合$P$。
\end{enumerate}

\item 用符号“$\in$”,“$\notin$”表示$b$, $c$, $d$与集合的关系。
\begin{enumerate}
    \item $A=\{x|x\text{是15的因数}\}$,$b=5$, $c=15$, $d=12$;
    \item $O=\{x|x\text{是小于16的质数}\}$,$b=2$, $c=3$, $d=7$。
\end{enumerate}

\item $C$是平面$P$上以$O$为圆心,半径为3cm的圆周上的所有点组
成的集合,$X$、$Y$、$Z$是平面$P$上的三个点,且$\overline{OX}=5{\rm cm}$, $\overline{OY}=2{\rm  cm}$, $\overline{OZ}=3{\rm cm}$, 试用符号“$\in$”和“$\notin$”表示$X$,
$Y$, $Z$与集合$C$的关系。

\item 试用特征性质描述法描述下列集合。
\begin{enumerate}
    \item 一元二次方程$x^2+2x-3=0$的两个根所构成的集合。
    \item 所有加7就大于15的实数所构成的集合。
    \item 所有大于或等于3而小于5的实数的集合。
\end{enumerate}

\item 试用特征性质描述第1题中的五个集合。

\item 把下列集合用列举法描述出来:
\begin{enumerate}
    \item $A=\{x|x\text{是整数且}|x|<5\}$
    \item $B=\{x|x\text{是英语中的元音字母}\}$
    \item $C=\{x|x\text{是整数且}1<x<10  \}$
    \item $D=\{x|x\text{是$a$或是$b$,或是$c$}\}$
\end{enumerate}
\end{enumerate} 
\end{ex}

\subsection{集合与集合的关系和集合的运算}

\subsubsection{集合与集合的关系}
如果集合$A$的每一个元素,也是集合$B$的元素,那么我们说$A$是$B$的\textbf{子集}。也可以说“$A$\textbf{含于}$B$”,或“$B$\textbf{包含}$A$”,我们用符号$A\subseteq B$, 或$B\supseteq A$来表示,请同学们注意:因为集合$A$的每个元素肯定是集合$A$的一个元素,所以每个集合$A$都是它本身的子集。

为了能够形象化地帮助我们理解集合,我们常用图来表示集合。最常用的方法是对给定的集合用圆形表示,圆形上的点表示该集合的元素,圆形外的点表示不是该集合的元素。不同的圆形表示不同的集合。这种圆形通常叫做维恩(Venn)图。

这样,集合$A$是集合$B$的子集,可形象地用图2.3所示的两圆形来表示。

如果集合$A$中的每一个元素都是集合$B$中的元素,而$B$中的元素却有不属于$A$的,这时我们说$A$是$B$的\textbf{真子集},记作$A\subset B$。
\begin{figure}[htp]\centering
    \begin{minipage}[t]{0.48\textwidth}
    \centering
\begin{tikzpicture}[>=latex, scale=.9]
\draw (0,0) circle (.6);
\draw (0,-.5) circle (1.3);
\node at (0,0){$A$};
\node at (-.7,-1){$B$};
    \end{tikzpicture}
    \caption{}
    \end{minipage}
    \begin{minipage}[t]{0.48\textwidth}
    \centering
    \begin{tikzpicture}[>=latex, scale=.9]
\draw[<->](-2,1)--(2,1);
\draw (-.8,1) [fill=black]circle (1.5pt)node[below]{$A$};
\draw (.8,1) [fill=black]circle (1.5pt)node[below]{$B$};
    \end{tikzpicture}
    \caption{}
    \end{minipage}
    \end{figure}

在图2.4中,$\overline{AB}$上的点都是直线$AB$上的点,但是直线$AB$上的点却还有很多不属于$\overline{AB}$, 所以$\overline{AB}$是直线$AB$的真子集,记作:
\[\overline{AB}\subset \text{直线}AB\]

这里我们要提醒同学们注意区分“属于”关系和“含于”关系。“属于”关系是集合的元素与集合本身的关系,但“含于”关系却是集合与集合之间的关系。

例如集合$A=\{3, 4, 5\},\; B=\{3, 4, 5, 6, 7\}$, 对于元素4来说,它和$A$或$B$的关系是“属于”关系,即$4\in A$, $4\in B$; 对于集合$A$与$B$来说,它们的关系却是“含于”关系,即$A\subset B$。

如果两个集合$A$、$B$是由共同的元素所构成的,我们称它们为相等的集合,记作$A=B$. 例如,$\{x,y,z\}=\{y,x, z\}$, $\{+, -, \x, \div \}=\{ \x, -,+,\div\}$等。

如果两个集合$A$、$B$, $A\subseteq B$且$B\subseteq A$, 这就是说$A$的每个元素都是$B$的元素,而$B$的每个元素也都是$A$的元素,显然这两个集合含有相同的元素,则$A=B$. 例如,$A=\{\text{偶数}\}$,$B=\{2n|n\in\mathbb{Z}\}$, 则$A=B$. 事实上$A$, $B$两个集合
就是同一个集合的两种不同描述法。

如果有三个集合$A$、$B$、$C$, $A\subseteq B$且$B\subseteq C$, 那么显然有$A\subseteq C$. 这就是说集合的含于关系具有传递性。

\subsubsection{集合的运算}

\paragraph{交集}
由集合$A$与集合$B$的公共元素所成的集合叫做集合$A$与集合$B$的交集(或交)。记为:
$A\cap B$,读为$A$交$B$.
\begin{figure}[htp]\centering
    \begin{minipage}[t]{0.48\textwidth}
    \centering
\begin{tikzpicture}[>=latex, scale=1]
    \draw (0,0)node{$A$} circle(.75);
    \draw (1.3,0)node{$B$} circle (1);
    \draw[->](.5,-1.3)node[below]{$A\cap B$}--(.5,-.7);
    \clip {(0,0) circle (.75)};
    \fill[pattern=north east lines] {(1.3,0) circle (1)};
    
    \end{tikzpicture}
    \caption{}
    \end{minipage}
    \begin{minipage}[t]{0.48\textwidth}
    \centering
    \begin{tikzpicture}[>=latex, scale=.9]
\draw (-2,.8)--(2,-.8)node[right]{$m$};
\draw (-1.5,-1)--(1.5,1)node[right]{$\ell$};
\node at (0,0)[above]{$A$};
    \end{tikzpicture}
    \caption{}
    \end{minipage}
    \end{figure}


两个集合$A$、$B$的交集用维恩图来示意,如图2.5中的阴影部分就表示$A\cap B$.

若$A=\{ a, b, c, d \}$, $B=\{ c, d, e\}$, 则$A\cap B=\{c,d\}$。

在图2.6中,直线$\ell$与直线$m$的交集是$A$点,即$\ell \cap m=\{A\text{点}\}$。
\begin{figure}[htp]\centering
    \begin{minipage}[t]{0.48\textwidth}
    \centering
\begin{tikzpicture}[>=latex, scale=.9]
\draw[<->](-2,1)--(2,1);
\draw (-.8,1) [fill=black]circle (1.5pt)node[below]{$A$};
\draw (.8,1) [fill=black]circle (1.5pt)node[below]{$B$};
    \end{tikzpicture}
    \caption{}
    \end{minipage}
    \begin{minipage}[t]{0.48\textwidth}
    \centering
    \begin{tikzpicture}[>=latex, scale=.9]

    \end{tikzpicture}
    \caption{}
    \end{minipage}
    \end{figure}

在图2.7中,$\overline{AB}$是射线$AB$和射线$BA$的交集,即:
\[\overline{AB}=\text{射线}AB\cap \text{射线}BA\]

一条直线把一个平面分成两部分,其中每一部分都叫做半平面,这条直线叫作半平面的界。图2.8中,$\angle AOB$的内部是以直线$OA$为界含有射线$OB$的半平面与以直线$OB$为界含有射线$OA$的半平面的交集(即图中的阴影部分)。

\begin{figure}[htp]\centering
    \begin{minipage}[t]{0.48\textwidth}
    \centering
\begin{tikzpicture}[>=latex, scale=1]
\fill[draw, pattern=north east lines] (0,0)node[fill=white]{$A$} circle(.75);
\fill[draw, pattern=north east lines]  (1.3,0)node[fill=white]{$B$} circle (1);
    \end{tikzpicture}
    \caption{}
    \end{minipage}
    \begin{minipage}[t]{0.48\textwidth}
    \centering
    \begin{tikzpicture}[>=latex, scale=.9]
\draw[<->](-2,1)--(2,1);
\draw (-.8,1) [fill=black]circle (1.5pt)node[below]{$A$};
\draw (.8,1) [fill=black]circle (1.5pt)node[below]{$B$};
    \end{tikzpicture}
    \caption{}
    \end{minipage}
    \end{figure}

显然,若$A\subseteq B$, 则$A\cap B=A$, 反之,若$A\cap B=A$, 则$A\subseteq B$.

\paragraph{并集}
由集合$A$的元素或集合$B$的元素合并而成的
集合叫做集合$A$与集合$B$的并集(或并)记为:
$A\cup B$,读为A并B。

集合$A$与集合$B$的并集用维恩图示意,如图2.9所示,图中的阴影部分就表示$A\cup B$.

如果$A=\{1, 2, 3\}$, $B=\{3, 4, 5\}$, 那么$A\cup B= \{1, 2, 3, 4, 5\}$.

\begin{rmk}
    在求上述$A$、$B$的并集时,虽然$A$与$B$含有共同的元素3, 但在$A\cup B$中3只取一次。
\end{rmk}

如果$A_+=\{x|x\text{是实数,且}x\ge 5\}$,$A_-=\{x|x\text{是实数,且}x\le -5\}$,$A=\{x|x^2\ge 25\}$,那么$A_+\cup A_-=A$。

在图2.10中,直线$AB$是射线$AB$和射线$BA$的并集,即
\[\text{直线}AB=\text{射线}AB \cup \text{射线}BA\]

\paragraph{空集}
为了使集合$A$和集合$B$的交集$A\cap B$在集合$A$与集合$B$不含有任何公共元素时仍有意义,我们自然想到:这时的$A\cap B$应是一个不含有任何元素的集合。因此,我们把这种不含任何元素的集合叫做\textbf{空集},并用符号$\emptyset$表示。

由空集的意义可知,对任何一个集合$P$, 都有$P\cap \emptyset=\emptyset$, $P\cup\emptyset=P$成立,并且空集是任何一个集合$P$的子集。即:$P\supseteq \emptyset$

如果$A=\{\text{奇数}\}$,$B=\{\text{偶数}\}$,那么$A\cap B=\emptyset$。

在图2.11中,若$\odot O_1$和$\odot O_2$相离,则$\odot O_1\cap \odot O_2=\emptyset$
\begin{figure}[htp]
    \centering
\begin{tikzpicture}
\draw (0,0)node[below]{$O_1$} circle (1);
\draw (2.5,0)node[below]{$O_2$} circle (.8);
\draw (0,0) [fill=black] circle(1pt); 
\draw (2.5,0) [fill=black] circle(1pt); 
\end{tikzpicture}
    \caption{}
\end{figure}

如果直线$a\parallel$直线$b$, 那么$a\cap b=\emptyset$.

\paragraph{基集与补集}

如果我们所讨论的集合都是一个给定集合的子集,我们就称这个给定集合为\textbf{基集},通常用符号$I$表示基集。

例如,我们所讨论的集合是$\emptyset$、$\{1\}$、$\{2\}$、$\{3\}$、$\{1, 2\}$、$\{2, 3\}$、$\{1, 3\}$和$\{1, 2, 3\}$时,因为上述集合都是集合$\{1, 2, 3\}$的子集,所以我们称$\{1, 2, 3\}$为基集。

又因为平面和平面上的几何图形都是点的集合,那么平面几何所讨论的图形都可以看作是某平面上所有点的集合的
子集,这时,我们把平面叫做基集。

在代数中,当我们讨论有理数运算时,全体有理数就是基集。

如果$A\subseteq I$, 在基集$I$中所有不属于$A$的元素所构成的集
合叫做$A$的\text{补集},以符号“$A^c$”表示,读为$A$补。因此,




\begin{figure}[htp]\centering
    \begin{minipage}[t]{0.48\textwidth}
    \centering
\begin{tikzpicture}[>=latex, scale=1.3]
\fill[pattern=north east lines, draw](0,0) rectangle (3.5,2.5);
\draw [fill=white](1.5,1)node{$A$} circle (.6);
\node at (3,2)[fill=white]{$I$};
\node at (3,1)[fill=white]{$A^c$};
    \end{tikzpicture}
    \caption{}
    \end{minipage}
    \begin{minipage}[t]{0.48\textwidth}
    \centering
    \begin{tikzpicture}[>=latex, scale=1.3]
\fill[pattern=north east lines, draw](0,0) rectangle (3.5,2.5);
\draw [fill=white](1.5,1)node[left]{$O$} circle (.6);
\node at (1.5,.75){$B^c=A$};
\draw [->](1.5,1)--node[above]{3}+(30:.6);
\node at (1.2,1.35){$A$};
\node at (3,2)[fill=white]{$P$};
\node at (2.8,1)[fill=white]{$A^c=B$};
\node at (.5,.5)[fill=white]{$B$};
    \end{tikzpicture}
    \caption{}
    \end{minipage}
    \end{figure}



\begin{figure}[htp]\centering
    \begin{minipage}[t]{0.48\textwidth}
    \centering
\begin{tikzpicture}[>=latex, scale=1.3]
\fill[pattern=north east lines, draw](0,0) rectangle (3.5,2.5);
\draw [fill=white](1.8,1.2)node{$A$} circle (1);
\draw [fill=white](2,1)node{$A$} circle (.5);
\node at (1.2,1.4){$B$};
\node at (3.1,.25)[fill=white]{$B^c$};
\node at (.5,2.2)[fill=white]{$I$};

    \end{tikzpicture}
    \caption{}
    \end{minipage}
    \begin{minipage}[t]{0.48\textwidth}
    \centering
    \begin{tikzpicture}[>=latex, scale=1.3]
\fill[pattern=north east lines, draw](0,0) rectangle (3.5,2.5);
\draw [fill=white](1.8,1.2) circle (1);
\draw [fill=white](2,1)node{$A$} circle (.5);
\node at (3.1,.25)[fill=white]{$B^c$};
\node at (.5,2.2)[fill=white]{$I$};
\node at (1.2,1.4){$A^c$};

    \end{tikzpicture}
    \caption{}
    \end{minipage}
    \end{figure}


























  \chapter{直线形}

在第一章里,我们从日常生活所熟悉的位置、通路、方
向、叠合出发,讨论了空间的几个重要的基本概念:点、直
线、平行、全等、相似,并通过观察、实验,分析归纳出了
空间的一些性质。在第二章中,我们把其中的某些性质作为
基本性质和定理。在本章中,我们将以这些基本性质和定理
为基础,运用第二章所介绍的演绎法去推演空间的其它性
质。演绎法不但是研究几何学的基本有效方法,在其它任何
科学的研究中也都是十分重要的方法。概括地说,对于科学
研究,实验归纳和论证推演是互相配合使用的两种基本科学
方法,它们是探索科学规律的两条腿。从这一章起,我们对
空间性质的探讨,主要用演绎法来进行。
\section{三角形}

\subsection{全等三角形}

\begin{blk}{定义}
    平面上顺次首尾端点相接且不在同一条直线上的
线段组成的封闭图形叫做\textbf{多边形}。这些线段叫做\textbf{多边形的边},
它们的端点叫做\textbf{多边形的顶点},每相邻两边的夹角叫做多边
形的\textbf{内角}。
\end{blk}


三角形是多边形中最简单的图形。有四条边的多边形叫
做四边形,有五条边的多边形叫五边形……等等。表示一个
多边形可用顶点的名称,沿周界顺次列出,如图3.1中的
$\triangle ABC$,四边形$ABCD$……等等。

如果多边形都在每边所在直线的同旁,我们称这种多边
形为\textbf{凸多边形}(图3.1中的三个图形都是凸多边形,图3.2
中的图形则不是)。以后我们说多边形时,都指的是凸多边
形。
\begin{figure}[htp]
    \centering
    \begin{tikzpicture}
\begin{scope}
\draw(0,0)node[below]{$B$}--(2,0)node[below]{$C$}--(1.5,1)node[above]{$A$}--(0,0);
\end{scope}
\begin{scope}[xshift=3cm]
\draw (0,0)node[left]{$B$}--(1,-1)node[below]{$C$}--(3,0)node[right]{$D$}--(1.2,1.5)node[above]{$A$}--(0,0);
\end{scope}
\begin{scope}[xshift=8cm, yshift=1.5cm]
\draw (0,0)node[above]{$A$}--(1.5,0)node[above]{$F$}--(2.5,-1)node[right]{$E$}--(1.4,-1.7)node[below]{$D$}--(0,-1.5)node[below]{$C$}--(-.4,-.7)node[left]{$B$}--(0,0);
\draw[thick](-1,0)--(3,0);
\end{scope}        
    \end{tikzpicture}
    \caption{}
\end{figure}

\begin{figure}[htp]
    \centering
\begin{tikzpicture}
\draw[dashed](-2,0)--(2,0);
\draw (-.5,0)--(0.2,0)--(.3,-1)--(.7,1.8)--(-.5,0);
\end{tikzpicture}
    \caption{}
\end{figure}

\begin{blk}{定义}
两个能够完全叠合的三角形叫做\textbf{全等三角形}。两
个全等三角形完全叠合时,互相叠合的顶点叫做\textbf{对应点},互
相叠合的边叫做\textbf{对应边},互相叠合的角叫做\textbf{对应角}。因此,
\textbf{全等三角形的对应边相等,对应角相等}。
\end{blk}
 
怎样判定两个三角形全等呢?
\begin{enumerate}
\item 有两边和它们的夹角对应相等的两个三角形全
等。(SAS)
\item 有两角和它们的夹边对应相等的两个三角形全
等。(ASA)
\item 有三边对应相等的两个三角形全等。(SSS)
\end{enumerate}

利用三角形的全等,是判断两条线段或两个角相等的一
种基本方法。

\begin{example}
    在图3.3中,已知$\overline{AB}=\overline{AC}$, $\angle B=\angle C$
    
    求证:$\overline{BD}=\overline{CE}$.
\end{example}



\begin{analyze}
    要证$\overline{BD}=\overline{CE}$, 从图
上看$\overline{BD}$, $\overline{CE}$分别是$\triangle ABD$和
$\triangle ACE$的边,因此只要证明
$\triangle ACE \cong \triangle ABD$就行了,由已
知条件$\overline{AC}=\overline{AB}$, $\angle B=\angle C$而
$\angle A$是公共角,所以$\triangle ABD$与
$\triangle ACE$全等是很显然的。
\end{analyze}

\begin{proof}
在$\triangle ABD$与$\triangle ACE$中,

$\because\quad \overline{AB}=\overline{AC},\quad \angle B=\angle C$(已知)。

而$\angle A=\angle A$(公共角),

$\therefore\quad \triangle ABD\cong \triangle ACE$ (ASA).

$\therefore\quad \overline{BD}=\overline{CE}$ (全等三角形的对应边相等)。
\end{proof}    


\begin{figure}[htp]\centering
    \begin{minipage}[t]{0.48\textwidth}
    \centering
\begin{tikzpicture}[>=latex, scale=.8]
       \draw (0,0)node[left]{$A$}--(5,0)node[right]{$B$};
\draw (0,0)--(45:5)node[right]{$C$};
\draw (2,0)node[below]{$E$}--(45:5);
\draw (45:2)node[above]{$D$}--(5,0);
    \end{tikzpicture}
    \caption{}
    \end{minipage}
    \begin{minipage}[t]{0.48\textwidth}
    \centering
    \begin{tikzpicture}[>=latex, scale=1]
        \draw (0,0)node[left]{$B$}--(3,0)node[right]{$C$}--(4,2)node[right]{$D$}--(1,2)node[left]{$A$}--(0,0);
        \draw[dashed](0,0)--(4,2);   
    \end{tikzpicture}
    \caption{}
    \end{minipage}
    \end{figure}


\begin{example}
    已知:在四边形$ABCD$中,$\overline{AD}=\overline{BC}$,
$\overline{AB}=\overline{CD}$(图3.4)。

求证:$\angle A=\angle C$。
\end{example}

\begin{analyze}
    要证明$\angle A=\angle C$, 
需要把四边形$ABCD$分成两个三
角形,为此,连结$B$、$D$. 这叫
做添\textbf{辅助线}。这样只需证
$\triangle ABD\cong \triangle CDB$就行了。
\end{analyze}

\begin{proof}
连结$B$、$D$, 在$\triangle ABD$与$\triangle CDB$中,

$\because\quad \overline{AD}=\overline{BC},\quad \overline{AB}=\overline{CD}$ (已知)

又$\because\quad \overline{BD}=\overline{BD}$ (公共边)

$\therefore\quad \triangle ABD\cong \triangle CDB$ (SSS)

$\therefore\quad \angle A=\angle C$(全等三角形的对应角相等)。
\end{proof}    

\begin{example}
在图3.5中,已知:$\overline{AB}=\overline{CD}$, $\angle B=\angle CDF$, $\overline{BD}=\overline{EF}$.

求证:$\overline{AE}=\overline{CF}$.
\end{example}

\begin{analyze}
    要证$\overline{AE}=\overline{CF}$, 只需证$\triangle ABE\cong \triangle CDF$. 由已
知,$\overline{AB}=\overline{CD}$, $\angle B=\angle CDF$, $\overline{BD}=\overline{EF}$, 虽然不能马上说
$\triangle ABE$和$\triangle CDF$全等,但只要注意到$\overline{BD}+\overline{DE}=\overline{DE}+\overline{EF}$, 
即$\overline{EB}=\overline{DF}$就行了。
\end{analyze}

\begin{proof}
    在图3-5中,$\because\quad \overline{BD}=\overline{EF}$ 已知

$\therefore\quad \overline{BD}+\overline{DE}=\overline{DE}+\overline{EF}$ (等量加等量和相等)。即:
\[\overline{BE}=\overline{DF}\]
又$\because\quad \overline{AB}=\overline{CD},\; \angle B =\angle CDF$ 已知

$\therefore\quad \triangle ABE\cong \triangle CDF$ (SAS).

$\therefore\quad \overline{AE}=\overline{CF}$ (全等三角形的对应边相等)。
\end{proof}

\begin{figure}[htp]\centering
    \begin{minipage}[t]{0.48\textwidth}
    \centering
\begin{tikzpicture}[>=latex, scale=1]
       \draw(0,0)node[left]{$B$}--(2,0)node[below]{$E$}--(.8,2)node[above]{$A$}--(0,0);
\draw(1.5,0)node[below]{$D$}--(3.5,0)node[right]{$F$}--(2.3,2)node[above]{$C$}--(1.5,0);
    \end{tikzpicture}
    \caption{ }
    \end{minipage}
    \begin{minipage}[t]{0.48\textwidth}
    \centering
    \begin{tikzpicture}[>=latex, xscale=.8]
      \draw(0,0)node[left]{$B$}--(5,0)node[right]{$D$};
      \draw(1,1.5)node[above]{$A$}--(2.5,0)node[above right]{$E$}--(1,-1.5)node[below]{$C$}--(0,0)--(1,1.5)--(5,0)--(1,-1.5);
    \end{tikzpicture}
    \caption{ }
    \end{minipage}
    \end{figure}

\begin{example}
    在图3.6中,已知:$\overline{AB}=\overline{BC}$, $\overline{AD}=\overline{CD}$, $E$点在$BD$上。

求证:$\overline{AE}=\overline{CE}$.
\end{example}

\begin{analyze}
    要证$\overline{AE}=\overline{CE}$, 只需证明$\triangle ABE\cong \triangle CBE$, 或者证明$\triangle ADE\cong \triangle CDE$, 假定我们证明$\triangle ABE\cong \triangle CBE$,
    已知$\overline{AB}=\overline{BC}$, $\overline{BE}=\overline{BE}$, 因此只需证明$\angle ABD=\angle CBD$;
    要证$\angle ABD=\angle CBD$, 只需证明$\triangle ABD\cong\triangle CBD$.
\end{analyze}

\begin{proof}
    在$\triangle ABD$与$\triangle CBD$中,

    $\because\quad \overline{AB}=\overline{CB},\quad \overline{AD}=\overline{CD}$ (已知)   $\overline{BD}=\overline{BD}$(公共边)

    $\therefore\quad \triangle ABD\cong \triangle CBD$ (SSS).

    $\therefore\quad \angle ABD=\angle CBD$ (全等三角形的对应角相等)

$\because\quad     \overline{AB}=\overline{BC}$ (已知) 
    $\overline{BE}=\overline{BE}$ (公共边)

$\therefore\quad \triangle ABE\cong \triangle CBE$ (SAS).

$\therefore\quad \overline{AE}=\overline{CE}$ (全等三角形的对应边相等)。
\end{proof}

    利用三角形全等,来证明两条线段或两个角相等,关键
    在于找出能够全等的三角形,并且使要证明的线段和角恰好
    成为它们的对应边和对应角。为了找出全等的三角形,必要
    时需要添加辅助线。  

\begin{ex}
\begin{enumerate}
    \item 已知:在四边形$ABCD$中,$AC$平分$\angle BAD$, $\overline{AB}=\overline{AD}$.
    
    求证:$\angle ACB=\angle ACD$.
    \item 已知:如图,$\overline{AC}$、$\overline{BD}$交于$O$点,    且$\overline{AO}=\overline{OC}$、$\overline{BO}=\overline{OD}$.
    
    求证:$\overline{AB}=\overline{CD}$.


\item 已知:如图,$\angle 1=\angle 4$, $\angle 2=\angle 3$.
求证:$\overline{AB}=\overline{CD}$.
\item 已知:如图,$\angle 1=\angle 2$, $\angle 3=\angle 4$, $\overline{AB}=\overline{AD}$. 

求证:$\overline{AE}=\overline{AC}$, $\angle E=\angle C$.

\item 已知:如图,$\angle 1=\angle 2$, $\angle 3=\angle 4$,
求证:$\overline{AC}=\overline{BD}$.
\item 已知:如图,在四边形$ABCD$中,$\overline{AB}=\overline{BC}$, $\overline{AD}=\overline{CD}$.

求证:$\angle A=\angle C$.
\item 已知:如图,$\overline{AD}=\overline{BE}$, $\overline{AE}=\overline{BD}$, AC、BC是直线。

求证:$\angle CDB=\angle CEA$.
\item 已知:如图,$\overline{AB}=\overline{CD}$, E、F分别是$\overline{AB}$、$\overline{CD}$的中点,
并且$\overline{BF}=\overline{CE}$.

求证:$\angle EBC=\angle FCB$, $\angle FBC=\angle ECB$.

\item 已知:如图,在四边形$ABCD$中,$\overline{AB}=\overline{CD}$, $\overline{AD}=\overline{BC}$, $\overline{EF}$过$\overline{BD}$的中点$O$. 

求证:$\overline{OE}=\overline{OF}$.
\end{enumerate}
\end{ex}

\begin{figure}[htp]\centering
    \begin{minipage}[t]{0.48\textwidth}
    \centering
\begin{tikzpicture}[>=latex, yscale=.8]
       \draw(0,2)node[above]{$A$}--(1,0)node[right]{$D$}--(0,-2)node[below]{$C$}--(0,2)--(-1,0)node[left]{$B$}--(0,-2);
    \end{tikzpicture}
    \caption*{第1题}
    \end{minipage}
    \begin{minipage}[t]{0.48\textwidth}
    \centering
    \begin{tikzpicture}[>=latex, scale=1.5]
      \draw(-1,.5)node[above]{$A$}--(1,-.5)node[below]{$C$}--(1.5,1)node[above]{$D$}--(-1.5,-1)node[left]{$B$}--(-1,.5);
      \node at (0,0)[below]{$O$};
    \end{tikzpicture}
    \caption*{第2题}
    \end{minipage}
    \end{figure}

\begin{figure}[htp]\centering
    \begin{minipage}[t]{0.48\textwidth}
    \centering
\begin{tikzpicture}[>=latex, scale=1.3]
\tkzDefPoints{1/2/A, 0/0/B, 3/0/C, 4/2/D}
\tkzDrawPolygon(A,B,C)
\tkzDrawPolygon(A,D,C)
\tkzMarkAngle[mark=none, size=.35](B,A,C)  
\tkzLabelAngle[pos=.5](B,A,C) {2}
\tkzMarkAngle[mark=none, size=.5](C,A,D)  
\tkzLabelAngle[pos=.7](C,A,D) {1}
 \tkzLabelAngle[pos=.75](A,C,B) {4}
\tkzMarkAngle[mark=none, size=.5](D,C,A)  
\tkzLabelAngle[pos=.25](D,C,A) {3}
\tkzMarkAngle[mark=none, size=.6](A,C,B) 
\tkzLabelPoints[left](A,B)
\tkzLabelPoints[right](C,D)
    \end{tikzpicture}
    \caption*{第3题}
    \end{minipage}
    \begin{minipage}[t]{0.48\textwidth}
    \centering
    \begin{tikzpicture}[>=latex, scale=1.3]
        \tkzDefPoint(0,0){A}
        \tkzDefPoint(-90:2){B}
        \tkzDefPoint(-60:2){D}
        \tkzDefPoint(0:2.75){E}
        \tkzDefPoint(-30:2.75){C}
        \tkzLabelPoints[left](A,B)
        \tkzDrawPolygon(A,B,D)
\tkzDrawLines[add=0 and 1.38](B,D) %\tkzGetPoint{C}
\draw (D)--(E)--(A)--(C);
\tkzLabelPoints[right](C,E)
\tkzLabelPoints[below](D)
\tkzMarkAngles[mark=none, size=.44](C,A,E B,A,D C,B,A E,D,A)  
\tkzLabelAngle[pos=.65](C,A,E) {2}
\tkzLabelAngle[pos=.65](B,A,D) {1}
\tkzLabelAngle[pos=.6](C,B,A) {3}
\tkzLabelAngle[pos=.6](E,D,A) {4}


    \end{tikzpicture}
    \caption*{第4题}
    \end{minipage}
    \end{figure}



\begin{figure}[htp]\centering
    \begin{minipage}[t]{0.48\textwidth}
    \centering
\begin{tikzpicture}[>=latex, scale=1]
\tkzDefPoints{-1/2/A, 1/2/D, -2/-1/B, 2/-1/C}
\tkzDrawPolygon(A,C,D) \tkzDrawPolygon(A,B,D)
\tkzLabelPoints[left](A,B)
\tkzLabelPoints[right](C,D)
\tkzMarkAngles[mark=none, size=.6](C,A,D A,D,B) 
\tkzMarkAngles[mark=none, size=.5](B,A,C B,D,C) 
\tkzLabelAngle[pos=.7](C,A,D) {1}
\tkzLabelAngle[pos=.7](A,D,B) {2}
\tkzLabelAngle[pos=.65](B,A,C) {3}
\tkzLabelAngle[pos=.65](B,D,C) {4}



    \end{tikzpicture}
    \caption*{第5题}
    \end{minipage}
    \begin{minipage}[t]{0.48\textwidth}
    \centering
    \begin{tikzpicture}[>=latex, scale=1.3]
        \tkzDefPoints{-1/0/B, 2/0/D, 1/1/A, 1/-1/C, 0/0/O} 
        \tkzDrawPolygon(A,B,C,D)
        \tkzAutoLabelPoints[center=O](A,B,C,D)

    \end{tikzpicture}
    \caption*{第6题}
    \end{minipage}
    \end{figure}




\begin{figure}[htp]\centering
    \begin{minipage}[t]{0.48\textwidth}
    \centering
\begin{tikzpicture}[>=latex, scale=1]
    \tkzDefPoints{-1.5/0/A, 1.5/0/B, 0/3/C, 0/1.5/O}  
    \tkzDefMidPoint(A,C)\tkzGetPoint{D}
    \tkzDefMidPoint(B,C)\tkzGetPoint{E}    
    \tkzDrawPolygon(A,B,C)
    \draw(B)--(D)node[left]{$D$};
    \draw (A)--(E)node[right]{$E$};
    \tkzAutoLabelPoints[center=O](A,B,C)
    \end{tikzpicture}
    \caption*{第7题}
    \end{minipage}
    \begin{minipage}[t]{0.48\textwidth}
    \centering
    \begin{tikzpicture}[>=latex, scale=.8]
        \tkzDefPoints{-2.5/0/B, 2.5/0/C, -1.5/3/A, 1.5/3/D,  0/1.5/O}  
        \tkzDefMidPoint(A,B)\tkzGetPoint{E}
        \tkzDefMidPoint(D,C)\tkzGetPoint{F}    
        \tkzDrawPolygon(A,B,C,D)
        \draw(B)--(F)node[right]{$F$};
        \draw (C)--(E)node[left]{$E$};
        \tkzAutoLabelPoints[center=O](A,B,C,D)
    \end{tikzpicture}
    \caption*{第8题}
    \end{minipage}
    \end{figure}


\begin{figure}[htp]\centering
    \begin{minipage}[t]{0.48\textwidth}
    \centering
\begin{tikzpicture}[>=latex, scale=.8]
    \tkzDefPoints{-2.5/0/B, 2.5/0/C, -1.5/3/A, 3.5/3/D}  
    \tkzDefMidPoint(D,B)\tkzGetPoint{O}
    \tkzDrawPolygon(A,B,C,D)
\draw (0,3)node[above]{$E$}--(1,0)node[below]{$F$};
\tkzAutoLabelPoints[center=O](A,B,C,D)
\draw(B)--(D);
\node at (O)[right]{$O$};

    \end{tikzpicture}
    \caption*{第9题}
    \end{minipage}
    \begin{minipage}[t]{0.48\textwidth}
    \centering
    \begin{tikzpicture}[>=latex, scale=1]
\tkzDefPoints{-1.5/0/B, 1.5/0/C, 0/3/A, 0/1.5/O}
\tkzDrawPolygon(A,B,C)
\tkzAutoLabelPoints[center=O](A,B,C)
\tkzMarkAngles[mark=none, size=.5](C,B,A A,C,B B,A,C) 
\node at (0,-.25){底};\node at (0,2.25){顶角};
\node at (-1,1.5){腰};\node at (1,1.5){腰};
\node(A) at (0,.25){底角};
\draw[<-](-1,.25)--(A);
\draw[<-](1,.25)--(A);
    \end{tikzpicture}
    \caption{}
    \end{minipage}
    \end{figure}

\subsection{等腰三角形}

\begin{blk}{定义}
    有两条边相等的三角形叫做\textbf{等腰三角形}。相等的两边
叫做\textbf{腰},另外的一边叫做\textbf{底},腰和底的夹角叫做\textbf{底角},两腰的夹
角叫\textbf{顶角},如图3.7所示。
\end{blk}

\begin{blk}{定义}
    三角形的一个角的平分线与对边相交,这个角的
    顶点和交点之间的线段叫做\textbf{三角形的角的平分线}。在图
    3.8(1)中,$\overline{AF}$平分$\angle A$, 交对边于$F$点,$\overline{AF}$就是$\triangle ABC$
    的$\angle A$的平分线。  

    连结三角形一个顶点和它的对边中点的线段叫做\textbf{三角形
的中线}。在图3.8(2)中,$E$点是$\overline{BC}$的中点,$\overline{AF}$就是$\triangle ABC$的$\overline{BC}$边上的中线。

从三角形一个顶点到它的对边所在直线作垂线,顶点和
垂足之间的线段叫做\textbf{三角形的高线}(简称\textbf{高})。在图3.8(3)
中,$\overline{AD}\bot$直线$BC$, $D$是垂足,$\overline{AD}$就是$\triangle ABC$的$\overline{BC}$边上
的高线。
\end{blk}

\begin{figure}[htp]
    \centering
\begin{tikzpicture}[scale=.8]
\begin{scope}
\tkzDefPoints{0/0/B, 3.6/0/F, 5.5/0/C, 4.5/3/A}
\tkzDrawPolygon(A,B,C)
\draw(A)--(F);
\tkzMarkAngles[mark=none, size=.5](B,A,F) 
\tkzMarkAngles[mark=none, size=.6](F,A,C)
\tkzLabelAngle[pos=.7](B,A,F) {1}
\tkzLabelAngle[pos=.75](F,A,C) {2}

\tkzLabelPoints[below](C, F, B)
\tkzLabelPoint(A){$A$}
\node at (2.7,-1){(1)};
\end{scope}
\begin{scope}[xshift=7cm]
    \tkzDefPoints{0/0/B, 2/0/E, 4/0/C, 4.5/3/A}
    \tkzDrawPolygon(A,B,C)
    \tkzLabelPoints[below](C, E, B)
    \tkzLabelPoint(A){$A$}
    \draw(E)--(A);
    \node at (2.2,-1){(2)};
\end{scope}    
\begin{scope}[yshift=-5cm]
\draw (0,0)node[below]{$B$}--(5.5,0)node[below]{$C$}--(4.5,3)node[above]{$A$}--(0,0);
\tkzDefPoints{4.5/3/A1, 4.5/0/D1, 5.5/0/C1}
\tkzMarkRightAngle(A1,D1,C1)
\draw(4.5,3)--(4.5,0)node[below]{$D$};
\draw(11,3)--(11,0)node[below]{$D$};
\draw[dashed](9,0)--(12,0);
\draw (7,0)node[below]{$B$}--(9.5,0)node[below]{$C$}--(11,3)node[above]{$A$}--(7,0);
\tkzDefPoints{11/3/A2, 11/0/D2, 9.5/0/C2}
\tkzMarkRightAngle(A2,D2,C2)
\node at (6,-1){(3)};
\end{scope}
\end{tikzpicture}
    \caption{}
\end{figure}

三角形的高线、中线、角平分线,一般是指一条线段,
但有时当我们不考虑其长度时,也把它们分别所在的直线叫
做三角形的高线、中线、角的平分线。


\begin{blk}
    {等腰三角形性质定理} 等腰三角形底角相等。
\end{blk}

已知:在$\triangle ABC$中;$AB=AC$.
求证:$\angle B=\angle C$.

\begin{figure}[htp]
    \centering
\begin{tikzpicture}
\draw (0,0)node[below]{$B$}--(3,0)node[below]{$C$}--(1.5,4)node[above]{$A$}--(0,0);
\draw[dashed](1.5,4)--(1.5,0)node[below]{$D$};
\end{tikzpicture}
    \caption{}
\end{figure}
\begin{proof}
    作$\angle BAC$的平分线$\overline{AD}$
(图3.9),在$\triangle ABD$和$\triangle ACD$
中

$\because\quad AB=AC$ (已知),$\overline{AD}=\overline{AD}$(公共边),$\angle BAD=\angle CAD$(角平分线定义)

$\therefore\quad \triangle ABD≤\triangle ACD$(SAS)

$\therefore\quad \angle B=\angle C$(全等三角形的对应角相等)。
\end{proof}


由于 $\overline{BD}=\overline{DC},\; \angle BDA=\angle CDA=90^{\circ}$

因此 $AD$平分$\overline{BC}$, 且$AD\bot BC$.

\begin{blk}{推论 }
    等腰三角形顶角的平分线垂直、平分底边。
\end{blk}

也就是说,等腰三角形的顶角平分线也是底边上的高线
和中线。(\textbf{三线合一})

\begin{blk}
    {等腰三角形的判定定理} 有两个角相等的三角形是等腰
三角形。
\end{blk}

已知:在$\triangle ABC$中,$\angle B=\angle C$(图3.10)。
求证:$\overline{AB}=\overline{AC}$.

\begin{figure}[htp]
    \centering
\begin{tikzpicture}
\draw(0,0)node[left]{$B$}--(2,0)node[right]{$C$}--(1,2)node[above]{$A$}--(0,0);
\draw(4,2)node[left]{$B'$}--(6,2)node[right]{$C'$}--(5,0)node[below]{$A'$}--(4,2);
\end{tikzpicture}
    \caption{}
\end{figure}


\begin{proof}
    根据翻转公理,我们可以把$\triangle ABC$翻转过来,
    设顶点$A$、$B$、$C$成为$A'$、$B'$、$C'$.
    
    $\because\quad \angle B=\angle C=\angle C',\quad \angle C=\angle B=\angle B'$

    又:$\because\quad \overline{BC}=\overline{C'B'}$

    $\therefore\quad \triangle ABC\cong \triangle A'C'B'$(ASA)

    $\therefore\quad \overline{AB}=\overline{A'C'}$(全等三角形的对应边相等)。

由于$\overline{AC}=\overline{A'C'}$,$\therefore\quad \overline{AB}=\overline{AC}$(等量代换)
\end{proof}

用逻辑语句说:等腰三角形的判定定理是其性质定理的
逆定理。这两个定理我们用“充要”条件可合写成一个定理:

\begin{blk}{}
   一个三角形是等腰三角形的充要条件是这个三角形有两
个角相等。
\end{blk}


\begin{blk}{定义}
    三条边都相等的三角形叫做\textbf{等边三角形},也叫做
    \textbf{正三角形}。
\end{blk}


同学们自己证明下面等边三角形的性质定理和判定定
理。

\begin{blk}{}
    \begin{itemize}
        \item 等边三角形的三内角相等。
        \item   三内角相等的三角形是等边三角形。
    \end{itemize}
\end{blk}

由等腰三角形及等边三角形的性质定理和判定定理可
知,在一个三角形中,由边的相等可以推知角的相等,反过
来由角的相等也可推知边的相等。下面举例说明它们在证题
中的应用。

\begin{example}
已知:在图3.11中,$\overline{AB}=\overline{EB}$, 
$\overline{AC}=\overline{DC}$, 
ADB、AEC是直线。

求证:$\angle ADC=\angle AEB$.
\end{example}

\begin{analyze}
    要证$\angle ADC=\angle AEB$,
只需证明$\angle ADC=\angle A$, $\angle AEB=\angle A$; 要证明
$\angle ADC=\angle A$, $\angle AEB=\angle A$, 只要知道$\overline{AC}=\overline{DC}$, 
$\overline{AB}=\overline{BE}$就行了。
\end{analyze}

\begin{proof}
    在$\triangle BAE$中,
$\because\quad \overline{AB}=\overline{EB}$(已知),

$\therefore\quad \angle AEB=\angle A$(等腰三角形的底角相等)。

在$\triangle CAD$中,
$\because\quad \overline{AD}=\overline{DC}$(已知),

$\therefore\quad \angle ADC=\angle A$(等腰三角形的底角相等)。

$\therefore\quad \angle ADC=\angle AEB$ (等量代换)。
\end{proof}

\begin{figure}[htp]\centering
    \begin{minipage}[t]{0.48\textwidth}
    \centering
\begin{tikzpicture}[>=latex, scale=1]
\draw(40:3)node[above]{$A$}--(0,0)node[left]{$B$}--(20:3)node[right]{$E$};
\draw(40:3)--(3.34,0.122)node[right]{$C$}--(40:2.1)node[left]{$D$};
    \end{tikzpicture}
    \caption{}
    \end{minipage}
    \begin{minipage}[t]{0.48\textwidth}
    \centering
    \begin{tikzpicture}[>=latex, scale=1]
\draw(-2,0)node[below]{$A$}--(0,2.5)node[above]{$C$}--(2,0)node[below]{$B$}--(-2,0);
\draw(.8,0)node[below]{$E$}--(0,2.5)--(-.8,0)node[below]{$D$};
\draw (-2+.4,0) arc (0:51.34:.4);
\draw (2-.4,0) arc (180:180-51.34:.4);
\draw (-.8+.3,0) arc (0:72.3:.3)node[right]{1};
\draw (.8-.3,0) arc (180:180-72.3:.3)node[left]{2};
    \end{tikzpicture}
    \caption{}
    \end{minipage}
    \end{figure}


\begin{example}
    如图3.12, 
己知:$\overline{AC}=\overline{BC}$, $\overline{AE}=\overline{DB}$. 

求证:$\overline{CD}=\overline{CE}$.
\end{example}

\begin{analyze}
    在$\triangle CDE$中,若要证$\overline{CD}=\overline{CE}$,
   只要证$\angle 1=\angle 2$即可, $\angle 1$和$\angle 2$分别在$\triangle BCD$与$\triangle ACE$中,如能证明$\triangle BCD\cong \triangle ACE$, 即可证明$\angle 1=\angle 2$.
\end{analyze}

\begin{solution}
在$\triangle ACE$与$\triangle BCD$中,

$\because\quad \overline{AC}=\overline{BC},\quad \overline{AE}=\overline{BD}$ (已知)

$\therefore\quad \angle A=\angle B$(等腰三角形两底角相等),

$\therefore\quad \triangle ACE\cong \triangle BCD$(SAS)

$\therefore\quad \angle 2=\angle 1$(全等三角形的对应角相等),

$\therefore\quad \triangle CDE$ 是等腰三角形(有两角相等的三角形是等腰三角形)。

$\therefore\quad \overline{CD}=\overline{CE}$
\end{solution}

\begin{ex}
\begin{enumerate}
    \item 画出$\triangle ABC$和$\triangle DEF$的三条边上的高线。
    \item 画一个三角形$ABC$, 然后画出$\triangle ABC$三个内角的平分线。
    \item 画一个三角形$DEF$, 然后画出$\triangle DEF$三边上的中线。
    \item 证明:全等三角形的对应角的平分线相等。
    \item 证明:全等三角形对应边上的中线相等。
    \item 已知:$A=\{\text{等腰三角形}\}$,$B=\{\text{两内角相等的三角
    形}\}$,指出集合$A$与集合$B$的关系。
    \item 若$\overline{AC}=\overline{BC},\quad \angle DCA=\angle ECB$, 则$\overline{CD}=\overline{CE}$.
    \item 若$\overline{AC}=\overline{BC},\quad \angle 1=\angle 2$, 则$\overline{AE}=\overline{BD}$.
    \item 若$\overline{AC}=\overline{BC}$, $\overline{AD}$、$\overline{BE}$分别是$\angle A$和$\angle B$的平分线,
    则$\overline{AD}=\overline{BE}$.
    \item 若$\overline{AC}=\overline{BC},\quad \overline{AD}=\overline{BE}$, $DE$是直线,则$\triangle DEC$是等腰
三角形。
\item 若$\overline{AC}=\overline{BC},\quad \angle ACD=\angle BCE$, $DE$是直线,则$\triangle DEC$
是等腰三角形。
\item 在等边$\triangle ABC$的三边上,分别取$D$、$E$、$F$(如图),
使$\overline{AD}=\overline{BE}=\overline{CF}$, 则$\triangle DEF$是等边三角形。
\item 设$\overline{DE}=\overline{EF}=\overline{FD}$, $\angle AFD=\angle BDE=\angle CEF$, 
则$\triangle ABC$是等边三角形。
\end{enumerate}
\end{ex}

\begin{figure}[htp]
    \centering
\begin{tikzpicture}
\begin{scope}
    \draw(0,0)node[below]{$B$}--(1.5,.2)node[below]{$C$}--(2,2)node[above]{$A$}--(0,0);
\end{scope}
\begin{scope}[xshift=5cm]
    \draw(0,0)node[below]{$E$}--(2,0)node[below]{$F$}--(.7,1.5)node[above]{$D$}--(0,0);
\end{scope}
\end{tikzpicture}
    \caption*{第1题}
\end{figure}

\begin{figure}[htp]\centering
    \begin{minipage}[t]{0.48\textwidth}
    \centering
\begin{tikzpicture}[>=latex, scale=1]
\draw(-1.5,0)node[below]{$A$}--(0,2.5)node[above]{$C$}--(1.5,0)node[below]{$B$}--(-1.5,0);
\draw(.8,0)node[below]{$E$}--(0,2.5)--(-.8,0)node[below]{$D$};
    \end{tikzpicture}
    \caption*{第7题}
    \end{minipage}
    \begin{minipage}[t]{0.48\textwidth}
    \centering
    \begin{tikzpicture}[>=latex, scale=1]
\draw(-1.5,0)node[below]{$A$}--(0,2.5)node[above]{$C$}--(1.5,0)node[below]{$B$}--(-1.5,0);
\draw(-.75,1.25)node[left]{$E$}--(1.5,0);
\draw(.75,1.25)node[right]{$D$}--(-1.5,0);
\draw(-1.5+.4,0) arc (0:29:.4)node[right]{1};
\draw(1.5-.4,0) arc (180:180-29:.4)node[left]{2};
    \end{tikzpicture}
    \caption*{第8--9题}
    \end{minipage}
    \end{figure}

\begin{figure}[htp]\centering
    \begin{minipage}[t]{0.48\textwidth}
    \centering
\begin{tikzpicture}[>=latex, scale=1]
    \draw(-1.8,0)node[below]{$D$}--(0,2.5)node[above]{$C$}--(1.8,0)node[below]{$E$}--(-1.8,0);
    \draw(.8,0)node[below]{$B$}--(0,2.5)--(-.8,0)node[below]{$A$};
    \end{tikzpicture}
    \caption*{第10--11题}
    \end{minipage}
    \begin{minipage}[t]{0.48\textwidth}
    \centering
    \begin{tikzpicture}[>=latex, scale=1]
\draw(-120:3)node[below]{$B$} --(0,0)node[above]{$A$}--(-60:3)node[below]{$C$}-- (-120:3) ;
\draw(-60:1.8)node[right]{$F$}--(-120:1.2)node[left]{$D$}--(-.3,-1.5*1.732)node[below]{$E$}--(-60:1.8);
    \end{tikzpicture}
    \caption*{第12--13题}
    \end{minipage}
    \end{figure}

\subsection{轴对称图形}
\begin{blk}{定义}
    在平面上有两个图形$F$和$F'$, 如果平面沿着某
条直线$\ell$折叠起来,$F$和$F'$叠合,就称$F$和$F'$关于$\ell$成\textbf{轴对
称}。(也称$F$和$F'$是以$\ell$为轴的\textbf{对称形})。$F$和$F'$上互相叠合
的点叫做\textbf{对称点},$\ell$叫做\textbf{对称轴}。
\end{blk}

在图3.13(1)中,平面上的$\triangle ABC$和$\triangle A'B'C'$关于直
线$\ell$成轴对称;$A$与$A'$、$B$与$B'$、$C$与$C'$都是对称点。

\begin{figure}[htp]
    \centering
\begin{tikzpicture}
\begin{scope}
    \draw(0,-.5)--(0,4)node[above]{$\ell$};
    \node at (0,-1){(1)};

\draw(-1.8,2)node[above]{$B$}--(-.4,3)node[above]{$A$}--(-1.4,0.5)node[below]{$C$}--(-1.8,2);
\draw(1.8,2)node[above]{$B'$}--(.4,3)node[above]{$A'$}--(1.4,0.5)node[below]{$C'$}--(1.8,2);

\end{scope}
\begin{scope}[xshift=6.5cm]
    \draw(0,-.5)--(0,4)node[above]{$\ell$};
\draw(-2,0)node[below]{$B$}--(2,0)node[below]{$C$}--(0,3)node[right]{$A$}--(-2,0);
\draw (0,0) rectangle (.2,.2);
\node at (0,-1){(2)};
\end{scope}
\end{tikzpicture}
    \caption{}
\end{figure}

\begin{blk}{推论}
     两个图形如果关于某直线成轴对称,那么这两个
图形是全等形。
\end{blk}

\begin{blk}{定义}
    如果一个图形可以分成两部分,这两部分关于某
一直线成轴对称,就把这个图形称为\textbf{轴对称形}。
\end{blk}

显然,等腰三角形关于它的顶角平分线成轴对称图形
(图3.13(2))。轴对称形在实际中应用非常广泛,图3.14中
的图形,都是轴对称形。
\begin{figure}[htp]
    \centering
\includegraphics[scale=.6]{fig/3-14.png}
    \caption{}
\end{figure}

轴对称图形有什么性质呢?这就要研究一下对称轴和对
称点的关系。

在图3.15中,设$A$与$A'$是关于直线$\ell$的轴对称点,因为
$A$、$A'$和$\ell$在同一平面内,并且$A$和$A'$在$\ell$的两侧,所以线段
$\overline{AA'}$与$\ell$必相交,设交点为$O$, 在$\ell$上取异于$O$的另一点$P$, 连
结$\overline{AP}$,$\overline{A'P}$.

由于$A$点所在的半平面沿直线$\ell$折叠过来,$A$和$A'$重合,
而$\ell$上的$P$和$O$重合于自身,所以$\overline{AP}=\overline{A'P}$, $\angle APO=\angle A'PO$, $\triangle PAA'$是等腰三角形,直线$\ell$是顶角的平分线,
所以$\ell$垂直平分底边$\overline{AA'}$.

由此得出轴对称图形的重要性质:
\begin{enumerate}
\item 对称轴上的任一点,与每一双对称点的距离相等。
\item 对称轴是每一双对称点所连线段的垂直平分线。
\end{enumerate}

在上述性质的证明中,我们所取$P$点异于$O$, 如果$P$点就
是$O$点,结论仍然一样。

由于一条线段的垂直平分线是唯一的,由性质2可知,
如果$\overline{AA'}$的垂直平分线是$\ell$, 那么$A$与$A'$是以$\ell$为轴的对称
点。

由此,我们就可以作出已知图形以某直线为轴的对称
形。

\begin{figure}[htp]\centering
    \begin{minipage}[t]{0.48\textwidth}
    \centering
\begin{tikzpicture}[>=latex, scale=.8]
    \draw(0,-.5)--(0,4)node[above]{$\ell$};
\draw(-2,0)node[below]{$A$}--(2,0)node[below]{$A'$}--(0,3)node[right]{$P$}--(-2,0);
\draw (0,0)node[below right]{$O$} rectangle (.2,.2);
    \end{tikzpicture}
    \caption{}
    \end{minipage}
    \begin{minipage}[t]{0.48\textwidth}
    \centering
    \begin{tikzpicture}[>=latex, scale=1]
\draw(0,-.5)--(0,4)node[above]{$\ell$};     
\tkzDefPoints{-1/3/A, 1/3/A', -2.5/2/B,2.5/2/B',-1.5/0/C,1.5/0/C',-.5/.8/D, .5/.8/D'}
\tkzDrawPolygon(A,B,C,D)\tkzDrawPolygon(A',B',C',D')
\foreach \x in {A,B,C,D}
{
    \draw(\x)--(\x');
}
\tkzDefPoints{0/2.1/O}
\tkzAutoLabelPoints[center=O](A,B,C,D,A',B',C',D')
\node at (0,3)[below right]{$O$};
\draw(0,3) rectangle (-.2,3+.2);
    \end{tikzpicture}
    \caption{}
    \end{minipage}
    \end{figure}



\begin{example}
已知:四边形$ABCD$及直线$\ell$. (图3.16)

求作:四边形$ABCD$以$\ell$为轴的轴对称形。

作法:
\begin{enumerate}
\item 由$A$点引$\ell$的垂线交$\ell$于$O$点,在射线$AO$上
取$OA'=AO$, 则$A'$是$A$点关于轴$\ell$的对称点。
\item 用同样的方法作点$B$、$C$、$D$关于$\ell$的对称点$B'$、
$C'$、$D'$.
\item 连结$\overline{A'B'}$, $\overline{B'C'}$, $\overline{C'D'}$, $\overline{D'A'}$, 则四边形
$A'B'C'D'$就是四边形$ABCD$以$\ell$为轴的轴对称图形。
\end{enumerate}

为什么呢?因为根据作法,如果把四边形$ABCD$和 四边
形$A'B'C'D'$所在平面,沿直线$\ell$折叠起来,则$A$与$A'$、
$B$与$B'$、$C$与$C'$、$D$与$D'$分别重合,所以四边形$ABCD$与
四边形$A'B'C'D'$完全重合,所以这两个四边形是以$\ell$为轴
的对称图形。
\end{example}

\begin{example}
    有公共底的两个等腰三角形,通过底所对的两个
顶点的直线是它们所组成图形的对称轴。

已知:在图3.17中,$BC$是等腰$\triangle ABC$与等腰$\triangle A'BC$
的公共底边。

求证:直线$AA'$是这个图形的对称轴。
\end{example}

\begin{figure}[htp]
    \centering
\begin{tikzpicture}[scale=.8]
\begin{scope}
\draw(0,-1.5)--(0,4);
\draw(-1.5,0)node[left]{$B$}--(1.5,0)node[right]{$C$}--(0,3)node[right]{$A$}--(-1.5,0)--(0,-1)node[right]{$A'$}--(1.5,0);

\end{scope}
\begin{scope}[xshift=7cm]
    \draw(0,-1)--(0,4);
\draw(-1.5,0)node[left]{$B$}--(1.5,0)node[right]{$C$}--(0,3)node[right]{$A$}--(-1.5,0)--(0,1.8)node[right]{$A'$}--(1.5,0);

\end{scope}
\end{tikzpicture}
    \caption{}
\end{figure}

\begin{proof}
$\because\quad \overline{AB}=\overline{AC},\quad \overline{A'B}=\overline{A'C}$(已知)

$\therefore\quad \angle ABC=\angle ACB,\quad 
\angle A'BC=\angle A'CB$(等腰三角形的两底角相等)。

两式相加(或相减)得:
$\angle ABA'=\angle ACA'$(等量加(或减)等量其和(或差)
相等)。

$\therefore\quad \triangle ABA'\cong \triangle ACA'$(SAS)

以$AA'$为轴折叠起来,$\triangle ABA'$与$\triangle ACA'$能够重
合,所以$AA'$是这个图形的对称轴。
\end{proof}

\begin{example}
    证明四条边相等的四边形的两条对角线互相垂直
平分,并且平分一双对角。

已知:在四边形$ABCD$中,$\overline{AB}=\overline{BC}=\overline{CD}=\overline{DA}$.
(图3.18)
\begin{figure}[htp]
    \centering
    \begin{tikzpicture}
    \draw(-2,0)--(0,1.2)--(2,0)--(0,-1.2)--(-2,0);
    \draw(-2,0)node[above]{$B$}--(2,0)node[above]{$D$};
    \draw(0,1.2)node[above]{$A$}--(0,-1.2)node[below]{$C$};
    \node at (0,0)[below right]{$O$};        
    \end{tikzpicture}
    \caption{}
\end{figure}

求证:$AC$、$BD$互相垂直平分,且$AC$平分$\angle A$和$\angle C$,
$BD$平分$\angle B$和$\angle D$.
\end{example}

\begin{proof}
在四边形$ABCD$中,
因为$\overline{AB}=\overline{BC}=\overline{CD}=\overline{DA}$, 所
以四边形$ABCD$可看成由两个
等腰三角形所拼成:等腰$\triangle ABD$
与等腰$\triangle CBD$, 或等腰$\triangle ABC$
与等腰$\triangle ADC$. 由例3.8可知,
对角线$\overline{AC}$、$\overline{BD}$所在的直线都是四边形$ABCD$的对称
轴。所以,$\overline{AC}$、$\overline{BD}$互相垂直平分,并且$\overline{AC}$平分$\angle A$和
$\angle C$, $\overline{BD}$平分$\angle B$和$\angle D$.
\end{proof}

\begin{ex}
\begin{enumerate}
    \item 下列各图形有多少个对称轴?对称轴是什么?
    \begin{multicols}{3}
        \begin{enumerate}
            \item 线段;\item 射线;\item 直线。
        \end{enumerate}
    \end{multicols}
    \item 已知直线$\ell$和$\ell$外面一点$A$, 只用圆规和直尺求作点$A$以
    直线$\ell$为对称轴的对称点$A'$.
    \item 求作两个已知点的对称轴。
    \item 已知$\triangle ABC$和直线$\ell$, 作$\triangle ABC$以$\ell$为对称轴的对称形。
    \item 求作与已知等边三角形$ABC$分别以$AB$、$AC$、$BC$为对称
    轴的对称图形。
    \item 作图。(只要求作出图形)
    \begin{enumerate}
    \item 画已知线段$\overline{AB}$的对称轴。
    \item 画已知$\angle A$的对称轴。
    \end{enumerate}
    \item 等腰三角形有几个对称轴?等边三角形有几个对称轴?任
    画一个等边三角形把它的对称轴都画出来。
\end{enumerate}
\end{ex}

\subsection{三角形中的不等关系}
\begin{blk}{定义}
    和三角形的内角相邻并且和它互补的角叫做三角
形的\textbf{外角}。
\end{blk}
 
如图3.19中的$\angle ACD$
就是$\triangle ABC$的一外个角。
这时$\angle ACB$称为$\angle ACD$
相邻的内角,$\angle A$和$\angle B$
分别称为$\angle ACD$不相邻的内角。

\begin{blk}{定理}
 三角形的外角大于和它不相邻的任一内角。
\end{blk}

\begin{figure}[htp]\centering
    \begin{minipage}[t]{0.48\textwidth}
    \centering
\begin{tikzpicture}[>=latex, scale=1]
\draw(0,0)node[left]{$B$}--(4.5,0)node[right]{$D$};
\draw(3,0)node[below]{$C$}--(1.7,2)node[above]{$A$}--(0,0);
\draw(3.4,0) arc (0:120:.4);
    \end{tikzpicture}
    \caption{}
    \end{minipage}
    \begin{minipage}[t]{0.48\textwidth}
    \centering
    \begin{tikzpicture}[>=latex, scale=1]
\draw(0,0)node[left]{$A$}--(3,0)node[below]{$B$};
\draw[dashed](3,0)--(4.5,0)node[right]{$D$};
\draw(3,0)--(1.7,2)node[above]{$C$}--(0,0);
\draw[dashed](3,0)--(4.7,2)node[right]{$F$}--(0,0); 
\node at (2.35,1)[above]{$E$};
    \end{tikzpicture}
    \caption{}
    \end{minipage}
    \end{figure}

已知:$\triangle ABC$和外角
$\angle CBD$(图3.20)。

求证:$\angle CBD>\angle C$
或$\angle A$.

\begin{proof}
    假定$E$是$\overline{BC}$
中点,引$AE$并延长到$F$, 
使$\overline{EF}=\overline{AE}$, 作$\overline{BF}$.

在$\triangle ACE$和$\triangle FBE$中,

$\because\quad \overline{AE}=\overline{EF}$ (作图),

$\because\quad E$是$\overline{BC}$中点(假定),

$\therefore\quad \overline{CE}=\overline{EB}$(线段中点定义),

$\because\quad \angle CEA=\angle BEF$(对顶角相等),

$\therefore\quad \triangle ACE\cong \triangle FBE$ (SAS),

$\therefore\quad \angle EBF=\angle C$(全等三角形的对应角相等)。

由于$\angle CBD>\angle EBF$ (全量大于它的任何一部分)。

$\therefore\quad \angle CBD>\angle C$ (等量代换)。

同理可证 $\angle CBD> \angle A$.
\end{proof}

\begin{blk}{定理}
 在一个三角形中,如果两条边不等,那么它们所
对的角也不等,大边所对的角较大;反之,如果在一个三角
形中两个角不等,那么它们所对的边也不等。大角所对的边
较大。
\end{blk}

已知:$\triangle ABC$(图3.21)

\begin{figure}[htp]\centering
    \begin{minipage}[t]{0.48\textwidth}
    \centering
\begin{tikzpicture}[>=latex, scale=1.3]
\tkzDefPoint(0,0){A}
\tkzDefPoint(-50:2){C}
\tkzDefPoint(-120:2){D}
\tkzDefPoint(-120:3){B}
\tkzDefPoint(0,-1.5){O}
\tkzDrawPolygon(A,C,B)
\draw[dashed](D)--(C);
\tkzMarkAngles[mark=none, size=.3](C,D,A A,C,D)
\tkzLabelAngle[pos=.6](C,D,A){2}
\tkzLabelAngle[pos=.6](A,C,D){1}
\tkzAutoLabelPoints[center=O](A,C,B,D)       
    \end{tikzpicture}
    \caption{}
    \end{minipage}
    \begin{minipage}[t]{0.48\textwidth}
    \centering
    \begin{tikzpicture}[>=latex, scale=1.3]
\tkzDefPoint(0,0){A}
\tkzDefPoint(30:1.5){D}
\tkzDefPoint(-150:2){B}
\tkzDefPoint(-30:1.5){C}
\tkzDrawPolygon(A,C,B)
\draw[dashed](A)--(D)--(C);
\tkzMarkAngles[mark=none, size=.2](D,C,A A,D,C)
\tkzLabelAngle[pos=.4](A,D,C){2}
\tkzLabelAngle[pos=.4](D,C,A){1}
\tkzAutoLabelPoints[center=A](C,B,D)  
\node at (0,0)[above left]{$A$};
    \end{tikzpicture}
    \caption{}
    \end{minipage}
    \end{figure}

求证:
\begin{enumerate}
    \item $\overline{AB}>\overline{AC}\quad \Rightarrow\quad \angle C>\angle B$
    \item $\angle C>\angle B\quad \Rightarrow\quad \overline{AB}>\overline{AC}$
\end{enumerate}

\begin{proof}
    先证$\overline{AB}>\overline{AC}\quad\Rightarrow\quad \angle C>\angle B$.

在$AB$上截$\overline{AD}=\overline{AC}$,
则$\triangle ADC$为等腰三角形。

$\therefore\quad \angle 1=\angle 2$.

由于$\angle ACB>\angle 1$ (不等量基本性质6)

$\therefore\quad \angle ACB>\angle 2$(等量代换)。

但$\angle 2>\angle B$ (三角形的外角大于和它不相邻的任一内
角),

$\therefore\quad \angle ACB>\angle B$(不等量基本性质5)

即:$\angle C>\angle B$.

我们再证$\angle C>\angle B\quad \Rightarrow\quad \overline{AB}>\overline{AC}$

如果$\overline{AB}$不大于$\overline{AC}$, 那么$\overline{AB}=\overline{AC}$或$\overline{AB}<\overline{AC}$. 
若$\overline{AB}=\overline{AC}$, 则$\angle B=\angle C$, 若$\overline{AB}<\overline{AC}$,则$\angle C<\angle B$, 这两
种结果都和$\angle C>\angle B$矛盾。

$\therefore\quad \angle C>\angle B\quad \Rightarrow\quad \overline{AB}>\overline{AC}$
\end{proof}

\begin{blk}{定理}
    三角形任意两边之和大于第三边。
\end{blk}
 
已知:$\triangle ABC$(图3.23)。

求证:$\overline{AB}+\overline{AC}>\overline{BC}$.

\begin{proof}
    在$BA$的延长线上取一点$D$, 使$AD=AC$. 则
    $\triangle ACD$为等腰三角形。
  
    $\therefore\quad \angle 1=\angle 2$

    $\because\quad \angle BCD>\angle 1$(不等量
    基本性质6)

$\therefore\quad \angle BCD>\angle 2$(等量代
    换)。

    在$\triangle BCD$中,由前面的定理可知:
    $\overline{BD}>\overline{BC}$, 但$\overline{BD}=\overline{AB}+\overline{AD}=\overline{AB}+\overline{AC}$, 
 
   $\therefore\quad  \overline{AB}+\overline{AC}>\overline{BC}$
\end{proof}

\begin{blk}{推论}
三角形任一边大于其他两边之差。
\end{blk}

下面我们举例说明上述定理的一些应用。


\begin{example}
    已知:$\triangle ABC$中$\overline{AB}=\overline{AC}$, $D$点在$\overline{BC}$上,$E$点
在$\overline{BC}$的延长线上(图3.23)。

求证:$\overline{AD}<\overline{AB}<\overline{AE}$
\end{example}

\begin{figure}[htp]\centering
    \begin{minipage}[t]{0.48\textwidth}
    \centering
\begin{tikzpicture}[>=latex, scale=1]
    \tkzDefPoints{0/1.5/A, -1.5/-1/B, -.25/-1/D, 1.5/-1/C, 2.5/-1/E}
\tkzDrawPolygon(A,C,B)
    \draw(D)node[below]{$D$}--(A)node[above]{$A$}--(E)node[below]{$E$};
\draw[dashed](C)--(3.5,-1);
\node at (B)[below]{$B$};
\node at (C)[below]{$C$};
\tkzMarkAngles[mark=none, size=.3](D,B,A  A,D,B  A,C,B)
\tkzLabelAngle[pos=.5](D,B,A){1}
\tkzLabelAngle[pos=.5](A,D,B){3}
\tkzLabelAngle[pos=.5](A,C,B){2}
    \end{tikzpicture}
    \caption{}
    \end{minipage}
    \begin{minipage}[t]{0.48\textwidth}
    \centering
    \begin{tikzpicture}[>=latex, scale=1.2]
\tkzDefPoint(0,0){D}
\tkzDefPoint(-60:2){C}
\tkzDefPoint(180-60:1){A}
\tkzDefPoint(-140:1.2){E}
\tkzDefPoint(-140:2.8){B}
\tkzDrawPolygon(A,C,B)
\draw(B)--(E)--(C);
\draw[dashed](E)--(D);
\foreach \x in {A,D,C}
{
    \node at (\x) [right]{$\x$};
}
\foreach \x in {B,E}
{
    \node at (\x) [left]{$\x$};
}
\tkzMarkAngles[mark=none, size=.2](E,D,C)
\tkzLabelAngle[pos=.4](E,D,C){1}
    \end{tikzpicture}
    \caption{}
    \end{minipage}
    \end{figure}

\begin{proof}
$\because\quad \overline{AB}=\overline{AC}$

$\therefore\quad \angle 1=\angle 2$

又$\because\quad \angle 3>\angle 2$

$\therefore\quad \angle 3>\angle 1$

$\therefore\quad \overline{AB}>\overline{AD}$(在一个三角形中,大角对大边。)

又:$\because\quad \angle 2>\angle E$

$\therefore\quad \angle 1>\angle E$

$\therefore\quad \overline{AE}>\overline{AB}$(在一个三角形中,大角对大边。)

因此有:$\overline{AD}<\overline{AB}<\overline{AE}$
\end{proof}


\begin{example}
    已知:$E$点在$\triangle ABC$内(图3.24)。

求证: \begin{enumerate}
    \item $\angle BEC>\angle A$
    \item $\overline{BE}+\overline{EC}<\overline{AB}+\overline{AC}$
\end{enumerate} 
\end{example}

\begin{proof}
    延长$\overline{BE}$交$\overline{AC}$于$D$点,
则$\angle BEC>\angle 1$,$\angle BEC>\angle 1$

$\therefore\quad \angle BEC>\angle A$(不等量基
本性质5)。

又$\because\quad \overline{BE}+\overline{ED}<\overline{AD}+\overline{AB}$,$\overline{EC}<\overline{ED}+\overline{DC}$(三角形两边之和大于第三边)。

$\therefore\quad \overline{BE}+\overline{ED}+\overline{EC}<\overline{AD}+\overline{AB}+\overline{ED}+\overline{DC}$(不等量基本性质3)。

$\therefore\quad \overline{BE}+\overline{EC}<\overline{AB}+\overline{AC}$(不等量基本性质1)。
\end{proof}

\begin{example}
如图3.25所示,$A$、$B$是平面上直线$\ell$同侧的两点,
试在直线$\ell$上求一点$P$, 使$\overline{AP}+\overline{PB}$最短。

在没有作这题前,让我们想一想该怎样着手,我们要求
$\overline{AP}+\overline{PB}$最短,但怎样才能最短呢?

我们知道两点间的直线段最短,但是我们却要在$\ell$上求
一点,使$\overline{AP}+\overline{PB}$最短。如果$A,B$两点在$\ell$的两侧,那么问
题要简单得多(作$\overline{AB}$与$\ell$交于$P$, $P$点就是所求的点)。但
我们曾学过,如果两点是轴对称点,那么它们的对称轴就是
两点间线段的垂直平分线,并且对称轴上的点到两对称点的
距离相等。我们只要作出$A$点关于轴$\ell$的对称点$A'$, 我们求
$\overline{AP}+\overline{PB}$的最小值,就可转化为求$\overline{A'P}+\overline{PB}$的最小值了。
而$A'$、$B$两点在$\ell$的异侧,这是很容易的。

\begin{figure}[htp]
    \centering
\begin{tikzpicture}[xscale=.6, rotate=5]
\draw (-2,0)--(8,0)node[right]{$\ell$};
\draw[dashed](-.5,1)node[left]{$A$}--(-.5,-1)node[left]{$A'$};
\draw[dashed](-.5,-1)--(5.5,2)node[right]{$B$};
\draw[very thick](-.5,1)--(1.5,0)node[below]{$P$}--(5.5,2);
\node at (-.5,0)[below left]{$D$};
\draw[dashed](-.5,1)  -- (6.5,0) node[below]{$C$}--(-.5,-1);
\draw[dashed](5.5,2)  -- (6.5,0);
\end{tikzpicture}
    \caption{}
\end{figure}
\end{example}


\begin{solution}
    作$A$点关于轴$\ell$的对称点$A'$, 作$\overline{A'B}$与$\ell$相交于$P$, 
    则$P$点使$\overline{AP}+\overline{PB}$最短。因为,如果在$\ell$上任找另外一点$C$, 
    在$\triangle A'BC$中,
    $\overline{A'B}<\overline{A'C}+\overline{CB}$(三角形两边之和大于第三边),
    但$\overline{A'B}=\overline{A'P}+\overline{PB}$

将    $\overline{A'P}=\overline{AP},\quad \overline{A'C}=\overline{AC}$ 代入上式,
    则得:$\overline{AP}+\overline{PB}<\overline{AC}+\overline{CB}$.

    这就证明了$P$点使$\overline{AP}+\overline{PB}$最短。
\end{solution}

\begin{rmk}
    如果$\ell$是镜子的话,那么从$A$点发出的光线,若反
射到$B$点,那么$\overline{AP}$、$\overline{PB}$就是光所走的路线。因为光线总是
走最近的路,在光学中就有:“入射角等于反射角”这条定
律。即$\angle APD=\angle BPC$. 在光学中由观测得到的结果和我
们用数学定理得出的结果一致。在自然界中我们有能力观测
到的结论是有限的,如何由这些有限的可观测的结论,得到
更进一步的结论,甚至有些是观测不到的结论,我们需要数
学这样有力的工具。
\end{rmk}
    
\begin{ex}
\begin{enumerate}
    \item 已知:如图,$\triangle ABC$内有一点$P$, 
    求证:$\angle BPC>\angle BAC$.
    \item 已知:如图,$\triangle ABC$内有一点P,
    求证:$\overline{AB}+\overline{AC}>\overline{PB}+\overline{PC}$.
    \item 将金属丝弯成三角形,如果要求一边长是25cm, 那么
    金属丝至少大于多少cm才能作成三角形?
    \item 证明四边形两条对角线之和小于周长而大于半周长。
 \item 证明三角形的一边小于它的周长的一半。
 \item  用整数表示三角形的各边,一边是3米,另一边是2米,第
    三边可以是多少米?
\item  在平面上$\overline{AB}$
    的垂直平分线的$A$点的一侧取一点$P$, 那
    么$\overline{PA}$和$\overline{PB}$哪一条线段长?
\item 已知:如图,$\triangle ABC$中,
$\angle A>\angle C$, $D$、$E$分别在$\overline{AB}$、
    $\overline{AC}$上并且$\overline{AD}>\overline{AE}$.
    求证:
    \begin{enumerate}
        \item $\angle BDE> \angle CED$
        \item $\overline{BC}>\overline{BE}$
    \end{enumerate}
\end{enumerate}
\end{ex}
    
\begin{figure}[htp]\centering
    \begin{minipage}[t]{0.48\textwidth}
    \centering
\begin{tikzpicture}[>=latex, scale=1]
    \tkzDefPoint(-60:2){C}
    \tkzDefPoint(180-60:1){A}
    \tkzDefPoint(-140:2.8){B}
    \tkzDefPoint(-.25,-.5){P}
    \tkzDrawPolygon(A,C,B)
    \draw(B)--(P)--(C);
\tkzAutoLabelPoints[center=P](A,B,C)
\node at (P)[above]{$P$};
    \end{tikzpicture}
    \caption*{第1--2题}
    \end{minipage}
    \begin{minipage}[t]{0.48\textwidth}
    \centering
    \begin{tikzpicture}[>=latex, scale=1.3]
        \tkzDefPoint(0,0){A}
        \tkzDefPoint(-45:2){C}
        \tkzDefPoint(-45:1){E}
        \tkzDefPoint(-140:2.8){B}
        \tkzDefPoint(-140:2){D}
        \tkzDrawPolygon(A,C,B)
        \tkzDrawPolygon(D,B,E)
        \tkzDefPoint(.1,-1){O}
        \tkzAutoLabelPoints[center=O](A,B,C,D,E)

    \end{tikzpicture}
    \caption*{第8题}
    \end{minipage}
    \end{figure}

\subsection*{习题3.1}

\begin{enumerate}
    \item 已知:如图,$\overline{AD}=\overline{AE}$, $\overline{BD}=\overline{EC}$, 
    求证:$\overline{DO}=\overline{EO}$.
    \item 已知:如图,$\overline{AB}=\overline{AD}$, $AC$平分$\angle BAD$, $E$是$\overline{AC}$上一点。
    求证:$\angle EBC=\angle EDC$.
    \item 求证:等腰三角形两腰上的中线相等。
    \item 已知:如图,$\angle B=\angle C$, $\overline{BD}=\overline{EC}$. 
    求证:$\angle 1=\angle 2$.

\begin{figure}[htp]\centering
    \begin{minipage}[t]{0.31\textwidth}
    \centering
\begin{tikzpicture}[>=latex, scale=1]
\tkzDefPoints{-1.5/0/B, 1.5/0/C, -.5/2/D, .5/2/E, 0/3/A}
\tkzDrawPolygon(A,B,C)
\tkzDrawSegments(B,E C,D)
\tkzInterLL(C,D)(B,E)  \tkzGetPoint{O}
\tkzLabelPoints[left](B,D)
\tkzLabelPoints[right](C,E)
\tkzLabelPoints[above](A)
\tkzLabelPoints[below](O)


    \end{tikzpicture}
    \caption*{第1题}
    \end{minipage}
    \begin{minipage}[t]{0.31\textwidth}
    \centering
    \begin{tikzpicture}[>=latex, scale=1]
\tkzDefPoints{-1.5/0/B, 1.5/0/D, 0/1.5/A, 0/-2/C, 0/.8/E}
\tkzDrawPolygon(A,B,C,D)
\tkzDrawSegments(B,E A,C E,D)
\tkzLabelPoints[left](B)
\tkzLabelPoints[right](D)
\tkzLabelPoints[above](A)
\tkzLabelPoints[below](C)
\tkzLabelPoints[below right](E)
    \end{tikzpicture}
    \caption*{第2题}
    \end{minipage}
        \begin{minipage}[t]{0.31\textwidth}
    \centering
    \begin{tikzpicture}[>=latex, scale=1]
\tkzDefPoints{-1.5/0/B, 1.5/0/C, -.75/0/D, .75/0/E, 0/3/A}
\tkzDrawPolygon(A,B,C)
\tkzDrawSegments(A,E A,D)
\tkzLabelPoints[left](B)
\tkzLabelPoints[right](C)
\tkzLabelPoints[above](A)
\tkzLabelPoints[below](D,E)
\tkzMarkAngles[mark=none, size=.3](C,D,A A,E,B)
\tkzLabelAngle[pos=.5](C,D,A){1}
\tkzLabelAngle[pos=.5](A,E,B){2}
    \end{tikzpicture}
    \caption*{第4题}
    \end{minipage}
    \end{figure}

    \item 已知:如图,在四边形$ABCD$中,
    $\overline{AD}=\overline{AB}$, $\overline{CD}=\overline{BC}$
    
    求证:$\angle 1=\angle 2$;$\angle 3=\angle 4$;$AC\bot DB$
  
    \item 如图,已知:$\angle 1=\angle 2$, $\angle 3=\angle 4$, 
求证:$\overline{AB}=\overline{AC}$.
\item 如图,已知:$\overline{BE}=\overline{ED}$, $\angle 1=\angle 2$.
求证:$\overline{AB}=\overline{CD}$.

\begin{figure}[htp]\centering
    \begin{minipage}[t]{0.48\textwidth}
    \centering
\begin{tikzpicture}[>=latex, scale=1]
    \tkzDefPoints{-2/0/A, 2/0/C, -1/1.5/D, -1/-1.5/B, -1/0/E}
    \tkzDrawPolygon(A,B,C,D)
    \tkzLabelPoints[left](A)
    \tkzLabelPoints[right](C)
    \tkzLabelPoints[above](D)
    \tkzLabelPoints[below](B)
    \tkzLabelPoints[above right](E)
    \tkzDrawSegments(A,C D,B)
    \tkzMarkAngles[mark=none, size=.4](C,A,D D,C,A)
    \tkzMarkAngles[mark=none, size=.5](B,A,C A,C,B)
    \tkzLabelAngle[pos=.7](C,A,D){1}
    \tkzLabelAngle[pos=.7](B,A,C){2}
    \tkzLabelAngle[pos=.7](D,C,A){3}
    \tkzLabelAngle[pos=.7](A,C,B){4}

    \end{tikzpicture}
    \caption*{第5题}
    \end{minipage}
    \begin{minipage}[t]{0.48\textwidth}
    \centering
    \begin{tikzpicture}[>=latex, scale=1]
        \tkzDefPoints{-1.5/0/B, 1.5/0/C, 0/3.5/A, 0/1/O}
        \tkzDrawPolygon(A,B,C)
        \tkzLabelPoints[left](B)
        \tkzLabelPoints[right](C)
        \tkzLabelPoints[above](A)
        \tkzLabelPoints[below](O)
        \tkzDrawSegments(A,O B,O C,O)        
        \tkzMarkAngles[mark=none, size=.4](C,B,O O,C,B)
        \tkzMarkAngles[mark=none, size=.3](A,O,B)
        \tkzLabelAngle[pos=.6](C,B,O){1}
        \tkzLabelAngle[pos=.6](O,C,B){2}
        \tkzLabelAngle[pos=.5](A,O,B){3}
        \tkzMarkAngles[mark=none, size=.35](C,O,A)
        \tkzLabelAngle[pos=.5](C,O,A){4}

    \end{tikzpicture}
    \caption*{第6题}
    \end{minipage}
    \end{figure}

\item 在$\triangle ABC$中,$\overline{AB}=\overline{AC}$, $D$、$E$
两点分别在$\overline{AB}$和$\overline{AC}$上,且$\overline{BE}$
和$\overline{CD}$交于$O$点,$\overline{BD}=\overline{CE}$. 求证:$AO$平分$\angle BAC$.
\item 在$\triangle ABC$中,$D$为$\overline{BC}$延长线上一点,且$\overline{CD}=\overline{AC}$, $F$
是
$\overline{AD}$中点,$\overline{CE}$是$\angle ACB$的平分线。求证$CE\bot CF$.

\item 在$\triangle ABC$中,$\angle B=2\angle C$, $AD$平分$\angle A$, 交$\overline{BC}$于$D$点,
求证:$\overline{AB}+\overline{BD}=\overline{AC}$.

\begin{figure}[htp]\centering
    \begin{minipage}[t]{0.48\textwidth}
    \centering
\begin{tikzpicture}[>=latex, scale=1]
    \tkzDefPoints{-1.5/0/A, 1.5/0/C, -1/2/B, 1/2/D}
    \tkzDrawPolygon(A,B,C)
    \tkzDrawPolygon(A,D,C)
    \tkzInterLL(B,C)(A,D) \tkzGetPoint{E}
    \tkzLabelPoints[left](A)
    \tkzLabelPoints[right](C)
    \tkzLabelPoints[above](B,D,E)
 
    \tkzMarkAngles[mark=none, size=.4](B,C,A C,A,D)
    \tkzLabelAngle[pos=.6](C,A,D){1}
    \tkzLabelAngle[pos=.6](B,C,A){2}

    \end{tikzpicture}
    \caption*{第7题}
    \end{minipage}
    \begin{minipage}[t]{0.48\textwidth}
    \centering
    \begin{tikzpicture}[>=latex, scale=1]
\tkzDefPoints{0/0/B, 3/0/C, 3/1.5/D}
\tkzDefPoint(70:3){A}
\tkzDrawPolygon(A,B,C,D)
\tkzLabelPoints[below](B,C)
\tkzLabelPoints[above](A,D)
    \end{tikzpicture}
    \caption*{第11题}
    \end{minipage}
    \end{figure}

\item 如图,已知四边形$ABCD$中,
$\overline{AB}=\overline{BC}$, $\overline{AD}>\overline{CD}$. 

求证:$\angle BCD>\angle BAD$.
\item 已知:$\triangle ABC$中,$D$是$\overline{AC}$上
一点,且$\overline{BD}>\overline{BC}$. 

求证:$\overline{AB}>\overline{AC}$.
\item 三角形的一边等于2米,另一边等于8米,如果已知第三
边能用被3整除的整数表示,求第三边。
\item 已知:$\triangle ABC$在平面$\alpha$上,点$P\in\alpha$, 并且$P$在$\triangle ABC$
的外部

求证:$\overline{PA}+\overline{PB}+\overline{PC}>\frac{1}{2}(\overline{AB}+\overline{BC}+\overline{AC})$
\item 已知:$\triangle ABC$在平面$\alpha$上,点$P\in\alpha$, 并且$P$在$\triangle ABC$
的内部

求证:$\overline{PA}+\overline{PB}+\overline{PC}<\overline{AB}+\overline{BC}+\overline{AC}$.
\item 求证:三角形一边上的中线小于其他两边和的一半。
\item 在两条公路$OX$和$OY$上,分别设邮筒$A$和$B$, 邮递员每
天由邮局$P$到邮筒$A$、$B$取信,然后返回邮局。问$A$、$B$的
位置确定在哪儿,才能使邮递员走的路程最短?

\begin{figure}[htbp]
    \centering
\begin{tikzpicture}
\tkzDefPoints{-2/0/X, 2/0/Y, .5/3/X1, -.5/3/Y1, 0/0/P}
\tkzDefPointWith[linear, K=.5](X,X1) \tkzGetPoint{A}
\tkzDefPointWith[linear, K=.3](Y,Y1) \tkzGetPoint{B}
\tkzDrawPolygon(A,B,P)
\tkzInterLL(X,X1)(Y,Y1)  \tkzGetPoint{O}
\tkzDrawSegments(X,X1 Y,Y1)
\tkzLabelPoints[below](X,Y,P)
\tkzLabelPoints[left](A)
\tkzLabelPoints[right](B,O)
\end{tikzpicture}
    \caption*{第17题}
\end{figure}
\end{enumerate}

\section{平行线与内角和定理}

\subsection{平行线}
\begin{blk}
    {定义}
平面上两条直线被第三条直线所截,截得八个
角,如图3.26, 其中的$\angle 1$和$\angle 5$, $\angle 4$和$\angle 8$, $\angle 2$和$\angle 6$, 
$\angle 3$和$\angle 7$, 都叫做\textbf{同位角},$\angle 4$和$\angle 6$, $\angle 3$和$\angle 5$都叫\textbf{内错
角},$\angle 4$和$\angle 5$, $\angle 3$和$\angle 6$都叫做\textbf{同旁内角}。
\end{blk}

\begin{figure}[htp]\centering
    \begin{minipage}[t]{0.48\textwidth}
    \centering
\begin{tikzpicture}[>=latex, scale=1]
\tkzDefPoints{-2/2/A, 2/2.5/B, -2/0/C, 2/1.5/D, 0/-.5/F, 0/4/E}
\tkzDrawSegments(A,B C,D E,F)
\tkzInterLL(A,B)(E,F)  \tkzGetPoint{P}
\tkzInterLL(C,D)(E,F)  \tkzGetPoint{Q}
\tkzMarkAngles[mark=none, size=.4](B,P,E A,P,F D,Q,E C,Q,F)
\tkzLabelAngle[pos=.6](B,P,E){1}
\tkzLabelAngle[pos=.6](A,P,F){3}
\tkzLabelAngle[pos=.6](D,Q,E){5}
\tkzLabelAngle[pos=.6](C,Q,F){7}
\tkzMarkAngles[mark=none, size=.3](E,P,A F,P,B E,Q,C F,Q,D)
\tkzLabelAngle[pos=.5](E,P,A){2}
\tkzLabelAngle[pos=.5](F,P,B){4}
\tkzLabelAngle[pos=.5](E,Q,C){6}
\tkzLabelAngle[pos=.5](F,Q,D){8}
    \end{tikzpicture}
    \caption{}
    \end{minipage}
    \begin{minipage}[t]{0.48\textwidth}
    \centering
    \begin{tikzpicture}[>=latex, scale=1]
\tkzDefPoints{-2/2/A, 2/2/B, -2/0/C, 2/0/D, -1/-1/F, 1/3/E}
\tkzDrawSegments(A,B C,D E,F)
\tkzInterLL(A,B)(E,F)  \tkzGetPoint{P}
\tkzInterLL(C,D)(E,F)  \tkzGetPoint{Q}
\tkzMarkAngles[mark=none, size=.4](B,P,E A,P,F)
\tkzLabelAngle[pos=.6](B,P,E){4}
\tkzLabelAngle[pos=.6](A,P,F){3}
\tkzLabelAngle[pos=.6](D,Q,E){2}
\tkzMarkAngles[mark=none, size=.3](F,P,B D,Q,E)
\tkzLabelAngle[pos=.5](F,P,B){1}
\tkzLabelPoints[above](A,B,C,D)
\tkzLabelPoints[below](Q)
\tkzLabelPoints[above left](P)
    \end{tikzpicture}
    \caption{}
    \end{minipage}
    \end{figure}


由平行线定义可知,两条直线平行的充要条件是这两条
直线被第三条直线所截出的同位角相等。这就是说:
\begin{enumerate}
\item 如
果同位角相等,则两条直线平行;   
\item 如果已知两条直线平
行,则同位角相等。 
\end{enumerate}
这就是平行线的判定方法和平行线的性
质。以此为根据又可推知以下定理:

\begin{blk}{定理}
两条直线被第三条直所线截,如果
内错角相等,或同旁内角互补,那么这两条
直线平行。
\end{blk}

\begin{proof}
   在图3.27中,设直线$AB$、$CD$被直线$PQ$所截。
$\angle 2$、$\angle 3$是内错角,$\angle 1$、$\angle 2$是同旁内角。
\begin{enumerate}
    \item 如果已知$\angle 2=\angle 3$. 

$\because\quad \angle 3=\angle 4$    (对顶角相等),

$\therefore\quad \angle 2=\angle 4$(等量代换)。

$\therefore\quad AB\parallel CD$(同位角相等,则两条直线平行)。

\item 如果已知$\angle 1+\angle 2=180^{\circ}$, 
则$\angle 2=180^{\circ}-\angle 1$(等量减等量差相等)。

$\because\quad PQ$是直线(已知)

$\therefore\quad \angle 4+\angle 1=180^{\circ}$.

$\therefore\quad \angle 4=180^{\circ}-\angle 1$.

$\therefore\quad \angle 2=\angle 4$(等量代换)。

$\therefore\quad AB\parallel CD$(同位角相等,则两条直线平行)。  
\end{enumerate}
\end{proof}

\begin{blk}{定理} 
    平面上两条不相交的直线是这两条直线互相平行
的充要条件。
\end{blk}


但是这个定理的证明比较麻烦,我们在这里就不证了,
有兴趣的同学可以自己去证明它。

\begin{blk}
   {推论1} 平行于同一直线的两条直线平行。 
\end{blk}

已知:直线$a\parallel$直线$c$, 直线$b\parallel$直线$c$ (图3.28).
 求证:$a\parallel b$.

\begin{proof}
    若$a$不平行$b$, $a$与$b$一定相交,设交点为$P$, 那么
过$P$点就有两条直线$a$、$b$平行于直线$c$, 但根据平行公理,
这是不可能的,所以$a\parallel b$.
\end{proof}

\begin{figure}[htp]
    \centering
\begin{tikzpicture}
\begin{scope}
    \foreach \x/\xtext in {0/c,1/b,2/a}
    {
        \draw(0,\x)--node[above]{$\xtext$}(3,\x);
    }
\end{scope}
\begin{scope}[xshift=5cm]
    \foreach \x/\xtext in {0/c,1/b,2/a}
    {
        \draw(0,\x)--node[above]{$\xtext$}(3,\x);
    }
\draw[dashed](3,1)--(4.5,1.5)node[right]{$P$}--(3,2);
\end{scope}
\end{tikzpicture}
    \caption{}
\end{figure}


上面这种证明方法,叫做反证法。我们先否定要证明的
结论,然后引出不合理的结果,从而说明结论非成立不可。
请同学们回想一下,我们在第二章讲基本逻辑语句的命题
时,若原命题是“若$\alpha$, 则$\beta$”,那么它的逆否命题是“若
非$\beta$, 则非$\alpha$.”我们知道这两个互为逆否的命题是同义
的,上面的反证法就是当我们证明原命题有困难时,我们可
去证它的逆否命题。

\begin{blk}
 {推论2} 过已知直线外一点,只可以引一条直线和已知
直线垂直。   
\end{blk}

请同学们用反证法证明这个推论。

\begin{example}
已知:在图3.29中,$\angle BED=\angle B+\angle D$.

求证:$AB\parallel CD$.
\end{example}

\begin{analyze}
    若能证明$AB$、$CD$都
与第三条直线平行,则$AB\parallel CD$.
\end{analyze}

\begin{proof}
    引直线$EF$, 使得$\angle 1=\angle B$, 则$AB\parallel EF$(内错角相等,则两条直线平行)。

又:$\because\quad \angle BED=\angle B+\angle D$(已知),

$\therefore\quad \angle BED-\angle 1=\angle D$(等量减等量差相等)。
即:$\angle FED=\angle D$.

$\therefore\quad CD\parallel EF$ (内错角相等,则两条直线平行)。

$\therefore\quad AB\parallel CD$ (平行于第三条直线的两条直线平
行)。
\end{proof}

\begin{figure}[htp]\centering
    \begin{minipage}[t]{0.48\textwidth}
    \centering
\begin{tikzpicture}[>=latex, scale=1]
\tkzDefPoints{0/0/C, 3/0/D, 2/1/E, 3/2/B, 0/2/A, 4/1/F}
\tkzDrawSegments(C,D D,E E,B A,B)
\tkzDrawSegments[dashed](E,F)
\tkzLabelPoints[below](C,D)
\tkzLabelPoints[above](A,B)
\tkzLabelPoints[left](E)
\tkzLabelPoints[right](F)
\tkzMarkAngles[mark=none, size=.4](F,E,B)
\tkzLabelAngle[pos=.6](F,E,B){1}

    \end{tikzpicture}
    \caption{}
    \end{minipage}
    \begin{minipage}[t]{0.48\textwidth}
    \centering
    \begin{tikzpicture}[>=latex, scale=1]
\tkzDefPoints{-.25/-.25/B, .25/.25/C, 1/-.25/F, -1/.25/E, 1.5/1.5/D, -1.5/-1.5/A}
\tkzDrawPolygon(A,E,C)
\tkzDrawPolygon(B,F,D)
\tkzLabelPoints[below](A,B,F)
\tkzLabelPoints[above](E,D,C)
\tkzMarkAngles[mark=none, size=.4](F,B,D E,C,A)
\tkzLabelAngle[pos=.6](F,B,D){2}
\tkzLabelAngle[pos=.6](E,C,A){1}

    \end{tikzpicture}
    \caption{}
    \end{minipage}
    \end{figure}

\begin{example}
    在图3.30中,已知$\overline{CD}=\overline{AB}$, $AD$是直线。
$\angle A=\angle D$, 并且$\overline{AE}=\overline{DE}$.

求证:$EC\parallel BF$.

\end{example}

\begin{analyze}
要证$EC\parallel BF$, 只需证明$\angle 1=\angle 2$即可,要证明
$\angle 1=\angle 2$, 只需证明$\triangle AEC\cong \triangle DFB$即可。
\end{analyze}

\begin{proof}
    $\because\quad \overline{AB}=\overline{CD}$(已知)
    $\therefore\quad \overline{AB}+\overline{BC}=\overline{BC}+\overline{CD}$(等量加等量和相等)。即:$\overline{AC}=\overline{BD}$.

又:$\because\quad \angle A=\angle D,\quad \overline{AE}=\overline{DF}$(已知),

$\therefore\quad \triangle AEC\cong \triangle DFB$ (SAS).

$\therefore\quad \angle 1=\angle 2$(全等三角形的对应角相等)。

$\therefore\quad CE\parallel BF$(内错角相等,则两条直线平行)。
\end{proof}

\begin{blk}
  {定理} 如果两条平行直线被一条直线所截,那么截出的
内错角相等;同旁内角互补。  
\end{blk}

这个定理同学们可以根据“两条直线平行则同位角相
等”这一性质推出。

\begin{example}
     在图3.31中,
已知:$OA\parallel O'A'$, $OB\parallel O'B'$.

求证:$\angle AOB=\angle A'O'B'$.   
\end{example}

\begin{proof}
    反向延长$O'A'$交$OB$于$C$点,

$\because\quad OB\parallel O'B'$(已知),

$\therefore\quad\angle A'O'B'=\angle A'CB$(两条直线平行,同位角相等)。

又$\because\quad OA\parallel O'A'$(已知),

$\therefore\quad \angle A'CB=\angle AOB$(两条直线平行,同位角相等)。

$\therefore\quad \angle AOB=\angle A'O'B'$(等量代换)。
\end{proof}

\begin{figure}[htp]\centering
    \begin{minipage}[t]{0.48\textwidth}
    \centering
\begin{tikzpicture}[>=latex, scale=1, rotate=-20]
\tkzDefPoints{0/0/O, 4/0/A, 2.5/1.5/O', 4.5/1.5/A', 0/1.5/C', 2.5+1.414/1.5+1.414/B', 2.5/2.5/B, 1.5/1.5/C}
\tkzDrawSegments(A,O A',O' B,O B',O')
\tkzDrawSegments[dashed](O',C')
\tkzLabelPoints[below](O,A,O',A')
\tkzLabelPoints[above](C,B,B')

    \end{tikzpicture}
    \caption{}
    \end{minipage}
    \begin{minipage}[t]{0.48\textwidth}
    \centering
    \begin{tikzpicture}[>=latex, scale=1]
\tkzDefPoints{0/0/B, 3/0/C, 1/2/A, 4/2/D}
\tkzDrawPolygon(A,B,C,D)
\tkzDrawSegments(A,C)
\tkzLabelPoints[left](A,B)\tkzLabelPoints[right](C,D)

\tkzMarkAngles[mark=none, size=.4](A,C,B C,A,D)
\tkzMarkAngles[mark=none, size=.3](B,A,C D,C,A)
\tkzLabelAngle[pos=.6](A,C,B){2}
\tkzLabelAngle[pos=.6](C,A,D){1}
\tkzLabelAngle[pos=.5](B,A,C){3}
\tkzLabelAngle[pos=.5](D,C,A){4}
    \end{tikzpicture}
    \caption{}
    \end{minipage}
    \end{figure}

上面的例子我们可以写成下面的定理:

\begin{blk}
    {定理} 如果一个角的两条边和另一个角的两条边分别同
向平行,那么这两个角相等。
\end{blk}

\begin{example}
    图3.32中,
    已知:$AD\parallel BC$, $\overline{AD}=\overline{BC}$.

    求证:$AB\parallel CD$.
\end{example}

\begin{analyze}
    要证明$AB\parallel CD$, 只要证明$\angle 3=\angle 4$就行
了,为此需要证明$\triangle ABC\cong \triangle CDA$.
\end{analyze}

\begin{proof}
$\because\quad AD\parallel BC$(已知),

$\therefore\quad \angle 1=\angle 2$(两条直线平行,则
内错角相等)。

又:$\because\quad \overline{AD}=\overline{BC}$(已知)
$\overline{AC}=\overline{AC}$(公共边)

$\therefore\quad \triangle ABC\cong \triangle CDA$ (SAS)

$\therefore\quad \angle  3=\angle 4$(全等三角形的对应角相等)。

$\therefore\quad AB\parallel CD$(内错角相等,则两条直线平行)。
\end{proof}

\begin{ex}
\begin{enumerate}
    \item 指出图中用数字标出的角中哪些是同位角?
    \item 如图,说出下列各对角的名称。
    $\angle 1$与$\angle 2$, $\angle 2$与$\angle 3$, $\angle 3$与$\angle 4$, $\angle 4$与$\angle 5$, $\angle 3$与
    $\angle 5$,$\angle 1$与$\angle 6$
    \item 试指出判定两条直线平行有哪些方法?并指出平行线
    有什么性质?
    \item 如图,已知:$\angle 1=60^{\circ}$, $\angle 2=60^{\circ}$. 
    求证:$AB\parallel CD$.

    \item 如图,已知:$\angle 1=\angle 2$, 求证:$AB\parallel CD$.
    \item 知图,已知$\angle 1=69^{\circ}$, $\angle 2=69^{\circ}$.
    求证:$AB\parallel CD$.
    \item 如图,已知:$\angle 1+\angle 2=180^{\circ}$, 求证:$AB\parallel CD$.
    \item 如图,已知:$\overline{AB}$、$\overline{CD}$相交于$E$点,且$\overline{EA}=\overline{EB}$, $\overline{EC}=\overline{ED}$
    
求证:$AC\parallel DB$.

\item 如图,已知:$\overline{AB}=\overline{DC}$, $\overline{AD}=\overline{BC}$. 求证:$AB\parallel CD$.
\item 已知:$\overline{AE}=\overline{DF}$, $\overline{AB}=\overline{CD}$, $\overline{EC}=\overline{BF}$. 
求证:$EC\parallel BF$.
\item 已知:$DE\parallel BC$, $\angle 1=\angle 2$. 求证:$\angle 3=\angle 4$.
\item 已知:$AB\parallel CD$, 求证:$\angle 1=\angle 2$.
\item 如图,已如:$\angle 1=\angle 2$. 求证:$\angle 3=\angle 4$.
\item 如图,已知:$AB\parallel CD$, 
求证:$\angle B+\angle D+\angle E=360^{\circ}$.

\item 在四边形$ABCD$中,$AD\parallel BC$, $AB\parallel CD$且$\overline{BE}=\overline{DF}$, 则$\overline{AE}\parallel \overline{FC}$.
\end{enumerate}
\end{ex}

\begin{figure}[htp]\centering
    \begin{minipage}[t]{0.48\textwidth}
    \centering
\begin{tikzpicture}[>=latex, scale=1]
\tkzDefPoints{0/0/A, 4/1/B, 0/1.5/C, 4/2.5/D, 2/-.5/F, 2/3.5/E}
\tkzDrawSegments(A,B C,D E,F)
\tkzInterLL(A,B)(E,F)  \tkzGetPoint{X1}
\tkzInterLL(C,D)(E,F)  \tkzGetPoint{X2}
\tkzDefPoints{3.5/2/Y1, 3.5/3.5/Y2}
\tkzDrawSegments(X1,Y1 X2,Y2)
\tkzMarkAngles[mark=none, size=.45](B,X1,Y1 D,X2,Y2)
\tkzLabelAngle[pos=.6](B,X1,Y1){4}  \tkzLabelAngle[pos=.6](D,X2,Y2){3}
\tkzMarkAngles[mark=none, size=.35](Y1,X1,E Y2,X2,E)
\tkzLabelAngle[pos=.6](Y1,X1,E){2}  \tkzLabelAngle[pos=.6](Y2,X2,E){1}
    \end{tikzpicture}
    \caption*{第1题}
    \end{minipage}
    \begin{minipage}[t]{0.48\textwidth}
    \centering
    \begin{tikzpicture}[>=latex, scale=1]
 \tkzDefPoints{0/0/A, 4/.8/B, 1.5/-1/C, 4/-.5/D, 0/2.5/E, 1.5/2.5/F}
 \tkzDrawSegments(A,B C,D A,E C,F)
 \tkzInterLL(A,B)(C,F)  \tkzGetPoint{G}     
 \tkzMarkAngles[mark=none, size=.3](B,A,E D,C,F B,G,F A,G,C)
 \tkzLabelAngle[pos=.5](B,A,E){1}  \tkzLabelAngle[pos=.5](D,C,F){4}
 \tkzLabelAngle[pos=.5](B,G,F){2} \tkzLabelAngle[pos=.5](A,G,C){3}
 \tkzMarkAngles[mark=none, size=.25](C,G,B F,G,A)
 \tkzLabelAngle[pos=.4](C,G,B){5}  \tkzLabelAngle[pos=.4](F,G,A){6}
    \end{tikzpicture}
    \caption*{第2题}
    \end{minipage}
    \end{figure}



\begin{figure}[htp]\centering
    \begin{minipage}[t]{0.32\textwidth}
    \centering
\begin{tikzpicture}[>=latex, scale=1]
    \tkzDefPoints{0/0/C, 2.5/0/D, 0/1/A, 2.5/1/B, .5/-1/E, 2.25/2.25/F}
\tkzInterLL(A,B)(E,F) \tkzGetPoint{X1}
\tkzInterLL(C,D)(E,F) \tkzGetPoint{X2}
\tkzLabelPoints[left](A,C)\tkzLabelPoints[right](B,D)
\tkzDrawSegments(A,B C,D E,F)
\tkzMarkAngles[mark=none, size=.4](A,X1,E C,X2,E)
\tkzLabelAngle[pos=.6](A,X1,E){1}  \tkzLabelAngle[pos=.6](C,X2,E){2}
    \end{tikzpicture}
    \caption*{第4题}
    \end{minipage}
    \begin{minipage}[t]{0.32\textwidth}
    \centering
    \begin{tikzpicture}[>=latex, scale=1]
\tkzDefPoints{0/0/C, 2.5/0/D, 0/1/A, 2.5/1/B, .5/-1/E, 2.25/2.25/F}
\tkzInterLL(A,B)(E,F) \tkzGetPoint{X1}
\tkzInterLL(C,D)(E,F) \tkzGetPoint{X2}
\tkzLabelPoints[left](A,C)\tkzLabelPoints[right](B,D)
\tkzDrawSegments(A,B C,D E,F)
\tkzMarkAngles[mark=none, size=.4](B,X1,F C,X2,E)
\tkzLabelAngle[pos=.6](B,X1,F){1}  \tkzLabelAngle[pos=.6](C,X2,E){2}
    \end{tikzpicture}
    \caption*{第5、7、12题}
    \end{minipage}
    \begin{minipage}[t]{0.32\textwidth}
    \centering
    \begin{tikzpicture}[>=latex, scale=1]
 \tkzDefPoints{0/0/C, 2.5/0/D, 0/1/A, 2.5/1/B, .5/-1/E, 2.25/2.25/F}
\tkzInterLL(A,B)(E,F) \tkzGetPoint{X1}
\tkzInterLL(C,D)(E,F) \tkzGetPoint{X2}
\tkzLabelPoints[left](A,C)\tkzLabelPoints[right](B,D)
\tkzDrawSegments(A,B C,D E,F)
\tkzMarkAngles[mark=none, size=.4](B,X1,F E,X2,D)
\tkzLabelAngle[pos=.6](B,X1,F){1}  \tkzLabelAngle[pos=.6](E,X2,D){2}
    \end{tikzpicture}
    \caption*{第6题}
    \end{minipage}
    \end{figure}

\begin{figure}[htp]\centering
    \begin{minipage}[t]{0.32\textwidth}
    \centering
\begin{tikzpicture}[>=latex, scale=1]
\tkzDefPoints{0/0/D, 2.5/0/B, -.5/1.5/A, 2/1.5/C}
\tkzInterLL(A,B)(C,D) \tkzGetPoint{E}
\tkzDrawPolygon(A,C,E)
\tkzDrawPolygon(B,D,E)
\tkzLabelPoints[above](A,C)
\tkzLabelPoints[below](B,D,E)
    \end{tikzpicture}
    \caption*{第8题}
    \end{minipage}
    \begin{minipage}[t]{0.32\textwidth}
    \centering
    \begin{tikzpicture}[>=latex, scale=1]
\tkzDefPoints{0/0/B, 2.5/0/C, .75/1.5/A, 3.25/1.5/D}
\tkzDrawPolygon(A,B,C,D)
\tkzLabelPoints[above](A,D)
\tkzLabelPoints[below](B,C)
    \end{tikzpicture}
    \caption*{第9题}
    \end{minipage}
    \begin{minipage}[t]{0.32\textwidth}
        \centering
        \begin{tikzpicture}[>=latex, scale=1]
\tkzDefPoints{0/0/A, 1/0/B,2/0/C,3/0/D, .5/1/E, 2.5/-1/F}
\tkzDrawPolygon(A,C,E)
\tkzDrawPolygon(B,D,F)
\tkzLabelPoints[above](E,D)
\tkzLabelPoints[below](A,B,C,F)
        \end{tikzpicture}
        \caption*{第10题}
        \end{minipage}
    \end{figure}



    \begin{figure}[htp]\centering
        \begin{minipage}[t]{0.46\textwidth}
        \centering
    \begin{tikzpicture}[>=latex, scale=.8]
\tkzDefPoints{0/0/B, 4/0/C, 2.2/4/A}
\tkzDefPointWith[linear, K=.6](A,B) \tkzGetPoint{D}
\tkzDefPointWith[linear, K=.6](A,C) \tkzGetPoint{E}
\tkzDrawPolygon(A,B,C)
\tkzDrawSegments(D,E)
\tkzLabelPoints[left](B,D)
\tkzLabelPoints[right](C,E)
\tkzLabelPoints[above](A)
\tkzMarkAngles[mark=none, size=.4](E,D,A C,B,A A,C,B A,E,D)
\tkzLabelAngle[pos=.6](E,D,A){3}  \tkzLabelAngle[pos=.6](C,B,A){1}
\tkzLabelAngle[pos=.6](A,C,B){2}  \tkzLabelAngle[pos=.6](A,E,D){4}
        \end{tikzpicture}
        \caption*{第11题}
        \end{minipage}
        \begin{minipage}[t]{0.46\textwidth}
        \centering
        \begin{tikzpicture}[>=latex, scale=1]
 \tkzDefPoints{0/0/C, 3/0/D, 0/1/A, 3/1/B, .5/-1/E, 1.5/2/F, 2.5/-1/G, 2/2/H}
 \tkzDrawSegments(A,B C,D E,F G,H)
\tkzLabelPoints[left](A,C)\tkzLabelPoints[right](B,D)
\tkzInterLL(A,B)(E,F) \tkzGetPoint{X1}
\tkzInterLL(C,D)(E,F) \tkzGetPoint{X2}
\tkzInterLL(A,B)(G,H) \tkzGetPoint{Y1}
\tkzInterLL(C,D)(G,H) \tkzGetPoint{Y2}

\tkzMarkAngles[mark=none, size=.25](A,X1,E D,X2,F G,Y1,B G,Y2,D)
\tkzLabelAngle[pos=.4](G,Y1,B){3}  \tkzLabelAngle[pos=.4](A,X1,E){1}
\tkzLabelAngle[pos=.4](D,X2,F){2}  \tkzLabelAngle[pos=.4](G,Y2,D){4}


        \end{tikzpicture}
        \caption*{第13题}
        \end{minipage}

        \end{figure}

    \begin{figure}[htp]\centering
        \begin{minipage}[t]{0.46\textwidth}
        \centering
    \begin{tikzpicture}[>=latex, scale=1]
\tkzDefPoints{0/0/C, 2.5/0/D, 0/1.5/A, 3/1.5/B, 4/1/E}
\tkzLabelPoints[above](A,B)
\tkzLabelPoints[below](C,D)
\tkzLabelPoints[right](E)
\tkzDrawSegments(A,B C,D D,E B,E)
        \end{tikzpicture}
        \caption*{第14题}
        \end{minipage}
        \begin{minipage}[t]{0.46\textwidth}
            \centering
            \begin{tikzpicture}[>=latex, scale=1]
\tkzDefPoints{0/0/D, 3/0/A, 0.75/2/C, 3.75/2/B}
\tkzLabelPoints[right](A,B)
\tkzLabelPoints[left](C,D)
\tkzDrawSegments(B,D)
\tkzDrawPolygon(A,B,C,D)
\tkzDefPointWith[linear, K=.3](B,D) \tkzGetPoint{E}
\tkzDefPointWith[linear, K=.7](B,D) \tkzGetPoint{F}
\tkzDrawSegments(A,E C,F)
\tkzLabelPoints[below](F)
\tkzLabelPoints[above](E)
            \end{tikzpicture}
            \caption*{第15题}
            \end{minipage}
        \end{figure}



\subsection{内角和定理}

\begin{blk}
  {三角形内角和定理} 三角形三内角和等于$180^{\circ}$.  
\end{blk}

这个定理在第一章用实验的办法验证过,现在我们应用
平行线的性质来证明它。

已知:任一$\triangle ABC$(图3.33). 求证:$\angle A+\angle B+
\angle C=180^{\circ}$.

\begin{figure}[htp]
    \centering
\begin{tikzpicture}
\tkzDefPoints{0/0/B, 4/0/C, 1/2.5/A, 0/2.5/D, 4/2.5/E}
\tkzDrawPolygon(A,C,B)
\tkzDrawSegments[dashed](D,E)
\tkzLabelPoints[left](B,D)\tkzLabelPoints[right](C,E)
\tkzLabelPoints[above](A)
\end{tikzpicture}
    \caption{}
\end{figure}

\begin{proof}
    过$A$点引直线
$DE\parallel BC$(平行公理)
则$\angle DAB=\angle B$, $\angle EAC=\angle C$(两条直线平行,
则内错角相等)。

$\because\quad DE$是过$A$点的直线(作图),

$\therefore\quad \angle DAB+\angle BAC+\angle CAE=180^{\circ}$(平角定义)。

$\therefore\quad \angle A+\angle B+\angle C=180^{\circ}$(等量代换)。
\end{proof}

由三角形内角和定理可以得到下列推论。

\begin{blk}
    {推论1} 一个三角形的两角若分别等于另一个三角形的
两角,则第三个角也相等。
\end{blk}


\begin{blk}
    {推论2} 两个三角形如果有两个角和其中一个角所对的
边分别相等,那么它们就全等。(AAS)
\end{blk}


\begin{blk}
    {推论3} 三角形的一个外角等于和它不相邻的两个内角
的和。
\end{blk}

\begin{blk}
    {推论4} 正三角形的各内角是$60^{\circ}$.
\end{blk}

\begin{blk}
    {推论5 }三角形中,至多有一个直角或一个钝角。
\end{blk}

\begin{blk}
   {定义} 每个内角都是锐角的三角形,叫做\textbf{锐角三角形};
有一个内角是直角的三角形叫做\textbf{直角三角形},夹直角的两边
叫做\textbf{直角边},直角的对边叫做\textbf{斜边};有一个内角是钝角的三
角形叫做\textbf{钝角三形角}(图3.34)。
\end{blk}

\begin{figure}[htp]
    \centering
\begin{tikzpicture}
\begin{scope}
    \tkzDefPoints{0/0/A, 3/0/B, 1.6/3/C}
    \tkzDrawPolygon(A,B,C)
\tkzMarkAngles[mark=none, size=.4](B,A,C C,B,A A,C,B)
\node at (1.5,-.5){锐角三角形};
\end{scope}
\begin{scope}[xshift=4cm]
        \tkzDefPoints{0/0/A, 2/0/B, 2/3/C}
    \tkzDrawPolygon(A,B,C)
    \node at (1,-.5){直角三角形};
    \tkzMarkRightAngles(C,B,A)
\end{scope}
\begin{scope}[xshift=7cm]
    \tkzDefPoints{0/0/A, 2/0/B, 3.6/2.8/C}
    \tkzDrawPolygon(A,B,C)
    \tkzMarkAngles[mark=none, size=.3](C,B,A)
    \node at (1.8,-.5){钝角三角形};
\end{scope}
\end{tikzpicture}
    \caption{}
\end{figure}

这样,三角形可以根据它的内角的大小分为三类:锐角
三角形、直角三角形和钝角三角形。

\begin{blk}
{推论6} 直角三角形中,非直角的两个角都是锐角,并
且这两个锐角互余。
\end{blk}

\begin{blk}
{定义}
三边都不相等的三角形叫做\textbf{不等边三角形}。
\end{blk}

这样三角形又可以按边长分成三类:等边三角形,底边
和腰不等的等腰三角形和不等边三角形。


\begin{example}
    已知:$O$是$\triangle ABC$内任一点(图3.35)。

求证:$\angle BOC=\angle A+\angle ABO+\angle ACO$.
\end{example}

\begin{proof}
 引$AO$交$\overline{BC}$于$D$点,
则$\angle DOB=\angle ABO+\angle BAO$, 
$\angle DOC=\angle ACO+\angle CAO$(三
角形外角等于不相邻内角和)

$\therefore\quad \angle DOB+\angle DOC=\angle ABO+\angle ACO+
\angle BAO+\angle CAO$(等量加等量和相等)。

$\therefore\quad \angle BOC=\angle A+\angle ABO+\angle ACO$
\end{proof}

\begin{figure}[htp]\centering
    \begin{minipage}[t]{0.48\textwidth}
    \centering
\begin{tikzpicture}[>=latex, scale=1]
\tkzDefPoints{0/0/B, 3/0/C, 2.5/3/A, 2/1.2/O}
\tkzDrawPolygon(A,B,C)
\tkzDrawSegments(B,O O,C)
\tkzInterLL(A,O)(B,C) \tkzGetPoint{D}
\tkzDrawSegments[dashed](A,D)
\tkzLabelPoints[below](B,D,C)
\tkzLabelPoints[above](A,O)
    \end{tikzpicture}
    \caption{}
    \end{minipage}
    \begin{minipage}[t]{0.48\textwidth}
    \centering
    \begin{tikzpicture}[>=latex, scale=1]
\tkzDefPoints{-1.5/0/B, 1.5/0/C, 0/3.2/A}
\tkzDrawPolygon(A,B,C)
\tkzDefPointBy[projection = onto A--B](C) \tkzGetPoint{D}
\tkzDefPointBy[projection = onto A--C](B) \tkzGetPoint{E}
\tkzDrawSegments(B,E C,D)
\tkzLabelPoints[left](D,B)
\tkzLabelPoints[right](E,C)
\tkzLabelPoints[above](A)
\tkzMarkRightAngles[size=.2](B,D,C C,E,B)


    \end{tikzpicture}
    \caption{}
    \end{minipage}
    \end{figure}


\begin{example}
    在图3.36中。
    已知:$\overline{CD}\bot\overline{AB}$于$D$, $\overline{BE}\bot\overline{AC}$于$E$, 且$\overline{BE}=\overline{CD}$.

    求证:$\angle ABC=\angle ACB$
\end{example}

\begin{proof}
    在$\triangle ABE$与$\triangle ACD$中,

$\because\quad  \overline{BE}\bot\overline{AC}$于$E$. $\overline{CD}\bot\overline{AB}$于$D$(已知)

$\therefore\quad \angle ADC=90^{\circ},\quad \angle AEB=90^{\circ}$
(垂直定义)。

$\therefore\quad \angle ADC=\angle AEB$(等量代换)。

又$\because\quad \angle A=\angle A$(公共角),$\overline{BE}=\overline{CD}$(已知)

$\therefore\quad \triangle ABE\cong \triangle ACD$ (AAS).

$\therefore\quad \overline{AB}=\overline{AC}$(全等三角形的对应边相等)。

$\therefore\quad \angle ABC=\angle ACB$. (等腰三角形的两底角相等)。
\end{proof}

\begin{example}
    试证明定理:角平分线上的点与角的两边距离
相等。

已知:$BD$是$\angle ABC$的平分线,$P\in BD$, $\overline{PE}\bot BA$于
$E$, $\overline{PF}\bot BC$于$F$ (图3.37). 
求证:$\overline{PE}=\overline{PF}$.
\end{example}

\begin{figure}[htp]
    \centering
\begin{tikzpicture}
\tkzDefPoints{0/0/B, 4/0/D, 3/0/P}
\tkzDefPoint(25:3.8){A}  \tkzDefPoint(-25:3.8){C}
\tkzDefPointBy[projection = onto A--B](P) \tkzGetPoint{E}
\tkzDefPointBy[projection = onto B--C](P) \tkzGetPoint{F}
\tkzMarkRightAngles[size=.2](B,E,P B,F,P)
\tkzDrawSegments(A,B B,C B,D E,P F,P)
\tkzLabelPoints[below](F,C,P) \tkzLabelPoints[above](E,A)
\tkzLabelPoints[left](B)\tkzLabelPoints[right](D)
\end{tikzpicture}
    \caption{}
\end{figure}

\begin{proof}
在$\triangle PEB$和$\triangle PFB$中,

$\because\quad BD$平分$\angle ABC$(已知),

$\therefore\quad \angle EBP=\angle FBP$(角平分线定义)。

又$\because\quad \overline{PE}\bot BA$于$E$, 
$\overline{PF}\bot BC$于$F$(已知),

$\therefore\quad \angle PEB=90^{\circ}$, $\angle PFB=90^{\circ}$(垂直定义),

$\therefore\quad \angle PEB=\angle PFB$(等量代换)。

又$\because\quad \overline{BP}=\overline{BP}$(公共边),

$\therefore\quad \triangle PEB\cong \triangle PFB$ (AAS)

$\therefore\quad \overline{PE}=\overline{PF}$(全等三角形的对应边相等)。
\end{proof}

\begin{blk}
    {定义} 连结多边形的不相邻两个顶点的线段叫做这个多
    边形的\textbf{对角线}。如图3.38中的四边形$ABCD$中就有两条对
    角线$\overline{AC}$、$\overline{BD}$.
\end{blk}

\begin{figure}[htp]\centering
    \begin{minipage}[t]{0.48\textwidth}
    \centering
\begin{tikzpicture}[>=latex, scale=1]
\tkzDefPoints{0/0/B, 3/.5/C, 2.8/1.8/D, 1.3/2.5/A}
\tkzDrawPolygon(A,B,C,D)
\tkzDrawSegments(A,C B,D)
\tkzLabelPoints[below](B,C)
\tkzLabelPoints[above](A,D)
    \end{tikzpicture}
    \caption{}
    \end{minipage}
    \begin{minipage}[t]{0.48\textwidth}
    \centering
    \begin{tikzpicture}[>=latex, scale=1]
        \tkzDefPoints{0/0/C, 1.7/.5/D, 2.5/1.5/E, 1/2.5/A, -.75/1.3/B}
        \tkzDrawPolygon(A,B,C,D,E)
        \tkzDrawSegments(A,C A,D)
        \tkzLabelPoints[left](B,C) \tkzLabelPoints[right](E,D)
        \tkzLabelPoints[above](A)
    \end{tikzpicture}
    \caption{}
    \end{minipage}
    \end{figure}

\begin{blk}
    {推论}
从$n$边形的一个顶点引对角线,可把这个$n$边形分
成$n-2$个三角形,如图3.39中的五边形$ABCDE$,从$A$点
引对角线$\overline{AC}$、$\overline{AD}$,则分成了三个三角形。
\end{blk}

由此推论不难得出多边形内角和定理:

\begin{blk}{多边形内角和定理}
    任意$n$边形的内角和等于$(n-2)\x180^{\circ}$.  
\end{blk}

这个定理请同学们参看图3.40或图3.41自己证明。

\begin{figure}[htp]\centering
    \begin{minipage}[t]{0.48\textwidth}
    \centering
\begin{tikzpicture}[>=latex, scale=1]
\tkzDefPoint(-20:1.5){A_4}
\tkzDefPoint(40:1.5){A}
\tkzDefPoint(80:1.5){B}
\tkzDefPoint(-60:1.5){A_3}
\tkzDefPoint(-130:1.5){A_2}
\tkzDefPoint(180:1.5){A_1}
\tkzDefPoint(130:1.5){A_n}
\tkzDrawSegments(A_1,A A_1,A_4 A_1,A_3)
\tkzDrawPolygon(A_1,A_2,A_3,A_4,A,B,A_n)
\tkzDefPoint(0,0){O}
\tkzAutoLabelPoints[center=O](A_1,A_2,A_3,A_4,A_n)


    \end{tikzpicture}
    \caption{}
    \end{minipage}
    \begin{minipage}[t]{0.48\textwidth}
    \centering
    \begin{tikzpicture}[>=latex, scale=1]
\tkzDefPoint(-20:1.5){A}
\tkzDefPoint(80:1.5){A_4}
\tkzDefPoint(-60:1.75){A_n}
\tkzDefPoint(-130:1.4){A_1}
\tkzDefPoint(170:1.5){A_2}
\tkzDefPoint(130:1.25){A_3}
\tkzDrawSegments(O,A_1 O,A_2 O,A_3 O,A_4 O,A_n O,A)
\tkzDrawPolygon(A_1,A_2,A_3,A_4,A,A_n)
\tkzDefPoint(0,0){O}
\tkzAutoLabelPoints[center=O](A_1,A_2,A_3,A_4,A_n)
\draw(0,0) circle(.5);
\tkzLabelPoints[below](O)

    \end{tikzpicture}
    \caption{}
    \end{minipage}
    \end{figure}

\begin{blk}{定义}
    多边形内角的一边和另一边的反向延长线所组
成的角叫做多边形的\textbf{外角}。
\end{blk}

对于任一个$n$边形每一内角取一个相应的外角,如图3.42中的$\alpha'_1, \alpha'_2,\ldots,\alpha'_n$, 很显然,一个内角与它相应的一个外角和为一个平角,因此:
\[\begin{split}
  (n-2)  \text{个平角外角和}+\text{外角和}&=\text{内角和}+\text{外角和}\\
  &=(\alpha_1+\alpha_2+\cdots+\alpha_n)+(\alpha'_1+\alpha'_2+\cdots+\alpha_n')\\
  &=n\text{个平角}
\end{split}\]
$\therefore\quad \text{外角和}=\text{2个平角}=360^{\circ}$

因而有多边形外角和定
理:

\begin{blk}{多边形外角和定理}
  任意多边形外角和都等
于$360^{\circ}$  
\end{blk}


\begin{blk}
  {定义} 各边都相等,各
角都相等的多边形叫做\textbf{正多
边形}。  
\end{blk}

\begin{figure}[htp]\centering
    \begin{minipage}[t]{0.48\textwidth}
    \centering
\begin{tikzpicture}[>=latex, scale=1]
\tkzDefPoint(20:1.5){B}
\tkzDefPoint(20+60:1.5){C}
\tkzDefPoint(20+120:1.5){D}
\tkzDefPoint(20+180:1.5){E}
\tkzDefPoint(20+240:1.5){F}
\tkzDefPoint(-40:1.5){A}
\tkzDrawPolygon(A,B,C,D,E,F)
\tkzDefPointWith[linear, K=1.75](A,B)  \tkzGetPoint{B1}
\tkzDefPointWith[linear, K=1.75](B,C)\tkzGetPoint{C1}
\tkzDefPointWith[linear, K=1.75](C,D)\tkzGetPoint{D1}
\tkzDefPointWith[linear, K=1.75](D,E)\tkzGetPoint{E1}
\tkzDefPointWith[linear, K=1.75](E,F)\tkzGetPoint{F1}
\tkzDefPointWith[linear, K=1.75](F,A)\tkzGetPoint{A1}

\tkzDrawSegments(A,A1 B,B1 C,C1 D,D1 E,E1 F,F1)
\tkzMarkAngles[mark=none, size=.4](B1,B,C1 C1,C,D1 D1,D,E1 E1,E,F1 F1,F,A A1,A,B1)
\tkzMarkAngles[mark=none, size=.25](F,E,D E,D,C D,C,B C,B,A B,A,F A,F,E)

\tkzLabelAngle[pos=.5](F,E,D){$\alpha_1$}
\tkzLabelAngle[pos=.5](A,F,E){$\alpha_2$}
\tkzLabelAngle[pos=.5](B,A,F){$\alpha_3$}
\tkzLabelAngle[pos=.5](C,B,A){$\alpha_4$}
\tkzLabelAngle[pos=.5](E,D,C){$\alpha_n$}
\tkzLabelAngle[pos=.7](E1,E,F1){$\alpha_1'$}
\tkzLabelAngle[pos=.7](F1,F,A){$\alpha_2'$}
\tkzLabelAngle[pos=.7](A1,A,B1){$\alpha_3'$}
\tkzLabelAngle[pos=.7](B1,B,C1){$\alpha_4'$}
\tkzLabelAngle[pos=.7](D1,D,E1){$\alpha_n'$}
    \end{tikzpicture}
    \caption{}
    \end{minipage}
    \begin{minipage}[t]{0.48\textwidth}
    \centering
    \begin{tikzpicture}[>=latex, scale=1]
\tkzDefPoint(180-18:2){B}
\tkzDefPoint(-126:2){C}
\tkzDefPoint(-54:2){D}
\tkzDefPoint(18:2){E}
\tkzDefPoint(90:2){A}
\tkzDrawPolygon(A,B,C,D,E)
\tkzDefPoint(0,0){O}
\tkzAutoLabelPoints[center=O](A,B,C,D,E)
\tkzDrawSegments(A,D C,E)
\tkzInterLL(A,D)(C,E)  \tkzGetPoint{F}
\tkzLabelPoints[left](F)
\tkzMarkAngles[mark=none, size=.4](F,A,E  E,D,F F,E,D)
\tkzMarkAngles[mark=none, size=.3](E,F,A A,E,F)
\tkzLabelAngle[pos=.7](E,D,F){$36^{\circ}$}
\tkzLabelAngle[pos=.7](F,E,D){$36^{\circ}$}
\tkzLabelAngle[pos=.6](A,E,F){$72^{\circ}$}
\tkzLabelAngle[pos=.7](F,A,E){$36^{\circ}$}
\node at (54:1.8){$a$};
    \end{tikzpicture}
    \caption{}
    \end{minipage}
    \end{figure}


\begin{example}
    在图3.43中,已知:$ABCDE$是正五边形,对角
    线$\overline{AD}$、$\overline{CE}$相交于$F$.

    求证:$\triangle AEF$、$\triangle DEF$、$\triangle CDF$都是等腰三角形。
\end{example}

\begin{proof}
由多边形内角和公式可知:正五边形的各内角
和
\[(5-2)\x180^{\circ}=540^{\circ}\]
由于正五边形各内角相等,

$\therefore\quad \angle AED=540^{\circ}\x\frac{1}{5}=108^{\circ}$

$\because\quad \triangle EAD$是等腰三角形,

$\therefore\quad \angle EAD=\angle EDA=(180^{\circ}-108^{\circ})\x\frac{1}{2}=36^{\circ}$

$\because\quad \triangle DCE$也是等腰三角形,
$\angle D=108^{\circ}$,

$\therefore\quad \angle DEF=36^{\circ}$.

$\therefore\quad \angle AEF=108^{\circ}-36^{\circ}=72^{\circ}$

$\angle AFE=180^{\circ}-36^{\circ}-72^{\circ}=72^{\circ}$, $\overline{AE}=\overline{AF}$.

$\therefore\quad \triangle AEF$是等腰三角形。

同理可证$\triangle CDF$也是等腰三角形,腰长等于正五边形的
边长;$\triangle DEF$也是等腰三角形。

请同学们用量角器和直尺,
划一个正五边形。
\end{proof}
    
\begin{example}
    已知:正六边形
    $ABCDEF$ (图3.44)。

    求证:对角线$\overline{AD}$、$\overline{BE}$
    $\overline{CF}$相交于一点,且把正六
    边形$ABCDEF$分割成六个
    正三角形。
\end{example}

\begin{figure}[htp]
    \centering
\begin{tikzpicture}
\tkzDefPoint(20:2){E}
\tkzDefPoint(80:2){F}
\tkzDefPoint(140:2){A}
\tkzDefPoint(200:2){B}
\tkzDefPoint(260:2){C}
\tkzDefPoint(320:2){D}
\tkzDefPoint(0,0){O}
\tkzAutoLabelPoints[center=O](A,B,C,D,E,F)
\tkzDrawSegments(A,D B,E C,F)
\tkzDrawPolygon(A,B,C,D,E,F)
\tkzLabelPoints[above right](O)
\end{tikzpicture}
    \caption{}
\end{figure}

\begin{proof}
    已知$ABCDEF$是正六边形,所以它的每一
    个内角
    \[(6-2)\x180^{\circ} \x\frac{1}{6}=120^{\circ}\]
    
    以$\overline{AB}$为边作正三角形$ABO$, 顶点$O$在正六边形$ABCD
    EF$内,连结$\overline{OC}$、$\overline{OD}$、$\overline{OE}$、$\overline{OF}$.

$\because\quad \angle B=120^{\circ},\quad \angle ABO=60^{\circ}$,
 
$\therefore\quad \angle OBC=60^{\circ}$.

又$\because\quad \overline{OB}=\overline{BC}$,

$\therefore\quad \angle BCO=\angle BOC=60^{\circ}$.

$\therefore\quad \triangle OBC$也是正三角形。

    同理可证:
$\triangle OCD$、$\triangle ODE$、$\triangle OEF$、$\triangle OFA$都是正三角形。
    由于$\angle AOB+\angle BOC+\angle COD=180^{\circ}$, 所以$A$、$O$、$D$
    在同一条直线上,同理,$B$、$O$、$E$,$C$、$O$、$F$也都分别在
    同一条直线上,所以对角线$\overline{AD}$、$\overline{BE}$、$\overline{CF}$相交于一点$O$, 
    且把正六边形分隔成六个全等的正三角形。
\end{proof}


    根据例3.21得出的正六边形的性质,请同学们想想,若已知
    正六边形的边长,如何用圆规和直尺画一个正六边形。
    


\begin{example}
    测量斜面的倾斜角,常用测倾仪(图3.45(1))。
测倾仪悬垂的指针所指的度数,就是倾角的度数,为什么?
\begin{figure}[htp]
    \centering
\includegraphics[scale=.6]{fig/3-45.png}
    \caption{}
\end{figure}
\end{example}


\begin{analyze}
    为了便于说明,我们将测倾仪指针所指倾角和斜面
    的斜角位置关系用图3.45(2)来表示,问题便转化为:

    已知:$AC\bot BC$, $AD\bot BD$. 求证:$\angle A=\angle B$.
\end{analyze}

\begin{proof}
    设$BC$、$AD$相交于$E$点。

$\because\quad AC\bot BC$  (已知),

$\therefore\quad \angle ECA=90^{\circ}$ (垂直定义)。

$\therefore\quad$ 在$\triangle ACE$中,$\angle A+\angle AEC=90^{\circ}$(直角三角形两锐角互余)。

同理:$\angle B+\angle BED=90^{\circ}$

又$\because\quad \angle AEC=\angle BED$(对顶角相等),

$\therefore\quad \angle A=\angle B$(等角的余角相等)。
\end{proof}
这个例题可以写成下面的定理:

\begin{blk}
   {定理} 如果一个锐角的两边分别垂直于另一个锐角的两
边,那么这两个锐角相等。 
\end{blk}

根据这个定理很容易证明下面的例题。

\begin{example}
已知:在图3.46中,
$\angle BAC=90^{\circ}$, $AD\bot BC$于
$D$点。

求证:$\angle B=\angle CAD$, $\angle C=\angle BAD$.
\end{example}


\begin{proof}
$\because\quad $在$\triangle ABC$中,
$\angle BAC=90^{\circ}$(已知),

$\therefore\quad \angle B$、$\angle C$都是锐角(直角三角形中,非直角的两
个角都是锐角)。

又$\because\quad AD$在$\angle BAC$的内部,

$\therefore\quad \angle CAD$也是锐角(全量大于它的任何一部分)。

由于,$AB\bot AC$, $AD\bot BC$, 

$\therefore\quad \angle B=\angle CAD$.

同理:$\angle C=\angle BAD$.(两边分别互相垂直的两个锐角相
等)
\end{proof}

\begin{figure}[htp]\centering
    \begin{minipage}[t]{0.48\textwidth}
    \centering
\begin{tikzpicture}[>=latex, scale=1]
\tkzDefPoints{0/0/B, 3.2/0/D, 5/0/C, 3.2/2.4/A}
\tkzDrawPolygon(A,B,C)
\tkzDrawSegments(A,D)
\tkzLabelPoints[below](B,C,D)
\tkzLabelPoints[above](A)
\tkzMarkAngles[mark=none, size=.3](C,B,A)
\tkzMarkRightAngles(A,D,C B,A,C)

    \end{tikzpicture}
    \caption{}
    \end{minipage}
    \begin{minipage}[t]{0.48\textwidth}
    \centering
    \begin{tikzpicture}[>=latex, scale=1]
\tkzDefPoints{-1.5/0/B, 1.5/0/B', 0/0/C, 0/2.5/A}
\tkzDrawPolygon(A,B,B')
\tkzDrawSegments(A,C)
\tkzMarkRightAngles(A,C,B)
\tkzLabelPoints[below](B,B')
\node at (C)[below]{$C(C')$};
\node at (A)[above]{$A(A')$};
    \end{tikzpicture}
    \caption{}
    \end{minipage}
    \end{figure}


\begin{example}
    已知:$\triangle ABC$和$\triangle A'B'C'$中,
$\angle ACB=\angle A'C'B'=90^{\circ}$, $\overline{AB}=\overline{A'B'}$,
$\overline{AC}=\overline{A'C'}$ (图3.47).

求证:$\triangle ABC\cong \triangle A'B'C'$
\end{example}

\begin{proof}
    根据移形公理,移动
$\triangle A'B'C'$, 因$\overline{AC}=\overline{A'C'}$,
所以可使$\overline{A'C'}$与$\overline{AC}$重合,
并使$B$、$B'$落于$\overline{AC}$的两侧。

由于$\angle ACB=\angle ACB'=90^{\circ}$.

$\therefore\quad \angle  BCB'=180^{\circ}$.

$\therefore\quad B$、$C$、$B'$在一条直线上。

在$\triangle ABB'$中,

$\because\quad \overline{AB}=\overline{A'B}=\overline{AB'}$,

$\therefore\quad \angle B=\angle B'$(等腰三角形的两底角相等)。

在$\triangle ABC$和$\triangle A'B'C'$中,

$\because\quad \angle ACB=\angle A'C'B'$, $\angle B=\angle B'$, $\overline{AB}=\overline{A'B'}$.

$\therefore\quad \triangle ABC\cong \triangle A'B'C'$ (AAS)    
\end{proof}

由上例可得如下定理:

\begin{blk}
  {定理} 斜边和一直角边对应相等的两个直角三角形全
等。  
\end{blk}

\begin{ex}
\begin{enumerate}
    \item 求下列各图中的$\angle x$、$\angle y$的度数。
    \item 如果等腰$\triangle ABC$的顶角是$\angle C$, 
\begin{enumerate}
\item 当$\angle C=40^{\circ}$ 时,求$\angle A$;
\item 当$\angle C=m^{\circ}$ 时,求$\angle B$;
\item 当$\angle A=40^{\circ}$ 时,求$\angle C$;
\item 当$\angle B=n^{\circ}$ 时,求$\angle C$.
\end{enumerate}

\item 已知$\triangle ABC$中,$\overline{AB}=\overline{AC}$, $AD$是$\angle BAC$的外角平分
    线,求证$AD\parallel BC$.
    \item 求证:三角形的两个外角的和减去第三个内角等于二个
    直角。
    \item 试证:若三角形的二外角和等于三个直角,此三角形必
    是直角三角形。
    \item 如果等腰三角形的顶角是$60^{\circ}$, 则此三角形为正三角
    形。
    \item 在$\triangle ABC$中,$\angle A=25^{\circ}$, $\angle B=68^{\circ}$, 经过这个三角
    形的顶点,作对边的平行线,求这些直线组成的三角形的
    三内角的度数。
    \item 三角形的两个角分别等于$62^{\circ}$ 和$72^{\circ}$, 从这两个角的顶
    点作
    对边的高线,求这两条高线的夹角。
    \item 如果一个三角形的任意两个角的和大于$90^{\circ}$, 这个三角
    形是怎样的三角形?
\item 已知六边形$ABCDEF$中,
$\angle A=a^{\circ}$, $\angle B=b^{\circ}$, 
$AF\parallel CD$, $AB\parallel DE$, 试
求图中$x$、$y$的度数。
\item 求七边形与十边形的内
角和。
\item 如果多边形的所有内角和等于:
\[1080^{\circ} ,\quad 1620^{\circ}  ,\quad 3960^{\circ}  ,\quad 1840^{\circ}\]那
么这些多边形各有几条边?
\item 如果多边形的每一个内角都等于:
\[144^{\circ} ,\quad 150^{\circ}  ,\quad 170^{\circ}  ,\quad 111^{\circ} 
,\quad 190^{\circ}\] 
那么这些多边形各有几条边?
\item 试证:两条平行线被一条直线所截时,两同旁内角的平
分线互相垂直。
\item 设$\overline{CD}$和$\overline{AE}$是$\triangle ABC$的高,求证$\angle BCD=\angle BAE$.
\item 试证:等腰三角形两腰上的高线相等。
\item 试证:等腰三角形中,从底的中点到两腰的距离相
等。
\item 试证:与角的两边距离相等的点在该角的平分线上。
\item 试证:与线段两端点距离相等的点在该线段的垂直平分
线上。
\end{enumerate}
\end{ex}

\begin{figure}[htp]
    \centering
\begin{tikzpicture}
\begin{scope}
\tkzDefPoints{0/0/B, 5/0/C, 3.5/3.5/A, 2.5/1.5/D}
\tkzDrawPolygon(A,B,C)
\tkzDrawSegments(B,D C,D)
\tkzMarkAngles[mark=none, size=.4](A,C,D B,A,C B,D,C)
\tkzMarkAngles[mark=none, size=1.2](D,B,A)
\tkzLabelAngle[pos=1.6](D,B,A){$23^{\circ}$}
\tkzLabelAngle[pos=.6](A,C,D){$39^{\circ}$}
\tkzLabelAngle[pos=.6](B,A,C){$70^{\circ}$}
\tkzLabelAngle[pos=.6](B,D,C){$x$}
\tkzLabelPoints[below](B,C)
\tkzLabelPoints[above](A)

\end{scope}
\begin{scope}[xshift=6cm, yshift=1cm]
    \tkzDefPoints{0/0/B, 4/-1/C, 0/1/A, 3/2.5/E, 4.5/1/D}
    \tkzDrawPolygon(A,B,C,D,E)
    \tkzLabelPoints[below](B,C)
    \tkzLabelPoints[above](E)
    \tkzLabelPoints[right](D)
    \tkzLabelPoints[left](A)
    \tkzMarkAngles[mark=none, size=.3](C,B,A B,A,E A,E,D E,D,C)
\tkzMarkRightAngle(D,C,B)
\tkzLabelAngle[pos=.6](C,B,A){$105^{\circ}$}
\tkzLabelAngle[pos=.6](A,E,D){$120^{\circ}$}
\tkzLabelAngle[pos=.6](B,A,E){$x$}
\tkzLabelAngle[pos=.6](E,D,C){$107^{\circ}$}
\tkzDefPointWith[linear, K=1.8](E,D) \tkzGetPoint{F}
\tkzDrawSegments[dashed](D,F)
\tkzMarkAngles[mark=none, size=.45](C,D,F)
\tkzLabelAngle[pos=.6](C,D,F){$y$}
\end{scope}
\end{tikzpicture}
    \caption*{第1题}
\end{figure}

\begin{figure}[htp]
    \centering
\begin{tikzpicture}[scale=.8]
\tkzDefPoints{0/0/C, 2/0/D, 3.5/1.5/E, -1.5/2.5/B, 0/4/A, 2.2/4/F}
\tkzLabelPoints[below](C,D)
\tkzLabelPoints[above](A,F)
\tkzLabelPoints[right](E)
\tkzLabelPoints[left](B)
\tkzDrawPolygon(A,B,C,D,E,F)
\tkzInterLL(A,B)(D,C) \tkzGetPoint{G}
\tkzLabelPoints[left](G)
\tkzDrawSegments[dashed](C,G B,G)
\tkzMarkAngles[mark=none, size=.3](C,B,A B,A,F D,C,B E,D,C)
\tkzLabelAngle[pos=.5](C,B,A){$b^{\circ}$}
\tkzLabelAngle[pos=.5](B,A,F){$a^{\circ}$}
\tkzLabelAngle[pos=.5](D,C,B){$x$}
\tkzLabelAngle[pos=.5](E,D,C){$y$}

\end{tikzpicture}
    \caption*{第10题}
\end{figure}

\subsection*{习题3.2}
\begin{enumerate}
    \item 在钝角三角形中,哪一条边最大?为什么?
    \item 试证明:三角形最小边总对着锐角。
    \item 在$\triangle ABC$中,$\overline{AB}$是最大的边,这个三角形的哪些角可
    能是锐角,可能是$\angle C$吗?

    \item    在等腰三角形中,如果它的腰长小于底边的长,它的顶
    角的大小应当在什么范围内?
    \item 证明:在三角形中最小的锐
    角不大于$60^{\circ}$.
    \item 如图,已知:$\angle B=90^{\circ}$, $D$点
    在$\overline{BC}$延长线上

    求证:$\overline{AD}>\overline{AC}$.

\begin{figure}[htp]
    \centering
\begin{tikzpicture}
\tkzDefPoints{0/0/D, 2/0/C, 3.5/0/B, 3.5/2/A}
\tkzDrawSegments(A,D)
\tkzDrawPolygon(A,B,C)
\tkzLabelPoints[below](B,C,D)
\tkzLabelPoints[above](A)
\tkzMarkRightAngle(A,B,C)
\tkzDrawSegments[dashed](C,D)
\end{tikzpicture}
    \caption*{第6题}
\end{figure}

    \item 证明:直角三角形中斜边比直角边长。
    \item 证明:从直线外一点和直线上各点所连线段中,和直线
    垂直的线段最短。
    \item 设$\triangle ABC$中的两角$\angle B$和$\angle C$的平分线相交于$O$, 过$O$
    作$\overline{BC}$的平行线,与$\overline{AB}$、$\overline{AC}$分别交于$D$、$E$, 则$\overline{DE}=
    \overline{BD}+\overline{CE}$.
    \item 在直角三角形$ABC$的斜边$\overline{AC}$上取两点$D$、$E$, 使$\overline{AD}=\overline{AB}$, $\overline{CE}=\overline{CB}$, 求$\angle DBE$的度数。
    \item 证明:在任一多边形中,锐角不能超过四个。
    \item 已知:$\triangle ABC$中,$AE$是$\angle A$外角的平分线,
    且$AE\parallel BC$. 
    求证:$\overline{AB}=\overline{AC}$.
    \item 已知:$\triangle ABC$中,$\overline{AB}=\overline{AC}$, $\angle A=36^{\circ}$, $\overline{BD}$是$\angle B$
    的平分线,
    求证:$\overline{AD}=\overline{BD}=\overline{BC}$, $\angle BDC=2\angle A$
    \item 过等腰三角形底边$\overline{BC}$上或$\overline{BC}$的延长线上任一点$D$($\overline{BC}$
    的中点除外),作底边$\overline{BC}$的垂线,分别交两腰$\overline{AB}$、$\overline{AC}$
    (或延长线)于$E$、$F$. 

    求证:$\triangle AEF$为等腰三角形。 
\begin{figure}[htp]
    \centering
\begin{tikzpicture}
\begin{scope}
    \tkzDefPoints{-2/0/B, 2/0/C, 0/2/A, .5/0/D, .5/.5/G}
\tkzDrawPolygon(A,B,C)
\tkzDefPointWith[linear, K=1.5](B,A) \tkzGetPoint{H}
\tkzDrawSegments[dashed](A,H)
\tkzInterLL(B,A)(D,G) \tkzGetPoint{E}
\tkzInterLL(A,C)(D,E) \tkzGetPoint{F}
\tkzLabelPoints[below](B,C,D)
\tkzLabelPoints[above](A,E)
\tkzLabelPoints[right](F)
\tkzDrawSegments(D,E)
\end{scope}
\begin{scope}[xshift=5cm, scale=.6]
    \tkzDefPoints{-2/0/B, 2/0/C, 0/2/A, 3/0/D, 3/.5/G}
    \tkzDrawPolygon(A,B,C)
    \tkzDefPointWith[linear, K=3](B,A)  \tkzGetPoint{H}
    \tkzDefPointWith[linear, K=1.5](B,C)  \tkzGetPoint{I}
    \tkzDefPointWith[linear, K=1.5](A,C)  \tkzGetPoint{J}
    \tkzInterLL(B,A)(D,G) \tkzGetPoint{E}
    \tkzInterLL(A,C)(D,E) \tkzGetPoint{F}
    \tkzLabelPoints[below](B,C)
    \tkzLabelPoints[above](A,E)
    \tkzLabelPoints[right](F)
    \tkzLabelPoints[above right](D)
    \tkzDrawSegments[dashed](A,H C,J C,I)
    \tkzDrawSegments(E,F)
\end{scope}
\end{tikzpicture}
    \caption*{第14题}
\end{figure}    
\item 已知:$\triangle ABC$中,$D\in \overline{AB}$, $\overline{DE}\parallel \overline{AC}$交$\overline{BC}$于$E$, 
并且$\overline{DE}$平分$\angle BDC$, 求证:$\overline{DA}=\overline{DC}$.
\item 求图中五角星形的五个角之和。
\begin{figure}[htp]\centering
    \begin{minipage}[t]{0.48\textwidth}
    \centering
\begin{tikzpicture}[>=latex, scale=1]
\tkzDefPoint(90-72:2){C}
\tkzDefPoint(90:2){D}
\tkzDefPoint(90+72:2){E}
\tkzDefPoint(90-144:2){B}
\tkzDefPoint(90+144:2){A}
\tkzDefPoint(0,0){O}
\tkzDrawSegments(B,D A,D E,C A,C E,B)
\tkzAutoLabelPoints[center=O](A,B,C,D,E)


    \end{tikzpicture}
    \caption*{第16题}
    \end{minipage}
    \begin{minipage}[t]{0.48\textwidth}
    \centering
    \begin{tikzpicture}[>=latex, scale=1.2]
\tkzDefPoints{0/0/B, 3/0/C, 1/1.8/A}
\tkzDefEquilateral(B,A)  \tkzGetPoint{D}
\tkzDefEquilateral(A,C)  \tkzGetPoint{E}
\tkzDrawPolygon(A,C,B)
\tkzDrawPolygon(D,A,E,C,B)
\tkzDrawSegments(B,E C,D)
\tkzLabelPoints[below](B,C)
\tkzInterLL(B,E)(C,D) \tkzGetPoint{O}
\tkzLabelPoints[above](D,A,E,O)
    \end{tikzpicture}
    \caption*{第19题}
    \end{minipage}
    \end{figure}

\item 已知$\triangle ABC$中,$\angle B$和$\angle C$的平分线交于$P$, 引$PH\parallel
AB$, $PG\parallel AC$, 设$PH$和$PG$交$\overline{BC}$于$H$、$G$, 则$\triangle PHG$
的周长等于$\overline{BC}$.
\item 试证:全等三角形的对应边上的高线相等。
\item 以$\triangle ABC$的$\overline{AB}$、$\overline{AC}$为一边在形外分别作正三角形
$ABD$和正三角形$ACE$, 设$\overline{DC}$与$\overline{BE}$交于$O$.

求证:$\overline{DC}=\overline{BE}$, $\angle BOC=120^{\circ}$.
\item 在线段$\overline{AB}$上,取一点$C$, 在$\overline{AB}$的同旁作正$\triangle DAC$和正$\triangle ECB$, $\overline{DB}$、$\overline{EA}$分别交$\overline{EC}$、$\overline{DC}$于$G$、$F$.
试证:
\begin{enumerate}
    \item $\angle BDC=\angle EAC$
    \item $\overline{GC}=\overline{FC}$
    \item $\triangle CFG$为正三角形
    \item $FG\parallel AB$
\end{enumerate}
\end{enumerate}

\begin{figure}[htp]\centering
    \begin{minipage}[t]{0.48\textwidth}
    \centering
\begin{tikzpicture}[scale=1.2]
\tkzDefPoints{0/0/A, 1.5/0/C, 4/0/B}
\tkzDefEquilateral(C,B)  \tkzGetPoint{E}
\tkzDefEquilateral(A,C)  \tkzGetPoint{D}
\tkzDrawPolygon(A,C,D)
\tkzDrawPolygon(B,C,E)
\tkzInterLL(C,D)(A,E)\tkzGetPoint{F}
\tkzInterLL(B,D)(C,E)\tkzGetPoint{G}
\tkzDrawSegments(A,E C,E B,D F,G)
\tkzLabelPoints[below](A,B,C,F,G)
\tkzLabelPoints[above](D,E)
\end{tikzpicture}
    \caption*{第20题}
    \end{minipage}
    \begin{minipage}[t]{0.48\textwidth}
    \centering
    \begin{tikzpicture}[>=latex, scale=1]
\tkzDefPoints{0/0/B, 4/0/C, 1/2.5/A}
\tkzDefPointsBy[translation = from B to A](C){D}
\tkzDrawPolygon(A,B,C,D)
\tkzInterLL(B,D)(A,C)  \tkzGetPoint{O}
\tkzDrawSegments(B,D A,C)
\tkzLabelPoints[below](O,B,C)
\tkzLabelPoints[above](A,D)
    \end{tikzpicture}
    \caption{}
    \end{minipage}
    \end{figure}


\section{特殊四边形}
\subsection{平行四边形}
\begin{blk}{定义}
两组对边分别平行的四边形叫做平行四边形。平
行四边形用符号“$\parallelogram$”表示,图3.48中的平行四边形记作
“$\parallelogram ABCD$”。
\end{blk}


我们分析多边形的性质时,基本方法是把它分解为
一些三角形,然后利用三角
形的性质来研究多边形的性质。

$\parallelogram ABCD$ 的每条对角
线都把平行四边形分成一对全等的三角形,并且
\[\triangle AOB\cong \triangle COD,\qquad \triangle BOC\cong \triangle DOA\]

请同学们利用全等三角形来证明下面平行四边形性质定
理:

\begin{blk}
    {平行四边形的性质定理}
平行四边形具有下列性质:
\begin{enumerate}
    \item 对边相等;
    \item 对角相等;
    \item 相邻两角互补;
    \item 对角线互相平分。
\end{enumerate}
\end{blk}

\begin{blk}
    {定义} 同时垂直于两条平行线的直线,叫做这两条平行
线的\textbf{公垂线}。公垂线夹在平行线间的线段叫做\textbf{公垂线段}。
\end{blk}

两条平行线间的公垂线段有无数条,但根据平行四边形
的性质可以证明:
夹在平行线间的公垂线段处处相等。


\begin{blk}
    {定义} 两条平行线间的公垂线
段的长叫做\textbf{这两条平行线间的
距离}(图3.49)。
\end{blk}

\begin{figure}[htp]
    \centering
\begin{tikzpicture}[>=latex]
    \draw[<->](2.5,0)--node[fill=white]{距离}(2.5,2);
\draw[very thick](0,0)--(5,0);
\draw[very thick](0,2)--(5,2);
\draw[thick](3,0)--(3,2);
\draw[thick](4,0)--(4,2);
\draw[](1,-1)--(1,3);
\draw[](2,-1)--(2,3);
\end{tikzpicture}
    \caption{}
\end{figure}



上面讨论了平行四边形的性
质,这些性质都是平行四边形的一
些必要条件。我们怎样判断一个四
边形是平行四边形呢?也就是说一个四边形是平行四边形的
充分条件有哪些呢?

\begin{blk}
    {平行四边形判定定理} 四边形溝足下列任一条件时,就
是平行四边形。
\begin{enumerate}
\item 两组对边分别相等;
\item 两组对角分别相等;
\item 一组对边平行且相等;
\item 对角线互相平分。
\end{enumerate}
\end{blk}

证明时,只要证明两组对边平行即可,我们证明1、2, 请同学们自己证明3、4.

已知:四边形$ABCD$中,
$\overline{AD}=\overline{BC}$, $\overline{AB}=\overline{CD}$ (图3.50).

求证:四边形$ABCD$为平行四
边形。

\begin{proof}
    连$\overline{BD}$, 在$\triangle BAD$与$\triangle DCB$中,

$\because\quad     \overline{AB}=\overline{CD}$、$\overline{AD}=\overline{BC}$(已知),$\overline{BD}=\overline{BD}$ (公共边),

$\therefore\quad \triangle BAD\cong \triangle DCB$ (SSS).

$\therefore\quad \angle 1=\angle 2,\quad \angle 3=\angle 4$(全等三角形的对应角相等)。

$\therefore\quad AD\parallel BC,\quad AB\parallel CD$ (内错角相等,则两条直线
平行)。

$\therefore\quad $四边形$ABCD$为平行四边形(平行四边形定义)。
\end{proof}

\begin{figure}[htp]\centering
    \begin{minipage}[t]{0.48\textwidth}
    \centering
\begin{tikzpicture}[>=latex, scale=1]
\tkzDefPoints{0/0/B, 3/0/C, -.5/1.5/A}
\tkzDefPointsBy[translation = from B to A](C){D}
\tkzDrawPolygon(A,B,C,D)
\tkzDrawSegments(B,D)
\tkzMarkAngles[mark=none, size=.4](D,B,A B,D,C)
\tkzMarkAngles[mark=none, size=.5](A,D,B C,B,D)
\tkzLabelAngle[pos=.6](D,B,A){4}
\tkzLabelAngle[pos=.6](B,D,C){3}
\tkzLabelAngle[pos=.7](A,D,B){1}
\tkzLabelAngle[pos=.7](C,B,D){2}
\tkzLabelPoints[below](B,C)
\tkzLabelPoints[above](A,D)
    \end{tikzpicture}
    \caption{}
    \end{minipage}
    \begin{minipage}[t]{0.48\textwidth}
    \centering
    \begin{tikzpicture}[>=latex, scale=1]
        \tkzDefPoints{0/0/B, 3/0/C, .9/1.5/A}
        \tkzDefPointsBy[translation = from B to A](C){D}
        \tkzDrawPolygon(A,B,C,D)
        \tkzLabelPoints[below](B,C)
\tkzLabelPoints[above](A,D)
    \end{tikzpicture}
    \caption{}
    \end{minipage}
    \end{figure}

已知:四边形$ABCD$中,$\angle A=\angle C$, $\angle B=\angle D$(图3.51)。

求证:四边形$ABCD$为平行四边形。

\begin{proof}
$\because\quad \angle A+\angle B+\angle C +\angle D=360^{\circ}$(多边形内角和定理),

又$\because\quad \angle A=\angle C,\quad\angle B=\angle D$(已知),

$\therefore\quad 2\angle A+2\angle B=360^{\circ}$.

$\therefore\quad \angle A+\angle B=180^{\circ}$.

$\therefore\quad AD\parallel BC$ (同旁内角互补则两条直线平行)。

同理可证,$AB\parallel CD$, 所以四边形$ABCD$为平行四边形。
\end{proof}

由平行四边形的性质定理和判定定理可知,在四边形
中:
\begin{enumerate}
    \item 两组对边分别平行;
    \item 两组对边分别相等;
    \item 两组对角分别相等;
    \item 一组对边平行且相等;
    \item 对角线互相平分。
\end{enumerate}
这些都分别是平行四边形的充分必要条
件。

\begin{example}
     已知:$\parallelogram ABCD$, 且$\angle A$、$\angle B$的平分线相交于
$O$, 直线$BO$和直线$AD$相交于$E$(图3.52).

求证:$\triangle AOB$是直角三角形;$\overline{OB}=\overline{OE}$.
\end{example}

\begin{proof}
四边形$ABCD$是平行四边形,
$\therefore\quad AD\parallel BC$, 即$AE\parallel BC$.

$\therefore\quad \angle 2=\angle AEB$ (两条直线平行,内错角相等)。

又$\because\quad BE$是$\angle ABC$的平分线(已知),

$\therefore\quad \angle 1=\angle 2$(角平分线定义)。

$\therefore\quad \angle 1=\angle AEB$(等量代换)。

$\therefore\quad \triangle ABE$是等腰三角形(两个
角相等的三角形是等腰三角形)。

又$\because\quad AO$是$\angle DAB$的平分线(已知),

$\therefore\quad AO$是$\overline{BE}$的垂直平分线。

$\therefore\quad \triangle AOB$是直角三角形;$\overline{BO}=\overline{OE}$.
\end{proof}

\begin{figure}[htp]\centering
    \begin{minipage}[t]{0.48\textwidth}
    \centering
\begin{tikzpicture}[>=latex, scale=1]
\tkzDefPoints{0/0/A, 3/0/B, 4/2/C}
\tkzDefPointsBy[translation = from B to C](A){D}
\tkzDefLine[bisector,normed](B,A,D) \tkzGetPoint{F}
\tkzDefLine[bisector,normed](C,B,A)  \tkzGetPoint{G}
\tkzInterLL(A,F)(B,G) \tkzGetPoint{O}
\tkzInterLL(A,D)(B,O) \tkzGetPoint{E}
\tkzDrawPolygon(A,B,C,D)
\tkzDrawSegments(D,E B,E A,O)
\tkzLabelPoints[left](A,D,E)
\tkzLabelPoints[right](B,C,O)
\tkzMarkAngles[size=.4, mark=none](C,B,E)
\tkzMarkAngles[size=.3, mark=none](E,B,A)
\tkzLabelAngle[pos=.6](C,B,E){2}
\tkzLabelAngle[pos=.6](E,B,A){1}
    \end{tikzpicture}
    \caption{}
    \end{minipage}
    \begin{minipage}[t]{0.48\textwidth}
    \centering
    \begin{tikzpicture}[>=latex, scale=1]
\tkzDefPoints{0/0/A, 4/0/B, 5/2.4/C}
\tkzDefPointsBy[translation = from B to C](A){D}
\tkzDrawPolygon(A,B,C,D)
\tkzDrawSegments(A,C B,D)
\tkzDefPointBy[projection= onto B--D](A)  \tkzGetPoint{F}
\tkzDefPointBy[projection= onto B--D](C)  \tkzGetPoint{G}
\tkzDefPointBy[projection= onto A--C](B)  \tkzGetPoint{E}
\tkzDefPointBy[projection= onto A--C](D)  \tkzGetPoint{H}
\tkzDrawSegments(D,H E,B A,F C,G E,F G,H)
\tkzDrawSegments[dashed](F,H E,G)
\tkzInterLL(A,C)(B,D)  \tkzGetPoint{O}
\tkzLabelPoints[left](A,D)
\tkzLabelPoints[right](B,C)
\tkzLabelPoints[above](E,F)
\tkzLabelPoints[below](G,H,O)
    \end{tikzpicture}
    \caption{}
    \end{minipage}
    \end{figure}

\begin{example}
    从$\parallelogram ABCD$的各顶点作对角线的垂线$AF$、$BE$、
$CG$、$DH$, $E$、$F$、$G$、$H$是垂足(图3.53)。

求证:$\overline{EF}=\overline{GH}$.
\end{example}

\begin{proof}
    设$\overline{AC}$、$\overline{BD}$交于$O$点,连$\overline{FH}$, $\overline{EG}$.

$\because\quad $四边形$ABCD$是平行四边形(已知),

$\therefore\quad \overline{AD}=\overline{BC}$(平行四边形的对边相等),
$AD\parallel BC$(平行四边形定义)。

$\therefore\quad \angle ADF=\angle CBG$ (两条直线平行,内错角相等),

又$\because\quad AF\bot BD,\quad CG\bot BD$(已知),

$\therefore\quad \triangle AFD\cong \triangle CGB$ (AAS),

$\therefore\quad \overline{DF}=\overline{BG}$

又$\because\quad \overline{BO}=\overline{OD}$(平行四边形的对角线互相平分),

$\therefore\quad \overline{FO}=\overline{GO}$ (等量减等量差相等)。

同理,$\overline{HO}=\overline{OE}$.

$\therefore\quad$ 四边形$FHGE$为平行四边形(对
角线互相平分的四边形是平行四边形)。

$\therefore\quad \overline{FE}=\overline{GH}$(平行四边形的对边相等)。
\end{proof}

\begin{blk}
{定理} 连结三角形任意两边中点的线段平行于第三边,
且等于第三边的一半。    
\end{blk}

已知:$D$、$E$分别为
$\triangle ABC$的$\overline{AB}$和$\overline{AC}$边上
的中点(图3.54)

求证:$\overline{DE}\parallel \overline{BC}$, $\overline{DE}=\frac{1}{2}\overline{BC}$.

\begin{figure}[htp]
    \centering
\begin{tikzpicture}
\tkzDefPoints{0/0/B, 3/0/C, 2/2.5/A}
\tkzDefMidPoint(A,B) \tkzGetPoint{D}
\tkzDefMidPoint(A,C) \tkzGetPoint{E}
\tkzDrawPolygon(A,B,C)
\tkzDrawSegments(D,E)
\tkzDefPointsBy[translation= from D to A](C){F}
\tkzDrawSegments[dashed](A,F E,F C,F C,D)
\tkzLabelPoints[left](A,B,D)
\tkzLabelPoints[right](C,F)
\tkzLabelPoints[above](E)
\end{tikzpicture}
    \caption{}
\end{figure}

\begin{proof}
 在$\overline{DE}$的延长
线上,截取$\overline{EF}=\overline{DE}$, 由
于$\overline{AE}=\overline{EC}$, 

$\therefore\quad $四边形$ADCF$是平行四边形
(对角线互相平分的
四边形是平行四边形)。

$\therefore\quad CF\parallel AD$, 即$CF\parallel BD$.

但,$\overline{CF}=\overline{AD},\quad \overline{AD}=\overline{BD}$,

$\therefore\quad \overline{CF}=\overline{BD}$(等量代换),

$\therefore\quad $四边形$BCFD$是平行四边形(一组对边平行且相
等的四边形是平行四边形)。

$\therefore\quad \overline{DF}\parallel \overline{BC}$, 即$\overline{DE}\parallel \overline{BC}$.

由于$\overline{DE}=\frac{1}{2}\overline{DF}$, $\overline{DF}=\overline{BC}$(平行四边形的对边相等),

$\therefore\quad \overline{DE}=\frac{1}{2}\overline{BC}$.
\end{proof}

\begin{blk}
    {定义}
连结三角形两边中点的线段,叫做\textbf{三角形的中位
线}。
\end{blk}

上面的定理又叫做三角形\textbf{中位线定理}。

\begin{blk}
    {定义} 如果四边形只有一组对边平行,另一组对边不平
行,这个四边形叫做\textbf{梯形}。平行的两边叫做梯形的\textbf{底},不平
行的两边叫做梯形的\textbf{腰},两腰中点的连线,叫做梯形的\textbf{中位
线}。两腰相等的梯形叫做\textbf{等腰梯形}。
\end{blk}

\begin{example}
    梯形中位线平行于两底且等于两底和的一半。
\end{example}

已知:$\overline{EF}$是梯形$ABCD$的中位线(图3.55)。

求证:$EF\parallel AD\parallel BC$,且$\overline{EF}=\frac{1}{2}(\overline{AD}+\overline{BC})$.

\begin{proof}
由同学自己完成(提示:射线$AF$交$\overline{BC}$的延长线
    于$G$, 则$\overline{EF}$转化为$\triangle ABG$的中位线)。
\end{proof}

\begin{figure}[htp]\centering
    \begin{minipage}[t]{0.48\textwidth}
    \centering
\begin{tikzpicture}[>=latex, scale=.8]
\tkzDefPoints{0/0/B, 4/0/C, 1/3/A, 3/3/D}
\tkzDefMidPoint(A,B)  \tkzGetPoint{E}
\tkzDefMidPoint(C,D)  \tkzGetPoint{F}
\tkzInterLL(A,F)(B,C) \tkzGetPoint{G}

\tkzDrawPolygon(A,B,C,D)
\tkzDrawSegments(E,F A,G)
\tkzDrawSegments[dashed](C,G)
\tkzLabelPoints[left](A,E,B)
\tkzLabelPoints[below](C,G)
\tkzLabelPoints[right](D,F)

    \end{tikzpicture}
    \caption{}
    \end{minipage}
    \begin{minipage}[t]{0.48\textwidth}
    \centering
    \begin{tikzpicture}[>=latex, scale=1]
\tkzDefPoints{0/0/B, 4/0/C, 3/3.5/A}
\tkzDefMidPoint(A,B) \tkzGetPoint{P}
\tkzDefMidPoint(A,C) \tkzGetPoint{N}
\tkzDefMidPoint(C,B) \tkzGetPoint{M}
\tkzInterLL(A,M)(B,N)  \tkzGetPoint{G}
\tkzDrawPolygon(A,B,C)
\tkzDrawSegments(A,M B,N C,P)
\tkzLabelPoints[left](P,B)
\tkzLabelPoints[above](A)
\tkzLabelPoints[below](M,C)
\tkzLabelPoints[right](N,G)
\tkzDefMidPoint(G,C) \tkzGetPoint{Q}
\tkzDefMidPoint(G,B) \tkzGetPoint{H}
\tkzDrawPolygon[dashed](P,N,Q,H)
\tkzLabelPoints[below](Q,H)
    \end{tikzpicture}
    \caption{}
    \end{minipage}
    \end{figure}


\begin{example}
    三角形的三条中线相交于一点,这点和各边中点
    的距离等于这边上中线的三分之一。
\end{example}

已知:$\overline{AM}$、$\overline{BN}$、$\overline{CP}$为$\triangle ABC$的三条中线(图3.56)

求证:$\overline{AM}$、$\overline{BN}$、$\overline{CP}$相交于一点$G$, 且$\overline{GM}=\frac{1}{3}\overline{AM}$, $\overline{GN}=\frac{1}{3}\overline{BN}$, $\overline{GP}=\frac{1}{3}\overline{CP}$.

\begin{proof}
设$\overline{BN}$与$\overline{CP}$相交于$G$点,作$\overline{PN}$, 则$\overline{PN}=\frac{1}{2}\overline{BC}$, 
$\overline{PN}\parallel \overline{BC}$(三角形中位线定理)。

设$H$、$Q$分别为$\overline{BG}$、$\overline{CG}$的中点,则
$\overline{HQ}=\frac{1}{2}\overline{BC}$, $\overline{HQ}\parallel\overline{BC}$(三角形中位线定理)。

$\therefore\quad \overline{PN}=\overline{HQ}$(等量代换),
$\overline{PN}\parallel \overline{HQ}$(平行于第三条直线的两条直线平行)。

$\therefore\quad$ 四边形$PHQN$为平行四边形(一组对边平行且相等
的四边形是平行四边形)。

$\therefore\quad \overline{GH}=\overline{GN},\quad \overline{GP}=\overline{GQ}$(平行四边形的对角线相
平分)

但 $\overline{GH}=\overline{HB}$, $\overline{GQ}=\overline{CQ}$

$\therefore\quad \overline{GN}=\overline{GH}=\overline{HB}=\frac{1}{3}\overline{BN}$,$\overline{GP}=\overline{GQ}=\overline{QC}=\frac{1}{3}\overline{CP}$.

这样,中线$\overline{CP}$和中线$\overline{BN}$相交在这两条中线的三分之一
处。用同样的推理又可证明中线$\overline{AM}$和中线$\overline{BN}$也相交在它
们的三分之一处。即点$G$. 

因此,$\overline{AM}$、$\overline{BN}$、$\overline{CP}$相交于$G$, 且$\overline{GM}=\frac{1}{3}\overline{AM}$, $\overline{GN}=\frac{1}{3}\overline{BN}$, $\overline{GP}=\frac{1}{3}\overline{CP}$.
\end{proof}

\begin{blk}
    {定义} 三角形三条中线的交点叫做三角形的\textbf{重心}。
\end{blk}

\begin{example}
    直角三角形的斜边的中点,与三个顶点的距离相等。
\end{example}
    
已知:$\triangle ABC$中,
$\angle ACB=90^{\circ}$, 
$D$是$\overline{AB}$的中点
(图3.57)。

求证:$\overline{CD}=\overline{DA}=\overline{DB}$.

\begin{proof}
    取$\overline{BC}$的中点$E$, 连
    结$\overline{DE}$,

$\because\quad D$是$\overline{AB}$的中点,(已知)

$\therefore\quad  \overline{DE}$是$\triangle ABC$的一条中位线。

$\therefore\quad DE\parallel AC$(三角形中位线定理)。

$\because\quad \angle ACB=90^{\circ}$(已知),

$\therefore\quad \angle DEB=90^{\circ}$. (两条直线平行,则同位角相等),

$\therefore\quad \angle DEB=\angle DEC$
    
又$\because\quad  \overline{DE}=\overline{DE},\quad \overline{BE}=\overline{EC}$

$\therefore\quad \triangle BDE\cong \triangle CDE$ (SAS)

$\therefore\quad \overline{CD}=\overline{DB}$. 但$\overline{DB}=\overline{DA}$,

$\therefore\quad \overline{CD}=\overline{DB}=\overline{DA}$.
\end{proof}

\begin{figure}[htp]\centering
    \begin{minipage}[t]{0.48\textwidth}
    \centering
\begin{tikzpicture}[>=latex, scale=1]
\tkzDefPoints{0/0/B, 3/0/C, 3/4/A}
\tkzDefMidPoint(A,B) \tkzGetPoint{D}
\tkzDrawPolygon(A,C,B)
\tkzDrawSegments(C,D)
\tkzDefPointBy[projection = onto B--C](D)\tkzGetPoint{E}
\tkzDrawSegments[dashed](E,D)
\tkzLabelPoints[below](B,E,C)
\tkzLabelPoints[above](A)
\tkzLabelPoints[left](D)
    \end{tikzpicture}
    \caption{}
    \end{minipage}
    \begin{minipage}[t]{0.48\textwidth}
    \centering
    \begin{tikzpicture}[>=latex, scale=1.2]
\tkzDefPoints{0/0/B, 2/0/C, 2/2*1.732/A}
\tkzDefMidPoint(A,B) \tkzGetPoint{D}
\tkzDrawPolygon(A,C,B)
\tkzDrawSegments[dashed](C,D)

\tkzLabelPoints[below](B,C)
\tkzLabelPoints[above](A)
\tkzLabelPoints[left](D)
\tkzMarkRightAngle[size=.3](A,C,B)
\tkzMarkAngle[size=.5, mark=none](B,A,C)
\tkzLabelAngle[pos=.7](B,A,C){$30^{\circ}$}

    \end{tikzpicture}
    \caption{}
    \end{minipage}
    \end{figure}

\begin{example}
    在直角三角形中,$30^{\circ}$角的对边等于斜边的一
半。
\end{example}

已知:$\triangle ABC$中,$\angle ACB=90^{\circ}$, $\angle A=30^{\circ}$(图3.58)

求证:$\overline{BC}=\frac{1}{2}\overline{AB}$.

\begin{proof}
    取$\overline{AB}$的中点$D$. 连结
    $\overline{CD}$, 则$\overline{CD}=\overline{DA}$(例3.29)

$\because\quad \angle A=30^{\circ}$,

$\therefore\quad \angle ACD=30^{\circ}$ (等腰三角形两底角相等)。

又$\because\quad \angle B=90^{\circ} -30^{\circ} =60^{\circ}$ (直角三角形两锐角互余),

$ \angle BDC=60^{\circ}$ (三角形的外角等于不相邻的两个
    内角的和)。

$\therefore\quad \triangle BCD$是正三角形(三内角相等的三角形是正三
    角形)。

$\therefore\quad  \overline{BC}= \overline{BD}=\frac{1}{3} \overline{AB}$.
 \end{proof}

以上两例的逆命题也是正确的,同学们可自己证明。   
    
\begin{blk}
   {定义} 取两个已知点$A$、$O$, 连结$\overline{AO}$, 并延长$\overline{AO}$使
$\overline{OA'}=\overline{AO}$, 这时我们说$A'$是$A$点关于$O$点的\textbf{中心对称点};
$A$、$A'$关于$O$点成\textbf{中心对称};$O$点叫
做两已知点$A$、$A'$的\textbf{对称中心}
(图3.59)。 
\end{blk}

\begin{figure}[htp]
    \centering
    \begin{tikzpicture}[>=latex, scale=1]
        \draw[|-|](-2,0)node[below]{$A$}--(0,0)node[below]{$O$}--(2,0)node[below]{$A'$};
        \draw(0,0)--(0,.1);
            \end{tikzpicture}
    \caption{}
\end{figure}

对于任何一个已知图形上的一点,
作出它关于某一点$O$的中心对称点,这些对称点的集合,构成
了一个新的图形,这个新的图形叫做已知图形的关于$O$点的
\textbf{中心对称形},$O$点叫做这两个图形的\textbf{对称中心}。

例如,已知$\triangle ABC$, 我们作顶点$A$、$B$、$C$关于$O$点的
中心对称点$A',B',C'$, 容易证明$\triangle A'B'C'$就是
$\triangle ABC$
的关于对称中心$O$的中心对称形(图3.60)。

\begin{figure}[htp]\centering
    \begin{minipage}[t]{0.48\textwidth}
    \centering
    \begin{tikzpicture}
        \tkzDefPoints{0/0/B, 1/1/A, 1/-1.5/C, 2.5/.6/O}
\foreach \x in {A,B,C}
{
    \tkzDefPointWith[linear, K=2](\x,O) \tkzGetPoint{\x'}
}
\tkzLabelPoints[above](A,C')
\tkzLabelPoints[below](C,O,A')
\tkzLabelPoints[left](B)
\tkzLabelPoints[right](B')
\tkzDrawPolygon(A,B,C,O)
\tkzDrawPolygon(A',B',C',O)
\tkzDrawSegments(B,B' A,C A',C')
    \end{tikzpicture}
    \caption{}
    \end{minipage}
    \begin{minipage}[t]{0.48\textwidth}
    \centering
    \begin{tikzpicture}[>=latex, scale=1]
\tkzDefPoints{0/0/B, 4/0/C, 5/2.75/D}
\tkzDefPointsBy[translation = from C to D](B){A}
\tkzDefPointWith[linear, K=.3](B,C) \tkzGetPoint{P'}
\tkzDefPointWith[linear, K=.3](D,A) \tkzGetPoint{P}
\tkzDrawPolygon(A,B,C,D)
\tkzDrawSegments(A,C D,B P,P')
\tkzInterLL(A,C)(B,D)  \tkzGetPoint{O}
\tkzLabelPoints[above](P)
\tkzLabelPoints[below](P')
\tkzLabelPoints[left](A,B)
\tkzLabelPoints[right](C,D,O)
\tkzMarkAngles[mark=none, size=.45](A,P,O P,O,A P',O,C C,P',O)
\tkzLabelAngle[pos=.7](A,P,O){3}
\tkzLabelAngle[pos=.7](P,O,A){1}
\tkzLabelAngle[pos=.7](P',O,C){2}
\tkzLabelAngle[pos=.7](C,P',O){4}
    \end{tikzpicture}
    \caption{}
    \end{minipage}
    \end{figure}

从任意两个中心对称形的定义,我们很容易看出:

\begin{blk}{}
   任意两个中心对称形是全等形。 
\end{blk}

\begin{blk}
   {定义} 如果一个图形上的每一点,关于某一点$O$的中心
对称点都还在这个图形上,我们说这个图形是\textbf{中心对称形},
$O$点叫做图形的\textbf{对称中心}。 
\end{blk}


\begin{blk}
    {定理} 平行四边形是中心对称形,它的对称中心是两条
对角线的交点。
\end{blk}

已知:$\parallelogram ABCD$的对角线$\overline{AC}$、$\overline{BD}$相交于$O$点(图3.61)。

求证:$\parallelogram ABCD$是中心对称形;$O$是对称中心。

\begin{proof}
    在$\parallelogram ABCD$上任取一点$P$, 设$P\in\overline{AD}$, 连结$\overline{PO}$, 
延长交$\overline{BC}$于$P'$, 在$\triangle AOP$和$\triangle COP'$中,

$\because\quad \overline{AO}=\overline{OC}$(平行四边形对角线互相平分),

$\angle 1=\angle 2$(对顶角相等),$\angle 3=\angle 4$(两条直线平行,则内错角相等),

$\therefore\quad \triangle AOP\cong \triangle COP'$ (AAS).

$\therefore\quad \overline{PO}=\overline{OP'}$(全等三角形的对应边相等),

$\therefore\quad P'$是$P$关于$O$点的中心对称点。

由于上述$P$点是任意选取的,$P$点关于$O$的对称点$P'$也在
$\parallelogram ABCD$上,所以$\parallelogram ABCD$是中心对称形。$O$点是对称中心。
\end{proof}

由中心对称形的定义,我们还可看出:

平面上的中心对称形,绕着它的对称中心,在平面上旋
转$180^{\circ}$ 后,它的新位置与原来位置重合。例如,我们把上例
中的$\parallelogram ABCD$绕$O$点旋转$180^{\circ}$后,则$A$与$C$对换位置,$B$与$D$
对换位置,这时$\parallelogram ABCD$与原来位置重合。

\begin{blk}
    {定义} 有一个角是直角的平行四边形叫做\textbf{矩形}或\textbf{长方
形}。

有一组邻边相等的平行四边形叫做\textbf{菱形}。

有一组邻边相等并且有一个角是直角的平行四边形叫做
\textbf{正方形}(图3.62)。
\end{blk}

\begin{figure}[htp]
    \centering
\begin{tikzpicture}
\begin{scope}
\tkzDefPoints{0/0/A, 3/0/B, 3/2/C}
\tkzDefPointsBy[translation = from B to C](A){D}
\tkzDrawPolygon(A,B,C,D)
\tkzDrawSegments(A,C D,B)
\tkzMarkRightAngle[size=.2](A,B,C)
\node at (1.5,-.5){矩形};
\end{scope}
\begin{scope}[xshift=4cm, yshift=1cm]
    \tkzDefPoints{0/0/A, 1.5/-1/B, 3/0/C, 1.5/1/D}
    \tkzDrawPolygon(A,B,C,D)
    \tkzDrawSegments(A,C D,B)
    \node at (1.5,-1.5){菱形};
\end{scope}
\begin{scope}[xshift=8cm]
    \tkzDefPoints{0/0/A, 2/0/B, 2/2/C}
\tkzDefPointsBy[translation = from B to C](A){D}
\tkzDrawPolygon(A,B,C,D)
\tkzDrawSegments(A,C D,B)
\tkzMarkRightAngle[size=.2](A,B,C)
\node at (1,-.5){正方形};
\end{scope}
\end{tikzpicture}
    \caption{}
\end{figure}

根据定义可知;矩形、菱形、正方形都是平行四边形,
所以它们都具有平行四边形的所有性质,另外它们还分别具
有下列特殊性质:

\begin{blk}
    {推论1} 矩形的四个角都是直角,并且两条对角线相等。
\end{blk}

\begin{blk}
{推论2} 菱形的四条边都相等,且对角线互相垂直并平
分每双对角。
\end{blk}

这个推论的后半部分,就是我们在例3.9中所
证的结论。

\begin{blk}
    {推论3} 正方形是矩形又是菱形。即
\[\{\text{矩形}\}\cap \{\text{菱形}\}=\{\text{正方形}\}\]
\end{blk}

以上三个推论,同学们作为练习,自己证明。并思考上
述三个推论的逆命题是否正确,并证明你的结论。

由上面的三个推论,我们还可推知:

矩形、菱形和正方形是中心对称形又是轴对称形。经过
矩形对边中点的两条直线是它的两条对称轴;菱形两条对角
线所在的直线是它的对称轴;正方形两条对角线和经过对边
中点的两条直线是它的对称轴(图3.63)。

\begin{figure}[htp]
    \centering
\begin{tikzpicture}[scale=.8]
\begin{scope}
\tkzDefPoints{0/0/A, 3/0/B, 3/2/C}
\tkzDefPointsBy[translation = from B to C](A){D}
\tkzDrawPolygon(A,B,C,D)
\tkzDefMidPoint(A,B) \tkzGetPoint{A'}
\tkzDefMidPoint(C,B) \tkzGetPoint{B'}
\tkzDefMidPoint(C,D) \tkzGetPoint{C'}
\tkzDefMidPoint(A,D) \tkzGetPoint{D'}

\tkzDrawLines(A',C' D',B')

\end{scope}
\begin{scope}[xshift=5cm, yshift=1cm]
    \tkzDefPoints{0/0/A, 1.5/-1/B, 3/0/C, 1.5/1/D}
    \tkzDrawPolygon(A,B,C,D)
    \tkzDrawLines(A,C D,B)

\end{scope}
\begin{scope}[xshift=10cm]
    \tkzDefPoints{0/0/A, 2/0/B, 2/2/C}
\tkzDefPointsBy[translation = from B to C](A){D}
\tkzDrawPolygon(A,B,C,D)
\tkzDrawLines(A,C D,B)
\tkzDefMidPoint(A,B) \tkzGetPoint{A'}
\tkzDefMidPoint(C,B) \tkzGetPoint{B'}
\tkzDefMidPoint(C,D) \tkzGetPoint{C'}
\tkzDefMidPoint(A,D) \tkzGetPoint{D'}

\tkzDrawLines(A',C' D',B')

\end{scope}
\end{tikzpicture}
    \caption{}
\end{figure}

\begin{example}
    已知:$ABCD$是矩形,设各外角的平分线相交于
    $E$、$F$、$G$、$H$(图3.64). 

    求证:$EFGH$是正方形。
\end{example}

\begin{figure}[htp]
    \centering
\begin{tikzpicture}
\tkzDefPoints{0/0/B, 3/0/C, 3/2/D, 0/2/A}
\tkzDrawPolygon[thick](A,B,C,D)
\tkzDrawLines[add=0 and .5, dashed](B,A A,D D,C C,B)
\tkzDefPointWith[linear, K=1.5](B,A)  \tkzGetPoint{A'}
\tkzDefPointWith[linear, K=1.5](A,D)  \tkzGetPoint{D'}
\tkzDefPointWith[linear, K=1.5](D,C)  \tkzGetPoint{C'}
\tkzDefPointWith[linear, K=1.5](C,B)  \tkzGetPoint{B'}

\tkzDefLine[bisector](D,A,A') \tkzGetPoint{A''}
\tkzDefLine[bisector](C,D,D') \tkzGetPoint{D''}
\tkzDefLine[bisector](B,C,C') \tkzGetPoint{C''}
\tkzDefLine[bisector](A,B,B') \tkzGetPoint{B''}
\tkzInterLL(A,A'')(D,D'')  \tkzGetPoint{H}
\tkzInterLL(D,D'')(C,C'')  \tkzGetPoint{G}
\tkzInterLL(B,B'')(C,C'')  \tkzGetPoint{F}
\tkzInterLL(A,A'')(B,B'')  \tkzGetPoint{E}
\tkzDrawPolygon[thick](E,F,G,H)
\tkzInterLL(A,E)(C,B)  \tkzGetPoint{P}
\tkzDrawSegments[dashed](E,P P,B)

\tkzLabelPoints[above](H,D)
\tkzLabelPoints[below](P,F,B)
\tkzLabelPoints[left](E,A)
\tkzLabelPoints[right](G,C)

\tkzMarkAngles[mark=none, size=.4](E,A,B A,B,E D,A,H B,C,F F,B,C H,D,A B,P,E)
\tkzLabelAngle[pos=.6](E,A,B){1}
\tkzLabelAngle[pos=.6](A,B,E){2}
\tkzLabelAngle[pos=.6](D,A,H){3}
\tkzLabelAngle[pos=.6](B,C,F){4}
\tkzLabelAngle[pos=.6](F,B,C){5}
\tkzLabelAngle[pos=.6](H,D,A){6}
\end{tikzpicture} 
    \caption{}
\end{figure}


\begin{proof}
延长$\overline{AE}$、$\overline{CB}$交于$P$点,

$\because\quad ABCD$是矩形,

$\therefore\quad AD\parallel BC$,

$\therefore\quad \angle P=\angle 3$(两条直线平行同位角相等)。

又$\because\quad \angle 3=\angle 4=45^{\circ}$ (矩形外角的一半是$45^{\circ}$),

$\therefore\quad EH\parallel FG$(内错角相等两条直线平行)。

同理:$EF\parallel HG$

$\therefore\quad $四边形$EFGH$是平行四边形

$\because\quad \angle 1=\angle 2=45^{\circ}$, (矩形外角的一半是$45^{\circ}$),

$\therefore\quad \angle AEB=90^{\circ}$,并且$\overline{AE}=\overline{BE}$(两个角相等的三角形是等腰三角形),

又$\because\quad \angle 3=\angle 5=\angle 4=\angle 6$, $\overline{AD}=\overline{BC}$(矩形的对边相等),

$\therefore\quad \triangle AHD\cong BFC$(ASA).

$\therefore\quad \overline{AH}=\overline{BF}$(全等三角形的对应边相等)。

$\therefore\quad \overline{AE}+\overline{AH}=\overline{BE}+\overline{BF}$(等量加等量和相等)。    

即:$\overline{EH}=\overline{EF}$,因此,四边形$EFGH$是正方形(正方形定
义)
\end{proof}

\begin{ex}
\begin{enumerate}
    \item $E$、$F$是$\parallelogram ABCD$对角线$\overline{BD}$上的两点,且$\overline{BE}=\overline{DF}$,  求证:$\overline{AE}=\overline{CF}$.
    \item 已知$\parallelogram ABCD$中,$E$、$F$是$\overline{BD}$上的两点,且$\angle BAE=\angle DCF$.  求证:$\overline{BE}=\overline{DF}$.
\item 已知:在$\parallelogram ABCD$中,$E$、$F$分别是$\overline{AB}$、$\overline{CD}$上的点,且$\overline{BE}=\overline{DF}$. 求证:$\overline{BF}\parallel\overline{DE}$

\item 已知:$\parallelogram ABCD$中$E$、$F$是$\overline{BD}$上的两点,且$\overline{BE}=\overline{DF}$. 求证:四边形$AECF$是平行四边形.
\item 已知:$E$、$F$是$\parallelogram ABCD$的对边$\overline{AB}$、$\overline{CD}$的中点,$\overline{EF}$
与$\overline{AC}$相交于$O$. 求证:$\overline{AO}=\overline{CO}$.
\item 求证:对角线相等的平行四边形是矩形。
\item 求证:对角线互相垂直的平行四边形是菱形。
\item 已知:$E$、$F$、$G$分别是$\triangle ABC$三边上的中点。
求证:四边形$AEFG$是平行四边形。
\item 已知:等腰$\triangle ABC$, $\overline{AB}=\overline{AC}$, $E$、$F$、$G$是三边上的中点。求证:四边形$AEFG$是菱形。
    
\item 证明:顺次连结任意四边形四边的中点所成的四边形是
平行四边形。
\item 顺次连结平行四边形四边中点所成的四边形是什么图
形?证明你的结论。
\item 顺次连结短形四边中点所成的四边形是什么图形?证明
你的结论。
\item 顺次连结菱形四边中点所成的四边形是什么图形?证明
你的结论。
\item 顺次连结正方形各边中点所成的四边形是什么图形?证
明你的结论。
\item 证明:三角形的三边中点间的线段把三角形分为四个全
等形。
\item 证明:等腰梯形的两底角相等。
\item 证明:等腰梯形的两条对角线相等。
\item 等腰梯形的腰等于中位线,周长为24cm. 求腰长。
\item 已知梯形$ABCD$中,$\overline{AD}\parallel \overline{BC}$, $\overline{BC}=2\overline{AD}$,$AF\parallel CD$交$\overline{BD}$于$O$点,$\overline{CD}=6$cm, 求$\overline{OF}$的长度。

\item 已知:正方形$ABCD$, $E$、$F$、$G$、$H$分别在$\overline{AB}$、$\overline{BC}$、$\overline{CD}$、$\overline{DA}$上,并且$\overline{AE}=\overline{BF}=\overline{CG}=\overline{DH}$, 求证:四边形
$EFGH$是正方形。
\end{enumerate}
\end{ex}



\begin{figure}[htp]\centering
    \begin{minipage}[t]{0.48\textwidth}
    \centering
\begin{tikzpicture}[>=latex, scale=1]
\tkzDefPoints{0/0/B, 3/0/C, 4.5/2/D}
\tkzDefPointsBy[translation=from C to D](B){A}
\tkzDrawPolygon(A,B,C,D)
\tkzDefPointWith[linear, K=.3](B,D) \tkzGetPoint{E}
\tkzDefPointWith[linear, K=.3](D,B) \tkzGetPoint{F}
\tkzDrawSegments(B,D A,E C,F)
\tkzLabelPoints[above](A,D,F)
\tkzLabelPoints[below](B,E,C)
    \end{tikzpicture}
    \caption*{第1、2题}
    \end{minipage}
    \begin{minipage}[t]{0.48\textwidth}
    \centering
    \begin{tikzpicture}[>=latex, scale=1]
  \tkzDefPoints{0/0/A, 3/0/B, 4/2/C}
\tkzDefPointsBy[translation=from B to C](A){D}
\tkzDrawPolygon(A,B,C,D)
\tkzDefPointWith[linear, K=.45](C,D) \tkzGetPoint{F}
\tkzDefPointWith[linear, K=.45](A,B) \tkzGetPoint{E}
\tkzDrawSegments(D,E F,B)
\tkzLabelPoints[above](C,D,F)
\tkzLabelPoints[below](B,E,A)    
    \end{tikzpicture}
    \caption*{第3题}
    \end{minipage}
    \end{figure}

\begin{figure}[htp]\centering
    \begin{minipage}[t]{0.48\textwidth}
    \centering
\begin{tikzpicture}[>=latex, scale=1]
    \tkzDefPoints{0/0/B, 3/0/C, 4/2.5/D}
    \tkzDefPointsBy[translation=from C to D](B){A}
    \tkzDrawPolygon(A,B,C,D)
    \tkzDefPointWith[linear, K=.3](B,D) \tkzGetPoint{E}
    \tkzDefPointWith[linear, K=.3](D,B) \tkzGetPoint{F}
    \tkzDrawSegments(B,D)
    \tkzDrawPolygon(A,F,C,E)
    \tkzLabelPoints[above](A,D,F)
    \tkzLabelPoints[below](B,E,C)
    \end{tikzpicture}
    \caption*{第4题}
    \end{minipage}
    \begin{minipage}[t]{0.48\textwidth}
    \centering
    \begin{tikzpicture}[>=latex, scale=1]
        \tkzDefPoints{0/0/A, 3/0/B, 4/2/C}
        \tkzDefPointsBy[translation=from B to C](A){D}
        \tkzDrawPolygon(A,B,C,D)
        \tkzDefPointWith[linear, K=.5](C,D) \tkzGetPoint{F}
        \tkzDefPointWith[linear, K=.5](A,B) \tkzGetPoint{E}
        \tkzDrawSegments(A,C E,F)
        \tkzLabelPoints[above](C,D,F)
        \tkzLabelPoints[below](B,E,A)   
        \tkzInterLL(A,C)(E,F) \tkzGetPoint{O}
        \tkzLabelPoints[right](O) 
    \end{tikzpicture}
    \caption*{第5题}
    \end{minipage}
    \end{figure}

\begin{figure}[htp]\centering
    \begin{minipage}[t]{0.48\textwidth}
    \centering
\begin{tikzpicture}[>=latex, scale=1]
\tkzDefPoints{0/0/B, 4/0/C, 2.4/2.5/A}
\tkzDefMidPoint(A,B) \tkzGetPoint{E}
\tkzDefMidPoint(A,C) \tkzGetPoint{G}
\tkzDefMidPoint(C,B) \tkzGetPoint{F}
\tkzDrawPolygon(A,B,C)
\tkzDrawSegments(E,F F,G)
\tkzLabelPoints[below](B,F,C)
\tkzLabelPoints[above](A)
\tkzLabelPoints[left](E)
\tkzLabelPoints[right](G)
    \end{tikzpicture}
    \caption*{第8题}
    \end{minipage}
    \begin{minipage}[t]{0.48\textwidth}
    \centering
    \begin{tikzpicture}[>=latex, scale=1]
        \tkzDefPoints{0/0/B, 3/0/C, 1.5/2.5/A}
        \tkzDefMidPoint(A,B) \tkzGetPoint{E}
        \tkzDefMidPoint(A,C) \tkzGetPoint{G}
        \tkzDefMidPoint(C,B) \tkzGetPoint{F}
        \tkzDrawPolygon(A,B,C)
        \tkzDrawSegments(E,F F,G)
        \tkzLabelPoints[below](B,F,C)
        \tkzLabelPoints[above](A)
        \tkzLabelPoints[left](E)
        \tkzLabelPoints[right](G)
    \end{tikzpicture}
    \caption*{第9题}
    \end{minipage}
    \end{figure}


\begin{figure}[htp]\centering
    \begin{minipage}[t]{0.48\textwidth}
    \centering
\begin{tikzpicture}[>=latex, scale=1]
\tkzDefPoints{0/0/B, 4/0/C, 3/2/D, 1/2/A, 2/0/E}
\tkzDrawPolygon(A,B,C,D)
\tkzDrawSegments[dashed](D,E)
\tkzLabelPoints[below](B,C,E)
\tkzLabelPoints[above](A,D)
\tkzMarkAngles[size=.4, mark=none](C,B,A D,C,B)

    \end{tikzpicture}
    \caption*{第16题}
    \end{minipage}
    \begin{minipage}[t]{0.48\textwidth}
    \centering
    \begin{tikzpicture}[>=latex, scale=1]
        \tkzDefPoints{0/0/B, 4/0/C, 3/2/D, 1/2/A}
\tkzDefMidPoint(B,C) \tkzGetPoint{F}
        \tkzDrawPolygon(A,B,C,D)
        \tkzDrawSegments(A,F B,D)
        \tkzInterLL(A,F)(B,D) \tkzGetPoint{O}
        \tkzLabelPoints[below](B,C,F)
        \tkzLabelPoints[above](A,D)     
        \tkzLabelPoints[left](O)     
    \end{tikzpicture}
    \caption*{第19题}
    \end{minipage}
    \end{figure}




\subsection{等分线段}

\begin{blk}
 {平行线等分线段定理} 如果一组平行线在一条直线上截
出一组等长的线段,那么在任一条与这组平行线相交的直线
上也被截为一组等长的线段。   
\end{blk}

已知:$\ell_1\parallel \ell_2\parallel \ell_3$, 直线$m$与$\ell_1$、$\ell_2$、$\ell_3$分别相交于$A_1,A_2,A_3$且$\overline{A_1A_2}=\overline{A_2A_3}$;任一条直线$n$分别与$\ell_1$、$\ell_2$、$\ell_3$相交于$B_1,B_2,B_3$(图3.65)。

求证:$\overline{B_1B_2}=\overline{B_2B_3}$.

\begin{figure}[htp]
    \centering
\begin{tikzpicture}[scale=1.5]
\tkzDefPoints{-2/1/A, 2/1/B, -2/2/A', 2/2/B', -2/3/A'', 2/3/B''}
\tkzDrawLines(A,B A',B' A'',B'')
\tkzDefPoints{-.5/4/m, 1/4/n, -2.5/0/M', 2/0/N'}

\tkzDrawSegments(m,M' n,N')
\tkzInterLL(m,M')(A'',B'')  \tkzGetPoint{A_1}
\tkzInterLL(m,M')(A',B')  \tkzGetPoint{A_2}
\tkzInterLL(m,M')(A,B)  \tkzGetPoint{A_3}
\tkzInterLL(n,N')(A'',B'')  \tkzGetPoint{B_1}
\tkzInterLL(n,N')(A',B')  \tkzGetPoint{B_2}
\tkzInterLL(n,N')(A,B)  \tkzGetPoint{B_3}

\tkzLabelPoints[above left](A_1,A_2,A_3)
\tkzLabelPoints[above right](B_1,B_2,B_3)
\tkzLabelPoints[above](m,n)
\node at (3,1){$\ell_3$};
\node at (3,2){$\ell_2$};
\node at (3,3){$\ell_1$};

\tkzDefPoints{-1.2/1/C_2}
\tkzDefPointsBy[translation= from A_2 to A_1](C_2){C_1}
\tkzLabelPoints[below](C_1,C_2)
\tkzDrawSegments(A_1,C_1 A_2,C_2)
\tkzMarkAngles[mark=none, size=.3](C_2,A_3,A_1 C_1,A_2,A_1 A_3,A_2,C_2  A_3,A_1,C_1)
\tkzLabelAngle[pos=.4](C_2,A_3,A_1){4}
\tkzLabelAngle[pos=.4](C_1,A_2,A_1){3}
\tkzLabelAngle[pos=.4](A_3,A_2,C_2){2}
\tkzLabelAngle[pos=.4](A_3,A_1,C_1){1}
\end{tikzpicture}
    \caption{}
\end{figure}

\begin{proof}
过$A_1$、$A_2$分别作直线$n$的平行线$A_1C_1$、$A_2C_1$,与
$\ell_2$、$\ell_3$分别交于$C_1,C_2$两点。在$\triangle A_1A_2C_1$和$\triangle A_2A_3C_2$中,

$\because\quad \overline{A_1A_2}= \overline{A_2A_4}$(已知),
$\ell_1\parallel \ell_2\parallel \ell_3$(已知)。$A_1C_1\parallel n_1$,  $A_2C_2\parallel n$,

$\therefore\quad A_1C_1\parallel A_2C_2$(平行于第三条直线的两条直线平行)。

$\therefore\quad \angle 1=\angle 2,\quad \angle 3=\angle 4$(两条直线平行同位角相等)。

$\therefore\quad \triangle A_1A_2C_1\cong \triangle A_2A_3C_2$ (ASA).

$\therefore\quad A_1C_1=A_2C_2$ (全等三角形的对应边相等)。
但四边形$A_1C_1B_2B_1$和$A_2C_2B_3B_2$都是平行四边形,由于
平行四边形对边相等,

$\therefore\quad \overline{A_1C_1}=\overline{B_1B2}$, $\overline{A_2C_2}=\overline{B_2B_3}$.

$\therefore\quad \overline{B_1B_2}=\overline{B_2B_3}$ (等量代换)。
\end{proof}    

\begin{blk}
    {推论} 经过三角形一边中点并和另一边平行的直线,平分第三边(图3.66)。
\end{blk}

应用平行线等分线段定理可以用直尺和圆规任意等分一
条线段。

\begin{figure}[htp]\centering
    \begin{minipage}[t]{0.48\textwidth}
    \centering
\begin{tikzpicture}[>=latex, scale=1]
\tkzDefPoints{0/0/B, 3/0/C, 2/3/A, 0/3/D, 3.54/3/E}
\tkzDrawPolygon(A,B,C)
\tkzDrawSegments[dashed](E,D)
\tkzDefMidPoint(A,B) \tkzGetPoint{G}
\tkzDefMidPoint(A,C) \tkzGetPoint{H}
\tkzDrawSegments(G,H)
\tkzLabelPoints[below](B,C)
\tkzLabelPoints[above](A)
    \end{tikzpicture}
    \caption{}
    \end{minipage}
    \begin{minipage}[t]{0.48\textwidth}
    \centering
    \begin{tikzpicture}[>=latex, scale=1]
\tkzDefPoints{0/0/A,5/0/B, 4.5/2.8/C}

\foreach \x/\xtext in {1/D,2/E,3/F,4/G,5/H}
{
    \tkzDefPointWith[linear, K=\x/6](A,C)  \tkzGetPoint{\xtext}
}
\foreach \x/\xtext in {1/D_1,2/E_1,3/F_1,4/G_1}
{
    \tkzDefPointWith[linear, K=\x/5](A,B)  \tkzGetPoint{\xtext}
}
\foreach \x in {D,E,F,G}
{
    \tkzDrawSegments(\x,\x_1)
}
\tkzDrawSegments(B,H A,B A,C)
\tkzLabelPoints[below](A,D_1,E_1,F_1,G_1,B)
\tkzLabelPoints[above](D,E,F,G,H,C)
    \end{tikzpicture}
    \caption{}
    \end{minipage}
    \end{figure}

\begin{example}
    已知:$\overline{AB}$(图3.67)。

求作:$\overline{AB}$上的点$D_1$、$E_1$、$F_1$、$G_1$,使$\overline{AD}_1=\overline{D_1E_1}=\overline{E_1F_1}=\overline{F_1G_1}=\overline{G_1B}$(将$\overline{AB}$五等分)。

作法:
\begin{enumerate}
    \item  作射线$AC$,
    \item  在射线$AC$上截取$\overline{AD}=\overline{DE}=\overline{EF}=\overline{FG}=\overline{GH}$,
    \item  连结$B$、$H$,
    \item  过各分点$D$、$E$、$F$、$G$分别作$BH$的平行线交
$\overline{AB}$于$D_1$、$E_1$、$F_1$、$G_1$, 则$D_1$、$E_1$、$F_1$、$G_1$把$\overline{AB}$五等分。
\end{enumerate}

证明:略。
\end{example}

\begin{ex}
\begin{enumerate}
    \item 任划一条$\overline{AB}$, 把它三等分。
    \item 任划一条$\overline{CD}$, 把它五等分。
\end{enumerate}
\end{ex}

\subsection*{习题3.3}
\begin{enumerate}
    \item 已知:梯形$ABCD$中,$\angle B=90^{\circ}$, $AD\parallel BC$, $\overline{CD}=12$cm, $\angle BCD=30^{\circ}$, 求一腰$\overline{AB}$之长。
\item 已知:矩形$ABCD$中,$\overline{AB}=2\overline{AD}$, $E\in \overline{CD}$, $\angle DAE=60^{\circ}$.

求证:$\angle CBE=15^{\circ}$
\item 已知:四边形$ABCD$中,$E$、$F$、$G$、$H$分别是$\overline{AB}$、$\overline{AC}$、
$\overline{CD}$、$\overline{BD}$的中点。

求证:四边形$EFGH$是平行四边形。
\begin{figure}[htp]\centering
    \begin{minipage}[t]{0.48\textwidth}
    \centering
\begin{tikzpicture}[>=latex, scale=1]
\tkzDefPoints{0/0/B, 4/0/C, 1.6/2.5/A, 3.6/3/D}
\tkzDefMidPoint(A,B)  \tkzGetPoint{E}
\tkzDefMidPoint(A,C)  \tkzGetPoint{F}
\tkzDefMidPoint(C,D)  \tkzGetPoint{G}
\tkzDefMidPoint(D,B)  \tkzGetPoint{H}
\tkzDrawPolygon(A,B,C,D) \tkzDrawPolygon(E,F,G,H)
\tkzDrawSegments(B,D A,C)
\tkzLabelPoints[right](C,D,G)
\tkzLabelPoints[left](E,B)
\tkzLabelPoints[above](H,A)
\tkzLabelPoints[below](F)
    \end{tikzpicture}
    \caption*{第3题}
    \end{minipage}
    \begin{minipage}[t]{0.48\textwidth}
    \centering
    \begin{tikzpicture}[>=latex, scale=1.4]
\tkzDefPoints{0/0/B, 2/0/C, 2/2/D, 0/2/A}
\tkzDefPointWith[linear, K=1.4](B,C)  \tkzGetPoint{E}
\tkzDefPointWith[linear, K=.4](C,D) \tkzGetPoint{F}
\tkzDrawPolygon(A,B,C,D)
\tkzDrawSegments[dashed](C,E)
\tkzDrawSegments(B,F D,E)
\tkzLabelPoints[right](F)
\tkzLabelPoints[above](A,D)
\tkzLabelPoints[below](B,C,E)
    \end{tikzpicture}
    \caption*{第4题}
    \end{minipage}
    \end{figure}

\item 已知:$E$是正方形$ABCD$的边$\overline{BC}$的延长线上的一点,
且$\overline{CE}=\overline{CF}$
求证:$\overline{DE}=\overline{BF}$;$DE\bot BF$.

\begin{figure}[htp]
    \centering
    \begin{tikzpicture}[>=latex, scale=1]
\tkzDefPoints{0/0/B, 3/0/C, 1.5/0/M, 2.2/2/A}
\tkzDrawPolygon(A,B,C)
\tkzDrawSegments(A,M)
\tkzLabelPoints[above](A)
\tkzLabelPoints[below](B,M,C)
\tkzDefSquare(B,A)
\tkzGetPoints{D}{E}
\tkzDefSquare(A,C)
\tkzGetPoints{F}{G}
\tkzLabelPoints[left](D,E)
\tkzLabelPoints[right](F,G)
\tkzDrawSegments(D,G)
\tkzDrawPolygon(A,B,E,D)
\tkzDrawPolygon(A,C,F,G)
    \end{tikzpicture}
    \caption*{第5题}
\end{figure}

\item 已知:$\overline{AM}$是$\triangle ABC$的中线,四边形$ABED$, $ACFG$都
是正方形。

求证:$\overline{AM}=\frac{1}{4}\overline{DG}$.

(提示:延长$\overline{AM}$到$N$, 使$\overline{MN}=\overline{AM}$, 考虑$\triangle ADG$与
$\triangle ABN$的关系)
\item 试证:对角线相等且互相平分的四边形是矩形。
\item 已知:$ABCD$为平行四边形,且$\overline{AE}=\overline{EB}$, $\overline{AF}=\overline{FD}$,

求证:$\overline{BK}=\overline{KL}=\overline{LD}$(提示:连$\overline{AC}$, 在$\triangle ABC$中应用重心定理)

\begin{figure}[htp]\centering
    \begin{minipage}[t]{0.48\textwidth}
    \centering
\begin{tikzpicture}[>=latex, scale=1]
\tkzDefPoints{0/0/A, 3/0/B, 4/2/C}
\tkzDefPointsBy[translation = from B to C](A){D}
\tkzDefMidPoint(A,D) \tkzGetPoint{F}
\tkzDefMidPoint(A,B) \tkzGetPoint{E}
\tkzDrawPolygon(A,B,C,D)
\tkzDrawSegments(B,D C,F C,E)
\tkzInterLL(B,D)(C,F)  \tkzGetPoint{L}
\tkzInterLL(B,D)(C,E)  \tkzGetPoint{K}

\tkzLabelPoints[below](A,E,B)
\tkzLabelPoints[above](C,D,L)
\tkzLabelPoints[left](F)
\tkzLabelPoints[right](K)
    \end{tikzpicture}
    \caption*{第7题}
    \end{minipage}
    \begin{minipage}[t]{0.48\textwidth}
    \centering
    \begin{tikzpicture}[>=latex, scale=1]
\tkzDefPoints{0/0/B}
\tkzDefPoint(25:2){C}
\tkzDefPoint(65:1.9){A}

\tkzDrawPolygon(A,B,C)
\tkzDefEquilateral(B,A)  \tkzGetPoint{D}
\tkzDrawPolygon(A,B,D)
\tkzDefEquilateral(A,C)  \tkzGetPoint{E}
\tkzDrawPolygon(A,C,E)
\tkzDefEquilateral(B,C)  \tkzGetPoint{F}
\tkzDrawPolygon(F,C,B)
\tkzDrawPolygon(A,E,F,D)
\tkzLabelPoints[below](C,B)
\tkzLabelPoints[above](A,D,F,E)


    \end{tikzpicture}
    \caption*{第8题}
    \end{minipage}
    \end{figure}

\item 在$\triangle ABC$的外侧,以$\overline{AB}$、$\overline{AC}$为边分别作正三角形$ABD$
和$ACE$, 在内侧以$BC$为边作正$\triangle BCF$.

求证:四边形$AEFD$为平行四边形。
\item 过平行四边形$ABCD$的对角线$\overline{AC}$和$\overline{BD}$的交点$O$作一直
线,交$\overline{BC}$、$\overline{AD}$于$E$、$F$, 已知$\overline{BE}=2$米,$\overline{AF=2.8}$米,求$\overline{BC}$的长。
\item 已知:$E$是正方形$ABCD$的对角线$\overline{AC}$上的一点,$\overline{CE}=\overline{BC}$, 过$E$作$EF\bot \overline{AC}$交$\overline{AB}$于$F$.

求证:$\overline{AE}=\overline{EF}=\overline{FB}$.

\item 设$P$点是正方形$ABCD$的$\overline{CD}$上的一点,$\angle BAP$的平分
线与$\overline{BC}$的交点为$Q$, 那么$\overline{AP}=\overline{DP}+\overline{BQ}$.

\begin{figure}[htp]\centering
    \begin{minipage}[t]{0.48\textwidth}
    \centering
\begin{tikzpicture}[>=latex, scale=1]
\tkzDefPoints{0/0/B, 3/0/C, 3/3/D, 0/3/A, 3/2.2/P}
\tkzDefLine[bisector](P,A,B) \tkzGetPoint{P'}
\tkzInterLL(A,P')(B,C) \tkzGetPoint{Q}
\tkzDrawSegments(A,P A,Q)
\tkzDrawPolygon(A,B,C,D)
\tkzLabelPoints[below](C,B,Q)
\tkzLabelPoints[above](A,D)
\tkzLabelPoints[right](P)
    \end{tikzpicture}
    \caption*{第11题}
    \end{minipage}
    \begin{minipage}[t]{0.48\textwidth}
    \centering
    \begin{tikzpicture}[>=latex, scale=1]
        \tkzDefPoints{0/0/A, 4/0/B, 2.5/2.5/C, 0.8/2.5/D}
        \tkzDrawPolygon(A,B,C,D)
\tkzDefMidPoint(A,C)  \tkzGetPoint{E}
\tkzDefMidPoint(B,D)  \tkzGetPoint{F}
\tkzDrawSegments(A,C B,D E,F)
\tkzInterLL(C,F)(A,B) \tkzGetPoint{M}
\tkzDrawSegments[dashed](C,M)

\tkzLabelPoints[below](A,M,B)
\tkzLabelPoints[above](C,D)
\tkzLabelPoints[right](F)
\tkzLabelPoints[left](E)
    \end{tikzpicture}
    \caption*{第12题}
    \end{minipage}
    \end{figure}

\item 已知:梯形$ABCD$中,$AB\parallel CD$, $E$、$F$分别是$\overline{AC}$、
$\overline{BD}$的中点,

求证:$EF\parallel AB\parallel DC$;$\overline{EF}=\frac{1}{2}(\overline{AB}-\overline{DC})$

(提示:引$\overline{CF}$交$\overline{AB}$于$M$, 先证明$\triangle CFD\cong \triangle MFB$)
\item 有一个角是直角的梯形叫\textbf{直角梯形},在直角梯形$ABCD$
中,$AD\parallel BC$, $\angle A=90^{\circ}$, $\angle D=45^{\circ}$, $\overline{CD}$的垂直
平分线交$\overline{CD}$于$E$, 交$\overline{BA}$的延长线于$F$, 若$\overline{AD}=a$, 求
$\overline{BF}$的长。
\item 在$\triangle ABC$中,$\overline{AB}=\overline{AC}$, $E$是$\overline{AB}$中点,$D$在$\overline{AB}$的延
长线上,且$\overline{BD}=\overline{AB}$, 求证:$\overline{CE}=\frac{1}{2}\overline{CD}$.
\item 在四边形$ABCD$中,对角线$\overline{AC}$、$\overline{BD}$相交于$E$, 且$\overline{AC}=\overline{BD}$, $M$、$N$分别是$\overline{BC}$、$\overline{AD}$中点,$\overline{MN}$分别交$\overline{AC}$、
$\overline{BD}$于$F$、$G$, 求证$\overline{EF}=\overline{EG}$.
\item 已知:$\overline{BD}$是直角三角形$ABC$的斜边$\overline{AC}$上的中线,过
$A$点作$AQ\parallel DB$交$\overline{CB}$延长线于$Q$.

求证:$\overline{AQ}=\overline{AC}$.

\end{enumerate}


\section{面积与勾股定理}


\subsection{面积的基本性质和面积单位}
在第一章里我们已经谈到长度,角度的概念,现在让我
们再来分析一下“面积”这个基本的几何量。度量面积通常
取边长为一个长度单位的正方形做面积单位。例如,我们把
每边长为一厘米的正方形的面积叫做一平方厘米。

\begin{figure}[htp]
    \centering
\begin{tikzpicture}[>=latex]
\begin{scope}
    \draw(0,0) rectangle (3,3);
    \draw[|<->|](0,-.35)--node[fill=white]{单位长}(3,-.35);
    \draw[|<->|](3.7,0)--node[fill=white]{单位长}(3.7,3);
\end{scope}
\begin{scope}[xshift=6cm]
    \draw(0,0) rectangle (1,1);
    \node at (.5,1.2){1厘米};
    \node at (1.5, .5){1厘米};
\end{scope}
\end{tikzpicture}
    \caption{}
\end{figure}

度量图形面积的大小,实际上就是求出这个图形包含多
少个面积单位。也就是求这个图形所含单位面积的量数,这
个量数是个正实数。

由面积单位的直观含义,我们可看到面积这个几何量应
该具有下列基本性质:
\begin{enumerate}
\item 设$R$和$R'$是两个可以完全叠合的“平面区域”
(图3.69),即$R$和$R'$的形状大小完全一致,则$R$和$R'$的
面积显然相等。
\begin{figure}[htp]
    \centering
\includegraphics[scale=.8]{fig/3-69.png}
    \caption{}
\end{figure}
\begin{figure}[htp]
    \centering
\includegraphics[scale=.8]{fig/3-70.png}
    \caption{}
\end{figure}
\item 面积的可加性:设区域$R_1$, $R_2$是由区域$R$分割而
成(图3.70),或者说区域$R_1$、$R_2$拼成区域$R$, 则$R$的面
积为$R_1$, $R_2$的面积之和。
\item 设区域$T$是区域$R$的一部分,也就是区域$R$包含区
域$T$, 则有:$T$的面积$<R$的面积(图3.71)。例如在图
3.72中,正方形$ABCD$的面积$<$圆的面积$<$正方形$A'B'C'D'$
的面积。
\end{enumerate}

\begin{figure}[htp]\centering
    \begin{minipage}[t]{0.48\textwidth}
    \centering
    \includegraphics[scale=.8]{fig/3-71.png}
    \caption{}
    \end{minipage}
    \begin{minipage}[t]{0.48\textwidth}
    \centering
    \begin{tikzpicture}[>=latex, scale=1]
\tkzDefPoints{0/0/A', 3/0/B', 3/3/C', 0/3/D', 1.5/1.5/O}
\tkzDefMidPoint(A',B') \tkzGetPoint{B}
\tkzDefMidPoint(A',D') \tkzGetPoint{A}
\tkzDefMidPoint(C',B') \tkzGetPoint{C}
\tkzDefMidPoint(C',D') \tkzGetPoint{D}
\tkzDrawPolygon(A,B,C,D)
\tkzDrawPolygon(A',B',C',D')
\tkzDrawCircle(O,B)
\tkzLabelPoints[left](D',A,A')
\tkzLabelPoints[right](C',C,B')
\tkzLabelPoints[above](D)
\tkzLabelPoints[below](B)
    \end{tikzpicture}
    \caption{}
    \end{minipage}
    \end{figure}


\begin{blk}{定义}
如果两个平面图形的面积相等,我们说它们是等
积图形。  
\end{blk}

以后我们说两条线段的积或比,都是指它们的长度的积
或比;我们说两个图形面积的比是这两个图形面积所对应的实数比。

\begin{ex}
\begin{enumerate}
    \item 请同学自己填写下面常用的面积单位:
\begin{enumerate}
\item     一平方公里($\rm km^2$)$=\underline{\qquad\qquad}$平方米($\rm m^2$)
\item     一平方米($\rm m^2$)$=\underline{\qquad\qquad}$平方分米($\rm dm^2$)
\item    一平方分米($\rm dm^2$)$=\underline{\qquad\qquad}$平方厘米($\rm cm^2$)
\item 一平方厘米($\rm cm^2$)$=\underline{\qquad\qquad}$平方毫米($\rm mm^2$)
\end{enumerate}

    \item 工程技术人员通常用“加半移三”的办法把平方米数换
    算成亩数,即把平方米数加上它的一半,小数点再向左移
    三位即得亩数,请同学把上面这段话用代数式表达出来,
    并给出证明。
    \item 把下列各平方米数用口算化成亩。
\[ 4500{\rm m^2};\quad     6000{\rm m^2};\quad     380{\rm m^2};
 \quad    28000{\rm m^2}\]
\end{enumerate}
\end{ex}

\subsection{长方形面积}
长方形的一组邻边分别叫做长和宽。

设长方形$ABCD$(图3.73), 长$\overline{AB}=a$, 宽$\overline{AC}=b$, 
则长方形面积$=$长$\x$宽$=ab$.

\begin{figure}[htp]
    \centering
\begin{tikzpicture}[>=latex]
\tkzDefPoints{0/0/A, 4/0/B, 4/3/C, 0/3/D}
\tkzDrawPolygon(A,B,C,D)
\draw(0,1)--(4,1);  \draw(0,2)--(4,2);
\draw(1,0)--(1,3);  \draw(2,0)--(2,3);  \draw(3,0)--(3,3);
\fill[pattern=north east lines](0,0) rectangle (1,1);
\tkzLabelPoints[left](A,D)
\tkzLabelPoints[above right](C)
\tkzLabelPoints[below right](B)
\draw[|<->|](0,-.3)--node[fill=white]{$a$}(4,-.3);
\draw[|<->|](4.3,0)--node[fill=white]{$b$}(4.3,3);
\end{tikzpicture}
    \caption{}
\end{figure}


上面这个公式,同学们在小学就已经学过了。虽然如
此,我们不妨问问同学:你们知道这个公式成立的理由吗?可
能会有许多同学觉得这个问题太简单了,其实要严格证明这
个公式并不那么简单。当$a,b$是正整数时,的确是对的。例
如,$a=4$, $b=3$, 那么$\overline{AB}$的长正好是4个单位长,$\overline{BC}$长正
好是3个单位长,$ABCD$正好被12个单位正方形填满。可
是,$\overline{AB}$、$\overline{BC}$的长不见得都能用整数表示,甚至有的线段长
不能用有理数来表示,那么上述公式当$a$、$b$不是正整数时是
否还成立呢:所以长方形面积的讨论并不是同学想像的那么
简单,以至我们现在还不能严格证明它。下面我们仅对$a$、
$b$都是整数或都是分数时加以证明。

\begin{enumerate}
    \item  $a=n$, $b=m$, $n$、$m$为正整数,长方形$ABCD$可
    分成$n\x m$个单位正方形,所以面积为$n\cdot m=a\cdot b$
    \item  $a$、$b$都是分数,不防设它们的分母相同,设
$a=\frac{m}{n}$, $b=\frac{\ell}{n}$,
    这时两个分数单位都是$\frac{1}{n}$单位长,$a=m$个$\frac{1}{n}$,$b=\ell$个$\frac{1}{n}$。

    我们把长度单位等分为$n$段(图3.74),单位正方形
    被分割为$n^2$个边长为$\frac{1}{n}$
    长度单位的小正方形。所以边长为$\frac{1}{n}$的小单位正方形面积是原单位正方形面积的$\frac{1}{n^2}$,
    长方形的长$a$为$m$
    个$\frac{1}{n}$长,宽为$\ell$个$\frac{1}{n}$
    长,我们把长方形$ABCD$
    分割成$m\x \ell$个边长为$\frac{1}{n}$
    的小正方形(图3.75),所以该
    长方形面积为$m\x\ell$个($\frac{1}{n^2}$单位面积),即$\frac{m}{n}\cdot\frac{\ell}{n}$个单
    位面积。所以长方形面积$=a\cdot b=\frac{m}{n}\cdot\frac{\ell}{n}$
 \item $a$、$b$都是实数时,长方形的面积仍等于长$\x$宽。
    这到高中我们才能证明它。
\end{enumerate}




\begin{figure}[htp]\centering
    \begin{minipage}[t]{0.4\textwidth}
    \centering
\begin{tikzpicture}[>=latex, scale=.7]
    \draw [help lines] grid (5,5);
    \fill[pattern = north east lines](0,0) rectangle (1,1);
\draw[|<->|](-1,0)--node[fill=white]{单位长}(-1,5);
\draw[|<->|](0,-.3)--node[below]{$\frac{1}{n}$}(1,-.3);
\node at (3,-.5){$n=5$};

    \end{tikzpicture}
    \caption{}
    \end{minipage}
    \begin{minipage}[t]{0.55\textwidth}
    \centering
    \begin{tikzpicture}[>=latex, scale=.6]
      \draw [help lines] grid (9,4);
      \fill[pattern = north east lines](0,0) rectangle (1,1);
\node at (0,0)[left]{$A$};\node at (0,4)[left]{$D$};
\node at (9,0)[below right]{$B$};\node at (9,4)[above right]{$C$};
\draw[|<->|](0,-.2)--node[below]{$\frac{m}{n}\; \left(\frac{9}{5}\right)$}(9,-.2);
\draw[|<->|](9.2,0)--node[right]{$\frac{\ell}{n}\; \left(\frac{4}{5}\right)$}(9.2,4);
    \end{tikzpicture}
    \caption{}
    \end{minipage}
    \end{figure}


    有了长方形面积公式后,一些其它与长方形有关连的图
    形的面积就很容易求得。

    平行四边形的一边叫底,这条边与其对边之间的公垂线
    段叫做高。

\begin{blk}
    {定理}
平行四边形的面积等于底$\x$高.
\end{blk}

已知:平行四边形$ABCD$中,设底$\overline{AB}=a$, 高$\overline{BE}=h$(图3.76)

求证:$\parallelogram ABCD$的面积$=a\cdot h$.

\begin{figure}[htp]
    \centering
    \begin{tikzpicture}
\tkzDefPoints{0/0/C, 3/0/D, 4/2/A}
\tkzDefPointsBy[translation = from D to A](C){B}
\tkzDefPointBy[projection = onto C--D](B)  \tkzGetPoint{E}
\tkzDefPointsBy[translation = from B to A](E){F}
\tkzDrawPolygon(A,B,C,D)
\tkzDrawSegments[dashed](B,E A,F D,F)
\tkzLabelPoints[below](C,E,D,F)
\tkzLabelPoints[above](B,A)
\fill[pattern=north east lines](B)--(C)--(E)--(B);
\fill[pattern=north east lines](A)--(D)--(F)--(A);        
    \end{tikzpicture}
    \caption{}
\end{figure}

\begin{proof}
    引$AF\bot AB$交$CD$的延长线于$F$, 
    则$AF\parallel BE$, 四边形$ABEF$是矩形。

    在直角三角形$ADF$与$BCE$中,

$\because\quad \overline{AF}=\overline{BE},\quad \overline{AD}=\overline{BC}$,

$\therefore\quad     \triangle ADF\cong \triangle BCE$

$\therefore\quad \triangle ADF$的面积$=\triangle BCE$的面积(性质1)

$\therefore\quad \parallelogram ABCD$的面积$=\triangle CBE$的面积$+$四边形$ABED$的面积,(性质2)

$\therefore\quad \parallelogram ABCD$的面积$=\triangle ADF$的面积$+$四边形$ABED$的面积$=$矩形$ABEF$的面积$=a\cdot h$.
\end{proof}


\begin{blk}
    {推论1} 等底等的平行四边形等积
\end{blk}

\begin{blk}
    {推论2} 任何两个平行四边形面积的比等于其底与高的
积的比。
\end{blk}

\begin{blk}
    {推论3}
等底平行四边形的面积的比等于其高的比。
\end{blk}

\begin{blk}
    {推论4} 等高的平行四边形面积的比,等于其底的比。
\end{blk}

从三角形某顶点向对边作高,则对边就叫做底。

\begin{blk}
    {定理} 三角形的面积等于底$\x$高的一半.
\end{blk}

已知:$\triangle ABC$中底$\overline{AB}=a$, $\overline{AB}$边上的高$\overline{CE}=h$(图3.77)。

求证:$\triangle ABC$的面积$=\frac{1}{2}ah$.

\begin{figure}[htp]
    \centering
\begin{tikzpicture}
\tkzDefPoints{0/0/A, 3/0/B, 1/2/C}
\tkzDefPointsBy[translation = from A to C](B){D}
\tkzDefPointBy[projection = onto A--B](C)    \tkzGetPoint{E}
\tkzDrawPolygon(A,B,C)
\tkzDrawSegments[dashed](C,D B,D)
\tkzDrawSegments(C,E)
\tkzMarkRightAngle[size=.24](C,E,B)
\tkzLabelPoints[below](A,B,E)
\tkzLabelPoints[above](C,D)
\end{tikzpicture}
    \caption{}
\end{figure}

\begin{proof}
    引$BD\parallel AC$, $CD\parallel AB$, $CD$、$BD$交于$D$, (图3.77)
则$ABDC$为平行四边形。

但$\parallelogram ABDC$的面积$=a\cdot h$,

$\parallelogram ABDC$的面积$=2\x\triangle ABC$的面积,

$\therefore\quad \triangle ABC$的面积$=\frac{1}{2}ah$.
\end{proof}

\begin{blk}
   {推论1} 等底等高的三角形等积。
\end{blk}

\begin{blk}
    {推论2}
任何两个三角形面积的比等于底与高的积的
比。
\end{blk}


\begin{blk}
    {推论3} 等底的三角形的面积的比等于其高的比;等高
的三角形的面积的比等于其底的比。
\end{blk}

梯形两底之间的任一条公垂线段都叫做\textbf{梯形的高}。

\begin{blk}
    {定理}梯形的面积等于两底和与高的积的一半。
\end{blk}

请同学给出证明。

\begin{blk}
{推论} 梯形的面积等于高与它的中位线的积。
\end{blk}

\begin{figure}[htp]\centering
    \begin{minipage}[t]{0.48\textwidth}
    \centering
\begin{tikzpicture}[>=latex, scale=1]
\tkzDefPoints{0/0/A, 3/0/B, 2.5/1.5/C, 1/1.5/D, 1/0/E}
\tkzDrawPolygon(A,B,C,D)
\tkzDrawSegments(D,E)
\tkzMarkRightAngle[size=.2](D,E,A)
\node at (1.5,0)[below]{$b$};
\node at (1.75,1.5)[above]{$a$};
\node at (1,.75)[right]{$h$};
\tkzDrawSegments[dashed](B,D)
    \end{tikzpicture}
    \caption{}
    \end{minipage}
    \begin{minipage}[t]{0.48\textwidth}
    \centering
    \begin{tikzpicture}[>=latex, scale=1]
\tkzDefPoints{0/0/A, 1.5/0/B, 2.5/1.4/C, 1/2.5/D, -.8/1.6/E}
\tkzDrawPolygon(A,B,C,D,E)
\tkzDefLine[parallel=through E](D,A) \tkzGetPoint{E'}
\tkzDefLine[parallel=through C](D,B) \tkzGetPoint{C'}
\tkzInterLL(E,E')(A,B) \tkzGetPoint{F}
\tkzInterLL(C,C')(A,B) \tkzGetPoint{G}
\tkzDrawSegments[dashed](E,F D,A D,B C,G)
\tkzDrawLines(F,G)
\tkzDrawSegments(D,F D,G)
\tkzLabelPoints[below](A,B,G,F)
\tkzLabelPoints[above](C,D,E)
    \end{tikzpicture}
    \caption{}
    \end{minipage}
    \end{figure}

\begin{example}
    已知:五边形$ABCDE$(图3.79).

    求作:一个三角形和五边形$ABCDE$等积。
\end{example}

作法:作$\overline{DB}$、$\overline{DA}$, 过顶点$C$、$E$, 引$CG$和$EF$, 分
别平行于$DB$和$DA$, 直线$AB$分别交$CG$和$EF$于$G$、$F$两
点,则$\triangle DFG$为所求的三角形。

\begin{proof}
    $\because\quad \triangle DBC$的面积$=\triangle DBG$的面积,
    $\triangle DAE$的面积$=\triangle DFA$的面积(等底等高)。

因此:
\[\begin{split}
    \text{五边形$ABCDE$的面积}&=(\triangle DAB+\triangle DAE+\triangle DBC)\text{的面积}\\
&=(\triangle DAB+\triangle DFA+\triangle DBG)\text{的面积}\\
&=\triangle DFG\text{的面积}
\end{split}\]
\end{proof}

\begin{rmk}
    本题解法不是唯一的(想想为什么?)
\end{rmk}

\begin{ex}
\begin{enumerate}
    \item 求证:如果两个三角形有公共底边,它们的另一对顶点
    同在底边的一条平行线上,那么这两个三角形等积。
    \item 求证:平行四边形的两条对角线将它分成四个等积三角
    形。
    \item 求证:如果两个三角形有两边对应相等,并且它们的夹
    角互补,那么这两个三角形等积.
    \item 证明一梯形两底的中点间的线段分梯形为两个等积形。
    \item 已知一等腰直角三角形的斜边是18米,求它的面积。
    \item 经过$\parallelogram ABCD$的对角线$\overline{AC}$上的一点$P$作两条边的平行线
    $\overline{EF}$, $\overline{GH}$(如图),则$\parallelogram GBEP$和$\parallelogram FPHD$等积。
    \item 经过$\parallelogram ABCD$的顶点$D$的任一直线设与$\overline{BC}$、$\overline{AB}$或者它
    们的延长线分别交于$Q$、$P$, 则$\triangle ABQ$和$\triangle PQC$等积。
    \item 求作一直角三角形和已知正方形等积。
    \item 求作一个三角形和已知矩形等积。
    \item 求作一个三角形和已知四边形等积。
    \item 求作一等腰三角形和已知不等边三角形等积。
\end{enumerate}
\end{ex}

\begin{figure}[htp]\centering
    \begin{minipage}[t]{0.48\textwidth}
    \centering
\begin{tikzpicture}[>=latex, scale=1]
\tkzDefPoints{0/0/B, 3/0/C, 3.7/2.3/D}
\tkzDefPointsBy[translation = from C to D](B){A}
\tkzDefPointWith[linear, K=.3](A,C) \tkzGetPoint{P}
\tkzDefPointWith[linear, K=.3](A,D) \tkzGetPoint{F}
\tkzDefPointWith[linear, K=.3](A,B) \tkzGetPoint{G}
\tkzDefPointWith[linear, K=.3](D,C) \tkzGetPoint{H}
\tkzDefPointWith[linear, K=.3](B,C) \tkzGetPoint{E}
\tkzDrawPolygon(A,B,C,D)
\tkzDrawSegments(E,F G,H A,C)
\tkzLabelPoints[left](A,G,B)
\tkzLabelPoints[right](D,H,C)
\tkzLabelPoints[above](F)  \tkzLabelPoints[below](E)
\tkzLabelPoints[below left](P)
    \end{tikzpicture}
    \caption*{第6题}
    \end{minipage}
    \begin{minipage}[t]{0.48\textwidth}
    \centering
    \begin{tikzpicture}[>=latex, scale=1]
        \tkzDefPoints{0/0/B, 3/0/C, 3.6/1.7/D}
        \tkzDefPointsBy[translation = from C to D](B){A}
        \tkzDrawPolygon(A,B,C,D)
\tkzDefPointWith[linear, K=.3](B,C) \tkzGetPoint{Q}
\tkzInterLL(D,Q)(A,B)  \tkzGetPoint{P}
\tkzDrawSegments(B,P C,P P,D A,Q)
\tkzLabelPoints[left](A,B,P)
\tkzLabelPoints[right](D,C)
\tkzLabelPoints[above](Q)
    \end{tikzpicture}
    \caption*{第7题}
    \end{minipage}
    \end{figure}




\subsection{勾股定理}

我国古代约二千年前就掌握了直角三角形三边之间的关
系。中国古算书记载有“勾三股四弦五”的规律。我国通常
“直角三角形中较短的直角边叫做勾,较长的直角边叫做股,斜边叫做弦(图3.80)。那么“勾三股四弦五”的意思是:如果一
个直角三角形的两条直角边长是3和4个长度单
位,那么斜边长一定是5个长度单位,3、4、5三个数之间满
足关系:
\[3^2+4^2=5^2\]
这就是说以3、4为边长的两个正方形的面积的和正好等
于以5为边长的正方形的面积(图3.81)

\begin{figure}[htp]\centering
    \begin{minipage}[t]{0.35\textwidth}
    \centering
\begin{tikzpicture}[>=latex, scale=.8]
\draw(0,0)node[below]{$C$}--node[below]{勾}(3,0)node[below]{$B$}--node[right]{弦}(0,4)node[above]{$A$}--node[left]{股}(0,0);
    \end{tikzpicture}
    \caption{}
    \end{minipage}
    \begin{minipage}[t]{0.58\textwidth}
    \centering
    \begin{tikzpicture}[>=latex, scale=.6]
\tkzDefPoints{0/0/B, 3/0/C, 0/4/A}
\tkzLabelPoints[below left](B)
\tkzLabelPoints[below right](C)
\tkzLabelPoints[above](A)

\tkzInit[xmin=0,xmax=3,ymin=-3, ymax=0]
\tkzGrid
\tkzDefSquare(B,A)  \tkzGetPoints{A'}{B'}
\tkzDrawPolygon[thick](A,B,B',A')
\tkzInit[xmin=-4,xmax=0,ymin=0, ymax=4]
\tkzGrid
\tkzDefSquare(C,B)  \tkzGetPoints{B''}{C'}
\tkzDrawPolygon[thick](C,B,B'',C')
\tkzDefSquare(A,C)  \tkzGetPoints{A''}{C''}
\tkzDrawPolygon[thick](A,C,A'',C'')
\foreach \x in {1,2,3,4}
{
    \tkzDefPointWith[linear, K=\x/5](A,C) \tkzGetPoint{A\x}
    \tkzDefPointWith[linear, K=\x/5](C'',A'') \tkzGetPoint{C\x}
    \tkzDefPointWith[linear, K=\x/5](C,A'') \tkzGetPoint{B\x}
    \tkzDefPointWith[linear, K=\x/5](A,C'') \tkzGetPoint{D\x}
}

\foreach \x in {1,2,3,4}
{
    \tkzDrawSegments[gray](A\x,C\x)
    \tkzDrawSegments[gray](B\x,D\x)
}

\node at (1.5,0)[above]{3};
\node at (0,2)[right]{4};
\node at (1.5,2)[left]{5};

    \end{tikzpicture}
    \caption{}
    \end{minipage}
    \end{figure}

由“勾三股四弦五”可以归纳出一般规律:

\begin{blk}
   {勾股定理} 直角三角形中,两条直角边的平方和等于斜
边的平方。 
\end{blk}

已知:直角三角形中,勾长是$a$, 股长是$b$, 斜边长是$c$
(图3.82)。

求证:$a^2+b^2=c^2$.

\begin{proof}
    以$a+b$为边长作正方形$ABCD$, 在$\overline{AB}$、$\overline{BC}$、
$\overline{CD}$、$\overline{DA}$各边上分别截取$\overline{AE}=\overline{BF}=\overline{CG}=\overline{DH}=a$, 于是,
$\overline{EB}=\overline{FC}=\overline{GD}=\overline{HA}=b$. 因为$\angle A=\angle B=\angle C=\angle D=90^{\circ}$
所以,$\triangle AEH\cong \triangle BFE\cong \triangle CGF\cong \triangle DHG$(图3.82).

并且,这四个全等的直角三角形的斜边全是$c$. 又因为直角
三角形中的两锐角互余,所以:
\[\alpha+\beta=90^{\circ}\]
但$\alpha+\beta+\gamma=180^{\circ}$

$\therefore\quad \gamma=90^{\circ}$.

由正方形的定义可知四边形$EFGH$是正方形。正方形
$ABCD$的面积是$(a+b)^2$或$4\x\frac{1}{2}ab+c^2$.

所以
\[(a+b)^2=4\x\frac{1}{2}ab+c^2\]
展开得:\[a^2+2ab+b^2=2ab+c^2\]
即
\[a^2+b^2=c^2\]
\end{proof}

\begin{figure}[htp]\centering
    \begin{minipage}[t]{0.38\textwidth}
    \centering
\begin{tikzpicture}[>=latex, scale=1]
\tkzDefPoints{0/0/B, 3/0/C, 3/3/D}
\tkzDefPointsBy[translation = from C to D](B){A}
\tkzDrawPolygon(A,B,C,D)
\tkzDefPointWith[linear, K=.4](A,B)\tkzGetPoint{E}
\tkzDefPointWith[linear, K=.4](B,C)\tkzGetPoint{F}
\tkzDefPointWith[linear, K=.4](C,D)\tkzGetPoint{G}
\tkzDefPointWith[linear, K=.4](D,A)\tkzGetPoint{H}
\tkzDrawPolygon(E,F,G,H)
\tkzLabelPoints[left](A,B,E)
\tkzLabelPoints[right](C,D,G)\tkzLabelPoints[above](H)
\tkzLabelPoints[below](F)
\tkzMarkAngles[mark=none, size=.4](H,E,A B,E,F)
\tkzMarkAngles[mark=none, size=.5](F,E,H)
\tkzLabelAngle[pos=.6](H,E,A){$\alpha$}
\tkzLabelAngle[pos=.6](B,E,F){$\beta$}
\tkzLabelAngle[pos=.8](F,E,H){$\gamma$}
\node at (.6,0)[below]{$a$};
\node at (2.4,0)[below]{$b$};
\node at (3,.6)[right]{$a$};
\node at (3,2.4)[right]{$b$};
\node at (2.4,3)[above]{$a$};
\node at (.9,3)[above]{$b$};
\node at (0,2.4)[left]{$a$};
\node at (0,.9)[left]{$b$};

    \end{tikzpicture}
    \caption{}
    \end{minipage}
    \begin{minipage}[t]{0.58\textwidth}
    \centering
    \begin{tikzpicture}[>=latex, scale=1]
    \tkzDefPoints{0/0/A, 2.5/0/C, 1.6/1.2/B}
\tkzDefSquare(A,B)  \tkzGetPoints{H}{K}
\tkzDefSquare(B,C)  \tkzGetPoints{F}{G}
\tkzDefSquare(C,A)  \tkzGetPoints{E}{D}
\tkzLabelPoints[above](H,B,G)
\tkzLabelPoints[below](D,E)
\tkzLabelPoints[left](A,K)
\tkzLabelPoints[right](C,F)
\tkzDrawPolygon(A,B,H,K)
\tkzDrawPolygon(A,C,D,E)
\tkzDrawPolygon(C,B,G,F)
\tkzDrawPolygon[pattern=north east lines](A,C,K)
\tkzDrawPolygon[pattern=north west lines](A,B,E)
\tkzDefPointBy[projection = onto D--E](B) \tkzGetPoint{L}
\tkzInterLL(B,L)(A,C)  \tkzGetPoint{I}
\tkzLabelPoints[below left](L,I)
\tkzDrawSegments(B,L)
\tkzMarkRightAngle[size=.23](B,L,D)
    \end{tikzpicture}
    \caption{}
    \end{minipage}
    \end{figure}

我国古代算学家对这个定理作了各种证明,上面的证法
只是其中的一种。作为例子,我们再介绍一种勾股定理的欧
几里得的证法。


如图3.83所示,在$\triangle ABC$中,$\angle ABC=90^{\circ}$. 以
$\triangle ABC$的各边为一边,分别在$\triangle ABC$的外侧作正方形
$AEDC$, $BCFG$, $ABHK$. 引$BL\bot ED$, 分别交$\overline{ED}$、$\overline{AC}$
于$L$、$I$, 则$BL\parallel AE$.

作$\overline{KC}$、$\overline{BE}$.

$\because\quad \overline{AB}=\overline{AK},\quad \overline{AC}=\overline{AE},\quad \angle KAC=\angle EAB$,

$\therefore\quad \triangle ABE\cong \triangle AKC$ (SAS)

$\because\quad \triangle ABE$与矩形$AELI$等底等高,
$\triangle AKC$与正方形$ABHK$等底等高,

$\therefore\quad \triangle ABE$的面积$=\frac{1}{2}$矩形$AELI$的面积

$\triangle AKC$的面积$=\frac{1}{2}$正方形$ABHK$的面积。

$\therefore\quad $矩形$AELI$的面积$=$正方形$ABHK$的面积。

同理可证:矩形$CILD$的面积$=$正方形$BCFG$的面积。

但:正方形$AEDC$的面积$=$矩形$AELI$的积$+$矩形$CILD$的面积。

$\therefore\quad $正方形$AEDC$的面积$=$正方形$ABHK$的面积$+$正方形
$BCFG$的面积。
即
\[\overline{AC}^2=\overline{AB}^2+\overline{BC}^2\]
这就证明了勾股定理。

比较这两种证法,我们看到我国古代的证明法用了代数
知识:$(a+b)^2=a^2+2ab+b^2$, 比较简捷明了,而古希
腊的证法,要用到等积变形,比较复杂。

勾股定理揭示了直角三角形三边之间的关系,如果知道
了直角三角形任意两边的长度,应用勾股定理便可计算出第
三边的长。

\begin{example}
    已知:直角$\triangle ABC$, $\angle B=90^{\circ}$, $AB=12$cm,
$BC=5$cm (图3.84).

求:$\overline{AC}$之长。
\end{example}


\begin{solution}
    设$\overline{AC}$的长为$x$cm, 根据勾股定理:
\[x^2=5^2+12^2\quad \Rightarrow\quad x^2=169\]
$\because\quad x>0,\qquad \therefore\quad x=13({\rm cm})$
\end{solution}

\begin{figure}[htp]\centering
    \begin{minipage}[t]{0.48\textwidth}
    \centering
\begin{tikzpicture}[>=latex, scale=.6]
\tkzDefPoints{0/0/B, 2.5/0/C, 0/6/A}
\tkzDrawPolygon(A,B,C)
\tkzLabelPoints[above](A)
\tkzLabelPoints[below](B,C)
\tkzMarkRightAngle[size=.3](A,B,C)
    \end{tikzpicture}
    \caption{}
    \end{minipage}
    \begin{minipage}[t]{0.48\textwidth}
    \centering
    \begin{tikzpicture}[>=latex, scale=.6]
\tkzDefPoints{0/0/B, -2/0/C, 0/5/A}
\tkzDrawPolygon(A,B,C)
\tkzLabelPoints[above right](A,B)
\tkzLabelPoints[above left](C)
\tkzMarkRightAngle[size=.3](A,B,C)
\fill[pattern=north east lines](-3.5,-.4) rectangle (1,0);
\draw(-3.5,0)--(1,0);
\tkzDrawLine[add=0 and .3](B,A)
    \end{tikzpicture}
    \caption{}
    \end{minipage}
    \end{figure}

\begin{example}
    将长为5米的梯子$\overline{AC}$斜靠在墙上,$\overline{BC}$长为2米,
求地面到梯子上端高$\overline{AB}$之长(精确到0.01米)(图3.85).
\end{example}



\begin{solution}
    设$\overline{AB}$的长为$x$(米),由于$\triangle ABC$是直角三角
形,且以$\overline{AC}$为斜边;根据勾股定理有,
\[x^2+2^2=5^2 \quad \Rightarrow\quad x^2=25-4=21\]

$\because\quad x>0$

$\therefore\quad x=\sqrt{21}\approx 4.58$(米)。

答:地面到梯子上端高$\overline{AB}$之长约
等于4.58米。
\end{solution}

\begin{blk}
    {勾股定理逆定理} 如果三角形的三
边$a$、$b$、$c$满足条件$a^2+b^2=c^2$, 那$c$边所对的角是直角。
\end{blk}


\begin{proof}
设$\triangle ABC$的三边满足$a^2+b^2=c^2$(图3.86). 
另作一个直角三角形$A'B'C'$,使$\angle C'=90^{\circ}$, $\overline{B'C'}=a$,
$\overline{A'C'}=b$. 根据勾股定理,
\[\overline{A' B'}^2=a^2+b^2\]
但
\[\overline{A'B'}^2=c^2\]

$\therefore\quad \overline{A'B'}=c$.

$\therefore\quad \triangle ABC\cong \triangle A'B'C'$ (SAS)

$\therefore\quad \angle C=\angle C'=90^{\circ}$
\end{proof}

\begin{figure}[htp]
    \centering
\begin{tikzpicture}[scale=.7]
\begin{scope}
\draw(0,0)node[below]{$B$}--node[below]{$a$}(3,0)node[below]{$C$}--node[right]{$b$}(3,4)node[above]{$A$}--node[left]{$c$}(0,0);
\end{scope}
\begin{scope}[xshift=6cm]
    \draw(0,0)node[below]{$B'$}--node[below]{$a$}(3,0)node[below]{$C'$}--(3,4)node[above]{$A'$}--node[left]{$c$}(0,0);
\end{scope}
\end{tikzpicture}
    \caption{}
\end{figure}

\begin{blk}
    {广义勾股定理} 平行四边形两条对角线的平方和等于它
的四边的平方和,或等于相邻两边平方和的二倍。
\end{blk}

已知:平行四边形$ABCD$ (图3.87). 

求证:$\overline{AC}^2+\overline{BD}^2=\overline{AB}^2+\overline{BC}^2+\overline{CD}^2+\overline{DA}^2=2\left(\overline{AB}^2+\overline{BC}^2\right)$

\begin{proof}
    设$\angle B$为锐角,分别由$A$、$D$向边$\overline{BC}$及其延长线
    作垂线$AE$、$DF$.

$\because\quad \overline{AB}=\overline{CD},\quad \angle B=\angle DCF,\quad \angle AEB=\angle DFC=90^{\circ}$,

$\therefore\quad \triangle ABE\cong \triangle DCF$ (AAS)

    在直角三角形$ABE$、$ACE$、$DBF$中,由勾股定理可以得到:
\begin{align}
    \overline{AB}^2&=\overline{AE}^2+\overline{BE}^2\\
    \overline{AC}^2&=\overline{AE}^2+\overline{EC}^2=\overline{AE}^2+\left(\overline{BC}-\overline{BE}\right)^2\\
    \overline{BD}^2&=\overline{BF}^2+\overline{DF}^2=\left(\overline{BC}+\overline{CF}\right)^2+\overline{DF}^2
\end{align}
$(3.2)+(3.3)$得:
\[\begin{split}
    \overline{AC}^2+\overline{BD}^2&=\left(\overline{BC}-\overline{BE}\right)^2+\left(\overline{BC}+\overline{CF}\right)^2+2\overline{AE}^2\\
    &=2\left(\overline{BC}^2+\overline{BE}^2+\overline{AE}^2\right)\\
    &=2\left(\overline{BC}^2+\overline{AB}^2\right)\\
    &=\overline{AB}^2+\overline{BC}^2+\overline{CD}^2+\overline{DA}^2
\end{split}\]
\end{proof}

\begin{figure}[htp]\centering
    \begin{minipage}[t]{0.48\textwidth}
    \centering
\begin{tikzpicture}[>=latex, scale=1]
\tkzDefPoints{0/0/B, 3/0/C, 4/2/D}
\tkzDefPointsBy[translation = from C to D](B){A}
\tkzDefPointBy[projection = onto B--C](A)  \tkzGetPoint{E}
\tkzDefPointBy[projection = onto B--C](D)  \tkzGetPoint{F}
\tkzDrawPolygon(A,B,C,D)
\tkzDrawSegments[dashed](A,E D,F)
\tkzDrawSegments(A,C B,D C,F)
\tkzMarkRightAngles[size=.2](A,E,B D,F,C)
\tkzMarkAngles[size=.3, mark=none](C,B,A F,C,D)

\tkzLabelPoints[below](B,E,C,F)
\tkzLabelPoints[above](A,D)
    \end{tikzpicture}
    \caption{}
    \end{minipage}
    \begin{minipage}[t]{0.48\textwidth}
    \centering
    \begin{tikzpicture}[>=latex, scale=1]
\tkzDefPoints{0/0/B, 3/0/C, 3/2/D, 0/2/A}
\tkzDrawPolygon(A,B,C,D)
\tkzDrawSegments(A,C B,D)
\tkzLabelPoints[below](B,C)
\tkzLabelPoints[above](A,D)
    \end{tikzpicture}
    \caption{}
    \end{minipage}
    \end{figure}

当平行四边形为矩形时(图3.88),由于矩形两条对
角线相等,上面的关系式转化为:
\[2\overline{AC}^2=2\left(\overline{AB}^2+\overline{BC}^2\right)\]
即
\[\overline{AC}^2=\overline{AB}^2+\overline{BC}^2\]
这恰好符合勾股定理。因
此,一般平行四边形对角线与边的关系可看成广义勾股定理。
    
\begin{example}
    已知:$\triangle ABC$的三边长$\overline{BC}=a$, $\overline{AC}=b$, $\overline{AB}=c$(图3.89)。

求:三角形三边上中线长$m_a$、$m_b$、$m_c$.
\end{example}


\begin{solution}
    以$\triangle ABC$的边$\overline{AB}$, $\overline{AC}$为邻边作平行四边形$ABDC$, 
对角线$\overline{AD}$、$\overline{BC}$相交于$M$, 则$\overline{AM}$为$\triangle ABC$中$\overline{BC}$边上
的中线,它为$\overline{AD}$长的一半,由广义勾股定理知:
\[\overline{AD}^2+\overline{BC}^2=2\left(\overline{AB}^2+\overline{AC}^2\right)\]
即
\[4\overline{AM}^2+a^2=2(b^2+c^2)\]
$\therefore\quad \overline{AM}^2=\frac{1}{4}\x [2(b^2+c^2)-a^2]$
\[AM=\frac{1}{2}\sqrt{2(b^2+c^2)-a^2}\]
即:$m_a=\frac{1}{2}\sqrt{2(b^2+c^2)-a^2}$

同理可得:
\[m_b=\frac{1}{2}\sqrt{2(a^2+c^2)-b^2},\qquad m_c=\frac{1}{2}\sqrt{2(a^2+b^2)-c^2}\]
\end{solution}

\begin{figure}[htp]\centering
    \begin{minipage}[t]{0.48\textwidth}
    \centering
\begin{tikzpicture}[>=latex, scale=1]
\tkzDefPoints{0/0/B, 3/0/C, 2/2/A}
\tkzDefPointsBy[translation = from A to C](B){D}
\tkzDrawPolygon(A,B,C)
\tkzDrawSegments[dashed](B,D C,D A,D)
\tkzInterLL(A,D)(B,C)   \tkzGetPoint{M}
\tkzLabelPoints[below](D,B,M,C)
\tkzLabelPoints[above](A)
    \end{tikzpicture}
    \caption{}
    \end{minipage}
    \begin{minipage}[t]{0.48\textwidth}
    \centering
    \begin{tikzpicture}[>=latex, scale=1]
      \tkzDefPoints{0/0/C, 2.5/0/B, 0/3/A}
      \tkzDrawPolygon(A,B,C)
      \tkzLabelPoints[below](B,C)
      \tkzLabelPoints[above](A)
    \end{tikzpicture}
    \caption*{第1题}
    \end{minipage}
    \end{figure}



\begin{ex}
\begin{enumerate}
    \item 在$\triangle ABC$中,$\angle C$是直角。
    \begin{enumerate}
        \item 已知$\overline{BC}=6$, $\overline{AC}=8$, 求$\overline{AB}$之长。
        \item 已知$\overline{AB}=25$, $\overline{BC}=14$, 求$\overline{AC}$之长。
        \item 已知$\overline{AC}=13$, $\overline{AB}=19$, 求$\overline{BC}$之长。
    \end{enumerate}
    
    
    \item 已知等腰直角三角形的直角边长是$a$, 求底边。
    \item 正方形一边长为10寸,求对角线长。
    \item 矩形相邻两边为$a$、$b$时,求对角线长。
    \item 已知等边三角形的边长为$a$, 求等边三角形一边上的高和
    这等边三角形的面积。
    \item 求出对角线为$d$的正方形的边长。
    \item 作一个三角形,使三边长为3cm, 4cm, 5cm, 边长为
    5cm的边所对的角是不是直角?为什么?
    \item 设三角形的三边等于下列各组數,判断各三角形是否是
    直角三角形:
\begin{multicols}{2}
    \begin{enumerate}
        \item 7、8、10;
        \item 3、5、4;
        \item 5、12、13;
        \item 4、8、10.
    \end{enumerate}
\end{multicols}
    \item 用勾股定理证明:斜边和一直角边对应相等的两直角三
    角形全等。
    \item 已知一等边三角形的高是$4\sqrt{3}$, 求面积。
    \item 等腰三角形的腰长是20厘米,底长是32厘米,求它的面
    积。
    \item 在$\triangle ABC$中关系式$\overline{AC}^2+\overline{BC}^2=\overline{AB}^2$是$\angle C=90^{\circ}$的
    什么条件?
\end{enumerate}
\end{ex}

\subsection*{习题3.4}
\begin{enumerate}
    \item 把梯形分割成直角三角形,矩形。由直角三角形、矩形
    的面积公式推导出梯形的面积公式。
    \item 如果过一个四边形的各顶点分别引直线平行于它的对角
    线,则由这些直线所组成的平行四边形的面积是已知四
    边形面积的2倍。
    \item 在直角三角形的各边上,分别向外侧作一个正方形,连
    结每相邻的两个正方形的最近的两个顶点,得三个三角
    形等积。
    \item 已知:$a$、$b$、$c$分别为直角三角形勾、股、弦长,
    \begin{enumerate}
        \item $a=3$, $b=4$, 求$c$.
        \item $c=25$, $b=14$, 求$a$.
     \item $a=13$, $c=19$, 求$b$.
    \end{enumerate}

    \begin{figure}[htp]
    \centering
\begin{tikzpicture}[scale=2]
\tkzDefPoints{-1/0/B_1, 0/0/A, -1/1/B_2}
\tkzDefPoint(180-80.26:1.732){B_3}
\tkzDefPoint(180-110.26:2){B_4}
\tkzDefPoint(180-136.82:2.236){B_5}
\tkzDrawSegments(A,B_1 A,B_2 A,B_3 A,B_4 A,B_5)

\tkzLabelPoints[above](B_3,B_4,B_5)
\tkzLabelPoints[left](B_1,B_2)
\tkzLabelPoints[right](A)

\draw(B_1)--node[left]{1}(B_2);
\draw(B_2)--node[above]{1}(B_3);
\draw(B_3)--node[above]{1}(B_4);
\draw(B_4)--node[above]{1}(B_5);

\end{tikzpicture}    
    \caption*{第5题}
\end{figure}

\item 象图中那样作出一系列的
直角三角形,$\overline{AB_2}$、$\overline{AB_3}$、
$\overline{AB_4}$、$\overline{AB_5}$分别是$\overline{AB_1}$的
多少倍?
\item 设$P$是矩形$ABCD$内的一点,则
$\overline{PA}^2+\overline{PC}^2=\overline{PB}^2+\overline{PD}^2$.

\item 设$P$是等腰直角三角形$ABC$的斜边$\overline{BC}$上的任意一点,
求证$\overline{BP}^2+\overline{CP}^2=2\overline{AP}^2$
\item 三角形的三边长分别是2cm、3cm、4cm, 求三条中线的
长。
\item 一矩形的两边分别是16米和12米,在它的对角线上作一
个三角形与这个矩形等积。并求出所作三角形的高(按
1:400画图)。
\item 一菱形两条对角线之和是24尺,它们的比是3:5, 求菱形
的面积。
\item 菱形的周长为16cm, 其中一个内角为$150^{\circ}$, 求它的面
积。
\item 菱形的周长为20cm, 面积是24cm, 求它的对角线的
长。
\item 在等腰梯形$ABCD$中,$AB\parallel CD$, $\angle A=60^{\circ}$, $BD$平分$\angle B$, 且梯形周长为20cm, 求梯形的面积。
\end{enumerate}

\section{相似形}
\subsection{成比例的线段}

\begin{blk}
  {定义} 用同样的长度单位去量两条已知线段,所得量数
的比叫做两条线段的比。它是一个正实数。  
\end{blk}

例如,$\overline{AB}=4$厘米,$\overline{CD}=2$厘米,
则 $\overline{AB}:\overline{CD}=4:2=2:1$.

上式也可写成:
\[\frac{\overline{AB}}{\overline{CD}}=\frac{4}{2}=\frac{2}{1}\]
$\overline{AB}$叫做比的\textbf{前项},$\overline{CD}$叫做比的\textbf{后项}。

根据分数的基本性质可知,两条线段的比和选择的长度
单位无关。


\begin{blk}
    {定义}
四条线段$a$、$b$、$c$、$d$, 如果满足等式$a:b=c:d$
(或$\frac{a}{b}=\frac{c}{d}$),
就说这\textbf{四条线段成比例}。$a$、$b$、$c$、$d$分别叫做
比例第一、二、三、四项。第一和第四两项叫做\textbf{比例外项},
第二第三两项叫做\textbf{比例内项}。
\end{blk}

\begin{blk}
    {定义}
三条线段$a$、$b$、$c$, 如果满足等式$a:b=b:c$
(或$\frac{a}{b}=\frac{b}{c}$),$b$就叫做$a$、$c$的\textbf{比例中项},$c$叫做$a$、$b$的
\textbf{比例第三项}。
\end{blk}

成比例的量有下面几条重要性质(下面所有字母都代表
不等于0的数)。

\begin{blk}{性质1:比例的基本性质}
\[\frac{a}{b}=\frac{c}{d}\quad \Rightarrow\quad ad=bc\]    
\end{blk}

\begin{blk}{性质3:反比定理}
    \[\frac{a}{b}=\frac{c}{d}\quad \Rightarrow\quad \frac{b}{a}=\frac{d}{c}\]       
\end{blk}

\begin{blk}{性质3:更比定理}
    \[\frac{a}{b}=\frac{c}{d}\quad \Rightarrow\quad \frac{a}{c}=\frac{b}{d}\] 
\end{blk}

\begin{blk}{性质4:合比定理}
    \[\frac{a}{b}=\frac{c}{d}\quad \Rightarrow\quad \frac{a+b}{b}=\frac{c+d}{d}\] 
\end{blk}

\begin{blk}{性质5:分比定理}
    \[\frac{a}{b}=\frac{c}{d}\quad \Rightarrow\quad \frac{a-b}{b}=\frac{c-d}{d}\] 
\end{blk}

\begin{blk}{性质6:等比定理}
    \[\frac{a}{b}=\frac{c}{d}=\frac{e}{f}=\cdots=\frac{r}{s}\quad \Rightarrow\quad \frac{a+c+e+\cdots+r}{b+d+f+\cdots+s}=\frac{a}{b}=\frac{c}{d}=\cdots=\frac{r}{s}\] 
其中:$b+d+f+\cdots+s\ne 0$
\end{blk}

我们只证明1和6,其余定理作为练习,同学
们自己证明。

\begin{proof}
证明性质1:
\begin{enumerate}
    \item 先证$\frac{a}{b}=\frac{c}{d}$是$ad=bc$的充分条件。

    因为已知$\frac{a}{b}=\frac{c}{d}$,所以两边同乘以$bd$, 即得$ad=bc$.
\item     再证$\frac{a}{b}=\frac{c}{d}$是$ad=bc$的必要条件。

因为已知$ad=bc$, 所以两边同除以$bd$, 即得$\frac{a}{b}=\frac{c}{d}$。
\end{enumerate}

由(a)、(b)所以有:
\[\frac{a}{b}=\frac{c}{d}\quad \Rightarrow\quad ad=bc\]    

证明性质6:

设$\frac{a}{b}=\frac{c}{d}=\frac{e}{f}=\cdots=\frac{r}{s}=k$,则
\[a=bk, c=dk, e=fk, \ldots, r=sk\] 
将上面诸式相加得:
\[a+c+e+\cdots+r=k(b+d+f+\cdots+s)\]
因为$b+d+f+\cdots+s\ne 0$,所以
\[\begin{split}
    \frac{a+c+e+\cdots+r}{b+d+f+\cdots+s}&=k\\
    &=\frac{a}{b}=\frac{c}{d}=\frac{e}{f}\\
    &=\cdots=\frac{r}{s}
\end{split}
    \]
\end{proof}


\begin{example}
    已知:在图3.90中,
$\frac{\overline{AE}}{\overline{AD}}=\frac{\overline{BE}}{\overline{BC}}$

求证:$\frac{\overline{AD}}{\overline{AE}}=\frac{\overline{BC}}{\overline{BE}}$; $\frac{\overline{AE}}{\overline{ED}}=\frac{\overline{BE}}{\overline{EC}}$; $\frac{\overline{AE}}{\overline{BE}}=\frac{\overline{ED}}{\overline{EC}}$
\end{example}

\begin{figure}[htp]
    \centering
\begin{tikzpicture}
\tkzDefPoints{0/0/C, 4/0/D, 2/2.5/A, 4.5/2.5/B}
\tkzInterLL(A,D)(B,C)  \tkzGetPoint{E}
\tkzDrawPolygon(A,B,E)
\tkzDrawPolygon(C,D,E)
\tkzLabelPoints[below](C,D)
\tkzLabelPoints[above](A,B)
\tkzLabelPoints[right](E)
\end{tikzpicture}
    \caption{}
\end{figure}

\begin{proof}
\begin{enumerate}
    \item $\because\quad \frac{\overline{AE}}{\overline{AD}}=\frac{\overline{BE}}{\overline{BC}}$
    
    $\therefore\quad \frac{\overline{AD}}{\overline{AE}}=\frac{\overline{BC}}{\overline{BE}}$ (反比定理)

    \item $\because\quad \frac{\overline{AD}}{\overline{AE}}=\frac{\overline{BC}}{\overline{BE}}$
    
    $\therefore\quad \frac{\overline{AD}-\overline{AE}}{\overline{AE}}=\frac{\overline{BC}-\overline{BE}}{\overline{BE}}$(分比定理)

    即:$\frac{\overline{ED}}{\overline{AE}}=\frac{\overline{EC}}{\overline{BE}}$

    $\therefore\quad \frac{\overline{AE}}{\overline{ED}}=\frac{\overline{BE}}{\overline{EC}}$ (反比定理)

    \item $\because\quad \frac{\overline{AE}}{\overline{AD}}=\frac{\overline{BE}}{\overline{BC}}$
    
$\therefore\quad \frac{\overline{AE}}{\overline{BE}}=\frac{\overline{ED}}{\overline{EC}}$(更比定理)
\end{enumerate}
\end{proof}



\begin{ex}
\begin{enumerate}
    \item 证明反比、更比、合比、分比定理。
    \item 求下列各数的比例中项。
\begin{multicols}{3}
    \begin{enumerate}
        \item 3、27;
        \item 5、5;
        \item 2、32.
    \end{enumerate}
\end{multicols}
\item 已知$ab=cd$, 试写出以$b$为首项的比例式,这样的比例式
    能有几个?
    \item 已知:$\frac{\overline{AB}}{\overline{AD}}=\frac{\overline{AC}}{\overline{AE}}$

    求证:$\frac{\overline{AD}}{\overline{AB}}=\frac{\overline{AE}}{\overline{AC}}$; $\frac{\overline{AD}}{\overline{DB}}=\frac{\overline{AE}}{\overline{EC}}$; $\frac{\overline{AD}}{\overline{AE}}=\frac{\overline{DB}}{\overline{EC}}$
\end{enumerate}
\end{ex}

\begin{figure}[htp]
    \centering
\begin{tikzpicture}
\tkzDefPoints{0/0/B, 4/0/C, 2.45/3/A}
\tkzDefPointWith[linear, K=.6](A,B)  \tkzGetPoint{D}
\tkzDefPointWith[linear, K=.6](A,C)  \tkzGetPoint{E}
\tkzDrawPolygon(A,B,C)
\tkzDrawSegments(D,E)

\tkzLabelPoints[below](B,C)
\tkzLabelPoints[above](A)
\tkzLabelPoints[left](D)
\tkzLabelPoints[right](E)
\end{tikzpicture}
    \caption*{第4题}
\end{figure}


\subsection{相似和相似比}
在前面,我们对比例线段有了明确的理解,现在我们进
一步研究相似图形的性质。

例如,图3.91所示的用不同比例尺绘成的同一机械零
件图就是相似的图形。

\begin{blk}
    {定义} 如果两个多边形的角对应相等,对应边成比例。
这两个多边形叫做\textbf{相似多边形}。
\end{blk}




\begin{example}
    
\end{example}


\begin{solution}
    
\end{solution}


\begin{blk}
    
\end{blk}

\begin{blk}
    
\end{blk}



\begin{blk}
    
\end{blk}

\begin{blk}
    
\end{blk}

\begin{blk}
    
\end{blk}

\begin{blk}
    
\end{blk}




\begin{example}
    
\end{example}

\begin{solution}
    
\end{solution}


\begin{solution}
    
\end{solution}
\begin{example}
    
\end{example}


\begin{solution}
    
\end{solution}
\begin{example}
    
\end{example}



\begin{solution}
    
\end{solution}

\begin{solution}
    
\end{solution}
\begin{example}
    
\end{example}


\begin{blk}
    
\end{blk}
\begin{proof}
    
\end{proof}

\begin{example}
    
\end{example}

\begin{example}
    
\end{example}

\begin{example}
    
\end{example}



















\begin{example}
    
\end{example}

\begin{example}
    
\end{example}

\begin{example}
    
\end{example}

\begin{example}
    
\end{example}

\begin{example}
    
\end{example}

\begin{example}
    
\end{example}

\begin{example}
    
\end{example}

\begin{example}
    
\end{example}

\begin{example}
    
\end{example}

\begin{example}
    
\end{example}

\begin{example}
    
\end{example}

\begin{example}
    
\end{example}

\begin{example}
    
\end{example}

\begin{example}
    
\end{example}

\begin{example}
    
\end{example}

\begin{example}
    
\end{example}




\begin{solution}
    
\end{solution}

\begin{solution}
    
\end{solution}

\begin{solution}
    
\end{solution}

\begin{solution}
    
\end{solution}

\begin{solution}
    
\end{solution}

\begin{solution}
    
\end{solution}

\begin{solution}
    
\end{solution}

\begin{solution}
    
\end{solution}

\begin{solution}
    
\end{solution}

\begin{solution}
    
\end{solution}

\begin{solution}
    
\end{solution}

\begin{solution}
    
\end{solution}

\begin{solution}
    
\end{solution}

\begin{solution}
    
\end{solution}

\begin{solution}
    
\end{solution}

\begin{solution}
    
\end{solution}

\begin{solution}
    
\end{solution}

\begin{solution}
    
\end{solution}

\begin{solution}
    
\end{solution}

\begin{solution}
    
\end{solution}

\begin{solution}
    
\end{solution}

\begin{solution}
    
\end{solution}

\begin{solution}
    
\end{solution}

\begin{solution}
    
\end{solution}

\begin{solution}
    
\end{solution}

\begin{solution}
    
\end{solution}

\begin{solution}
    
\end{solution}

\begin{solution}
    
\end{solution}










































































\subsection*{实习作业}
\begin{enumerate}
    \item 在田野里选择有障碍物的两点(如两棵树),用这节介绍
    的方法测量它们之间的距离。
    \item 选择一块多边形空地,测绘它的平面图并求出它的面积。
\end{enumerate}

\subsection*{习题3.5}
\begin{enumerate}
    \item 在$\triangle ABC$中,已知$\overline{BC}=36$cm, 高$\overline{AD}=30$cm, 在距
    离$\overline{BC}$边10cm的地方作一条平行于$BC$的直线,交$\overline{AB}$
    于$E$, 交$\overline{AC}$于$F$, 求$\overline{EF}$的长。
    \item 已知梯形两底的长为36cm和60cm, 高为32cm, 求这个梯
    形两腰延长线的交点到两底的距离各是多少?
    \item 两个相似三角形对应边的比是$7:5$, 第一个三角形的周长
    是14cm, 求另一个三角形的周长。
    \item 已知$\triangle ABC\backsim \triangle A'B'C$, $\triangle A'B'C'$的面积是$\triangle ABC$
    面积的4倍,并且$\overline{AB}=4$cm, $\overline{BC}=7$cm, $\overline{AC}=8$cm, 求
    $\triangle A'B'C'$各边的长。
    \item 已知梯形$ABCD$的面积为90平方厘米,两底$\overline{AB}$、$\overline{CD}$的长各
    为12厘米和8厘米,延长梯形两腰相交于$M$点,求$\triangle MDC$
    的面积。
    \item 设$AD$是$\triangle ABC$的中线,一条与
    $\overline{BC}$边平行的直线与
    $\overline{AB}$、$\overline{AD}$、$\overline{AC}$的交点分别为$P$、$Q$、$R$, 那么$\overline{PQ}=\overline{QR}$
    \item 过梯形$ABCD$的对角线$AC$、$BD$的交点$E$, 作两底的平
    行线,与两腰$\overline{AB}$、$\overline{CD}$分别相交于$M$、$N$, 求证:
    $\overline{ME}=\overline{EN}$.
    \item 在梯形$ABCD$中,作与两底互相平行的直线,交两腰
    $\overline{AB}$、$\overline{CD}$于$P$、$Q$两点,与对角线的交点是$R$、$S$, 那么
    $\overline{PR}=\overline{SQ}$
\end{enumerate}

\begin{figure}[htp]\centering
    \begin{minipage}[t]{0.48\textwidth}
    \centering
\begin{tikzpicture}[>=latex, scale=1]
    \tkzDefPoints{-2/0/B, 2/0/C, 1/2.5/D, -1/2.5/A}
    \tkzInterLL(A,C)(B,D)  \tkzGetPoint{E}
\tkzDrawSegments(A,C B,D)
\tkzDrawPolygon(A,B,C,D)

\tkzDefPointWith[linear, K=.3333](A,B)  \tkzGetPoint{M}
\tkzDefPointWith[linear, K=.3333](D,C)  \tkzGetPoint{N}
\tkzLabelPoints[left](A,B,M)
\tkzLabelPoints[right](C,D,N)
\tkzLabelPoints[below](E)
\tkzDrawSegments(M,N)
    \end{tikzpicture}
    \caption*{第7题}
    \end{minipage}
    \begin{minipage}[t]{0.48\textwidth}
    \centering
    \begin{tikzpicture}[>=latex, scale=1]
        \tkzDefPoints{-2/0/B, 2/0/C, 1/2.5/D, -1/2.5/A}
    \tkzDrawSegments(A,C B,D)
    \tkzDrawPolygon(A,B,C,D)

\tkzDefPointWith[linear, K=.7](A,B)  \tkzGetPoint{P}
\tkzDefPointWith[linear, K=.7](D,C)  \tkzGetPoint{Q}
\tkzInterLL(A,C)(P,Q)  \tkzGetPoint{S}
\tkzInterLL(P,Q)(B,D)  \tkzGetPoint{R}
\tkzLabelPoints[below](R,S)
\tkzLabelPoints[left](A,B,P)
\tkzLabelPoints[right](C,D,Q)
\tkzDrawSegments(P,Q)
    \end{tikzpicture}
    \caption*{第8题}
    \end{minipage}
    \end{figure}

\section*{复习题三}
\begin{enumerate}
    \item 已知不共线三点,能不能作一个等边三角形,使这三个
    点分别在三边上?
    \item 求证:三角形的三条角平分线相交于一点。
    \item 求证:等腰三角形底边上任意一点到两腰的距离的和一
    定。它等于什么?如果在底边的延长线上取点,你能得出
    合什么结论?
    \item $O$点是等边三角形内任一点,试证明:$O$点到三边距离的
    和与$O$点的位置无关,这个和等于什么?
    \item 已知四边形,求这样一点,使这点到各顶点的距离的和
    为最小。
    \item 已知梯形$ABCD$中,$\overline{AD}\parallel\overline{BC}$, $\angle B=90^{\circ}$, 
    $\angle C=60^{\circ}$, 腰$\overline{CD}=42$cm, 求另一腰$\overline{AB}$等于多少厘米?
    \item 从平行四边形$ABCD$的各顶点到形外一条直线作垂线
    $AE$、$BF$, $CG$、$DH$, 设$E$、$F$、$G$、$H$为垂足。求证:
    $\overline{AE}+\overline{CG}=\overline{BF}+\overline{DH}$.
    \item 如果两正方形对角线相等,那么这两个正方形全等。
\item 已知一边和一条对角线,求作矩形。
\item 已知高,求作等边三角形。
\item 如图,$DE\parallel BC$, $BE$、$CD$相交于$F$, 射线$AF$和$\overline{BC}$
相交于$G$. 求证:$\overline{BG}=\overline{CG}$.
\item 如图,已知:$\triangle ABC$中$DE\parallel CA$, $DF\parallel BA$.

求证:$\frac{\overline{DE}}{\overline{AC}}+\frac{\overline{DF}}{\overline{AB}}=1$

\begin{figure}[htp]\centering
    \begin{minipage}[t]{0.48\textwidth}
    \centering
\begin{tikzpicture}[>=latex, scale=1]
\tkzDefPoints{0/0/B, 4/0/C, 3/3.5/A}
\tkzDefPointWith[linear, K=.6](A,B)  \tkzGetPoint{D}
\tkzDefPointWith[linear, K=.6](A,C)  \tkzGetPoint{E}
\tkzDrawPolygon(A,B,C)
\tkzDrawSegments(D,E C,D B,E)
\tkzInterLL(C,D)(B,E)  \tkzGetPoint{F}
\tkzInterLL(A,F)(B,C)  \tkzGetPoint{G}
\tkzDrawSegments(A,G)
\tkzLabelPoints[left](D,B)
\tkzLabelPoints[right](E,C,F)
\tkzLabelPoints[below](G)
\tkzLabelPoints[above](A)
    \end{tikzpicture}
    \caption*{第11题}
    \end{minipage}
    \begin{minipage}[t]{0.48\textwidth}
    \centering
    \begin{tikzpicture}[>=latex, scale=1]
        \tkzDefPoints{0/0/B, 4/0/C, 1.5/3/A}
        \tkzDrawPolygon(A,B,C)
        \tkzDefPointWith[linear, K=.6](A,B)  \tkzGetPoint{E}
        \tkzDefPointWith[linear, K=.6](C,A)  \tkzGetPoint{F}
        \tkzDefPointWith[linear, K=.6](C,B)  \tkzGetPoint{D}
\tkzDrawSegments(D,E D,F)
\tkzLabelPoints[below](B,C,D)
\tkzLabelPoints[left](E)
\tkzLabelPoints[right](F)
\tkzLabelPoints[above](A)
    \end{tikzpicture}
    \caption*{第12题}
    \end{minipage}
    \end{figure}

\item 已知:$D$、$D'$是$\triangle ABC$的$\overline{BC}$边上两点,$\overline{BD}=\overline{D'C}$, 
$\overline{DE}$, $\overline{D'E'}$和$\overline{AC}$平行,并交$\overline{AB}$于$E,E'$, $\overline{DF}$, $\overline{D'F'}$ 和$AB$
平行交$\overline{AC}$于$F$、$F'$, 求证:$E'F\parallel EF'$.
\item 已知:$E$是正方形$ABCD$的$\overline{AB}$边上的一点,
$\overline{AE}=\frac{1}{n}\overline{AB}$, $\overline{DE}$和$\overline{AC}$相交于$F$. 求证;$\overline{AF}=\frac{1}{n+1}\overline{AC}$.
\item 如图,已知
$\overline{EF}\parallel \overline{BC}$, $\overline{FD}\parallel\overline{AB}$, $\overline{AE}=6.4$厘米,
$\overline{BE}=7.2$厘米,$\overline{CD}=9$厘米,
求四边形$BDFE$的周长。
\item 已知$\triangle ABC$的三边是
$\overline{AB}=11$cm, $\overline{BC}=6$cm,
$\overline{AC}=7$cm, $\overline{AD}$, $\overline{AD'}$分别是$\angle A$和它的外角平分线,求
$\overline{DD'}$的长。
\item 已知$P$是$\angle A$内的一个定点,求经过$P$作一条直线,分
别交$\angle A$的两边于$B$、$C$, 使$\overline{AB}:\overline{AC}=3:2$.
\item $OC$是$\angle AOB$内的一条射线,求证从$OC$上任意两 点到
$\angle AOB$的两边的距离的比一定。

\begin{figure}[htp]\centering
    \begin{minipage}[t]{0.48\textwidth}
    \centering
\begin{tikzpicture}[>=latex, scale=1]
\tkzDefPoints{0/0/B, 4/0/C, 2.3/2/A}
\tkzDefPointWith[linear, K=.4](A,B) \tkzGetPoint{E}
\tkzDefPointWith[linear, K=.4](A,C) \tkzGetPoint{F}
\tkzDefPointWith[linear, K=.6](C,B) \tkzGetPoint{D}
\tkzDrawPolygon(A,B,C)
\tkzDrawSegments(E,F D,F)
\tkzLabelPoints[below](B,C,D)
\tkzLabelPoints[left](E)
\tkzLabelPoints[right](F)
\tkzLabelPoints[above](A)
    \end{tikzpicture}
    \caption*{第15题}
    \end{minipage}
    \begin{minipage}[t]{0.48\textwidth}
    \centering
    \begin{tikzpicture}[>=latex, scale=1]
\tkzDefPoints{0/0/B, 4/0/C, 3/3/A}
\tkzDefPointWith[linear, K=.45](A,B) \tkzGetPoint{D}
\tkzDefPointWith[linear, K=.55](A,C) \tkzGetPoint{F}
\tkzDefPointWith[linear, K=.45](C,B) \tkzGetPoint{E}
\tkzDrawPolygon(A,B,C)
\tkzDrawSegments(E,F D,E)
\tkzLabelPoints[below](B,C,E)
\tkzLabelPoints[left](D)
\tkzLabelPoints[right](F)
\tkzLabelPoints[above](A)
    \end{tikzpicture}
    \caption*{第19题}
    \end{minipage}
    \end{figure}

\item 如图,$D$、$E$、$F$分别是$\triangle ABC$的边$\overline{AB}$、$\overline{BC}$、$\overline{CA}$
上的点,$ADEF$
是菱形,$\overline{AB}=14$cm, $\overline{BC}=12$cm, $\overline{AC}=10$cm, 求$\overline{BE}$和$\overline{EC}$
的长。

\item 已知:$E$是平行四边形
$ABCD$的边$\overline{DA}$延长线上的一点,$\overline{EC}$交$\overline{AB}$于$G$, 交对角线$\overline{DB}$于$F$.

求证:$\overline{FC}^2=\overline{FG}\cdot \overline{FE}$
\item 设$\overline{AB}$的中点为$M$, 从$\overline{AB}$上另一点$C$向直线$AB$的一
侧引线段$\overline{CD}$. 令$\overline{CD}$的中点为$N$, $\overline{BD}$的中点为$P$, $\overline{MN}$
的中点为$Q$, 求证:直线$PQ$平分$\overline{AC}$.
\item 已知梯形的面积是$Q$, 上下底的比是$\frac{m}{n}$,
求两腰的延长线和上底所组成的三角形的面积。
\item 在直角三角形$ABC$中,$\overline{AD}$是斜边$\overline{BC}$上的高,作
$DE\bot AB$, $DF\bot AC$, $E,F$是垂足.

求证:$\frac{\overline{BE}}{\overline{CF}}=\frac{\overline{AC}^3}{\overline{AB}^3}$

\item 如图,已知$\overline{OM}:\overline{MP}=\overline{ON}:\overline{NR}$, 
求证:$\triangle PQR$是等腰三角形。

\begin{figure}[htp]\centering
    \begin{minipage}[t]{0.48\textwidth}
    \centering
\begin{tikzpicture}[>=latex, scale=.8]
\tkzDefPoints{0/0/O, 3/0/R, 4/3/N}
\tkzDefPointWith[linear, K=.4](O,N) \tkzGetPoint{M}
\tkzDefPointWith[linear, K=1.6](N,R) \tkzGetPoint{Q}
\tkzInterLL(O,R)(M,Q)  \tkzGetPoint{P}
\tkzLabelPoints[right](N,R,Q)
\tkzLabelPoints[left](O,M)
\tkzLabelPoints[below](P)
\tkzDrawPolygon(O,N,R)
\tkzDrawPolygon(N,Q,M)
    \end{tikzpicture}
    \caption*{第24题}
    \end{minipage}
    \begin{minipage}[t]{0.48\textwidth}
    \centering
    \begin{tikzpicture}[>=latex, scale=1]
\tkzDefPoints{0/0/B, 3/0/C, 2.4/3.5/A, 5/0/X, 4/.5/X1}
\tkzInterLL(X,X1)(A,C)  \tkzGetPoint{Y}
\tkzInterLL(X,X1)(A,B)  \tkzGetPoint{Z}
\tkzDrawPolygon(A,B,C)
\tkzDrawPolygon(X,B,Z)
\tkzLabelPoints[right](Y)
\tkzLabelPoints[left](A,Z)
\tkzLabelPoints[below](B,C,X)
    \end{tikzpicture}
    \caption*{第25题}
    \end{minipage}
    \end{figure}

\item $\triangle ABC$被一条直线所截,设与三边的交点为$X$、$Y$、$Z$, 

试证:$\frac{\overline{BX}}{\overline{XC}}\cdot \frac{\overline{CY}}{\overline{YA}}\cdot \frac{\overline{AZ}}{\overline{ZB}}=1$

\item 设$O$是$\triangle ABC$内任一点,射线$AO$、$BO$、$CO$与对边
的交点分别是$X$、$Y$、$Z$, 那么
$\frac{\overline{BX}}{\overline{XC}}\cdot \frac{\overline{CY}}{\overline{YA}}\cdot \frac{\overline{AZ}}{\overline{ZB}}=1$


\begin{figure}[htp]\centering
    \begin{minipage}[t]{0.48\textwidth}
    \centering
\begin{tikzpicture}[>=latex, scale=1]
    \tkzDefPoints{0/0/B, 4/0/C, 2.9/3.2/A, 2.5/1.5/O}
\tkzInterLL(A,O)(B,C)  \tkzGetPoint{X}
\tkzInterLL(B,O)(A,C)  \tkzGetPoint{Y}
\tkzInterLL(C,O)(B,A)  \tkzGetPoint{Z}
\tkzDrawPolygon(A,B,C)
\tkzDrawSegments(A,X B,Y C,Z)
\tkzLabelPoints[right](Y)
\tkzLabelPoints[left](A,Z)
\tkzLabelPoints[below](B,C,X)
\tkzLabelPoints[below left](O)
    \end{tikzpicture}
    \caption*{第26题}
    \end{minipage}
    \begin{minipage}[t]{0.48\textwidth}
    \centering
    \begin{tikzpicture}[>=latex, scale=1]
\tkzDefPoints{0/0/B, 2.6/0/C, 2.4/3.5/A, 5/0/D, 4/.25/X1}
\tkzInterLL(D,X1)(A,C)  \tkzGetPoint{E}
\tkzInterLL(D,X1)(A,B)  \tkzGetPoint{F}
\tkzDrawPolygon(A,B,C)
\tkzDrawPolygon(D,B,F)
\tkzLabelPoints[right](E)
\tkzLabelPoints[left](A,F)
\tkzLabelPoints[below](B,C,D)      
    \end{tikzpicture}
    \caption*{第27题}
    \end{minipage}
    \end{figure}

\item 从$\triangle ABC$的一边$\overline{BC}$的延长线上的一点$D$, 引直线与
其它二边$\overline{AC}$、$\overline{AB}$相交于$E$、$F$, 若$\angle AEF=\angle AFE$, 则$\overline{BD}:\overline{CD}=\overline{BF}:\overline{CE}$.


\item 在$\triangle ABC$之边$\overline{AB}$、$\overline{AC}$上取$E$、$F$两点,使$\overline{EB}=2\overline{EA}$, $\overline{AF}=2\overline{FC}$, 延长$\overline{EF}$、$\overline{BC}$交于$H$, 求$\overline{BH}:\overline{CH}$.

\begin{figure}[htp]\centering
    \begin{minipage}[t]{0.48\textwidth}
    \centering
\begin{tikzpicture}[>=latex, scale=1]
    \tkzDefPoints{0/0/B, 4/0/C, 3/3/A}
\tkzDefPointWith[linear, K=0.333](A,B) \tkzGetPoint{E}
\tkzDefPointWith[linear, K=0.333](C,A) \tkzGetPoint{F}
\tkzInterLL(B,C)(E,F)  \tkzGetPoint{H}
\tkzDrawPolygon(A,B,C)
\tkzDrawSegments(E,H C,H)
\tkzLabelPoints[right](F)
\tkzLabelPoints[left](A,E)
\tkzLabelPoints[below](B,C,H)  
    \end{tikzpicture}
    \caption*{第28题}
    \end{minipage}
    \begin{minipage}[t]{0.48\textwidth}
    \centering
    \begin{tikzpicture}[>=latex, scale=1]
        \tkzDefPoints{0/0/A, 2/0/B, 4/0/D, 2.8/3/C}
\tkzDrawPolygon(A,B,C)
\tkzDefPointWith[linear, K=.45](C,A) \tkzGetPoint{E}
\tkzInterLL(D,E)(B,C)  \tkzGetPoint{F}
\tkzLabelPoints[right](F)
\tkzLabelPoints[left](C,E)
\tkzLabelPoints[below](B,A,D)  
\tkzDrawSegments(E,D B,D)
    \end{tikzpicture}
    \caption*{第29题}
    \end{minipage}
    \end{figure}

\item 在$\triangle ABC$中,$\overline{AB}<\overline{AC}$, 延长$\overline{AB}$至$D$, 使$\overline{BD}=\overline{AB}$, 在$\overline{AC}$上取$E$点,使$\overline{CE}=\overline{BD}$, 连接$\overline{DE}$与$\overline{BC}$交于
$F$, 则$\overline{AB}:\overline{AC}=\overline{EF}:\overline{FD}$.

\item 延长$\triangle ABC$的$\overline{BC}$边至$D$, 使$\overline{CD}=\overline{BC}$, 从$D$点引直
线通过$\overline{AC}$的中点$E$, 与$\overline{AB}$边相交于$F$, 求$\overline{FE}:\overline{ED}$.
\item 从$\triangle ABC$的底边$\overline{BC}$中点
$D$引直线,与过顶点$A$平行
$BC$的直线相交于$G$, 与$\overline{BA}$
的延长线相交于$E$, 又与$AC$
相交于$F$, 求证:$\overline{DF}:\overline{FG}=\overline{DE}:\overline{EG}$.

\begin{figure}[htp]\centering
    \begin{minipage}[t]{0.48\textwidth}
    \centering
\begin{tikzpicture}[>=latex, scale=1]
    \tkzDefPoints{0/0/B, 2/0/C, 4/0/D, 2.2/3.5/A}
    \tkzDrawPolygon(A,B,C)
    \tkzDefMidPoint(A,C)  \tkzGetPoint{E}
    \tkzInterLL(D,E)(B,A)  \tkzGetPoint{F}
    \tkzLabelPoints[right](E)
    \tkzLabelPoints[left](A,F)
    \tkzLabelPoints[below](B,C,D)   
    \tkzDrawSegments(C,D D,F)
    
    \end{tikzpicture}
    \caption*{第30题}
    \end{minipage}
    \begin{minipage}[t]{0.48\textwidth}
    \centering
    \begin{tikzpicture}[>=latex, scale=.8]
    \tkzDefPoints{0/0/B, 2/0/D, 4/0/C, 1.6/2.5/A, 3.5/2.5/X1, 2.2/1/X2}
    \tkzDrawPolygon(A,B,C)
    \tkzInterLL(D,X2)(B,A)  \tkzGetPoint{E}
    \tkzInterLL(D,X2)(X1,A)  \tkzGetPoint{G}
    \tkzInterLL(D,X2)(C,A)  \tkzGetPoint{F}
    \tkzLabelPoints[right](F)
    \tkzLabelPoints[left](A,E)
    \tkzLabelPoints[above right](G)
    \tkzLabelPoints[below](B,C,D)   
    \tkzDrawSegments(A,X1 A,E E,D)
    \end{tikzpicture}
    \caption*{第31题}
    \end{minipage}
    \end{figure}



\end{enumerate}
% \chapter{圆}
在工农业生产和日常生活中,圆的应用相当广泛。过去
我们初步掌握了一些圆的知识,这一章我们将在复习这些知
识的基础上,把圆和直线形结合起来,进一步学习有关圆的
一些性质。

\section{圆的基本性质}
\subsection{圆的概念}
在一个平面上和某一定点的距离等于定长的点的集合叫
做\textbf{圆周},简称为\textbf{圆};其中定点叫做圆的\textbf{圆心},连结圆心与圆
上任一点的线段叫做\textbf{半径}.通常以点$O$为圆心的圆记作$\odot O$; 
以点$O$为圆心,半径长是$r$的圆记作$\odot(O,r)$。


显然,\textbf{同圆的半径都相等}(图4.1)。而当一个圆的圆
心确定了,半径$r$的大小也确定了,这个圆的位置与大小也
就完全确定了。

圆上任意两点间的部分叫做\textbf{弧};连结圆上任意两点间的
线段叫做这个圆的弦;通过圆心的弦叫做圆的\textbf{直径}(图4.2)。
显然,\textbf{一个圆的直径等于它的半径的二倍}。
\begin{figure}[htp]\centering
    \begin{minipage}[t]{0.48\textwidth}
    \centering
\begin{tikzpicture}[>=latex, scale=.8]
    \draw (0,0) circle (2);
\draw[very thick](0,0)node[left]{$O$}--node[above]{$r$}(30:2)node[right]{$P$};
\draw[very thick](0,0)--node[below]{$r$}(150:2)node[left]{$Q$};
\draw[very thick](0,0)--node[right]{$r$}(-80:2)node[below]{$R$};
    \end{tikzpicture}
    \caption{}
    \end{minipage}
    \begin{minipage}[t]{0.48\textwidth}
    \centering
    \begin{tikzpicture}[>=latex, scale=.8]
        \draw  (0,0) circle (2);
\draw[very thick] (60:2)--node[below]{弦}(135:2);
\draw[very thick] (-10:2)--node[below]{直径}(170:2);
\draw (-135:2) [fill=black]circle(1.5pt)node[left]{$E$};
\draw (-45:2) [fill=black]circle(1.5pt)node[right]{$F$};
\draw[very thick]  (-135:2) arc (-135:-45:2);
\node at (0,-2) [below=2pt]{弧};


    \end{tikzpicture}
    \caption{}
    \end{minipage}
    \end{figure}

从圆的定义,不难直接推知:
\begin{itemize}
    \item 两个圆能够重合的充要条件是两个圆的半径相等。
    \item 半径相等的圆叫做\textbf{等圆},\textbf{等圆的半径相等直径相等}。
\end{itemize}

从圆的定义,我们还可以看出,一个圆把它所在的平面
分为三部分(图4.3):
\begin{enumerate}
    \item 圆本身,即与圆心的距离等于半径的点所构成的集
合。其中任何一点都叫做圆上的点。
\item 圆的内部,与圆心的距离小于半径的点所构成的集
合。圆的内部又简称\textbf{圆内};其中任何一点都叫做圆内的点。
\item 圆的外部:与圆心的距离大于半径的点所构成的集
合;圆的外部又简称\textbf{圆外},其中任何一点都叫做圆外的点。
\end{enumerate}

\begin{figure}[htp]\centering
    \begin{minipage}[t]{0.48\textwidth}
    \centering
\begin{tikzpicture}[>=latex, scale=.8]
    \draw[pattern=north west lines] (0,0) circle (2);
    \draw[thick] (0,0)--node[left=3.5pt, fill=white]{$r$}(60:2);
    \end{tikzpicture}
    \caption{}
    \end{minipage}
    \begin{minipage}[t]{0.48\textwidth}
    \centering
    \begin{tikzpicture}[>=latex, scale=.8]
        \draw  (0,0) circle (2);
\draw[very thick](-.8,-.7)node[left]{$P_2$}--(0,0)node[below]{$O$}--(3,0)node[right]{$P_3$};
\draw[very thick](0,0)--(0,2)node[above]{$P_1$};

    \end{tikzpicture}
    \caption{}
    \end{minipage}
    \end{figure}

通常我们说的圆面,指的是由圆所围成的平面部分,也
就是与圆心的距离小于或等于半径的点所构成的集合。如图
4.3中阴影部分。

由上述定义可知,$\odot(O,r)$与平面上任一点P的位置关
系,有下述的性质(图4.4)。
\begin{enumerate}
    \item 点$P$在$\odot(O,r)$上的充要条件是$\overline{OP}=r$;
    \item 点$P$在$\odot(O,r)$内的充要条件是$\overline{OP}<r$;
    \item 点$P$在$\odot(O,r)$外的充要条件是$\overline{OP}>r$。
\end{enumerate}

\begin{ex}
\begin{enumerate}
    \item 根据下述条件画圆
\begin{enumerate}
\item 已知定点$O$, 以$O$为圆心画一圆使半径等于2厘
米。
\item 已知两个定点$O$、$P$, 画$\odot(O,\overline{OP})$.
\item 先画一条$\overline{AB}$, 再画出以$\overline{AB}$
为直径的圆。
\end{enumerate}

\item 把以下命题写成“若一则”形式
\begin{enumerate}
\item 点$P$在$\odot(O,r)$上的充分条件是
$\overline{OP}=r$;
\item 点$P$在$\odot(O,r)$内的必要条件是
$\overline{OP}<r$;
\item 点$P$在$\odot(O,r)$外的充分条件是
$\overline{OP}>r$。
\end{enumerate}


\item 以点$O$为圆心,$r_1$、$r_2$为半径画两个圆。说出满足下列条
件的点$X$在平面上的位置范围。
\begin{multicols}{2}
\begin{enumerate}
    \item $\overline{OX} >r_2$
    \item $\overline{OX} \le r_1$
    \item $r_1<\overline{OX}<r_2 $
    \item $\overline{OX}=r_1 $
    \item $\overline{OX}<r_1 $
\end{enumerate}
\end{multicols}

\item 已知一个$\odot O$的直径长是4cm. 说出满足下列条件的$P$点
的可能位置:
\begin{multicols}{2}
    \begin{enumerate}
        \item $\overline{OP} >2$cm
        \item $\overline{OP} \ge 2$cm
        \item $\overline{OP} <2$cm
        \item $\overline{OP} =0$
    \end{enumerate}
    \end{multicols}

\item 求证一个圆的直径
是这个圆中最长的弦。

(提示:按图中所示证$\overline{AB}>\overline{CD}$)。
\begin{center}
    \begin{tikzpicture}
        \draw (0,0) circle (1.5);
\draw[thick] (-150:1.5)node[left]{$A$}--node[above]{$O$}(30:1.5)node[right]{$B$};
\draw[thick]  (-135:1.5)node[left]{$C$}--(-45:1.5)node[right]{$D$};
\draw[dashed](-135:1.5)--(0,0)--(-45:1.5);
    \end{tikzpicture}
\end{center}
\end{enumerate}
\end{ex}

\subsection{不共线的三点确定一圆}
我们已知,如果知道了圆心的位置和半径长,那么圆的
位置和大小也就确定了。现在我们来研究经过一个点;经过
两个点;经过三个点可分别作出几个圆?
    
已知一个点$A$, 很明显,以$A$点以外的任何点为圆心,
以这点到$A$点的距离为半径所作的圆都经过$A$点(图4.5)。
因此,\textbf{经过一点可以作无数个圆}。

经过两个已知点$A$、$B$,可以作多少个圆呢(图4.6)?
由于经过$A$、$B$两点的圆的圆心到$A$点与$B$点的距离应相等,而
和$A$、$B$两点距离相等的点仅在$AB$的垂直平分线上,所以,
以$AB$的垂直平分线上任一点为圆心,以这点到$A$点(或$B$
点)的距离为半径所作的圆都经过$A$、$B$两点。因此,\textbf{经过
两点也可以作无数个圆,且圆心都在连结这两点的线段的垂
直平分线上}。

现在我们来研究,经过$A$、$B$、$C$三点可以作多少个圆的
问题?





























 % \chapter{轨迹与作图}
在前面的几章中,我们学习了直线形和圆的有关性质。
学习的途径主要是根据图形的定义和已知性质去推演图形的
其它性质。这一章,我们将把图形看成点的集合(点集),
研究如何根据点所具有的某种性质来求出点集在平面上的形
状和位置。

\section{轨迹}
\subsection{轨迹的概念}
我们知道,物体在运动中都要经过一定的路线。例如,
人在雪地里行走会留下明显的足迹,飞机飞行有一定的航
线,地球运行也有它的轨道等等。一般,我们常把物体按某
种规律运动的路线叫做物体运动的\textbf{轨迹}。在几何中,我们用
点表示物体在空间的位置。这样,一个点在空间按某种规律
运动的路线,我们就把它叫做这个点运动的轨迹,这个点就
叫做\textbf{动点}。例如,我们用圆规画圆时,圆规的一个脚尖固定
不动,而另一个脚上装上的铅笔尖端就可看作一个动点。它
和固定的脚尖保持一定的距离运动,所画出的图形就是这个
动点的轨迹。我们知道,圆是“同一平面上和某定点的距离
等于定长的点的集合”。由此可见,按某种规律运动的点的
轨迹,也就是具有某种性质的点的集合。

\begin{blk}{定义}
具有性质$\alpha$的所有点构成的集合,叫做具有性
质$\alpha$的点的轨迹。
\end{blk}

设$X=\{\text{具有性质$\alpha$的点}\}$。由上述定义,当我们要证明
某图形$A$是具有某种性质$\alpha$的点的轨迹时,也就是要证明
集合$A=X$. 要证明$A=X$, 就必须从以下两方面进行证
明:
\begin{enumerate}
\item $P$点$\in A\Rightarrow P$点具有性质$\alpha$ $(P\in X)$.
\item $P$点具有性质$\alpha$ $(P\in X)\Rightarrow P$点$\in A$.
\end{enumerate}

按上述两个方面证明,这是缺一不可的。如果我们只证
了第一条,实际上只是说明$A$是$X$的一个子集,并不能断
定$A=X$; 如果只证了第二条,也只是说$X$是$A$的一个子
集,同样不能断定$A=X$, 只有当我们证明了第一条:$A\subseteq
X$, 又证明了第二条:$X\subseteq A$, 我们才能断定$A=X$.

第一条证明了$A\subseteq X$, 这就是说在图形$A$上的点,都具
有性质$\alpha$. 没有一点是鱼目混珠的,通常把证这一条叫做证
\textbf{轨迹的纯粹性}。第二条证明了$X\subseteq A$, 这就是说,具有性质
$\alpha$的点都在图形$A$上,没有一点被遗漏掉。通常又把证这一
条叫做证\textbf{轨迹的完备性}。

由于原命题与逆否命题等价,所以也可以分别去证上述
两条的逆否命题,即要证轨迹的纯粹性也可证:
\[P\text{点不具有性质}\alpha\Rightarrow P\notin A\]
要证轨迹的完备性时,也可证:
\[P\text{点}\notin A\Rightarrow P\text{点不具有性质}\alpha\]

\begin{ex}
\begin{enumerate}
\item 叙
述两个集合相等的定义。
\item 在证轨迹命题时,为什么即要证轨迹的纯粹性,又要证
轨迹的完备性?
\item 如果我们证明了
$\overline{AB}$的垂直平分线上的任一点到$A$、
$B$两
点的距离相等,能否就说与$A$、$B$两点距离相等的点
的轨迹是$\overline{AB}$的垂直平分线?
\end{enumerate}
\end{ex}


\subsection{基本轨迹}
这一小节,我们来学习六个平面上的点的基本轨迹,我
们只证了1和4, 其它四个由同学们自证。

\begin{blk}{基本轨迹1}
与两个已知点距离相等的点的轨迹是连结
这两点的线段的垂直平分线。
\end{blk}

已知:两定点$A$、$B$, 直线$MN$是$\overline{AB}$的垂直平分线(图5.1)。

求证:与$A$、$B$两点距离相等的点的轨迹是直线$MN$.

\begin{figure}[htp]
    \centering
\begin{tikzpicture}[scale=.7]
\draw(-2,0)node[left]{$A$}--(2,0)node[right]{$B$};
\draw (0,3)node[above]{$M$}--(0,-2.5)node[below]{$N$};
\draw[dashed](-2,0)--(0,2.5)node[left]{$P$}--(2,0)--(0,-1.8)node[left]{$Q$}--(-2,0);
\node at (.25,.25){$O$};

\end{tikzpicture}
    \caption{}
\end{figure}


\begin{proof}
\begin{enumerate}
    \item 
设$P$是直线$MN$上的任一点,作$\overline{PA}$、$\overline{PB}$,
在$\triangle AOP$与$\triangle BOP$中,

$\because\quad \overline{AO}=\overline{BO},\quad \angle AOP=\angle BOP,\quad \overline{OP}=\overline{OP}$

$\therefore\quad \triangle AOP\cong \triangle BOP$ (SAS),
$\overline{PA}=\overline{PB}$.

这就说明了直线$MN$上的点,都与两点的距离相等。

\item 设$Q$为与$A$、$B$等距的点,即$\overline{QA}=\overline{QB}$. 过$AB$的
中点$O$与$Q$作直线$OQ$, 根据等腰三角形的性质,直线$OQ$
垂直平分$\overline{AB}$, 但$\overline{AB}$的垂直平分线只有一条,

$\therefore\quad MN$与$OQ$重合,$Q\in MN$.

\end{enumerate}

于是由1、2可知,与$A$、$B$两点距离相等的点的
轨迹是直线$MN$.
\end{proof}

\begin{blk}
{基本轨迹2} 与已知角的两边距离相等的点的轨迹是这
个已知角的平分线。
\end{blk}

\begin{blk}
{基本轨迹3} 与两条平行线等距离的点的轨迹是和这两
条平行线平行且平分它们的公垂线段的直线。
\end{blk}

\begin{blk}
    {基本轨迹4}与一条直线的距离等于定长的点的轨迹,是
    平行于这条直线,并和这条直线的距离等于定长的两条直线。
\end{blk}

已知:直线$CD\parallel$直线$AB$; 直线$EF\parallel$直线$AB$; $CD$、$EF$和$AB$之间的距离都是$d$(图5.2)。

求证:与$AB$的距离等于
$d$的点的轨迹是$CD$和$EF$.

\begin{figure}[htp]
    \centering
\begin{tikzpicture}
\draw (0,0)node[left]{$E$}--(4,0)node[right]{$F$};
\draw (0,1)node[left]{$A$}--(4,1)node[right]{$B$};
\draw (0,2)node[left]{$C$}--(4,2)node[right]{$D$};
\draw[dashed, thick] (1.5,-1)--(1.5,3);
\draw[thick] (3,1)node[below]{$L$}--(3,2)node[above]{$P$};
\node at (1.8,0)[below]{$N'$};
\node at (1.8,1)[above]{$L'$};
\node at (1.8,2)[above]{$M'$};
\draw (1.5,.6) [fill=black] circle (1.5pt) node[right]{$P'$};
\end{tikzpicture}
    
    \caption{}
\end{figure}


\begin{proof}
\begin{enumerate}
    \item 设$P$是$CD$或$EF$ 上的任一点。作$PL\bot AB$于$L$点。

$\because\quad CD\parallel AB$且和$AB$的距离等于$d$

$\therefore\quad \overline{PL}$是$AB$和$CD$的公垂线段,且$\overline{PL}=d$.

这就是说$CD$上的任一点和$AB$的距离都等于$d$, 同理
可证$EF$上的任一点和$AB$的距离也都等于$d$.
\item 设$P'$点是不在$CD$或$EF$上的任一点。
经过$P'$点作垂直于$AB$的直线,分别交$AB$、$CD$、$EF$于
$L'$、$M'$、$N'$, 则$\overline{M'L'}=\overline{N'L'}=d$.

$\because\quad P'$不在$CD$或$EF$上

$\therefore\quad P'$不和$M'$、$N'$重合

$\because\quad $在直线$M'N'$上和$L'$距离等于$d$的点只有$M'$、$N'$

$\therefore\quad \overline{P'L'}\ne d$

这就是说,不在$CD$和$EF$上的任何一点和$AB$的距离都
不等于$d$.
\end{enumerate}

于是由1、2可知,和$AB$的距离等于$d$的点的轨
迹是$CD$和$EF$。
\end{proof}

\begin{blk}
    {基本轨迹5} 与一个定点的距离等于定长的点的轨迹,
是以定点为圆心,定长为半径的一个圆。
\end{blk}

\begin{blk}
    {基本轨迹6}与一条定线段的两端连线所夹的角等于定角的点的轨迹,是以这条定线段为弦,所含的圆周角等于定
    角的两条弧。
\end{blk}

以上六个基本轨迹是研究其它轨迹问题的基础,同学们
一定要熟记。

\begin{example}
    求已知圆内等于定长的弦的中点的轨迹。

    已知$\odot (O,r)$和定长$a$, 且$a<2r$(图5.3)。

    求$\odot (O,r)$内等于定长$a$的弦的中点的轨迹。
\end{example}

\begin{figure}[htp]
    \centering
\begin{tikzpicture}[scale=1.6]
\draw[very thick] (0,0) circle (1.414)  ;  
\draw[very thick, dotted] (0,0) circle (1)  ;  
\draw (-45:1.414)node[right]{$B$}--(-45-90:1.414)node[left]{$A$};
\draw[dashed] (80:1.414)--(0,0)--node[left]{$r$}(-45-90:1.414);
\draw [thick] (0,0)--node[right]{$r'$}(0,-1)node[below]{$M$};
\draw (80:1.414)node[above]{$C$}--(-10:1.414)node[below]{$D$};
\draw [thick] (0,0)--(35:1)node[right]{$P$};

\draw[|-|](-4,1)--node[above]{$a$}(-2,1);
\end{tikzpicture}
    \caption{}
\end{figure}

\begin{solution}
    如图5.3, 设$\overline{AB}$是$\odot (O,r)$内等于定长$a$的弦,$M$
是它的中点,作$\overline{OM}$, 那么,$\overline{OM}\bot \overline{AB}$。
\[\overline{AM}=\frac{1}{2}\overline{AB}=\frac{a}{2}\]
所以
\[\overline{OM}=\sqrt{\overline{OA}^2-\overline{AM}^2}=\sqrt{r^2-\left(\frac{a}{2}\right)^2}\]
设$r'=\sqrt{r^2-\left(\frac{a}{2}\right)^2}$,则$r'$
为定长,以$O$为圆心,$r$为
半径画$\odot (O,r')$, 那么,$\odot (O,r)$内等于定长$a$的弦的中
点都在$\odot (O,r')$上。另外,在$\odot (O,r')$上任取一点$P$, 作
$\overline{OP}$, 再作弦$\overline{CD}\bot\overline{OP}$于$P$点,则$P$点是$\overline{CD}$弦的中点,且
\[\overline{CD}=2\overline{CP}=2\sqrt{r^2-{r'}^2}=2\sqrt{r^2-\left[r^2-\left(\frac{a}{2}\right)^2\right]}=2\cdot\frac{a}{2}=a\]
这就是说$\odot (O,r')$上的任一点都是$\odot (O,r')$内等于定长$a$
的一条弦的中点,所以我们所求的轨迹就是$\odot (O,r')$。
\end{solution}

\begin{example}
    过定圆外一定点引圆的割线,求割线被圆截下的弦的中点的轨迹。

    已知:定$\odot (O,r)$和$\odot O$外定点$P$(图5.4)。
    
    求:过$P$点引$\odot O$的割线,被$\odot O$截下的弦的中点
    的轨迹。
\end{example}


\begin{figure}[htp]
    \centering
\begin{tikzpicture}[scale=1.2]
\tkzDefPoints{0/0/O, 2.5/0/O', 5/0/P}
\draw[dashed] (P)--(-2,0)node[left]{$E'$};
\draw (0,0) circle (2);
\draw[dashed] (O') circle (2.5);
\tkzDefTangent[from with R=P](O, 2cm)
\tkzGetPoints{A}{B}
\tkzDrawSegments[add=0 and .2](P,A  P,B)
\node at (-.25,-.25){$O$};\node at (.25+2.5,.25){$O'$};
\node at (2.2,0)[above]{$E$};
\tkzDrawPoints(O',O)
\tkzDefShiftPoint[O'](150:2.5){M}
\tkzDefShiftPoint[O'](-142:2.5){N}
\tkzDefShiftPoint[O'](-155:2.5){X}
\tkzAutoLabelPoints[center=O'](M,N,X,A,B,P)
\tkzInterLC(P,M)(O,A) \tkzGetPoints{C'}{C}
\tkzInterLC(P,X)(O,A)\tkzGetPoints{D}{D'}
\tkzInterLC(P,N)(O,A)\tkzGetPoints{F}{F'}
\tkzAutoLabelPoints[center=O](D,F,C,C',D',F')
\draw(C')--(P)--(F');
\draw(D')--(P);
\draw[dashed](M)--(O)--(X);
\draw[dashed](A)--(O);
\end{tikzpicture}
    \caption{}
\end{figure}

\begin{solution}
    由于轨迹是具有某种性质$\alpha$的点的集合,求轨迹时,可先按照“性质”画出一些点,看看这些点可
能构成什么样的图形,如图
5.4, 过$P$点作$\odot O$的割线,与$\odot O$相交于$C$、$C'$, 作弦
$\overline{CC'}$的中点$M$, 我们证割线$PCC'$绕$P$点旋转,看这条变动
的割线被$\odot O$截下的弦的中点经过什么路线。大概可以看
出,可能是一段圆弧,究竟是不是圆弧,如果是圆弧,又如
何把它作出来,还要进一步分析。

$\because\quad M$是弦$\overline{CC'}$的中点,作$\overline{OM}$

则$\overline{OM}\bot\overline{CC'}$,即:$\angle OMP$是直角。

这就是说,过$P$点作$\odot O$的任一条 割线
被$\odot O$截下的弦的中点与$O$、$P$的连线的夹角等于直角,因
此,符合题中条件的弦的中点都在以$\overline{OP}$
为直径的圆上,以$\overline{OP}$
为直径作$\odot O'$, 我们所作的$\odot O'$是不是就是所求的轨迹呢?
这还要看$\odot O'$上有没有不符合条件的点。设$\odot O'$与$\odot O$相
交于$A$、$B$两点。显然,在$\odot O'$上$\wideparen{AOB}$外的点都是不合条
件的(包括$A$、$B$),我们再来看$\wideparen{AOB}$上的点是不是都是合
条件的点。在
$\wideparen{AOB}$上任取一点$X$, 设$PX$与$\odot O$相交于$D$、
$D'$, 作$\overline{OX}$, 则$\overline{OX}\bot \overline{DD'}$,所以$X$是$\overline{DD'}$的中点,这就
是说$\wideparen{AOB}$上的点都是过$P$点的某条割线被$\odot O$截下的弦的中
点。
\end{solution}

综合以上分析我们可得:过定圆外一定点引圆的割线,
割线被圆截下的弦的中点的轨迹是以定点与圆心间的线段为
直径的圆被夹在定圆内的一段弧。

\begin{ex}
    说出下列的点的轨迹是什么图形?并把它们分别画出
来。
\begin{enumerate}
    \item 到一条5cm长的线段的两端距离相等的点的轨迹。
    \item 通过两定点的圆的圆心的轨迹。
    \item 到一个等于60$^{\circ}$的已知角的两边距离相等的点的轨述:
    \item 与两条相交直线等距离的点的轨迹。
    \item 与$\angle AOB$的两边都相切的圆的圆心的轨迹。
    \item 与距离是3cm的两条平行直线$AB$、$CD$的距离相等的点
的轨迹。
\item 与距离是3cm的两条平行直线都相切的圆的圆心的轨
迹。
\item 和已知直线$AB$的距离等于2cm的点的轨迹。
\item 和已知直线$AB$相切,并且半径等于1.5cm的圆的圆心
的轨迹。
   \item 和一条已知直线切于已知点的圆的圆心的轨迹。
\item 与定点$A$的距离等于2cm的点的轨迹。
\item 和一条长是3cm的$AB$的两端连线所夹的角是直角的点
的轨迹。
\item 和一条长是4cm的已知线段$AB$的两端连线所夹的角等
于60$^{\circ}$的点的轨迹。

\end{enumerate}
\end{ex}


\section*{习题5.1}
\addcontentsline{toc}{subsection}{习题5.1}
\begin{enumerate}
    \item 求下列轨迹
\begin{enumerate}
    \item 以已知$\overline{AB}$为一边的三角形的外心的轨迹。
    \item 以已知$\overline{AB}$为一边,并且这边上的中线的长等于定长$m$的三角形的重心的轨迹。
\item 以3cm长的已知$\overline{AB}$为一边,并且面积等于6平方厘
米的三角形的顶点$C$的轨迹。
\item 和一个半径等于定长$r$的$\odot O$外切,并且半径等于
$r'$的圆的圆心的轨迹。
\item 以已知$\overline{BC}$为斜边的直角$\triangle ABC$的顶点$A$的轨迹。
\end{enumerate}

\item 叙述符合下列条件的点的轨迹。
\begin{enumerate}
    \item 平行于三角形的一边而夹在其余两边之间的线段的
中点的轨迹。
\item 平行于已知直线而在已知圆内的弦的中点的轨迹。
\end{enumerate}
\item 作出下列各题中给出的两个点集,向它们的交集各含有
几个元素。

\begin{enumerate}
    \item 和距离等于3cm的两条已知平行线$AB$、$CD$的距离
相等的点集;和直线$AB$上的一定点$E$的距离是2cm的点
集。
\item 和已知$\angle AOB$的两边的距离相等的点集;和边$OA$
的距离等于$d$的点集。
\item 和一条长3cm的已知$\overline{AB}$的两端连线所夹的角是直角
的点集;和$\overline{AB}$所在直线距离等于2cm的点集。
\end{enumerate}


\item 求通过$\odot (O,r)$内一定点$P$的弦的中点的轨迹。
\item 求到$\odot(O,3{\rm cm})$的圆面等于4cm的点的轨迹。
\end{enumerate}

\section{作图}
\subsection{基本作图}
在前几章中,我们曾用直尺、圆规解过不少作图题。在
这一节里,我们将进一步学习解作图题的一些重要方法,下
面列出我们已经学过的一些作图题(具体作法不再写出,由
同学自己复习、研究),这些作图题一般叫做\textbf{基本作图题},
它们是进一步解较复杂的作图题的基础。
\begin{enumerate}
\item 作一条线段等于已知线段,作一条线段等于$n$条线
段的和($n\ge 2$ 且 $n\in\mathbb{N}^+$)。
\item 作一条线段等于两条已知线段的差。
\item 作一个角等于已知角。
\item 平分一个已知角。
\item 过已知直线上或已知直线外一点,作已知直线的垂
线。
\item 作已知线段的垂直平分线。
\item 等分已知线段。
\item 按已知条件作三角形。
\begin{enumerate}
 \item 已知三边。
\item 已知两边及其夹角。
\item 已知两角及其中一角的对边。
\item 已知两角及其夹边。 
\end{enumerate}
\item 作三角形的外接圆和内切圆。
\item 已知斜边和一直角边,作直角三角形。
\item 已知线段$a$, 作一线段$x=\frac{m}{n}a$. (其中$\frac{m}{n}$
是正有理数)。
\item 作已知三条线段$a$、$b$、$c$的比例第四项。
\item 已知线段$a$、$b$, 作$a$、$b$的比例中项。
\item 已知线段$a$、$b$ $(a>b)$, 作线段$x=\sqrt{a^2+b^2}$
或$x=\sqrt{a^2-b^2}$。
\item 已知线段$a$, 作线段$x=\sqrt{\frac{m}{n}}a$ ($\frac{m}{n}$为正有理数)。
\item 过已知圆上一点作圆的切线。
\item 过已知圆外一点作圆的切线。
\item 作两个已知圆的公切线。
\item 过已知直线外的一个已知点,作这条直线的平行
线。
\end{enumerate}

\begin{ex}
\begin{enumerate}
    \item 作一直角三角形$ABC$, 使$\angle C=90^{\circ}$, $AB=3$cm, $BC=
    2$cm. 你能想出几种作法?
    \item 已知线段$a$、$b$,你能用几种方法作线段$x=\sqrt{ab}$。
    \item 已知线段$a$, 作线段$x=\sqrt{\frac{2}{3}a}$
        \item 过圆外一点,你能用几种方法作这个圆的切线。
\end{enumerate}
\end{ex}

\subsection{轨迹法作图}
我们在解作图题时,常常归结为要确定某些点的位置,
而这些点所要满足的条件又往往不是一个,我们只要根据点
所满足的各个条件,分别作出相应的轨迹,那么,这些轨迹
交集中的点,就是我们所要求作的点。

例如,已知两个定点$B$、$C$, 且
$\overline{BC}=3$cm (图5.5),以
$B$、$C$为两个顶点,求作一三角形,使第三个顶点与$B$的距
离是2cm, 与$C$的距离是4cm, 这个作图题,实际上就是确定
三角形的第三个顶点的位置。第三个顶点要满足两个条件:
\begin{enumerate}
    \item 和$B$点的距离是2cm
    \item 和$C$点的距离是4cm
\end{enumerate}
满足第一个条件的点的轨迹是$\odot(B,2{\rm cm})$,满足第二个条件的
点的轨迹是$\odot (C,4{\rm cm})$,所以第三个顶点就应该是$\odot(B,2{\rm cm})$与$\odot (C,4{\rm cm})$的交集中的点。由作图可知:
\[\odot(B,2{\rm cm})\cap \odot (C,4{\rm cm})=\{A, A'\}\]
于是,$A$点和$A'$点都是我们所
要求作的点,$\triangle ABC$与$\triangle A'BC$都是我们所要求作的三角
形。像这样应用轨迹的交集来确定点的位置,从而来解作图
题的方法,就叫做\textbf{轨迹法}。

\begin{figure}[htp]\centering
    \begin{minipage}[t]{0.48\textwidth}
    \centering
\begin{tikzpicture}[>=latex, scale=.8, rotate=-15]
\tkzDefPoints{0/0/B, 3/0/C, -.5/1.94/A,-.5/-1.94/A'}
\draw(A)--(B)--(C)--(A);
\draw(B)--(A')--(C);
\tkzDrawArc[delta=10](C,A)(A')
\tkzCompass(B,A)
\tkzCompass(B,A')
\tkzLabelPoints[above](A,C)
\tkzLabelPoints[left](A',B)

    \end{tikzpicture}
    \caption{}
    \end{minipage}
    \begin{minipage}[t]{0.48\textwidth}
    \centering
    \begin{tikzpicture}[>=latex, scale=1.7]
\draw (0,0) circle(1);
\tkzDefPoints{0/0/O}
\tkzDrawPoints(O)
\tkzLabelPoints[below](O)
\tkzDefPoint(-5:1){C}
\tkzDefPoint(45:1){D}
\tkzDrawLines[add=.5 and .5](O,C O,D)
\draw[dashed](15:1)node[right]{$B$}--(0,0)--(75:1)node[above]{$A$};
\draw (75:1)--(15:1);
\node at (-5:1.5)[right]{$\ell$};
\node at (45:1.5)[right]{$m$};

    \end{tikzpicture}
    \caption{}
    \end{minipage}
    \end{figure}



\begin{example}
    已知一条定直线和直线外两个定点,求作一个
圆,使圆心在这条直线上,并且经过这两个定点。
\end{example}

已知:定直线$\ell$和两定点$AB$, 且$A\notin \ell$, $B\notin \ell$(图5.6)。

求作:一圆使圆心在$\ell$上,且经过$A$、$B$两点。

分析:假定$\odot O$为所求作的圆;那么,圆心$O$应满足两
个条件:
\begin{enumerate}
    \item $O\in \ell$;
    \item $O$点到$A$、$B$两点等距离。
\end{enumerate}
因为$\ell$
是已知的直线,而到$A$、$B$两点等距离的点的轨迹是$\overline{AB}$的
垂直平分线,所以,圆心$O$应是$\ell\cap \overline{AB}$的垂直平分线中的点,于是得作法如下:

作法:
\begin{enumerate}
\item 作$\overline{AB}$;
\item 作$\overline{AB}$的垂直平分线$m$与$\ell$相交于$O$点,
\item 以$O$为圆心,$OA$为半径作$\odot O$, 则$\odot O$即为所求
作的圆。
\end{enumerate}

\begin{proof}
    作$\overline{OA}$、$\overline{OB}$,

$\because\quad m$是$\overline{AB}$的垂直平分线,且$O\in m$.

$\therefore\quad \overline{OA}=\overline{OB}$

\begin{multicols}{2}
  $\because\quad A$点$\in \odot O$,

$\therefore\quad B$点$\in \odot O$,

又知$O$点$\in\ell$,

$\therefore \odot O$为所求作的圆。  
\end{multicols}

\end{proof}

讨论:
\begin{enumerate}
    \item 当直线$\ell$与AB所在的直线不垂直时,$\overline{AB}$的垂直平分线$m$与$\ell$一定有一个交点,且只有一个交点。这时,问
题有一解。
\item 当$\ell$与直线$\overline{AB}$垂直时,如果$\ell$与$m$重合时,问题
有无穷多解;如果$\ell\parallel m$时问题无解。
\end{enumerate}

\begin{example}
    已知三角形的一边和这条边上的中线及高,求作
三角形。
\end{example}

已知:线段$a$、$m$、$h$(图5.7)。

\begin{figure}[htp]
    \centering
\begin{tikzpicture}[scale=1.7]
\begin{scope}
    \draw(0,0)--node[above]{$h$}(1.4,0);
    \draw(0,.5)--node[above]{$m$}(1.6,.5);
    \draw(0,1)--node[above]{$a$}(2,1);
\end{scope}
\begin{scope}[xshift=3cm]
    

\draw(0,0)node[left]{$B$}--node[below]{$a$}(2,0)node[right]{$C$};
\draw(-.5,1.4)--(1.8,1.4);
\tkzDefPoints{1/0/M, 0.23/1.4/A}
\tkzCompass(M,A)
\draw(M)--node[right]{$m$}(A)--(2,0);
\draw(0,0)--(A)--node[right]{$h$}(0.23,0)node[below]{$D$};
\tkzLabelPoints[above](M,A)
\node (D) at (0.23,0){};
\tkzMarkRightAngle[size=.1](C,D,A)
\end{scope}
\end{tikzpicture}
    \caption{}
\end{figure}

求作:三角形使它的一边等于$a$, 且这边上的中线等于
$m$, 高等于$h$.

分析:画一草图(图5.7),假设$\triangle ABC$为所求作的三
角形,且$\overline{BC}=a$, $\overline{BC}$上的中线$\overline{AM}=m$, 高$\overline{AD}=h$, $B$、
$C$两个顶点由条件$\overline{BC}=a$, 很容易确定;$A$点的位置应该
满足条件:
\begin{enumerate}
    \item 与$\overline{BC}$的中点$M$的距离等于$a$的长;
    \item 与$\overline{BC}$所在直线的距离等于$h$.
\end{enumerate}
分别满足条件1和2的轨迹
都可作,故$A$的位置可作出,于是所要求作的三角形可作。

作法:(图5.8)
\begin{figure}[htp]
    \centering
\begin{tikzpicture}[scale=1.7]
    \draw(0,0)node[left]{$B$}--(2,0)node[right]{$C$};
\draw(-.5,1.4)--(2,1.4)node[right]{$\ell$};
\tkzDefPoints{1/0/M, 0.23/1.4/A}
\tkzCompass(M,A)
\draw(M)node[below]{$M$}--node[right]{$m$}(A)--(2,0);
\draw(0,0)--(A)--(0.23,0)node[below]{$D$};
\draw[dashed](1.5,0)--node[right]{$h$}(1.5,1.4);
\tkzLabelPoints[above](A)
\node (D) at (0.23,0){};
\tkzMarkRightAngle[size=.1](C,D,A)
\end{tikzpicture}
    \caption{}
\end{figure}

\begin{enumerate}
    \item 任取一点$B$, 作$\overline{BC}=a$,
    \item 作直线$\ell\parallel$直线$BC$,
    且与直线$BC$的距离等于$h$.
    \item 作$\overline{BC}$的中点$M$, 作$\odot(M,m)$交$\ell$于$A$点
    \item  作$\overline{AB}$、$\overline{AC}$, 则$\triangle
    ABC$为所求作的三角形。
\end{enumerate}

\begin{proof}
    (略)
\end{proof}

讨论
\begin{enumerate}
    \item 当$m\ge h$时,由于$\odot(M,m)$与直线$\ell$能够相
交,所以问题有一解。
\item 当$m<h$时,$\odot(M,m)$与直线$\ell$不相交,所以
这时问题无解。
\end{enumerate}

\begin{rmk}
    上题中所求作的三角形,并没有指定它在平面上
    的确切位置(叫做\textbf{不定位}),我们可在不同的位置分别作出很
    多满足已知条件的三角形,而且这些三角形都是全等形,遇到
    这种情况,我们任作一个满足条件的图形就可以了,并说问
    题只有一解;如果作出的满足已知条件的三角形不是全等
    形,那么要把这些图形都作出来,作出几个,我们就说问题
    有几解。如果在问题里是要求在固定位置作图(叫做\textbf{定位}),
    那么不管作出的是不是全等形,都要把它们作出来,作出几
    个我们就说问题有几解。 
\end{rmk}

\begin{example}
   求作和定角$\angle ABC$的两边都相切,并且半径等于
$r$的圆。 
\end{example}

已知:$\angle ABC$, 线段$r$(图5.9)。

求作:和$\angle ABC$的两边都相切,并且半径等于$r$的圆。

分析:画一草图(图5.10),假设$\odot O$为所求作的圆,
且$\odot O$的半径等于$r$, 那么圆心$O$应满足条件:
\begin{enumerate}
    \item 与$\angle ABC$的两边等距离;
    \item 与边$BA$或$BC$的距离等于$r$.
\end{enumerate}
满足1或2的轨迹都可作,故所求的圆的圆心可以作出。

作法:(图5.9)
\begin{enumerate}
    \item 作$\angle ABC$的平分线$BF$;
    \item 作直线$\ell\parallel BC$, 且与$BC$的距离等于$r$, 设$\ell$与$BF$
    相交于$O$点;
    \item 以$O$为圆心,$r$为半径作$\odot(O,r)$, $\odot(O,r)$即为所求作的圆。
\end{enumerate}

\begin{figure}[htp]\centering
    \begin{minipage}[t]{0.48\textwidth}
    \centering
\begin{tikzpicture}[>=latex, scale=1]
\foreach \x in {0,20,40}
{
    \draw(0,0)--(\x:5);
}
\draw(2,1)--(5,1)node[right]{$\ell$};
\draw[dashed](4.5,0)node[below]{$C$}--node[right]{$r$}(4.5,1);
\draw(2.75,1) circle(1);
\draw(2.75,0)node[below]{$E$}--(2.75,1)node[above]{$O$}--(40:2.75)node[above]{$D$};
\node at (0,0) [left]{$B$};
\node at (20:5) [right]{$F$};
\node at (40:5) [right]{$A$};
\draw(0,2)--node[above]{$r$}(1,2);
    \end{tikzpicture}
    \caption{}
    \end{minipage}
    \begin{minipage}[t]{0.48\textwidth}
    \centering
    \begin{tikzpicture}[>=latex, scale=1]
\foreach \x/\xtext in {0/C,20/F,40/A}
{
    \draw(0,0)--(\x:5)node[right]{$\xtext$};
}
\draw(2,1)--(5,1);
\draw(2.75,1) circle(1);
\draw(2.75,0)--node[right]{$r$}(2.75,1)node[above]{$O$}--(40:2.75);
\node at (0,0) [left]{$B$};
\tkzDefPoint(40:2.75){D}
\tkzDefPoint(0:2.75){E}
\tkzDefPoint(2.75,1){O}\tkzDefPoint(0,0){B}
\tkzMarkRightAngles[size=.1](B,D,O B,E,O)


    \end{tikzpicture}
    \caption{}
    \end{minipage}
    \end{figure}

\begin{proof}
    作$\overline{OD} \bot BA$于$D$点,$\overline{OE}\bot BC$于$E$点。

$\because\quad EF$平分$\angle ABC$, 且$O\in BF$,

$\therefore\quad \overline{OD}=\overline{OE}$

又$\because\quad O\in\ell$, 而$\ell$与$BC$的距离是$r$的长。

$\therefore\quad \overline{OD}=\overline{OE}=r,\quad D\in \odot O,\quad E\in \odot O$

$\odot O$与$BA$和$BC$都相切,$\odot O$为所求作的圆。
\end{proof}

讨论:由于$\angle ABC$的平分线$BF$与直线$\ell$不会平行,所
以它们总有一个唯一的交点,所以此题只有一解。

从例5.5中可以看到,$\wideparen{DE}$把$\angle ABC$的两边在切点$D$、$E$
处连接起来了,这种连结叫做\textbf{直线与圆弧的平滑连接}。

\begin{example}
已知:$\odot (O_1,r_1)$和$\odot (O_2,r_2)$外离,且$r_1=
    10$mm, $r_2=8$mm(图5.11)。

    求作:一圆与$\odot (O_1,r_1)$外切,与$\odot (O_2,r_2)$内切,并
    且半径是20mm. 
\end{example}

\begin{figure}[htp]\centering
    \begin{minipage}[t]{0.48\textwidth}
    \centering
\begin{tikzpicture}[>=latex, scale=.7]
\draw (0,0) circle (2);
\draw (4,-1.2) circle (1.2);
\draw[thick](4,-3) circle (3);
\tkzDefPoints{0/0/O_1, 4/-1.2/O_2, 4/0/C1, 4/-3/O, 1.6/-1.2/B, 0/2/A, 1/-3/C}
\tkzLabelPoints[right](O_1,O_2,O)
\tkzLabelPoints(B,A,C)
\tkzDrawPoints(O_1,O_2,O, B,A,C)
\draw[dashed](O_1)--(O)--(4,0);
\tkzCompass(O_1,O)
\tkzCompass(C1,O)
    \end{tikzpicture}
    \caption{}
    \end{minipage}
    \begin{minipage}[t]{0.48\textwidth}
    \centering
    \begin{tikzpicture}[>=latex, scale=.5]
   \draw (0,0) circle (2);
\draw (4,-1.2) circle (1.2);
\draw[thick](4,-3) circle (3);
\tkzDefPoints{0/0/O_1, 4/-1.2/O_2, 4/-3/O}
\tkzLabelPoints[right](O_1,O_2,O)
\tkzDrawPoints(O_1,O_2,O)
\draw[dashed](O_1)--(O)--(4,0);   
    \end{tikzpicture}
    \caption{}
    \end{minipage}
    \end{figure}

\begin{analyze}
    画一草图(图5.12),设$\odot (O,20{\rm mm})$为要求作的
圆.由于$\odot (O,20{\rm mm})$与$\odot (O_1,r_1)$外切,并与$\odot (O_2,r_2)$内切,
所以
\[\begin{split}
   \overline{OO_1}&=r_1+20{\rm  mm}=30{\rm mm}\\
   \overline{OO_2}&=20{\rm  mm}-r_2=12{\rm  mm} 
\end{split}\]

因此,$O$点是$\odot (O_1,30{\rm mm})$和$\odot (O_2,12{\rm  mm})$的交点,
故$O$点可作出。
\end{analyze}

作法:(图5.11)。
\begin{enumerate}
    \item 作$\odot (O_1,30{\rm mm})$和$\odot (O_2,12{\rm mm})$, 设两圆有
交点$O$;
\item 以$O$为圆心,以20mm为半径作$\odot O$, 即$\odot O$为
所求作的圆。
\end{enumerate}

\begin{proof}
(略)
\end{proof}

讨论:
\begin{enumerate}
    \item 当$(r_1+20{\rm mm})+(20{\rm mm}-r_2)\ge \overline{O_1O_2}$时,
即当$ \overline{O_1O_2}\le 42{\rm mm}$时,问题有一解。
\item 当$ \overline{O_1O_2}> 42{\rm mm}$时,问题无解。
\end{enumerate}

从例5.6中的图5.11, 我们可看出$\odot O_1$的
$\wideparen{AB}$与$\odot O_2$的$\wideparen{BC}$
在切点$B$处连接起来了,这种连结叫做\textbf{圆弧与圆弧的平
滑连接}。

\begin{ex}
\begin{enumerate}
    \item 已知两定点$A$、$B$, 且直线$AB$与定直线$\ell$平行,在$\ell$上求作与$A$、$B$两点直线$\ell$距离相等的点。
    \item 求作和两条相交直线$\ell$和$m$的距离相等,且和它们的交点的距离等于定长$d$的点。
    \item 求作和已知直线$\ell$的距离等于定长$d$,并且和$\ell$上的一个定点的距离等于$2d$的点。
    \item 求作和$\overline{AB}$的两端的距离相等,且和$\overline{AB}$的两端的连线所夹的角等于定角$\alpha$的点。
    \item 求作等腰三
    角形,使它的底边等于定长$a$, 顶角等于定
    角$\alpha$.
    \item 求作一圆,使它的半径等于定长$a$, 且经过一定点并和
    一条定直线相切。
    \item 已知三角形的二边和其中一边上的高,求作三角形。
    \item 已知$\odot O(0,15{\rm mm})$和直线$\ell$, 并且$\ell$与$\odot O$相离,画半
    径为10mm的圆,使其与$\odot O$外切并和$\ell$相切。
\end{enumerate}
\end{ex}

\subsection{代数法作图}
解作图题时,有时也常常归结为要求作一条线段的长度
问题。这时我们可根据给出的条件,求出这条线段的代数表
达式,根据线段的代数表达式,把线段作出,使问题得到解
决。这种解作图题的方法,就叫做\textbf{代数分析法}。

\begin{example}
    已知:正方形$ABCD$(图5.13).

    求作:一点$P$使$P\in \overline{CD}$, 且$\overline{AP}=\overline{BC}+\overline{CP}$.
\end{example}

\begin{figure}[htp]\centering
    \begin{minipage}[t]{0.48\textwidth}
    \centering
\begin{tikzpicture}[>=latex, scale=1]
\tkzDefPoints{0/0/A, 0/2/D, 2/2/C, 2/0/B, 1/2/E, 1.5/2/P}
\tkzDrawPolygon(A,B,C,D)
\tkzLabelPoints[left](A,D)
\tkzLabelPoints[right](B,C)
\tkzLabelPoints[above](E,P)
\draw(A)--(P);
\tkzDrawPoints(E)
    \end{tikzpicture}
    \caption{}
    \end{minipage}
    \begin{minipage}[t]{0.48\textwidth}
    \centering
    \begin{tikzpicture}[>=latex, scale=1]
      \tkzDefPoints{0/0/A, 0/2/D, 2/2/C, 2/0/B,  1.5/2/P}
      \tkzDrawPolygon(A,B,C,D)
\tkzLabelPoints[left](A,D)
\tkzLabelPoints[right](B,C)
\tkzLabelPoints[above](P)
\draw(A)--(P);
\tkzDrawPoints(E)
    \end{tikzpicture}
    \caption{}
    \end{minipage}
    \end{figure}

\begin{analyze}
    画一草图(图5.14),已知正方形$ABCD$, 假设$P$点是所求作的点,即$P\in \overline{CD}$, 且$\overline{AP}=\overline{BC}+
    \overline{CP}$, 设
    $\overline{CP}=x$, $\overline{BC}=a$, 故$\overline{AP}=8+x$, $\overline{DP}=\overline{DC}-\overline{CP}=\overline{BC}-\overline{CP}=a-x$, 又因$\triangle ADP$是直角三角形,所以根据
    勾股定理有:$\overline{AP}^2=\overline{AD}^2+\overline{DP}^2$, 即:$(a+x)^2=a^2+(a-x)^2$.
    解此方程得
    $x=\frac{a}{4}$;$a$为已知,$\frac{a}{4}$
    可作出,故$P$点也可作
    出。
\end{analyze}

作法:(图5.13)。

\begin{enumerate}
    \item 作$\overline{CD}$的中点$E$.
    \item 作$\overline{CE}$的中点$P$, $P$点即为所求作的点。
\end{enumerate}

\begin{proof}
    由作图知$\overline{CP}=\frac{a}{4}$,

$\therefore\quad \overline{DP}=\overline{CD}-\overline{CP}=\frac{3}{4}a$
\[\overline{AP}=\sqrt{\overline{AD}^2+\overline{DP}^2}=\sqrt{a^2+\left(\frac{3}{4}a\right)^2}=\frac{5}{4}a=a+\frac{a}{4}\]
\[\overline{AP}=\overline{BC}+\overline{CP}\]
\end{proof}


讨论:由分析知,$\overline{DP}:\overline{PC}=3:1$, 因为分点$P$是唯一的,所以此题只有一解。


\begin{example}
    在已知线段上求作一点,分已知线段为两部分,
    使其中一部分是全线段和另一部分的比例中项。

    已知:$\overline{AB}$(图5.15).

    求作:$\overline{AB}$的内分点$G$, 使$\overline{AG}^2=\overline{AB}\cdot \overline{GB}$.
\end{example}

\begin{figure}[htp]\centering
    \begin{minipage}[t]{0.48\textwidth}
    \centering
\begin{tikzpicture}[>=latex, scale=.8]
    \draw(0,0)node[left]{$A$}--(0.618*4,0)node[below]{$G$}--(4,0)node[right]{$B$};
    \tkzDrawPoint(0,0)
    \tkzDrawPoint(0.618*4,0)
    \tkzDrawPoint(4,0)
\tkzDefPoints{4/2/D, 4/0/B, 0/0/A, 2.472/0/G}
\tkzDefPoint(26.57: 2.472){E}
\tkzDrawArc[delta=10](D,E)(B)
\tkzDrawArc[delta=10](A,G)(E)
\tkzDrawSegments(A,D D,B)
\tkzLabelPoints[above](D,E)

    \end{tikzpicture}
    \caption{}
    \end{minipage}
    \begin{minipage}[t]{0.48\textwidth}
    \centering
    \begin{tikzpicture}[>=latex, scale=1]
\draw(0,0)node[left]{$A$}--node[above]{$x$}(0.618*4,0)node[below]{$G$}--(4,0)node[right]{$B$};
\tkzDrawPoint(0,0)
\tkzDrawPoint(0.618*4,0)
\tkzDrawPoint(4,0)
    \end{tikzpicture}
    \caption{}
    \end{minipage}
    \end{figure}



\begin{analyze}
    画一草图(图5.16),已知$\overline{AB}$, 假定$G$为所求
    之点,设$\overline{AG}=x$, 则
    $\overline{BG}=a-x$,
    $x$满足方程
\[    x^2=a(a-x)\]
    即:$x^2+ax-a^2=0$. 解此方程得:
\[x_1=\frac{-a+\sqrt{4a^2+a^2}}{2},\qquad x_2=\frac{-a-\sqrt{4a^2+a^2}}{2}\quad \text{舍去}\]
$\therefore\quad x=\frac{-a+\sqrt{4a^2+a^2}}{2}=\sqrt{a^2+\left(\frac{a}{2}\right)^2}-\frac{a}{2}$

此线段$x$可作,故$G$点也可作出来。
\end{analyze}

作法:(图5.15)。
\begin{enumerate}
\item 作$\overline{ED}\bot\overline{AB}$, 使$\overline{BD}=\frac{1}{2}\overline{AB}$
\item 在$\overline{DA}$上截取$\overline{DE}=\overline{DB}$,\item 在$\overline{AB}$上截取$\overline{AG}=\overline{AE}$, $G$点即为所求作的点。
\end{enumerate}

\begin{proof}
    设$\overline{AB}=a$, 由作法有,
\[\begin{split}
    \overline{AG}&=\overline{AE}=\overline{AD}-\overline{ED}\\
&=\sqrt{\overline{AB}^2+\overline{BD}^2  }-\overline{BD}=\sqrt{a^2+\left(\frac{a}{2}\right)^2}-\frac{a}{2}\\
\overline{GB}&=\overline{AB}-\overline{AG}=a-\left(\sqrt{a^2+\left(\frac{a}{2}\right)^2}-\frac{a}{2}\right)\\
&=\frac{3}{2}a-\sqrt{a^2+\left(\frac{a}{2}\right)^2}
\end{split}\]
$\therefore\quad \overline{AB}\cdot \overline{GB}=a\left(\frac{3}{2}a-\sqrt{a^2+\left(\frac{a}{2}\right)^2}\right)$

由于:
\[\begin{split}
    \overline{AG}^2&=\left(\sqrt{a^2+\left(\frac{a}{2}\right)^2}-\frac{a}{2}\right)^2\\
&=a^2+\left(\frac{a}{2}\right)^2-2\x\frac{a}{2}\x \sqrt{a^2+\left(\frac{a}{2}\right)^2}+\left(\frac{a}{2}\right)^2\\
&=\frac{3}{2}a^2-a\sqrt{a^2+\left(\frac{a}{2}\right)^2}\\
&=a\left(\frac{3}{2}a-\sqrt{a^2+\left(\frac{a}{2}\right)^2}\right)
\end{split}
    \]
$\therefore\quad \overline{AG}^2=\overline{AB}\cdot \overline{BG}$

故$G$点为所求之点。
\end{proof}

讨论:由分析可知,满足条件的$\overline{AG}$总有一个,所以$G$
点总能作出一个,故此题有一解也只有一解。

由于
\[\overline{AG}=\sqrt{a^2+\left(\frac{a}{2}\right)^2}-\frac{a}{2}=\frac{\sqrt{5}}{2}a-\frac{1}{2}a=\frac{\sqrt{5}-1}{2}a\approx 0.618 a\]
所以$\overline{AG}$是$\overline{AB}$被$G$点分成的两段中,较长的一段,这种作图通常叫做分已知线段成“\textbf{中
外比}”,又叫做\textbf{黄金分割}。

\begin{example}
    在已知圆中,作内接正十边形。

已知:$\odot(O,r)$(图5.17).

求作:$\odot O$的内接正十边形。
\end{example}

\begin{figure}[htp]\centering
    \begin{minipage}[t]{0.48\textwidth}
    \centering
\begin{tikzpicture}[>=latex, scale=1]
\draw (0,0) circle (2.5);
\tkzDefPoint(0,0){O}
\foreach \x/\xtext in {0/D,1/E,2/F,3/G,4/H,5/I,6/J,7/A,8/B,9/C}
{
    \tkzDefPoint(\x*36:2.5){\xtext}
    \tkzDrawPoint(\xtext)
    \tkzAutoLabelPoints[center=O](\xtext)
}
\tkzDrawPolygon(A,B,C,D,E,F,G,H,I,J)
\draw(O)--(A);
\tkzDefPoint(-108:1.545){C'}  
\tkzLabelPoints[left](O, C') \tkzDrawPoints(O, C')
\tkzCompass[delta=10](A,B)
\tkzCompass[delta=10](B,C)
\tkzCompass[delta=10](C,D)
\tkzCompass[delta=10](D,E)
\tkzCompass[delta=10](E,F)
\tkzCompass[delta=10](F,G)
\tkzCompass[delta=10](G,H)
\tkzCompass[delta=10](H,I)
\tkzCompass[delta=10](I,J)
\tkzCompass[delta=10](J,A)
    \end{tikzpicture}
    \caption{}
    \end{minipage}
    \begin{minipage}[t]{0.48\textwidth}
    \centering
    \begin{tikzpicture}[>=latex, scale=.8]
        \draw (0,0) circle (2.5);
        \tkzDefPoint(0,0){O}
        \tkzDefPoint(7*36:2.5){A}
        \tkzDrawPoint(A)
        \tkzAutoLabelPoints[center=O](A)
        \tkzDefPoint(8*36:2.5){B}
        \tkzDrawPoint(B)
        \tkzAutoLabelPoints[center=O](B)
        \draw(O)--(A)--(B);
\tkzDefPoint(-108:1.545){C'}  
\tkzLabelPoints[left](O, C') \tkzDrawPoints(O, C')
\draw(C')--(B)--(O);
    \end{tikzpicture}
    \caption{}
    \end{minipage}
    \end{figure}







\begin{analyze}
    画一草图(图5.18),设$\overline{AB}$是$\odot O$的内接正十
边形的一边,则
\[\angle AOB=36^{\circ},\qquad \angle OAB=\angle OBA=72^{\circ}\]
作$\angle OBA$的平分线交$\overline{OA}$于$C'$点,则
\[\angle OBC'=\angle ABC'=\frac{1}{2}\angle OBA=36^{\circ}\]
又知$\angle BC'A=180^{\circ}-\angle ABC'-\angle OAB$,

$\therefore\quad \angle BC'A=72^{\circ},\quad 
\overline{OC'}=\overline{BC'}=\overline{AB}$, 且
$$\triangle OAE\backsim \triangle BAC',\quad \overline{OA}:\overline{AB}=\overline{AB}:\overline{AC'}$$

即:$\overline{AB}^2=\overline{OA}\cdot \overline{AC'}$

$\therefore\quad \overline{OC'}^2=\overline{OA}\cdot \overline{AC'}$
\end{analyze}

这个结果告诉我们,$C'$点恰好把半径$\overline{OA}$分成中外比且
$\overline{OC'}$是较长的一段。故应用黄金分割法由已知圆的半径作出
正十边形的边长,从而圆内接正十边形可以作出来。

作法 (图5.17).
\begin{enumerate}
\item 作半径$\overline{OA}$.
\item 作$C'$点分$\overline{OA}$成中外比,使$\overline{OC'}$为较长的一段,
\item 以$A$为起点,顺次作$\odot O$的弦$\overline{AB}$、$\overline{BC}$、$\overline{CD}$、
$\overline{DE}$、$\overline{EF}$、$\overline{FG}$、$\overline{GH}$、$\overline{HI}$、$\overline{IJ}$、$\overline{JA}$, 且使它们都
等于$\overline{OC'}$, 则$ABCDEFGHIJ$为所求作的圆内接正十边形。
\end{enumerate}



\begin{proof}
    (略)
\end{proof}

讨论:由分析可知一个圆内接正十边形的边长等于一个
定值,所以此问题有一解。

我们把圆十等分后,把相
间的五个分点用弦顺次连结,
就可作出圆内接正五边形,把
正五边形的五条对角线都作出
来,然后去掉各边剩下的图形
就是正五角星了(图5.19)。

\begin{figure}[htp]
    \centering
\begin{tikzpicture}[scale=.8]
\draw (0,0) circle (2.5);
\tkzDefPoint(0,0){O}
\foreach \x/\xtext in {0/A,1/B,2/C,3/D,4/E}
{
    \tkzDefPoint(18+\x*72:2.5){\xtext}
    \tkzDrawPoint(\xtext)
 %   \tkzAutoLabelPoints[center=O](\xtext)
}
\tkzDrawPolygon[dashed](A,B,C,D,E)
\tkzDrawSegments(A,C C,E B,D B,E A,D)

\foreach \x/\xtext in {0/A1,1/B1,2/C1,3/D1,4/E1}
{
    \tkzDefPoint(54+\x*72:2.5){\xtext}
   % \tkzAutoLabelPoints[center=O](\xtext)
}
\tkzCompass[delta=10](A,A1)
\tkzCompass[delta=10](A1,B)
\tkzCompass[delta=10](B,B1)
\tkzCompass[delta=10](B1,C)
\tkzCompass[delta=10](C,C1)
\tkzCompass[delta=10](C1,D)
\tkzCompass[delta=10](D,D1)
\tkzCompass[delta=10](D1,E)
\tkzCompass[delta=10](E,E1)
\tkzCompass[delta=10](E1,A)

\end{tikzpicture}
    \caption{}
\end{figure}


通过以上三例,我们看出,用代数法解作图题的一般步
骤:
\begin{itemize}
    \item 首先要分析解决这个问题需要作出哪条线段,并用$x$
表示;
\item 其次依照题中所给的条件和图形的性质,列出关于$x$
的方程;
\item 第三步,解这个方程(不合题意的根舍去);
\item 第四
步,依照方程的根的表示式,作出未知线段$x$; 
\item 第五步,完
成作图。即得所求作的图形。
\end{itemize}



\begin{ex}
\begin{enumerate}
    \item 已知线段$a$、$b$ ($a>b$), 求作下列线段$x$
\begin{multicols}{2}
    \begin{enumerate}
        \item $x=\frac{ab}{a+b}$
        \item $x=\sqrt{4a^2+b^2}$
        \item $x=\sqrt{a^2-\frac{b^2}{4}}$
        \item $x=\sqrt{a^2+3b^2}-\frac{b}{2}$
    \end{enumerate}
\end{multicols}
    \item 求作已知三角形的相似形,使它的面积等于已知三角形
    面积的三分之二。
    \item 从圆外一定点求作圆的一条割线,使它的圆外部分同圆
    内部分相等。
    \item 求作一正方形,使它同已知长方形等积。
    \item 在已知梯形中,求作底的平行线,平分已知梯形的面积。
    \item 作一圆的内接正五边形(只用尺、规)。
    \item 经过圆内一点作一条弦,使这个点是这条弦的一个三等
分点。
\end{enumerate}
\end{ex}


\section*{习题5.2}
\begin{enumerate}
    \item 已知线段$a$、$b$、$c$且$a>b$, 求作线段$x=\sqrt{a^2-b^2}+c$.
    \item 已知五条线段$a$、$b$、$c$、$d$、$e$, 求作线段$x=\frac{abc}{de}$。
    \item 已知线段$a$, 求作一条线段$x=\frac{\sqrt{5}-1}{2}a$
    \item 已知线段$a$、$b$, 且$a>b$, 求作$a+b$和$a-b$的比例中
    项。
    \item 求作与一个已知圆相切于已知点,且经过已知圆外的一
    个定点的圆。
    \item 已知三角形的一边长是3cm, 这边上的中线长4cm, 这边
    的对角是$65^{\circ}$, 求作这个三角形。
    \item 已知一边和这边上的中线及另一边上的高线,求作这个
    三角形。
    \item 求作一个三角形,使它同已知三角形等积,又同另一个
    已知的三角形相似。
    \item 求作直径是$d$的圆内接矩形,使它的面积等于每边是$a$
    的已知正方形面积(提示:设矩形的长、宽各为$x$、$y$,
    列出含有$x$、$y$的方程组解之)。
    \item 从已知圆外的一个已知点,作圆的割线,使它在圆外的
    部分与圆内的部分的比是$1:2$.
    \item 求作过两定点,且在一条定直线上截取定长弦的圆
    (提示:利用切割线定理)。
\end{enumerate}

\section*{复习题五}

\begin{enumerate}
    \item 连结一定直线上的点和线外一定点的线段,求这线段中
    点的轨迹。
    \item 求具有公共底边,且这边上的高相等的三角形顶点的轨
    迹。
    \item 三定点$A$、$B$、$C$在一直线上,且$\overline{AB}=\overline{BC}$, 求与$A$、$B$
    两点和与$B$、$C$两点连线夹角相等点的轨迹。
    \item 从一定点$A$向通过另一定点$B$的动直线引垂线,求垂足
    $P$的轨迹。
    \item 求到两定点$A$、$B$的距离的平方和等于$\overline{AB}^2$
    的点的轨迹。
    \item 求到一定圆引切线,切线长等于定长的点的轨迹。
    \item 求对相交的两定圆有等幂的点的轨迹。
    \item 求到两定点$A$、$B$平方差等于$\overline{AB}^2$的点的轨迹。
    \item 已知三点$A$、$B$、$C$在一条直线上,求与$A$、$B$两点和$B$、
   $ C$两点连线夹角相等的点的轨迹。
    \item 求作一个圆使和两条已知平行线都相切,并且经过两平
    行线间一个已知点。
    \item 已知$\odot(O_1,1.5{\rm cm})$, $\odot(O_2,1{\rm cm})$, 圆心间的距离
    $\overline{O_1O_2}=4{\rm cm}$, 求作一圆,使它的半径等于1.3cm, 并且
    和$\odot O_1$与$\odot O_2$都外切。
    \item 已知$\odot(O_1,1.4{\rm cm})$, $\odot(O_2,1{\rm cm})$, $\overline{O_1O_2}=3{\rm cm}$, 求
    作一个圆,使它的半径为3.2cm, 并且和$\odot O_1$与$\odot O_2$都相内切。
    \item 已知线段$a$, 求作线$x=\sqrt{3a}$, $y=\sqrt{12a}$.
    \item 已知线段$x$与$y$满足方程组:
    \[\begin{cases}
        \frac{x+y}{2}=a\\
        \sqrt{xy}=b
    \end{cases}\qquad (a>b)\]
    求作线段$x$和$y$. 
    \item 作一个三角形具有已知的周长,并和一个已知三角形相
似。
\item 求作$\triangle ABC$的内接正方形,使正方形的一边位于$\overline{AB}$边
上,另外两个顶点各在$\overline{AC}$边和$\overline{BC}$边上。
\item 在已知正方形$ABCD$内作内接正方形,使内接正方形的
四个顶点分别在正方形的各边上,且使它的边长等于定
长$b$.
\item 把任一个三角形改为等边三角形,使它的面积不变。
\item 求作三个两两外切的圆,使它们的半径分别等于1cm,
 2cm, 3cm.
\item 作一个正五角星(只画出图形).
\item 已知$A$、$B$、$C$为定直线$\ell$上的三个定点,求作$\odot A$、
$\odot B$、$\odot C$使它们两两均相切,但不切于同一点。
\item 在已知矩形内作两个互相外切的等圆,使各切于这矩形
一组对角的两边。
\end{enumerate}












\chapter{三角比与角边关系}

人们为了要确定空间各点之间的相互位置,就得做一番
测量,测量是几何学的起源,也是几何学最直接的实践。

测量学的最基本原理,就是相似形的性质及三角形的边
角关系。例如,我们在第三章末用相似形性质测量两点间的,
距离,物体的高度、测绘具有多边形形状的地段的平面图
等。我们知道,在两个直角三角形中,只要有一个锐角对应
相等,它们就相似了,这就是说,一个直角三角形的各边之。
间的比是被它的一个锐角的大小所决定,例如在图6.1中,
一些含有$30^{\circ}$角的直角三角形,$30^{\circ}$角所对的直角边与斜边的
比都是1:2.

\begin{figure}[htp]
    \centering
\begin{tikzpicture}
\begin{scope}
\tkzDefPoint(30:2){B_1}
\tkzDefPoints{0/0/A_1, 1.732/0/C_1}
\tkzDrawPolygon(A_1,B_1,C_1)
\tkzMarkAngle[mark=none, size=.5](C_1,A_1,B_1)
\tkzLabelPoints[right](C_1,B_1)
\tkzLabelPoints[left](A_1)
\tkzMarkRightAngle(B_1,C_1,A_1)
\end{scope}
\begin{scope}[xshift=3.5cm]
    \tkzDefPoint(30:3){B_2}
    \tkzDefPoints{0/0/A_2, 2.6/0/C_2}
    \tkzDrawPolygon(A_2,B_2,C_2)
    \tkzMarkAngle[mark=none, size=.5](C_2,A_2,B_2)
    \tkzLabelPoints[right](C_2,B_2)
    \tkzLabelPoints[left](A_2)
    \tkzMarkRightAngle(B_2,C_2,A_2)
\end{scope}
\begin{scope}[xshift=11cm]
    \tkzDefPoint(150:4){B_3}
    \tkzDefPoints{0/0/A_3, -3.464/0/C_3}
    \tkzDrawPolygon(A_3,B_3,C_3)
    \tkzMarkAngle[mark=none, size=.6](B_3,A_3,C_3)
    \tkzLabelPoints[below](C_3,A_3)
    \tkzLabelPoints[left](B_3) 
    \tkzMarkRightAngle(A_3,C_3,B_3)
\end{scope}

\end{tikzpicture}
    \caption{}
\end{figure}

这一章,我们首先向同学介绍的就是直角三角形中,边
与边的比与它所含锐角之间的关系。这些边与边的比值叫做
\textbf{锐角三角比},它们是进行测量计算时的常用数据,也是从数
量方面研究几何学的基本工具。

\section{锐角三角比}

\subsection{定义}

\begin{figure}[htp]\centering
    \begin{minipage}[t]{0.48\textwidth}
    \centering
\begin{tikzpicture}[>=latex, scale=.8]
\draw(30:6)node[right]{$Y$}--(0,0)node[left]{$A$}--(5.5,0)node[right]{$X$};
\draw(30:5)node[above]{$B$}--node[right]{$a$}+(0,-2.5)node[below]{$C$};
\node at (1.25*1.732,0)[below]{$b$};
\node at (30:2.5)[above]{$c$};
\draw(4.33,0) rectangle (4.33-.2,.2);
    \end{tikzpicture}
    \caption{}
    \end{minipage}
    \begin{minipage}[t]{0.48\textwidth}
    \centering
    \begin{tikzpicture}[>=latex, scale=.8]
\draw(30:5.5)node[right]{$Y$}--(0,0)node[left]{$A$}--(5,0)node[right]{$X$};
\draw(30:2.5)node[above]{$B$}--+(0,-1.25)node[below]{$C$};
\draw(30:3.5)node[above]{$B'$}--+(0,-1.75)node[below]{$C'$};
\draw(30:4.5)--+(0,-2.25);
\draw(2.165,0) rectangle (2.165-.2,.2);
\draw(3.03,0) rectangle (3.03-.2,.2);
\draw(3.9,0) rectangle (3.9-.2,.2);
    \end{tikzpicture}
    \caption{}
    \end{minipage}
    \end{figure}


取任意锐角$\angle XAY$, 在边$AY$上任取一点$B$, 作$\overline{BC}\bot AX$
于$C$(图6.2). 在直角$\triangle ABC$中,设$\angle A$、$\angle B$、$\angle C$的
对边分别用$a$、$b$、$c$表示,对
锐角$A$来说,$a$叫做$\angle A$的\textbf{对
边},$b$叫做$\angle A$的相邻的直角
边(简称\textbf{邻边})我们定义:
\begin{enumerate}
    \item $\angle A$的对边与斜边的比值,叫做$\angle A$的正弦,用符
号$\sin A$来表示,即
\[\sin A=\frac{\angle A\text{的对边}}{\text{斜边}}=\frac{a}{c}\]
\item $\angle A$的邻边与斜边的比值,叫做$\angle A$的余弦。用符
号$\cos A$来表示,即
\[\cos A=\frac{\angle A\text{的邻边}}{\text{斜边}}=\frac{b}{c}\]
\item $\angle A$的对边与邻边的比值,叫做$\angle A$的正切,用符
号$\tan A$来表示,即
\[\tan A=\frac{\angle A\text{的对边}}{\angle A\text{的邻边}}=\frac{a}{b}\]
\item $\angle A$的邻边与对边的比值,叫做 $\angle A$的余切,用符
号$\cot A$来表示,即
\[\cot A=\frac{\angle A\text{的邻边}}{\angle A\text{的对边}}=\frac{b}{a}\]
\end{enumerate}

我们知道,只要$\angle A$的大
小定了,不管$B$点在边$AY$上
的位置如何(图6.3),以上
的四个比值都是不变的,只有
当$\angle A$变化时,这些比值才随着变化。这四个比都叫做锐角$A$的三角比。

有了以上定义,我们就在直角三角形的角与边之间建立
了联系,知道了角的大小,相应的四个三角比就被唯一地确
定了。反过来,如果我们知道了一个角的四个三角比中的任
何一个,我们也就能确定这个角的大小。

\begin{example}
    在直角$\triangle ABC$中,$\angle C=90^{\circ}$, $\overline{BC}=3$cm, $\overline{AC}=
    4$cm, 求$\angle A$的4个三角比(图6.4)。
\end{example}

\begin{figure}[htp]\centering
    \begin{minipage}[t]{0.48\textwidth}
    \centering
\begin{tikzpicture}[>=latex, scale=1]
\draw(0,0)node[left]{$B$}--node[below]{3}(3,0)node[right]{$C$}--node[right]{4}(3,4)node[above]{$A$}--node[left]{5}(0,0); 
\draw(3,0) rectangle (3-.2,.2);
\draw(3,4-.5) arc (-90:-36.9-90:.5);

    \end{tikzpicture}
    \caption{}
    \end{minipage}
    \begin{minipage}[t]{0.48\textwidth}
    \centering
    \begin{tikzpicture}[>=latex, scale=.8]
\draw(40:7)node[right]{$Y$}--(0,0)node[left]{$A$}--(5,0)node[right]{$X$};
\draw(4.2,0)node[below]{$C$}--node[right]{$1.8$}(4.2,3.56)node[above]{$B$};
\node at (40:3)[left]{$3$};
\node at (2.1,0)[below]{$2.1$};
\draw(4.2,0) rectangle (4,.2);
\draw(.5,0) arc (0:40:.5);
    \end{tikzpicture}
    \caption{}
    \end{minipage}
    \end{figure}

\begin{solution}
    根据勾股定理,
  \[  \overline{AB}=\sqrt{\overline{BC}^2+\overline{AC}^2}=\sqrt{3^2+4^2}=5{\rm (cm)}\]
  根据各三角比的定义有,
\[\begin{split}
    \sin A=\frac{\overline{BC}}{\overline{AB}}=\frac{3}{5},&\qquad \cos A=\frac{\overline{AC}}{\overline{AB}}=\frac{4}{5}\\
    \tan A=\frac{\overline{BC}}{\overline{AC}}=\frac{3}{4},&\qquad \cot A=\frac{\overline{AC}}{\overline{BC}}=\frac{4}{3}\\
\end{split}\]
\end{solution}


\begin{example}
      求$40^{\circ}$角的四个三角比。
\end{example}

\begin{solution}
 用量角器画$\angle XAY=40^{\circ}$(图6.5). 在边$AY$上截
取$\overline{AB}=3$cm(为计算方便,我们尽量取整数),作$\overline{BC}\bot AY$于$C$点,量得$\overline{BC}=1.8$cm, $\overline{AC}=2.1$cm, 在直角
$\triangle ABC$中,根据三角比的定义可得:
\[\sin40^{\circ}=\frac{1.8}{3}\approx 0.6,\qquad \cos40^{\circ}=\frac{2.1}{3}\approx 0.7 \]
\[\tan40^{\circ}=\frac{1.8}{2.1}\approx 0.8,\qquad \cot40^{\circ}=\frac{2.1}{1.8}\approx 1.1\]
\end{solution}

\begin{example}
    已知$\tan A=\frac{1}{2}$, 求$\angle A$.
\end{example}

\begin{figure}[htp]
    \centering
\begin{tikzpicture}
\draw(0,0)node[left]{$A$}--(4,0)node[right]{$C$}--(4,2)node[above]{$B$}--(0,0);
\draw (4,0)rectangle(4-.2,.2);
\end{tikzpicture}
    \caption{}
\end{figure}

\begin{solution}
    作一个直角$\triangle ABC$, 使直角边$\overline{AC}=2$个单位长,
$\overline{CB}=1$个单位长(图6.6),于是,
\[\tan A=\frac{1}{2}\]
用量角器量$\angle A$, 得之$\angle A\approx 26^{\circ}$。
\end{solution}

\begin{ex}
\begin{enumerate}
    \item 已知直角$\triangle ABC$, $\angle C=90^{\circ}$, $\overline{BC}=5$个单位长,$\overline{AC}=
    12$个单位长,求$\angle A$与$\angle B$的四个三角比。
    \item 用作图法求出表中各角的四个三角比的近似值,填入表
    中:
\begin{center}
    \begin{tabular}{c|cccc}
\hline
        $\alpha$ & $20^{\circ}$ & $40^{\circ}$ & $50^{\circ}$ & $80^{\circ}$\\
\hline
$\sin\alpha$\\
$\cos\alpha$\\
$\tan\alpha$\\
$\cot\alpha$\\
\hline
    \end{tabular}
\end{center}
\item 已知 $\sin\alpha=\frac{3}{4}$,
用作图法求$\angle\alpha$.
\item 已知$\cos\beta=\frac{2}{5}$,
用作图法求$\angle \beta$.
\item 已知$\tan A=\frac{3}{5}$, 
用作图法求$\angle A$.
\end{enumerate}
\end{ex}

\subsection{$0^{\circ}$到$90^{\circ}$角的三角比的变化}
半径等于1个单位长的圆叫做\textbf{单位圆}。下面我们利用单
位圆来研究锐角三角比的变化规律:

\begin{figure}[htp]
    \centering
\begin{tikzpicture}[>=latex,scale=1.5]
\draw[->](-1.5,0)--(1.5,0)node[right]{$x$};
\draw[->](0,-1.5)--(0,2)node[right]{$y$};    

\draw(0,0)node[below left]{$O$} circle (1);
\draw(1,0)node[above right]{$A$}--(1,2)node[above]{$A'$};
\draw[very thick](0,0)--(60:1)node[below left]{$P$}--(60:2)node[right]{$Q$};
\draw[very thick](60:1)--(0.5,0)node[below]{$M$};
\draw[very thick](0,1)node[above left]{$B$}--(1,1)node[right]{$B'$}--(1,0)node[below right]{1};
\node at (.5,1)[above]{$T$};
\draw (.25,0) arc (0:60:.25)node[right]{$\alpha$};
\end{tikzpicture}
    \caption{}
\end{figure}


画单位圆$\odot O$(图6.7), 通过单位圆的圆心$O$作互相
垂直的两条直线,其中一条是
水平的,另一条是铅直的,以
$O$为原点,单位圆的半径为长
度单位,在两条直线上建立数
轴,其中水平轴向右为正,铅
直轴向上为正;水平轴用$x$表
示,又叫做$x$轴,铅直的轴用
$y$表示,又叫做$y$轴。以$O$为
顶点,$x$轴的正方向为一边,作$\angle AOP$等于已知角$\alpha$, $\angle AOP$
的两边分别与单位圆相交于$A$、$P$两点,过$P$点作$\overline{PM}\bot OA$于$M$点,

$\because\quad \overline{OP}=1$

$\therefore\quad \sin\alpha=\frac{\overline{MP}}{\overline{OP}}=\overline{MP}\text{的量数},\quad \cos\alpha=\frac{\overline{OM}}{\overline{OP}}=\overline{OM}\text{的量数}$

这样,对于任一锐角$\alpha$, 我们可直接用$\overline{MP}$和$\overline{OM}$的量
数来分别表示$\sin\alpha$, $\cos\alpha$的值。我们把$\overline{MP}$和$\overline{OM}$分别叫做
角$\alpha$的\textbf{正弦线}和\textbf{余弦线}。下面我们用正弦线和余弦线来研究
$\sin\alpha$与$\cos\alpha$的变化规律。

\begin{itemize}
\item 当$\alpha=0^{\circ}$时,$\overline{MP}=0$, 
$\overline{OM}=1$
\item 当$\alpha=90^{\circ}$时,$\overline{MP}=1$, 
$\overline{OM}=0$
\end{itemize}
我们就说,
\[\sin0^{\circ}=0,\qquad
\cos0^{\circ}=1,\qquad
\sin90^{\circ}=1,\qquad
\cos90^{\circ}=0.\]
我们使角$\alpha$从$0^{\circ}$逐渐增加到$90^{\circ}$, 于是从角$\alpha$的正弦线
和余弦线的变化规律可以看到,\textbf{当
$\alpha$增大时,$\sin\alpha$随着增
大,而$\cos\alpha$随着减小;反之,当$\alpha$减小时,$\sin\alpha$随着减小,
而$\cos\alpha$随着增大。}

在图6.7中,过$A$点作$AA'\bot OA$, 与角$\alpha$的一边$OP$相
交于$Q$点,于是,
\[\tan\alpha=\frac{\overline{AQ}}{\overline{OA}}=\overline{AQ}\text{的量数}\]
$\overline{AQ}$叫做角$\alpha$的\textbf{正切线}。

在图6.7中,过单位圆与$y$轴的交点$B$作$BB'\bot OB$, 角
$\alpha$的一边$OP$与$BB'$相交于$T$点,于是,
\[\cot\alpha=\frac{\overline{BT}}{\overline{OB}}=\overline{BT}\text{的量数}\]
$\overline{BT}$叫做角$\alpha$的\textbf{余切线}。

下面我们用正切线和余切线来说明角$\alpha$的正切和余切
随着角$\alpha$的变化规律。

\begin{itemize}
    \item 当$\alpha =0^{\circ}$时,$\overline{AQ}=0$, 边$OP$与$BB'$不相交,我们就说$\tan0^{\circ}=0$,
    $\cot 0^{\circ}$不存在。
    \item 当$\alpha =90^{\circ}$时,$AQ$与$OP$不相交,$\overline{BT}=0$, 
    我们就说,
    $\tan90^{\circ}$不存在,
    $\cot 90^{\circ}=0$.
\end{itemize}

我们使角$\alpha$从$0^{\circ}$增加到$90^{\circ}$, 于是从角$\alpha$的正切线和余
切线的变化规律可以看到,\textbf{当$\alpha$增大时,$\tan\alpha$也随着增大,
而$\cot\alpha$测随着减小。反之当$\alpha$减小时,$\tan\alpha$也随着减小,
而$\cot\alpha$则随着增大。}


\begin{ex}
\begin{enumerate}
    \item 在横线上填入不等号($\alpha$、$\beta$都是锐角)。
    \begin{enumerate}
    \item 当$\alpha>\beta$时,$\sin\alpha\underline{\quad}\sin\beta$,$\cos\alpha\underline{\quad}\cos\beta$,
    $\tan\alpha\underline{\quad}\tan\beta$,$\cot\alpha\underline{\quad}\cot\beta$.
    \item 当$\alpha<\beta$时,$\sin\alpha\underline{\quad}\sin\beta$,$\cos\alpha\underline{\quad}\cos\beta$,
    $\tan\alpha\underline{\quad}\tan\beta$,$\cot\alpha\underline{\quad}\cot\beta$.
    \end{enumerate}

    \item 指出下列差的符号:
\begin{multicols}{2}
\begin{enumerate}
    \item $\sin34^{\circ}-\sin33^{\circ}$
    \item $ \sin27^{\circ}-\sin26^{\circ}$
    \item $\cos83^{\circ}-\cos84^{\circ}$
    \item $\cos10^{\circ}-\cos9^{\circ}$
    \item $\tan 5^{\circ}-\tan 6^{\circ}$
    \item $\cot 14^{\circ}-\cot 13^{\circ}$
    \item $\tan 46^{\circ}-\tan 44^{\circ}$
    \item $\cot 44^{\circ}-\cot 47^{\circ}$
\end{enumerate}
\end{multicols}
\end{enumerate}
\end{ex}

\subsection{$30^{\circ}$、$45^{\circ}$、$60^{\circ}$角的三角比}
我们根据锐角三角比的定义和直角三角形中的一些边角
特殊关系,可以计算出$30^{\circ}$、$45^{\circ}$、$60^{\circ}$角的三角比的精确
值。

作$\triangle ABC$, 使$\angle C=90^{\circ}$,
$\angle A=30^{\circ}$ (图6.8), 那么
$\angle B=60^{\circ}$, 设$\overline{BC}=a$, 则$\overline{AB}=2a$, $\overline{AC}=\sqrt{\overline{AB}^2-\overline{BC}^2}=
\sqrt{(2a)^2-a^2}=\sqrt{3}a$. 
由此得:
\[\begin{split}
    \sin 30^{\circ}=\frac{a}{2a}=\frac{1}{2},&\qquad \sin 60^{\circ}=\frac{\sqrt{3}a}{2a}=\frac{\sqrt{3}}{2}\\
    \cos 30^{\circ}=\frac{\sqrt{3}a}{2a}=\frac{\sqrt{3}}{2},&\qquad \cos 60^{\circ}=\frac{a}{2a}=\frac{1}{2}\\  
\end{split}\]
\[\begin{split}
    \tan 30^{\circ}=\frac{a}{\sqrt{3}a}=\frac{1}{\sqrt{3}}=\frac{\sqrt{3}}{3},&\qquad \tan 60^{\circ}=\frac{\sqrt{3}a}{a}=\sqrt{3}\\
    \cot 30^{\circ}=\frac{\sqrt{3}a}{a}=\sqrt{3},&\qquad \cot 60^{\circ}=\frac{a}{\sqrt{3}a}=\frac{1}{\sqrt{3}}=\frac{\sqrt{3}}{3}\\  
\end{split}\]

\begin{figure}[htp]\centering
    \begin{minipage}[t]{0.48\textwidth}
    \centering
\begin{tikzpicture}[>=latex, scale=1]
     \draw(0,0)node[left]{$A$}--node[below]{$\sqrt{3}a$}(2*1.732,0)node[right]{$C$}--node[right]{$a$}(2*1.732,2)node[above]{$B$}--node[left]{$2a$}(0,0);  
     \draw(2*1.732,0)rectangle (2*1.732-.2,.2);
    \end{tikzpicture}
    \caption{}
    \end{minipage}
    \begin{minipage}[t]{0.48\textwidth}
    \centering
    \begin{tikzpicture}[>=latex, scale=1]
 \draw(0,0)node[left]{$A$}--node[below]{$a$}(3,0)node[right]{$C$}--node[right]{$a$}(3,3)node[above]{$B$}--node[left]{$\sqrt{2}a$}(0,0);     
 \draw(3,0)rectangle (3-.2,.2);
    \end{tikzpicture}
    \caption{}
    \end{minipage}
    \end{figure}

作$\triangle ABC$, 使$\angle C=90^{\circ}$, $\angle A=45^{\circ}$ (图6.9), 那么,
$\angle B=45^{\circ}$, 设$\overline{BC}=a$, 则$\overline{AC}=a$, $\overline{AB}=\sqrt{a^2+a^2}=\sqrt{2}a$

由此得:
\[\sin 45^{\circ}=\frac{a}{\sqrt{2}a}=\frac{1}{\sqrt{2}}=\frac{\sqrt{2}}{2},\qquad \cos 45^{\circ}=\frac{a}{\sqrt{2}a}=\frac{1}{\sqrt{2}}=\frac{\sqrt{2}}{2}\]
\[\tan 45^{\circ}=\frac{a}{a}=1,\qquad \cot 45^{\circ}=\frac{a}{a}=1\]
为了便于记忆,我们把$30^{\circ},45^{\circ},60^{\circ}$的三角比列成表。
\begin{center}
    \begin{tabular}{c|ccc}
        \hline
$\alpha$&$30^{\circ}$&$45^{\circ}$&$60^{\circ}$\\
\hline
$\sin\alpha$  &  $\frac{1}{2}$ & $\frac{\sqrt{2}}{2}$& $\frac{\sqrt{3}}{2}$\\
$\cos\alpha$  &  $\frac{\sqrt{3}}{2}$ & $\frac{\sqrt{2}}{2}$& $\frac{1}{2}$\\
$\tan\alpha$  &  $\frac{\sqrt{3}}{3}$ &1&$\sqrt{3}$\\
$\cot\alpha$  &  $\sqrt{3}$ &1&$\frac{\sqrt{3}}{3}$\\
\hline
    \end{tabular}
\end{center}

\begin{example}
    计算 $4\cot 30^{\circ}-2\sin60^{\circ}+2\cos60^{\circ}$
\end{example}

\begin{solution}
\[4\cot 30^{\circ}-2\sin60^{\circ}+2\cos60^{\circ} =4\x \sqrt{3}-2\x \frac{\sqrt{3}}{2}+2\x \frac{1}{2}=3\sqrt{3}+1
\]
\end{solution}


\begin{example}
    计算
$\cos^2 30^{\circ}+\sin^2 45^{\circ}-\tan^2 45^{\circ}$
其中:$\sin^\alpha$、$\cos^2\alpha$
$\tan^2\alpha$、$\cot^2\alpha$分别表示$(\sin\alpha)^2$、$(\cos \alpha)^2$、$(\tan\alpha)^2$、$(\cot \alpha)^2$
\end{example}


\begin{solution}
\[\cos^2 30^{\circ}+\sin^2 45^{\circ}-\tan^2 45^{\circ}=\left(\frac{\sqrt{3}}{2}\right)^2+\left(\frac{\sqrt{2}}{2}\right)^2-1^2=\frac{3}{4}+\frac{2}{4}-1=\frac{1}{4}\]
\end{solution}

\begin{ex}
    求下列各式之值:
\begin{enumerate}
\item $\sin^2 60^{\circ}+\cos^2 30^{\circ}$
\item $\sin^2 60^{\circ}+\cos^2 60^{\circ}$
\item $\sin^2 45^{\circ}+\cos^2 45^{\circ}$
\item $2\sin30^{\circ}+2\cos60^{\circ}+4\tan 45^{\circ}$
\item $5\tan 30^{\circ}+\cot 45^{\circ}-2\tan 45^{\circ}+2\cos 60^{\circ}$  
\item $\frac{2\sin30^{\circ}}{2\cos30^{\circ}-1}$
\item $\frac{\sin60^{\circ}-\sin30^{\circ}}{\sin60^{\circ}+\sin30^{\circ}}$
\end{enumerate}
\end{ex}

\subsection{三角比值表}
在前面的内容中,我们讲的只是特殊角的三角比,为了应用方
便,人们早已制定了任意锐角的三角比值表,下面就来介绍
四位三角比值表的用法。

\subsubsection{正弦表,余弦表}
在正弦、余弦表里左右各有一列排度数,左列上端和右
列下端都有$A$字,在左列$A$的下面,由上到下排着度数,
在右列$A$的上面,由下到上排着度数,在顶行$A$的右边,
由左至右依次排着$0',6',\ldots,60'$, 在表的顶上写着正弦表,
说明查正弦时用左列$A$下面的度数和顶行的分数,表的底下
写着余弦表,说明查余弦时,用右列$A$上的度数和底行的分
数。例如要查$26^{\circ}18'$的正弦,在表中左列$A$的下面先找到
$26^{\circ}$, 顺着$26^{\circ}$所在的这一行往右,在顶行$18'$所在的这一列
里找到了一个数0.4431, 就是$26^{\circ}18'$的正弦,即$\sin26^{\circ}18'=
0.4431$. 换句话说左列$A$下面的$26^{\circ}$所在的行和顶行$18'$所
在的列的交点处的0.4431, 就是$\sin26^{\circ}18'$. 要查$\cos27^{\circ}24'$, 
在表中右列$A$的上面找到$27^{\circ}$, 底行里找到$24'$, $27^{\circ}$所在的
行和$24'$所在的列的交点处的0.8878, 就是$\cos27^{\circ}24'$, 即
$\cos27^{\circ}24'=0.8878$.

\subsubsection{正切表、余切表}
正切的查法和正弦相同,余切的查法和余弦相同,例如
我们在正切表、余切表中可以查到:
\[\tan 54^{\circ}30'=1.4019,\qquad \cot 4^{\circ}6'=13.95\]

\begin{ex}
    查表求下列各三角比:
\begin{enumerate}
    \item $\sin14^{\circ},\quad \sin20^{\circ}24',\quad \sin65^{\circ}30',\quad \sin82^{\circ}12'$
    \item $\cos7^{\circ},\quad  \cos32^{\circ}6',\quad  \cos60^{\circ}54',\quad \cos83^{\circ}18'$
    \item $\tan 18^{\circ},\quad  \tan 78^{\circ}36',\quad  \tan 80^{\circ}24',\quad  \tan 83^{\circ}$
    \item $\cot 42^{\circ}42',\quad   \cot20^{\circ}48', \quad  \cot9^{\circ}36', \quad  \cot 5^{\circ}30'$
\end{enumerate}
\end{ex}

在三角比值表中,最右边的三列是修正值,它是用来求
在左边表里找不到的角的三角比,这三列的上端和下端都标
有$1'$、$2'$、$3'$, 三列中的数是小数的简写,每一个数都代表
一个小数,它的末位数相当于表中间同一行小数的末位数,
$1'$、$2'$、$3'$各列中的各数,分别是它所在的行的角度分别相
差$1'$、$2'$、$3'$时的三角比的修正值,下面举列说明查法:

例如,要求$\sin20^{\circ}19'$, 先从表中查得$\sin20^{\circ}18'$的值是
0.3469, 因为$20^{\circ}19'$比$20^{\circ}18'$大$1'$, 查修正值是0.0003. 因
为角度大,它的正弦值也大,所以$\sin20^{\circ}19'$就比$\sin20^{\circ}18'$
大0.0003, 因此,$\sin20^{\circ}19'=0.3469+0.0003=0.3472$. 要
求$\sin20^{\circ}46'$, 先从表中查得$\sin20^{\circ}48'$的值是0.3551, $20^{\circ}46'$
比$20^{\circ}48'$小$2'$, 查得$2'$的修正值是0.0005, 角度小,正弦值
也小,所以$\sin20^{\circ}46'=0.3551-0.0005=0.3546$. 要求
$\cos28^{\circ}26'$, 先从表中查得$\cos28^{\circ}24'=0.8796$, $28^{\circ}26'$比
$28^{\circ}24'$大$2'$, 查表得修正值是0.0003, 角大余弦值反而小,
所以$\cos28^{\circ}26'=0.8796-0.0003=0.8793$. 

正切的查法和正弦相同,余切的查法和余弦相同。

例如,我们可以求得
\[\tan 69^{\circ}25'=2.662,\qquad \cot 70^{\circ}45'=0.3492\]

\begin{ex}
    查表求下列各三角函数:
\begin{multicols}{2}
    \begin{enumerate}
        \item $\sin18^{\circ}19',\quad  \sin63^{\circ}40'$
        \item $\cos65^{\circ}2',\quad  \cos10^{\circ}34'$
        \item $\tan 9^{\circ}19',\quad  \tan64^{\circ}10'$
        \item $\cot25^{\circ}28',\quad  \cot10^{\circ}25'$
    \end{enumerate}
\end{multicols}
\end{ex}

从三角比值表里不但可以查得任何锐角的三角比,反过
来,也可以根据已知的三角比值查到未知的锐角。

\begin{example}
    已知$\sin x=0.9966$, 求锐角
    $x$.
\end{example}

\begin{solution}
    在正弦表里找到0.9966, 因为它是正弦的值,要用到左
    列的度和顶行的分,在0.9966这一行的左端是$85^{\circ}$, 在0.5966
    的上端是$18'$. 所以
   $ \sin85^{\circ}18'=0.9966$, 因此$x=85^{\circ}18'$.
\end{solution}

\begin{example}
已知$\cos y=0.9966$, 求锐角$y$.
\end{example}

\begin{solution}   
    在表中找到0.9966, 因为它是余弦的值,余弦要用到右
    列的度数和底行的分,在0.9966这一行的右端是$4^{\circ}$, 在
    0.9966这一列的下端是$42'$.

    所以$\cos4^{\circ}42'=0.9966$, 因此$y=4^{\circ}42'$.
\end{solution}

\begin{example}
   $\tan x=14.30$, $\cot y=1.4715$, 求$x$、$y$. 
\end{example}

\begin{solution}
    倒查正切表(查法和例6.6相同)可得:$x=86^{\circ}$.

倒查余切表(查法和例6.7相同)可得:$y=34^{\circ}12'$.
\end{solution}

由三角比值求角度,有时要用到修正值,用修正值时,
必须注意到,对于正弦、正切的值越大,角度也越大;对于
余弦、余切的值越大,角度反而越小,下面举例说明用修正
值查法。

\begin{example}
      $\sin x=0.2493$, 求$x$.
\end{example}

\begin{solution}
在正弦表里,和0.2493最接近的正弦值是0.2487, 它是
$14^{\circ}24'$的正弦,$0.2493-0.2487=0.0006$, 在$14^{\circ}$这一行里
正弦值相差0.0006时,角度的修正值是$2'$, $\sin x$比$\sin14^{\circ}24'$
大0.0006, $x$就比$14^{\circ}24'$大$2'$, 因此
\[x=14^{\circ}24'+2'=14^{\circ}26'\]
\end{solution}

\begin{example}
    $\cos y=0.9841$, 求$y$.
\end{example}

\begin{solution}
    在余弦表中和0.9841最接近的余弦值是0.9842, 它是
$10^{\circ}12'$的余弦,$0.9842-0.9841=0.0001$, 在$10^{\circ}$这一行里
余弦值相差0.0001时,角度的修正值是$1'$或$2'$, $\cos y$比
$\cos10^{\circ}12'$小0.0001, $y$比$12^{\circ}12'$大$1'$或$2'$, 因此
\[y=10^{\circ}12'+1'=10^{\circ}13'\]
或\[y=10^{\circ}12'+2'=10^{\circ}14'\]
这里$y$有两个答案,
一个是不足近似值,一个过剩近似值。
\end{solution}

\begin{example}
    $\tan x=1.3773$, $\cot=0.1950$, 求$x$、$y$.
\end{example}

\begin{solution}
倒查正切表,得$x=54^{\circ}1'$(查法和例6.9相同)。

倒查余切表,得$y=78^{\circ}51'$(查法和例6.10相同)。
\end{solution}

\begin{ex}
    由三角比值表求锐角$x$:
\begin{enumerate}
    \item $\sin x=0.9816,\quad
    \sin x=0.6639,\quad
    \tan x=9.595,\quad
    \tan x=0.1890$
    \item $\cos x=0.8607,\quad
    \cos x=0.9893,\quad
    \cot x=2.106,\quad
    \cot x=67.40$
    \item $\sin x=0.2476,\quad
    \sin x=0.9709,\quad
    \cos x=0.3372$
    
    $
    \cos x=0.4174,\quad
    \tan x=0.365,\quad
    \cot x=0.1614$
\end{enumerate}
\end{ex}

\subsection{互为余角的三角比间的关系}
\begin{figure}[htp]
    \centering
\begin{tikzpicture}[>=latex, scale=.8]
     \draw(0,0)node[left]{$A$}--node[below]{$b$}(2,0)node[right]{$C$}--node[right]{$a$}(2,2*1.732)node[above]{$B$}--node[left]{$c$}(0,0);  
     \draw(2,0)rectangle (2-.2,.2);
    \end{tikzpicture}
    \caption{}
\end{figure}

在直角$\triangle ABC$中,如果$\angle C=90^{\circ}$(图6.10), 则
\[\angle A+\angle B=90^{\circ},\qquad \angle B=90^{\circ}-\angle A\]
由于$\angle A$的对边是$\angle B$的邻边,
$\angle B$的对边是$\angle A$的邻边,那么,根
据三角比的定义,我们便得出$\angle A$和
$\angle B$这两个互为余角的三角比之间有
下面的关系:
\[\begin{split}
    \sin A=\frac{a}{c}=\cos B=\cos(90^{\circ}-A),&\qquad \cos A=\frac{b}{c}=\sin B=\sin(90^{\circ}-A)\\
    \tan A=\frac{a}{b}=\cot B=\cot(90^{\circ}-A),&\qquad \cot A=\frac{b}{a}=\tan B=\tan(90^{\circ}-A)\\
\end{split}\]

这就是说,\textbf{互为余角的两个角中,任一角的正弦等于另
一角的余弦;任一角的正切等于另一角的余切。}

有了这个关系,我们就可把任意大于$45^{\circ}$的锐角的三角
比化为小于$45^{\circ}$的锐角的三角比。

\begin{example}
    把下面各角的三角比化为小于$45^{\circ}$的锐角的三角
    比。
\begin{multicols}{2}
\begin{enumerate}
\item $\sin75^{\circ}$
\item $\cos62^{\circ}22'$
\item $\tan 80^{\circ}$
\item $\cot56^{\circ}18'$
\end{enumerate}
\end{multicols}
\end{example}

\begin{solution}
\begin{enumerate}
    \item $\sin75^{\circ}=\cos(90^{\circ}-75^{\circ})=\cos 15^{\circ}$
    \item $\cos62^{\circ}22'=\sin(90^{\circ}-62^{\circ}22')=\sin27^{\circ}38'$
    \item $\tan 80^{\circ}=\cot(90^{\circ}-80^{\circ})=\cot10^{\circ}$
    \item $\cot 56^{\circ}18'=\tan(90^{\circ}-56^{\circ}18')=\tan 33^{\circ}42'$
\end{enumerate}
\end{solution}


\begin{example}
    下列等式是否成立:
\begin{enumerate}
\item $\sin(60^{\circ}+\alpha )=\cos(30^{\circ}-\alpha ) \qquad (0\le \alpha \le 30^{\circ})$
\item $\sin(45^{\circ}+\alpha )=\cos(45^{\circ}-\alpha ) \qquad (0\le \alpha \le 45^{\circ})$
\item $\tan(50^{\circ}+\alpha )=\cot (40^{\circ}-\alpha ) \qquad (0\le \alpha <40^{\circ})$
\end{enumerate}
\end{example}


\begin{solution}
由于
\[\begin{split}
    (60^{\circ}+\alpha )+(30^{\circ}-\alpha )&=90^{\circ}\\
(45^{\circ}+\alpha )+(45^{\circ}-\alpha )&=90^{\circ}\\
(50^{\circ}+\alpha )+(40^{\circ}-\alpha )&=90^{\circ}
\end{split}\]

而$\alpha $角的取值范围使得各式中的三角比都有意义,所以
根据互为余角的三角比之间的关系可知,各等式都成立。
\end{solution}

\begin{ex}
    \begin{enumerate}
        \item 为什么三角比值表中,锐角$\alpha$的正弦和$90^{\circ}-\alpha$角的余弦共
        用一个表。
\item 把下列各角的三角比化为小于$45^{\circ}$的锐角的三角比。
\begin{multicols}{2}
\begin{enumerate}
    \item $\sin73^{\circ},\qquad \sin77^{\circ}18'$
    \item $\cos57^{\circ},\qquad \cos52^{\circ}38'$
    \item $\tan78^{\circ},\qquad \tan79^{\circ}5'$
    \item $\cot48^{\circ},\qquad \cot78^{\circ}31'$
\end{enumerate}
\end{multicols}
\item 下列各式中的$x$应为多少度?
\begin{multicols}{2}
\begin{enumerate}
    \item $\sin75^{\circ}=\cos x$
    \item $\cos18^{\circ}=\sin x$
    \item $\tan5^{\circ}=\cot x$
    \item $\cot83^{\circ}=\tan x$
\end{enumerate}
\end{multicols}
\item 下列等式是否成立($x$、$\alpha$的取值都使各三角比有意义)。
\begin{multicols}{2}
\begin{enumerate}
    \item $\sin(75^{\circ}+\alpha )=\cos(15^{\circ}-\alpha )$
    \item $\sin(15^{\circ}-\alpha )=\sin(30^{\circ}+\alpha )$
    \item $ \tan (30^{\circ}+x)=\cot (60^{\circ}-x)$
    \item $ \cot (89^{\circ}+\alpha )=\tan (1^{\circ}-\alpha )$
\end{enumerate}
\end{multicols}
    \end{enumerate}
\end{ex}

\subsection{同一锐角的各三角比间的关系}
\begin{blk}{定理}
    同一锐角$\alpha$ 的四个三角比之间有下列关系:
\begin{enumerate}
    \item $\sin^2\alpha  +\cos^2\alpha =1$
    \item $\tan\alpha=\frac{\sin\alpha}{\cos\alpha },\qquad \cot\alpha=\frac{\cos\alpha }{\sin\alpha }$
    \item $\tan\alpha\cdot \cot\alpha =1$
\end{enumerate}
\end{blk}

\begin{proof}
    作直角$\triangle ABC$, 使$\angle C=90^{\circ}$, $\angle A=\alpha$ (图6.11).
    \begin{figure}[htp]
        \centering
    \begin{tikzpicture}[>=latex, scale=1.2]
         \draw(0,0)node[left]{$A$}--node[below]{$b$}(2,0)node[right]{$C$}--node[right]{$a$}(2,1.5)node[above]{$B$}--node[left]{$c$}(0,0);  
         \draw(2,0)rectangle (2-.2,.2);
         \draw (.5,0) arc (0:36.87:.5)node[right]{$\alpha$};
        \end{tikzpicture}
        \caption{}
    \end{figure}
    
\begin{enumerate}
    \item $\because\quad  a^2+b^2=c^2$
    
    $\therefore\quad \left(\frac{a}{c}\right)^2+\left(\frac{b}{c}\right)^2=1$

    又$\because\quad \frac{a}{c}=\sin\alpha,\quad \frac{b}{c}=\cos\alpha$,

    $\therefore\quad \sin^2\alpha  +\cos^2\alpha =1$,即:
\textbf{一锐角的正弦和余弦的平方和等于1.}
该式也就是勾股定理的三角比的表示式。

\item $\because\quad \tan\alpha=\frac{a}{b},\quad \frac{\sin\alpha}{\cos\alpha}=\frac{\frac{a}{c}}{\frac{b}{c}}=\frac{a}{b}$,

$\therefore\quad \tan\alpha=\frac{\sin\alpha}{\cos\alpha}$

又$\because\quad \cot\alpha=\frac{b}{a},\quad \frac{\cos\alpha}{\sin\alpha}=\frac{\frac{b}{c}}{\frac{a}{c}}=\frac{b}{a}$,

$\therefore\quad \cot\alpha=\frac{\cos\alpha}{\sin\alpha}$

\item $\because\quad \tan\alpha=\frac{a}{b},\quad \cot\alpha=\frac{b}{a}$

$\therefore\quad \tan\alpha\cdot \cot\alpha=\frac{a}{b}\cdot \frac{b}{a}=1$
\end{enumerate}
\end{proof}

上面的定理,表示了同一锐角的三角比之间的关系,我
们可利用定理中的三个公式,由已知锐角的一个三角比,去
计算这个角的其它的三个三角比;也可以利用它们来化简含
有三角比的式子。


\begin{example}
    已知:$\sin\alpha=\frac{3}{5}$, 
    求$\alpha$角($\alpha$为锐角)的其它
    的三个三角比。
\end{example}


\begin{solution}
 从公式$\sin^2\alpha +\cos^2\alpha =1$ 可得
   \[ \cos\alpha =\pm\sqrt{1-\sin^2\alpha}\] 
   由于锐角的三角比都是正
    数,所以根号前应取正号,把$\sin\alpha =\frac{3}{5}$
    代入上式,得
\[\cos\alpha=\sqrt{1-\left(\frac{3}{5}\right)^2}=\sqrt{\frac{25-9}{5^2}}=\frac{4}{5}\]
根据$\tan\alpha=\frac{\sin\alpha}{\cos\alpha}$,可得
\[\tan\alpha=\frac{\frac{3}{5}}{\frac{4}{5}}=\frac{3}{4}\]
再根据$\tan\alpha\cdot \cot\alpha=1$,可得
\[\cot\alpha=\frac{4}{3}\]
\end{solution}

\begin{example}
    化简
    $\sin^2 54^{\circ}+\sin^2 36^{\circ}-\tan 45^{\circ}$
\end{example}

\begin{solution}
\[\begin{split}
    \sin^2 54^{\circ}+\sin^2 36^{\circ}-\tan 45^{\circ}&=
    \cos^2(90^{\circ}-54^{\circ})+\sin^2 36^{\circ}-\tan 45^{\circ}\\
    &=\cos^2 36^{\circ}+\sin^2 36^{\circ}-1\\
    &=1-1=0
\end{split}\]
\end{solution}


\begin{example}
    化简$\frac{\sqrt{1-\sin^2\alpha}}{\sin\alpha}\cdot \tan\alpha$
\end{example}

\begin{solution}
    \[\frac{\sqrt{1-\sin^2\alpha}}{\sin\alpha}\cdot \tan\alpha=\frac{\cos\alpha}{\sin\alpha}\cdot \tan\alpha=\cot\alpha\cdot \tan\alpha=1\]
\end{solution}

\begin{example}
    化简$(\sin\alpha+\cos\alpha)^2+(\sin\alpha-\cos\alpha)^2$
\end{example}

\begin{solution}
\[\begin{split}
&    (\sin\alpha+\cos\alpha)^2+(\sin\alpha-\cos\alpha)^2\\  
&\quad =\sin^2\alpha+2\sin\alpha\cdot \cos\alpha+\cos^2\alpha
+\sin^2\alpha-2\sin\alpha\cdot\cos\alpha+\cos^2\alpha\\
&\quad =1+1=2
\end{split}\]    
\end{solution}

\begin{ex}
\begin{enumerate}
    \item 证明$\sin^2\alpha+\cos^2\alpha=1$, $\tan\alpha=\frac{\sin\alpha}{\cos\alpha}$
    \item 已知$\sin\alpha=\frac{5}{13}$,求$\cos\alpha, \tan\alpha, \cot\alpha$
    \item 已知$\cos\alpha=\frac{4}{5}$,求$\sin\alpha, \tan\alpha, \cot\alpha$
    \item 已知$\tan\alpha=\frac{3}{4}$,求$\sin\alpha,\cos\alpha$。
    
提示:$\tan^2\alpha =\frac{\sin^2\alpha}{1-\sin^2\alpha}=\frac{9}{16}$,令$\sin \alpha =x$,解方程。

\item 化简

\begin{enumerate}\begin{multicols}{2}
    \item $\frac{1-\sin^2\alpha}{\cos^2\alpha}$
    \item $\tan\alpha\cdot \cos\alpha$
    \item $\frac{\cos\alpha}{\sqrt{1-\cos^2\alpha}}$
    \item $\frac{\sin^2\alpha}{1+\cos\alpha}$
    \item $\frac{\cos^2\alpha}{1-\sin\alpha}$\end{multicols}
    \item $(\tan\alpha+\cot\alpha)^2-(\tan\alpha-\cot\alpha)^2$
\end{enumerate}

\end{enumerate}
\end{ex}

\subsection*{习题6.1}
\begin{enumerate}
    \item 在直角$\triangle ABC$中,$\angle C=90^{\circ}$, $a=3$, $b=2$, 求$\angle A$的四
    个三角比。
    \item 在直角$\triangle ABC$中,$\angle C=90^{\circ}$,$\overline{AB}=10$, $\overline{BC}=8$, 求:
   $ \sin A$、$\cos A$.
    \item 在直角$\triangle ABC$中,$\angle C=90^{\circ}$, $\overline{CD}$为$\overline{AB}$边上的高,问
    图中哪些线段的比可以表
    示$\angle A$的正弦?哪些线段
    的比可以表示$\angle B$的正弦?
\begin{figure}[htp]
    \centering
    \begin{tikzpicture}
\draw(0,0)node[below]{$A$}--(5,0)node[below]{$B$}--(3.2,2.4)node[above]{$C$}--(0,0);
\draw(3.2,2.4)--(3.2,0)node[below]{$D$};
\draw(3.2,0) rectangle (3.2+.2,.2);        
\draw (0.4,0) arc (0:32:.4);
\draw (3.2,2) arc (-90:-90+32:.4);
    \end{tikzpicture}
    \caption*{第3题}
\end{figure}

    \item 判断下列等式是否正确:
\begin{enumerate}
\item $\sin35^{\circ} 30'=\cos54^{\circ} 30'$
\item $\sin(30^{\circ} -\alpha )=\cos(60^{\circ} +\alpha )$
\item $\cos(2\alpha +14^{\circ} )=\sin(76^{\circ} -2\alpha )$
\end{enumerate}

\item 化简下列各式:
\begin{enumerate}
\item $\sin^4\alpha  +2\sin^2\alpha \cos^2\alpha +\cos^4\alpha $
\item $\sin^4\alpha -\cos^4\alpha +\cos^2\alpha $
\item $\sin(45^{\circ} +\alpha )-\cos(45^{\circ} -\alpha )$    
\item $\frac{1-\sin^2\alpha }{1-\cos^2\alpha }$
\item $\frac{\tan \alpha +\tan \beta}{\cot\alpha +\cot \beta}$
\item $\sin(90^{\circ} -\alpha )\cdot \cot(90^{\circ} -\alpha )$
\end{enumerate}

\item 已知 $\tan \alpha =2$, 求 $\tan(90^{\circ} -\alpha )$.
\item 已知
$\sin A=\frac{1}{3}$
求 $\cos A$, $\tan A$.
\item 已知互为余角的两角的正切的和等于3, 求两角正切的
平方和。
\item 求下列各题中的未知锐角$x$:
\begin{multicols}{2}
\begin{enumerate}
    \item $\sin x=\frac{1}{2}$
    \item $\cos x=\frac{\sqrt{2}}{2}$
    \item $\tan x=\sqrt{3}$
    \item $\sin x=\frac{\sqrt{3}}{2}$
    \item $\cos x=\frac{1}{2}$
    \item $\tan x=\frac{1}{\sqrt{3}}$
\end{enumerate}
\end{multicols}

\item 已知 $\cos^2 x=\frac{1}{4}$,求$x$.
\item 已知 $\cos^2x-\sin^2x=\frac{1}{2}$, 
求$x$.
\item 回答下列问题:
\begin{enumerate}
\item 当$0^{\circ}\le \alpha\le 90^{\circ}$时,$\sin\alpha$的最大值是多少?最小值
是多少?
\item 当$0^{\circ}\le \alpha\le 90^{\circ}$时,$\cos\alpha$的最大值是多少?最小值
是多少?
\end{enumerate}


\item 当$0^{\circ}\le \alpha\le 45^{\circ}$, $\alpha=45^{\circ}$, $45^{\circ}<\alpha\le 90^{\circ}$时,分别比较
$\sin\alpha$与$\cos\alpha$的大小。
\end{enumerate}

\section{解直角三角形}
\subsection{直角三角形中的边角关系}

我们把学过的直角三角形中
的边角基本关系总结如下:
\begin{figure}[htp]
    \centering
\begin{tikzpicture}[>=latex, scale=.8]
     \draw(0,0)node[left]{$A$}--node[below]{$b$}(2,0)node[right]{$C$}--node[right]{$a$}(2,2.5)node[above]{$B$}--node[left]{$c$}(0,0);  
     \draw(2,0)rectangle (2-.2,.2);
    \end{tikzpicture}
    \caption{}
\end{figure}

已知$\triangle ABC$, $\angle C=90^{\circ}$ (图6.12).
\begin{enumerate}
    \item 勾股定理:$a^2+b^2=c^2$
    \item 两锐角互余:$\angle A+\angle B=90^{\circ}$
    \item 四个三角比:
\[\sin A=\frac{\angle A\text{的对边}}{\angle A\text{斜边}},\qquad \cos A=\frac{\angle A\text{的邻边}}{\angle A\text{斜边}}\]
\[\tan A=\frac{\angle A\text{的对边}}{\angle A\text{的邻边}},\qquad \cot A=\frac{\angle A\text{的邻边}}{\angle A\text{的对边}}\]
\end{enumerate}
    为了应用方便,我们把3中的四个公式写成下面的形
    式:
\[\angle A\text{的对边}=\text{斜边}\x \sin A,\qquad \angle A\text{的邻边}=\text{斜边}\x \cos A\]
\[\angle A\text{的对边}=\text{邻边}\x \tan A,\qquad \angle A\text{的邻边}=\text{对边}\x \cot A\]

改用语言叙述,就是:
\begin{blk}{}
 在直角三角形中:
\begin{enumerate}
\item 一条直角边等于斜边乘上这条直角边所对锐角的
正弦。
\item 一条直角边等于斜边乘上这条直角边相邻锐角的
余弦。
\item 一条直角边等于另一条直角边乘上这条直角边所
对锐角的正切。
\item 一条直角边等于另一条直角边乘上这条直角边相
邻锐角的余切。
\end{enumerate}
\end{blk}

\begin{ex}
    \begin{enumerate}
        \item 在直角$\triangle ABC$中,$\angle C=90^{\circ}$, 求证:
\[\overline{AB}=\frac{\overline{BC}}{\sin A},\qquad \overline{AB}=\frac{\overline{AC}}{\cos A}\]
       并把这两个式子用语言叙述出来。
        \item 在直角$\triangle ABC$中,$\angle C=90^{\circ}$,求证:
\begin{enumerate}
    \item $\sin A=\cos B,\qquad      \sin B=\cos A$
    \item $\tan A=\cot B,\qquad \cot A=\tan B$
\end{enumerate}
    \end{enumerate}

\end{ex}

\subsection{解直角三角形}
根据三角形的某些已知元素求出它的未知元素,这种过
程叫做\textbf{解三角形}。在直角三角形中,直角总是已知的。除直
角外,只要再知道两个元素,其中至少有一边,就可以求出
直角三角形的其它各元素。因此,解直角三角形,只有下面
四种情况:
\begin{enumerate}
    \item 已知斜边与一锐角,
    \item 已知一条直角边与一锐角;
    \item 已知斜边与一条直角边;
    \item 已知两条直角边。
\end{enumerate}

上述四种情况,都可用前面中所列出的关系式,并利用
三角比值表来求解。

\begin{example}
    已知$c=18$, $\angle A=62^{\circ}20'$ (图6.13),
求:$\angle B, a, b$
\end{example}

\begin{solution}
\[\begin{split}
    \angle B&=90^{\circ}-\angle A=90^{\circ}-62^{\circ}20'
=27^{\circ}40'\\
a&=c\sin A=18\x \sin62^{\circ}20'=18\x0.8857
\approx 15.9\\
b&=c\cos A=18\x\cos62^{\circ}20'=18\x0.4643\approx 8.3600
\end{split}\]
\end{solution}

\begin{figure}[htp]\centering
    \begin{minipage}[t]{0.48\textwidth}
    \centering
\begin{tikzpicture}[>=latex, scale=2]
    \draw(0,0)node[left]{$A$}--node[below]{$b$}(.8358,0)node[right]{$C$}--node[right]{$a$}(62.33:1.8)node[above]{$B$}--node[left]{$c$}(0,0);  
    \draw(.8358,0)rectangle (.8358-.1,.1);
    \draw(.2,0) arc (0:62.33:.2);
    \end{tikzpicture}
    \caption{}
    \end{minipage}
    \begin{minipage}[t]{0.48\textwidth}
    \centering
    \begin{tikzpicture}[>=latex, scale=1.3]
\draw(0,0)node[left]{$A$}--node[below]{$b$}(1.567,0)node[right]{$C$}--node[right]{$a$}(55.27:2.75)node[above]{$B$}--node[left]{$c$}(0,0);  
    \draw(1.567,0)rectangle (1.567-.2,.2);
    \end{tikzpicture}
    \caption{}
    \end{minipage}
    \end{figure}

\begin{example}
    已知$c=27.5$, $a=22.6$, 求$b$、$\angle A$、$\angle B$ (图6.14).
\end{example}

\begin{solution}
\[b=\sqrt{c^2-a^2}=\sqrt{27.5^2-22.6^2}\approx 15.7\]
由于
$\sin A=\frac{a}{c}=\frac{22.6}{27.5}\approx 0.8218$,
因此:
\[\begin{split}
    \angle A&\approx 55^{\circ}16'\\
\angle B&=90^{\circ}-\angle A=90^{\circ}-55^{\circ}16'=34^{\circ}44'
\end{split}\]
\end{solution}


\begin{example}
    已知$a=3.8$, $\angle A=42^{\circ}$, 求$\angle B$, $C$, $b$ (图6.15).
\end{example}

\begin{solution}
\[\begin{split}
    B&=90^{\circ} -\angle A=48^{\circ} \\
b&=a \cot A=3.8\x\cot 42^{\circ}\approx 4.22\\
c&=\frac{a}{\sin A}=\frac{3.8}{\sin 42^{\circ}}\approx 5.68 
\end{split}\]
    
\end{solution}

\begin{figure}[htp]\centering
    \begin{minipage}[t]{0.48\textwidth}
    \centering
\begin{tikzpicture}[>=latex, scale=.7]
    \draw(0,0)node[left]{$A$}--node[below]{$b$}(4.22,0)node[right]{$C$}--node[right]{$a$}(42:5.68)node[above]{$B$}--node[left]{$c$}(0,0);  
    \draw(4.22,0)rectangle (4.22-.2,.2);
    \draw(.6,0) arc (0:42:.6);
    \end{tikzpicture}
    \caption{}
    \end{minipage}
    \begin{minipage}[t]{0.48\textwidth}
    \centering
    \begin{tikzpicture}[>=latex, scale=.8]
\draw(0,0)node[left]{$B$}--node[below]{$c$}(5,0)node[right]{$A$}--node[right]{$b$}(90-36.87:3)node[above]{$C$}--node[left]{$a$}(0,0);  
  
    \end{tikzpicture}
    \caption{}
    \end{minipage}
    \end{figure}


\begin{example}
    已知$a=3$、$b=4$, 求$c$、$\angle A$、$\angle B$ (图6.16)
\end{example}

\begin{solution}
\[c=\sqrt{a^2+b^2}=\sqrt{3^2+4^2}=5\]
由于$\tan A=\frac{3}{4}=0.75$,
因此:$$\angle A\approx 36^{\circ} 52',\qquad 
\angle B\approx 90^{\circ} -36^{\circ} 52'=53^{\circ} 8'$$ 
\end{solution}

\begin{ex}
\begin{enumerate}
    \item 解下列直角三角形:
    \begin{enumerate}
    \item 已知$c=58.5,\quad \angle A=45^{\circ} 13'$
    \item 已知$c=14,\quad \angle B=62^{\circ}$
    \item 已知$c=28,\quad \angle A=34^{\circ} $
    \item 已知$c=195,\quad \angle B=78^{\circ} 47'$
    \item 已知$a=87,\quad \angle A=55^{\circ} $
    \item 已知$b=99,\quad \angle B=83^{\circ} $
    \item 已知$c=32,\quad a=18$
    \item 已知$a=14,\quad \angle B=78^{\circ}$
    \item 已知$c=79,\quad b=56$
    \item 已知$a=12.8,\quad b=15.6$
    \item 已知$c=73,\quad \angle B=66.2^{\circ}$ 
    \item 已知$c=350,\quad \angle A=3.8^{\circ} $
\end{enumerate}

\item 已知直角$\triangle ABC$, $\angle C=90^{\circ}$, $\angle A=\alpha$和$\angle A$对的直角边是
$a$

求证:直角$\triangle ABC$的面积$S=\frac{a^2}{2}\tan\alpha$

\item  已知直角三角形的一个锐角为$\alpha$, 面积等于$S$, 
求它的外接圆的面积。
\end{enumerate}
\end{ex}

\subsection{解直角三角形的应用}
下面我们举例说明解直角三角形在实际中的应用。
\begin{example}
如图6.17, 已知在测点$C$处,测得一铁塔顶端$A$的
仰角$\angle ACE=\aleph$, $\overline{BD}=a$, 仪器的高度$\overline{CD}=b$, 求铁塔的高$\overline{AB}$.

\end{example}


\begin{solution}
\[\overline{AB}=\overline{AE}+\overline{EB}\]
已知$\overline{EB}=\overline{CD}=b$, 在$\triangle ACE$中,
\[\overline{AE}=\overline{CE}\x\tan\alpha\]
又$\overline{CE}=\overline{BD}=a$.

$\therefore\quad \overline{AB}=a\cdot \tan\alpha+b$
\end{solution}

\begin{figure}[htp]\centering
    \begin{minipage}[t]{0.48\textwidth}
    \centering
\includegraphics[scale=.5]{fig/6-17.png}
    \caption{}
    \end{minipage}
    \begin{minipage}[t]{0.48\textwidth}
    \centering
    \begin{tikzpicture}[>=latex, scale=1]
\begin{scope}
\draw(30:2)node[right]{$B$}--(0,0)node[left]{$A$}--(2,0)node[right]{$C$};
\node at (1,-.5){(1)};
\draw(0.5,0) arc (0:30:.5);
\end{scope}
\begin{scope}[xshift=3cm]
    \draw(2,1)node[right]{$C$}--(0,1)node[left]{$A$}--+(-30:2)node[right]{$B$};
    \node at (1,-.5){(2)};
    \draw(0.5,1) arc (0:-30:.5);
\end{scope}
    \end{tikzpicture}
    \caption{}
    \end{minipage}
    \end{figure}

\begin{rmk}
    如图6.18(1)连结测点$A$和目的物$B$, 并且经过$A$点画和
$AB$在同一铅直平面内的水平线$AC$, 如$AB$在$AC$的上方,那么
$\angle BAC$叫做仰角;如图6.18(2), 如果$AB$在$AC$的下方,那么
$\angle CAB$叫做俯角。
\end{rmk}  


\begin{example}
    如图6.19, 已知$C$、$D$两点与物体“$\overline{AB}$”的底端$B$
点共线,且
$\overline{CD}=a$, 仪器$\overline{CC'}$、$\overline{DD'}$的高都等于$b$, 在$C'$、
$D'$两点测得物体的顶端$A$的仰角分别是$\alpha$、$\beta$, 求物体的高
$\overline{AB}$
\end{example}

\begin{figure}[htp]\centering
    \begin{minipage}[t]{0.48\textwidth}
    \centering
    \includegraphics[scale=.5]{fig/6-19.png}
    \caption{}
    \end{minipage}
    \begin{minipage}[t]{0.48\textwidth}
    \centering
    \begin{tikzpicture}[>=latex, scale=1]
\draw(0,0)--node[below]{$\ell$}(4,0)--node[right]{$h$}(4,.8)--(0,0);
\draw(1,0) arc (0:11.3:1)node[right]{$\alpha$};
    \end{tikzpicture}
    \caption{}
    \end{minipage}
    \end{figure}


\begin{solution}
    在直角$\triangle AEC'$与直角$\triangle AED'$中,
\begin{align}
    \overline{C'E}&=\overline{AE}\x \cot\alpha\\
    \overline{D'E}&=\overline{AE}\x \cot\beta
\end{align}
$(6.1)-(6.2)$得:
\[\overline{D'E}-\overline{C'E}=\overline{AE} \x(\cot\beta-\cot\alpha)\]
即:$\overline{C'D'}=\overline{AE} \x(\cot\beta-\cot\alpha)$

已知$\overline{C'D'}=\overline{CD}=a$

$\therefore\quad \overline{AE}=\frac{a}{\cot\beta-\cot\alpha}$

但$\overline{AB}=\overline{AE}+\overline{EB}=\overline{AE} +b$, 

$\therefore\quad \overline{AB}=b+\frac{a}{\cot\beta-\cot\alpha}$

\end{solution}

如果要测底部可以到达的物体的高度,可用例6.22所介绍
的方法,如要测底部不能到达的的物体的高度,可用例6.23所
介绍的方法。

\begin{example}
    一条公路的路面升高$h$与水平距离$\ell$的比值$i$叫做
    路面的坡度。即$i=h:\ell$ (图6.20). 已知某段公路,每前进
    100米就升高4米,求路面的坡度及路面对水平面的倾角$\alpha$.
\end{example}

\begin{solution}
    路面的坡度$i=\frac{h}{\ell}$
    ,已知$h=4$米,$\ell=100$米,

$\therefore\quad  i=\frac{4}{100}=0.04$

    从图6.20中的直角三角形可
    见,$h:\ell$正好是角$\alpha$的正切,即:$i=\tan\alpha$,
    
    $\therefore\quad \tan\alpha=0.04$. 倒查正切表得:
    $\alpha=2^{\circ}17'$.

    答:路面的坡度是0.04, 路面对水平面的倾角是$2^{\circ}17'$.
\end{solution}

\begin{example}
   已知一门式起重
机(图6.21),机身高21
米,吊杆长$\overline{AB}=36$米,吊
杆的倾角$A$(即吊杆与水平
线的夹角)可以从$30^{\circ}$转到
$80^{\circ}$, 求这门起重机工作时
的最大高度和最大水平距
离。
\end{example}
    
\begin{figure}[htp]
    \centering
\includegraphics[scale=.6]{fig/6-21.png}
    \caption{}
\end{figure}

\begin{solution}
    当吊杆$\overline{AB}$的倾角
    $A$达到最大限度$80^{\circ}$时,这时
    起重机吊的最高位置如图6.
    21中$\overline{AB}$的位置.在直角
    $\triangle ABC$中,$\angle A=80^{\circ}$, 
   $\overline{AB}=36$米,因此:
\[    \overline{EC}=\overline{AB} \x \sin80^{\circ}=36\x\sin80^{\circ}
    =36\x0.9848\approx 35.45{\rm m}\]
\[\text{起重机的最大高度}=\text{机身高}+\overline{BC}=21+35.45=56.45{\rm m}\]

当倾角$A$达到最小角时,起吊的水平距离最远,这时,
如图6.21中$\overline{AB'}$的位置。在直角$\triangle AB'C'$中,$\angle A=30^{\circ}$, 
$\overline{AB'}=36$米。

因此:起重机起吊的最远水平距离
\[\overline{AC'}=\overline{AB'}\x\cos30^{\circ}=36\x0.8660
\approx 31.18{\rm m}\]

答:起重机工作的最大高度是56.45米,最远水平距离
是31.18米。
\end{solution}

\begin{ex}
\begin{enumerate}
    \item 测量学校旗杆的高度。
    \item 选择底部不能到达的建筑物或树本,测量它的高度。
    \item 如图,要求河两岸$B$、$C$两点间的距离,在$B$点这一岸垂
    直于$BC$方向上找一点$A$, 测出$\angle BAC=58^{\circ}12'$,
    $\overline{AB}=25$米,求$B$、$C$两点间的距离。
    \item 如图,从山顶$D$测得地平面上同一方向的两点$A$和$B$的俯
角分别是$18^{\circ}$和$23^{\circ}$, 已知$\overline{AB}=140$米,求山高
$\overline{CD}$(得数保留整数米)。
\item 已知传送带和地面的夹角是$25^{\circ}$, 它把物件从地面运到
离地面9米高的地方,求物件所走的路程。
\item 工件上有V形槽,测出上口宽200mm,深19.2mm,
求V形角$\alpha$多大。
\item 如图,在离地面高5米处引拉线固定电线杆,拉线和地
面成$60^{\circ}$角,求每根拉线多长,拉线底端离杆底多远。
\end{enumerate}
\end{ex}

\begin{figure}[htp]\centering
    \begin{minipage}[t]{0.3\textwidth}
    \centering
\includegraphics[scale=.7]{fig/6-3ti.PNG}
    \caption*{第3题}
    \end{minipage}
    \begin{minipage}[t]{0.6\textwidth}
    \centering
    \includegraphics[scale=.7]{fig/6-4ti.PNG}
    \caption*{第4题}
    \end{minipage}
    \end{figure}

\begin{figure}[htp]\centering
    \begin{minipage}[t]{0.48\textwidth}
    \centering
\begin{tikzpicture}[>=latex, scale=1]
\tkzDefPoints{0/0/A, 4/0/B, 4/2/C, 2.7/2/D, 2/1/E, 1.3/2/F, 0/2/G}
\tkzDrawPolygon(A,B,C,D,E,F,G)
\draw[dashed](F)--node[above]{20}(D);
\draw[<->](3.5,1)--node[fill=white]{19.2}(3.5,2);
\draw(E)--(3.7,1);
\tkzMarkAngle[mark=none, size=.3](D,E,F)
\node at (E)[above=.2]{$\alpha$};
    \end{tikzpicture}
    \caption*{第6题}
    \end{minipage}
    \begin{minipage}[t]{0.48\textwidth}
    \centering
    \begin{tikzpicture}[>=latex, scale=1]
\fill[pattern=north east lines](-2,-.25) rectangle (2.5,0);
\draw(-2,0)--(2.5,0);
\draw[very thick](-1.75,0)--(0,3.03)--(1.75,0);
\draw(-.04,0) rectangle (.04,3.5);
\draw(0,3.03)--(2.5,3.03);
\draw[<->](2.25,0)--node[fill=white]{5m}(2.25,3.03);
\draw(-1.25,0) arc (0:60:.5)node[right]{$60^{\circ}$};
    \end{tikzpicture}
    \caption*{第7题}
    \end{minipage}
    \end{figure}

\subsection*{习题6.2}
\begin{enumerate}
    \item 已知等腰$\triangle ABC$, 腰长为5cm, 底边长为8cm, 求
    $\angle A$、$\angle B$、$\angle C$.
    \item 把2米的竹杆,垂直于地面的时候,影长1.6米,这时
    太阳的仰角大约是多少度?
    \item  已知等腰$\triangle ABC$, 腰长$\overline{AB}=\overline{AC}=10$cm, 底角为$40^{\circ}$, 
    求高$\overline{AD}$, 底边$\overline{BC}$和面积$S$.
    \item 在$\triangle ABC$中,$\angle B$、
    $\angle C$是锐角,$\overline{AB}=c$, $\overline{AC}=b$, $\overline{AD}$是
    $\overline{BC}$边上的高,求证:
\begin{enumerate}
    \item $\overline{AD}=b\sin C=c\sin B$
    \item $\frac{b}{\sin B}=\frac{c}{\sin C}$
\end{enumerate}
\item 已知正$n$边形的半径是$r$,周长是$\ell$,求证:
\[\ell=2nr\sin\frac{180^{\circ}}{n}\]
\item  一架战斗机从3300米高空以每秒150米的速度向轰炸目
标俯冲,俯冲角为$42^{\circ}$, 在离地面1300米时投弹,击中
目标,问飞机开始俯冲到投弹共用几秒钟?
\begin{figure}[htp]\centering
    \begin{minipage}[t]{0.48\textwidth}
    \centering
\begin{tikzpicture}[>=latex, scale=1]
\draw(-2,0)--(4,0);
\draw(-1.5,0) rectangle (-.8,.5);
\draw(-1.3,.5) rectangle (-1,.9);
\draw[dashed](0,3.3)--(4,3.3);
\draw[dashed](0,1.3)--(3,1.3);
\draw[dashed](3,0)--(3,3.3);
\draw[very thick, ->](3,3.3)--(0,1.3);
\draw[<->] (3.5,3.3)--node[fill=white]{3300}(3.5,0);
\draw[<->] (1,1.3)--node[fill=white]{1300}(1,0);
    \end{tikzpicture}
    \caption*{第6题}
    \end{minipage}
    \begin{minipage}[t]{0.48\textwidth}
    \centering
    \begin{tikzpicture}[>=latex, scale=.8]
\draw(0,0)node[left]{$A$}--node[above]{2km}(4,0)node[right]{$B$};
\draw[dashed](0,-2)--(0,0)--(4,-4.767)node[below]{$C$}--(4,0);
\draw(0,-.5) arc (-90:-50:.5)node[below=4pt]{$40^{\circ}$};
    \end{tikzpicture}
    \caption*{第7题}
    \end{minipage}
    \end{figure}

\item 东西两炮台$A$、$B$相距2km, 同时发现入侵敌舰$C$, 炮台
$A$测得敌舰$C$在它的南偏东$40^{\circ}$的方向,炮台$B$测得敌舰$C$
在它的正南方,试求敌舰与两炮台的距离。

\item 
从圆外一点向圆引两条切线,已知切线长等于21.8cm, 
圆的半径等于10.6cm, 求这两条切线间的夹角。
\end{enumerate}

\section{任意三角形中的边角关系}
\subsection{正弦定理}

为了解任意的三角形,下面我们用三角比来研究任意三
角形中的边角关系。

我们约定在$\triangle ABC$中,$\angle A$、$\angle B$、$\angle C$分别用$\alpha$、$\beta$、$\gamma$
来表示;$\angle A$、$\angle B$、$\angle C$的对边分别用$a$、$b$、$c$来表示;外
接圆的半径用$r$来表示。

假定$\alpha$是锐角(图6.22),作$\triangle ABC$的外接圆$O$, 再作$\odot O$的直径$\overline{CA'}$, 弦$\overline{BA'}$, 在$\triangle A'BC$中,$\angle A'BC$是直角,因此$a=2r\sin A'$

$\because\quad \angle A'=\angle A=\alpha$

$\therefore\quad a=2r\sin\alpha$,因此:
\begin{equation}
\frac{a}{\sin\alpha}=r
\end{equation}

\begin{figure}[htp]\centering
    \begin{minipage}[t]{0.48\textwidth}
    \centering
\begin{tikzpicture}[>=latex, scale=1]
\draw[thick](0,0) circle(2);
\tkzDefPoints{0/0/O}
\tkzDefPoint(40:2){B}\tkzDefPoint(-30:2){C}\tkzDefPoint(-170:2){A}\tkzDefPoint(150:2){A'}
\tkzDrawPolygon(A,B,C)
\tkzDrawPolygon[dashed](A',B,C)
\tkzAutoLabelPoints[center=O](A,B,C,A')
\tkzLabelPoints[right](O)
\tkzDrawPoints(O)
\tkzMarkAngles[mark=none, size=.5](C,A,B C,A',B)
\node at (5:2)[left]{$a$};

    \end{tikzpicture}
    \caption{}
    \end{minipage}
    \begin{minipage}[t]{0.48\textwidth}
    \centering
    \begin{tikzpicture}[>=latex, scale=1]
  \draw[thick](0,0) circle(2);
\tkzDefPoint(120:2){B}\tkzDefPoint(-120:2){C}\tkzDefPoint(180:2){A}\tkzDefPoint(60:2){A'}    
\tkzLabelPoints[right](O)
\tkzDrawPoints(O)
\tkzDefPoints{0/0/O}
\tkzDrawPolygon(A,B,C)
\tkzDrawPolygon[dashed](A',B,C)
\tkzAutoLabelPoints[center=O](A,B,C,A')
\tkzMarkAngles[mark=none, size=.4](C,A,B B,A',C)

    \end{tikzpicture}
    \caption{}
    \end{minipage}
    \end{figure}

假定$\angle A$是钝角(图6.23),作$\triangle ABC$的外接圆$\odot O$, 再
作直径$\overline{CA'}$, 弦$\overline{BA'}$

$\because\quad \angle A+\angle A'=180^{\circ}$

$\therefore\quad \angle A'=180^{\circ}-\angle A$, $\angle A'$是一个锐角,

在直角$\triangle A'BC$中,$\angle A'BC$是直角。

$\therefore\quad a=2r\sin A'=2r\sin(180^{\circ}-A)=2r\sin(180^{\circ}-\alpha)$,因此:
\begin{equation}
    \frac{a}{\sin(180^{\circ}-\alpha)}=2r
\end{equation}

我们比较(6.3)、(6.4)两式,可以
看出,如果我们定义一个钝角的正弦
等于它的补角的正弦,即
\[\sin\alpha  =\sin(180^{\circ}-\alpha )\]
其中:$\alpha$ 是钝角。
那么(6.4)式在形式上就与(6.3)式相同
了,这就是说,$\alpha$ 不论是锐角还是钝角
都有:
\[\frac{a}{\sin\alpha}=2r\]

\begin{figure}[htp]
    \centering
\begin{tikzpicture}
    \draw[thick](0,0) circle(2);
    \tkzDefPoint(100:2){B}\tkzDefPoint(-80:2){C}\tkzDefPoint(-130:2){A}
    \tkzDefPoints{0/0/O}
    \tkzLabelPoints[right](O)    \tkzDrawPoints(O)
    \tkzDrawPolygon(A,B,C)
    \tkzAutoLabelPoints[center=O](A,B,C)
    \tkzMarkRightAngle[size=.2](C,A,B)
\end{tikzpicture}
    \caption{}
\end{figure}

当$\alpha$是直角时(图6.24),作直角$\triangle ABC$的外接圆$\odot O$
这时斜边$\alpha$正好是$O$的直径。

$\because\quad \sin90^{\circ}=1$

$\therefore\quad \frac{a}{\sin\alpha}=2r$
仍然成立。

同理,我们还可得出,
\[\frac{b}{\sin\beta}=2r,\qquad \frac{c}{\sin\gamma}=2r\]

总结上面的讨论,我们得到:

\begin{blk}
    {正弦定理}
在任意$\triangle ABC$中,边角关系满足:
\[\frac{a}{\sin\alpha}=\frac{b}{\sin\beta}=\frac{c}{\sin\gamma}=2r\]
\end{blk}

\begin{example}
    已知等边$\triangle ABC$的边长是$a$, 求$r$。
\end{example}

\begin{solution}
$\because\quad \frac{a}{\sin\alpha}=2r,\quad \alpha=60^{\circ}$

$\therefore\quad \frac{a}{\sin60^{\circ}}=2r$
\[r=\frac{a}{2\sin60^{\circ}}=\frac{a}{2\cdot \frac{\sqrt{3}}{2}}=\frac{\sqrt{3}}{3}a\]
\end{solution}

\begin{example}
    在$\triangle ABC$中,已知$b\sin\beta=c\sin\gamma$, 求证$\triangle ABC$是等腰三角形。
\end{example}   

\begin{proof}
$\because\quad b\sin\beta=c\sin\gamma$

$\therefore\quad \frac{b}{c}=\frac{\sin\gamma}{\sin\beta}$

又$\because\quad \frac{b}{\sin\beta}=\frac{c}{\sin\gamma}$

$\therefore\quad \frac{c}{b}=\frac{\sin\gamma}{\sin\beta},\quad \frac{b}{c}=\frac{c}{b},\quad b^2=c^2$

$\because\quad b>0,\quad c>0$

$\therefore\quad b=c$,$\triangle ABC$是等腰三角形。
\end{proof}

\begin{example}
    已知$P$点是$\triangle ABC$的$\overline{BC}$边上的任一点,求证:
\[\frac{\overline{BP}}{\overline{PC}}=\frac{c\sin \angle PAB}{b\sin\angle PAC}\]
\end{example}

\begin{proof}
    如图6.25所示,在$\triangle ABP$与$\triangle APC$中,根据正弦
    定理有:
\[\frac{\overline{BP}}{\sin \angle PAB}=\frac{c}{\sin \angle APB},\qquad \frac{\overline{PC}}{\sin \angle PAC}=\frac{b}{\sin\angle APC}\]
即:    
\[\overline{BP}=\frac{c\sin \angle PAB}{\sin \angle APB},\qquad \overline{PC}=\frac{b\sin \angle PAC}{\sin \angle APC}\]

$\because\quad \angle APB+\angle APC=\pi$    

$\therefore\quad \sin \angle APB=\sin\angle APC,\qquad \frac{\overline{BP}}{\overline{PC}}=\frac{c\sin\angle PAB}{b\sin\angle PAC}$
\end{proof}

\begin{figure}[htp]
    \centering
\begin{tikzpicture}[scale=.8]
\tkzDefPoints{0/0/B, 5/0/C, 2/3/A, 1.5/0/P}
\tkzDrawPolygon(A,B,C)
\tkzDrawSegments(A,P)
\tkzLabelPoints[above](A)
\tkzLabelPoints[below](B,C,P)
\node at (1,1.5)[left]{$c$};
\node at (3.5,1.5)[right]{$b$};
\end{tikzpicture}
    \caption{}
\end{figure}

\begin{ex}
\begin{enumerate}
    \item 求证在$\triangle ABC$中,
\begin{enumerate}
    \item $\sin\alpha=\frac{a}{b}\sin\beta=\frac{a}{c}\sin\gamma$
    \item $a+b+c=2r(\sin\alpha +\sin\beta+\sin\gamma)$
    \item $\sin\alpha+\sin\beta<\sin\gamma$
\end{enumerate}

\item 在$\odot O$中,量得一个$30^{\circ}$的圆周角所对的弦长是4cm,
求$\odot O$的半径$r$.
\item 求下列各角的正弦:
\[90^{\circ},\quad 120^{\circ},\quad 150^{\circ},\quad 135^{\circ},\quad 132^{\circ}\]
\item 在半径是5cm的圆中,$120^{\circ}$的圆心角所对的弦长是多
少?
\item 在半径是15cm的圆中,一条长18cm的弦所对的圆心角
是多少度(精确到度)。
\item 应用正弦定理证明三角形内角平分线定理(提示:模仿
例6.28的证法)。
\end{enumerate}
\end{ex}

\subsection{余弦定理}
在$\triangle ABC$中,如果$\beta$、$\gamma$都是锐角(图6.26),作$\overline{BC}$边
上的高$\overline{AD}$, 则$a=\overline{BD}+\overline{DC}$.

但$\overline{BD}=c\cos\beta$, $\overline{DC}=b\cos\gamma$,因此:
\begin{equation}
    a=c\cos\beta+b\cos\gamma
\end{equation}

如果$\beta$、$\gamma$中有一个是钝角,设$\gamma$是钝角(图6.27)作
$\overline{BC}$边上的高$\overline{AD}$, 这时垂足$D$落在$\overline{BC}$的延长线上,则$a=\overline{BD}-\overline{CD}$。但$\overline{BD}=c\cos\beta$, $\overline{CD}=b\cos(180^{\circ}-\gamma)$,因此:
\begin{equation}
    a=c\cos\beta -b\cos(180^{\circ}-\gamma)
\end{equation}

比较(6.5)、(6.6)两式,为了使它们在形式上得到一致,
我们定义钝角Y的余弦为:
\[\cos\gamma=-\cos(180^{\circ}-\gamma)\qquad \text{($\gamma$为钝角)}\]
这样,不论$\gamma$是锐角还是钝角都有关系式
\[a=c\cos\beta +b\cos\gamma\]
如果$\beta$ 是钝角,由$\cos\beta =-\cos(180^{\circ}-\beta )$同样可以得到:
\[a=c\cos\beta +b\cos\gamma\]

\begin{figure}[htp]\centering
    \begin{minipage}[t]{0.32\textwidth}
    \centering
\begin{tikzpicture}[>=latex, scale=1]
\tkzDefPoints{0/0/B, 3/0/C, 2/2/A, 2/0/D}
\tkzLabelPoints[below](B,C,D)
\tkzLabelPoints[above](A)   
\tkzDrawPolygon(A,B,C)
\node at (1,1)[left]{$c$};  \node at (2.5,1)[right]{$b$};
\node at (1,0)[below]{$a$};  \draw(A)--(D);
\tkzMarkRightAngle[size=.2](A,D,B)
\tkzMarkAngles[size=.3, mark=none](A,C,D C,B,A)
\tkzLabelAngle[pos=.5pt](A,C,D){$\gamma$}
\tkzLabelAngle[pos=.5pt](C,B,A){$\beta$}

    \end{tikzpicture}
    \caption{}
    \end{minipage}
    \begin{minipage}[t]{0.32\textwidth}
    \centering
\begin{tikzpicture}[>=latex, scale=1]
    \tkzDefPoints{0/0/B, 2/0/C, 3/2/A, 3/0/D}     
    \tkzLabelPoints[below](B,C,D)
\tkzLabelPoints[above](A)   
\tkzDrawPolygon(A,B,C)
\draw[dashed](A)--(D)--(C);
\tkzMarkRightAngle[size=.2](A,D,B)
\node at (1.5,1)[left]{$c$};  \node at (2.5,1)[right]{$b$};
\node at (1,0)[below]{$a$};  
\tkzMarkAngles[size=.3, mark=none](A,C,B)
\tkzLabelAngle[pos=.5pt](A,C,B){$\gamma$}
    \end{tikzpicture}
    \caption{}
    \end{minipage}
    \begin{minipage}[t]{0.32\textwidth}
    \centering
    \begin{tikzpicture}[>=latex, scale=1]
\tkzDefPoints{0/0/B, 2/0/C, 2/2.5/A}
\tkzDrawPolygon(A,B,C)
\tkzLabelPoints[below](B,C)
\tkzLabelPoints[above](A)   
\tkzMarkRightAngle[size=.2](A,C,B)

    \end{tikzpicture}
    \caption{}
    \end{minipage}
    \end{figure}


如果$\beta$、$\gamma$中有一个是直角,设$\gamma=90^{\circ}$ (图6.28), 

$\because\quad \cos\gamma=\cos90^{\circ}=0$

$\therefore\quad a=c\cos\beta +b\cos\gamma$ 仍然成立。

若分别作$\overline{AB}$、$\overline{AC}$边上的高,同样还可证明
\[\begin{split}
  b&=a\cos\gamma+c\cos\alpha \\
c&=b\cos\alpha+a\cos\beta  
\end{split}\]

总结上面的讨论,我们得到:

\begin{blk}
 {定理}
在任意$\triangle ABC$中,边角关系满足:
\begin{align}
 a&=c\cos\beta +b\cos\gamma\\
b&=a\cos\gamma+c\cos\alpha \\
c&=b\cos\alpha +a\cos\beta 
\end{align}  
\end{blk}

由上面的定理,我们就可推出余弦定理

\begin{blk}
    {余弦定理}
在任意的$\triangle ABC$中,边角关系满足
\[\begin{split}
  a^2&=b^2+c^2-2bc\cos\alpha\\
b^2&=c^2+a^2-2ca \cos\beta\\
c^2&=a^2+b^2-2ab\cos\gamma  
\end{split}\]
\end{blk}

\begin{proof}
    由上面的定理的(6.7)式得,
\[\cos\beta=\frac{a-b\cos\gamma}{c}\]
由(6.8)式得
\[\cos\alpha=\frac{b-a\cos\gamma}{c}\]
把$\cos\beta$、$\cos\alpha$代入(6.9)得
\[c=a\frac{a-b\cos\gamma}{c}+b\frac{b-a\cos\gamma}{c}=\frac{a^2+b^2-2ab\cos\gamma}{c}\]
$\therefore\quad c^2=a^2+b^2-2ab\cos\gamma$

用同样的办法我们可证:
\[\begin{split}
   b^2&=c^2+a^2-2ca \cos\beta\\ 
   a^2&=b^2+c^2-2bc\cos\alpha   
\end{split}\]
\end{proof}

请同学们自证如下两个推论:

\begin{blk}{推论1}
在$\triangle ABC$中,
\[\begin{split}
    \cos\alpha&=\frac{b^2+c^2-a^2}{2bc}\\
    \cos\beta&=\frac{a^2+c^2-b^2}{2ac}\\
        \cos\gamma&=\frac{a^2+b^2-c^2}{2ab}
\end{split}\]
\end{blk}

\begin{blk}{推论2}
     一个三角形两边平方的和如果等于第三边的平
方,那么第三边所对的角是直角;如果小于第三边的平方,
那么第三边所对的角是钝角;如果大于第三边的平方,那么
第三边所对的角是锐角。
\end{blk}


\begin{example}
    分别求出$120^{\circ}$、$135^{\circ}$、$150^{\circ}$、$140^{\circ}$的余弦。
\end{example}

\begin{solution}
由定义知,一个钝角的余弦等于它的补角的余弦的
相反数,所以有:
\[\begin{split}
  \cos120^{\circ}&=-\cos(180^{\circ}-120^{\circ})=-\cos60^{\circ}=-\frac{1}{2}\\
  \cos135^{\circ}&=-\cos(180^{\circ}-135^{\circ})=-\cos45^{\circ}=-\frac{\sqrt{2}}{2}\\  
  \cos150^{\circ}&=-\cos(180^{\circ}-150^{\circ})=-\cos30^{\circ}=-
\frac{\sqrt{3}}{2}\\
\cos140^{\circ}&=-\cos(180^{\circ}-140^{\circ})=-\cos40^{\circ}\approx -0.7660  
\end{split}\]
\end{solution}

\begin{example}
    用余弦定理证明广义勾股定理。

    已知:$\parallelogram ABCD$ (图6.29)

    求证:
    $\overline{AC}^2+\overline{BD}^2=2(\overline{AB}^2+\overline{AD}^2)$
\end{example}

\begin{figure}[htp]
    \centering
\begin{tikzpicture}
 \tkzDefPoints{0/0/A, 3/0/B, 3.75/1.5/C, .75/1.5/D}
\tkzDrawPolygon[thick](A,B,C,D)
\tkzDrawSegments[thick](A,C B,D)
\tkzLabelPoints[below](A,B)
\tkzLabelPoints[above](C,D)   
\end{tikzpicture}
    \caption{}
\end{figure}

\begin{proof}
在$\triangle ABC$与$\triangle ABD$中,
\begin{align}
   \overline{AC}^2&=\overline{AB}^2+\overline{BC}^2-2\overline{AB}\cdot \overline{BC} \cos\angle ABC \\
    \overline{BD}^2&=\overline{AB}^2+\overline{AD}^2-2\overline{AB}\cdot \overline{AD}\cos\angle BAD
\end{align}

$\because\quad \angle BAD+\angle ABC =180^{\circ}$

$\therefore\quad \cos \angle ABC=-\cos\angle BAD$

又$\because\quad \overline{BC}=\overline{AD}$

$(6.10)+(6.11)$则得:
\[\overline{AC}^2+\overline{BD}^2=\overline{AB}^2+\overline{BC}^2+\overline{AB}^2+\overline{AD}^2=2\left(\overline{AB}^2+\overline{AD}^2\right)\]
\end{proof}

\begin{ex}
\begin{enumerate}
    \item 在$\triangle ABC$中,已知$a=4$, $b=7$, $\gamma=40^{\circ}$, 求$c$.
    \item 在$\triangle ABC$中,已知$a=2$, $b=3$, $c=4$, 求$\cos\alpha$、$\cos\beta$、
    $\cos\gamma$.
    \item 已知$\triangle ABC$的三边:
\begin{enumerate}
    \item 56, 65, 33, 求最大角;
    \item 7、$4\sqrt{3}$、$\sqrt{13}$, 求最小角。
\end{enumerate}

    \item 在$\triangle ABC$中,$\angle A=120^{\circ}$, 求证:
\begin{enumerate}
    \item $a^2-b^2=c(b+c)$
    \item $ b(a^2-b^2)=c(a^2-c^2)$
\end{enumerate}

\item  在$\triangle ABC$中,已知$\cos\beta=\frac{\sin\alpha}{2\sin\gamma}$, 求证$\triangle ABC$是等腰三角形。
\item  在$\triangle ABC$中,求证:
\[\frac{\cos A}{a}+\frac{\cos B}{b}+\frac{\cos C}{c}=\frac{a^2+b^2+c^2}{2abc}\]
\end{enumerate}    
\end{ex}

\subsection{解斜三角形}
在一个三角形中,如果没有一个角是直角,那么,这个三角
形叫做斜三角形。斜三角形的解法可以分成下面的四种情形:
\begin{enumerate}
\item 已知一边和两角;
\item 已知两边和它们的夹角;
\item 已知三边;
\item 已知两边和其中一边的对角。
\end{enumerate}

下面我们分别举例说明每种情形的解法。
    
\begin{example}
    已知:$c=100$、$\alpha=40^{\circ}$, $\beta=60^{\circ}$ (图6.30), 求
$\gamma$、$a$、$b$.
\end{example}

\begin{figure}[htp]
    \centering
\begin{tikzpicture}
\tkzDefPoints{0/0/A, 4/0/B, 2.8/2/C}
\tkzDrawPolygon(A,B,C)
\tkzMarkAngles[mark=none, size=.3](B,A,C C,B,A)
\tkzLabelPoints[below](A,B)
\tkzLabelPoints[above](C)
\end{tikzpicture}
    \caption{}
\end{figure}

\begin{solution}
\[\begin{split}
    \gamma&=180^{\circ}-(40^{\circ}+60^{\circ})=80^{\circ}\\
    a&=\frac{c\sin\alpha}{\sin\gamma}\approx \frac{100\x \sin 40^{\circ}}{\sin 80^{\circ}}=\frac{100\x 0.6428}{0.9848}=65.27\\
    b&=\frac{c\sin\beta}{\sin\gamma}=\frac{100\x \sin 60^{\circ}}{\sin 80^{\circ}}=\frac{100\x 0.8660}{0.9848}=87.94
\end{split}\]
\end{solution}

例6.31告诉我们,如果已知两角和一边,先用内角和定理
求出未知的第三个角,然后再用正弦定理便可求出未知的两
边了。

\begin{ex}
\begin{enumerate}
    \item 已知下列条件,解三角形,并求其外接圆的半径。
\begin{enumerate}
    \item $b=4,\qquad \alpha=30^{\circ} ,\qquad \beta=120^{\circ} $
    \item $c=13,\qquad \alpha=45^{\circ},\qquad \beta=60^{\circ} $
\end{enumerate}
    \item 解三角形:
\begin{enumerate}
    \item $\alpha=62^{\circ} ,\qquad \beta=48^{\circ} ,\qquad c=24$
    \item $\alpha=55^{\circ} 12',\qquad \beta=29^{\circ} 18',\qquad c=18$
\end{enumerate}
\end{enumerate}
\end{ex}


\begin{example}
    已知$b=60$、$c=34$、$\alpha=41^{\circ}$ (图6.31)。求$\alpha$, $\beta$,
$\gamma$.
\end{example}

\begin{figure}[htp]
    \centering
\begin{tikzpicture}
\tkzDefPoints{0/0/B, 4/0/C, 1.5/2/A}
\tkzDrawPolygon(A,B,C)
\tkzMarkAngles[mark=none, size=.3](B,A,C)
\tkzLabelPoints[below](C,B)
\tkzLabelPoints[above](A)
\node at (2,0)[below]{$a$};
\node at (5.5/2,1)[right]{$b$};
\node at (.75,1)[left]{$c$};
\end{tikzpicture}
    \caption{}
\end{figure}

\begin{solution}
\[\begin{split}
    a^2&=b^2+c^2-2bc\cos\alpha\\
&=602+342-2\x60\x34\cos41^{\circ}\\
&=3600+1156-4080\x0.755\approx 1676
\end{split}\]
查平方根表,得:$a\approx 41$。
\[\sin\gamma=\frac{c\sin\alpha}{a}=\frac{34\x\sin41^{\circ}}{41}=\frac{34\x 0.655}{41}\approx 0.5441\]

$\therefore\quad \gamma\approx 32^{\circ}58'$

\[\beta=180^{\circ}-(\alpha+\gamma)
=180^{\circ}-(41^{\circ}+32^{\circ}58')
=106^{\circ}2'\]    
\end{solution}


例6.32告诉我们,如果已知两边及夹角,可由余弦定理先求
出未知的一边,然后由正弦定理就可算出未知的一个角了,
再由内角和定理,就可求出未知的另一个角了。再求出未知
边后,最好先求较短边所对的角,因为较短边所对的角一定
是锐角,这样就可避免求钝角的情况。

\begin{ex}
    已知下列条件,解三角形
\begin{enumerate}
\item  $a=22,\qquad b=26,\qquad \gamma=78^{\circ}$
\item  $b=10,\qquad c=10,\qquad \alpha=102^{\circ}$
\item  $a=0.8,\qquad c=0.6,\qquad \beta=50^{\circ}$
\item  $a=4,\qquad b=5,\qquad \gamma=60^{\circ}$
\end{enumerate}
\end{ex}







\begin{example}
    
\end{example}


\begin{solution}
    
\end{solution}











































\begin{ex}
\begin{enumerate}
    \item 要测量一条河两岸的两点$A$、$B$之间的距离,在$A$点所在
    的岸边选择一条基线$\overline{AC}=380$米,测出
    $\angle BAC=75^{\circ}32'$, $\angle BCA=45^{\circ}35'$, 
    求$A$、$B$间的距离。
    \item 如图,$A$、$B$两地不能直接测量,选同时能看到$A$、$B$的一点
    $C$, 测得$\overline{AC}=280$米,$\overline{BC}=470$米,$\angle ACB=80^{\circ}21'$, 
    求$A$、$B$两地的距离。
    \item 河对岸有两个目标$A$、$B$, 若
    不准过河,如何测量才能
    算出$A$、$B$两点间的距离?
\end{enumerate}
\end{ex}



\subsection*{习题6.3}
\begin{enumerate}
    \item 在$\triangle ABC$中,
\begin{enumerate}
    \item 已知$a=8$、$\beta=40^{\circ}$、$\gamma=90^{\circ}$, 求$b$.
    \item 已知$a=25$、$b=31$、$\gamma=90^{\circ}$, 求$a$、$c$.
    \item 已知$a=6$、$\beta=40^{\circ}$、$\gamma=80^{\circ}$, 求b.
    \item 已知$a=8$、$b=6$、$\alpha=75^{\circ}$, 求$\beta$、$c$.
    \item 已知$a=3$、$b=4$、$c=5$, 求$\gamma$.
\end{enumerate}
\item  已知下列条件,求$S_{\triangle}$.
\begin{enumerate}
    \item $a=6,\qquad c=12,\qquad \beta =135^{\circ}$
    \item $a=10,\qquad \beta =45^{\circ},\qquad \gamma=60^{\circ}$
    \item $a=3,\qquad b=5,\qquad c=10$
\end{enumerate}

\item  在$\triangle ABC$中,$a=10$cm, $\beta =54^{\circ}16'$, $\alpha=63^{\circ}6'$, 求$\triangle ABC$的面积和$\angle C$的平分线长。
\item  设$D$把$\triangle ABC$的边$\overline{BC}$分成$m:n$, 求证:
\[n\overline{AB}^2+m \overline{AC}^2=(m+n)\overline{AD}^2+\overline{BD}^2+m\overline{CD}^2\]
\item  已知$\triangle ABC$, 且
$\frac{a+b}{6}=\frac{b+c}{4}=\frac{c+a}{5}$,
求证:
\begin{enumerate}
    \item $\frac{\sin\alpha}{7}=\frac{\sin\beta}{5}=\frac{\sin\gamma}{3}$
    \item $\frac{\cos\alpha}{-7}=\frac{\cos\beta}{11}=\frac{\cos\gamma}{13}$
    \item $\alpha=120^{\circ}$
\end{enumerate}
\item  已知在$\triangle ABC$中,$\gamma=60^{\circ}$, 求证:$a^2+b^2=c^2+ab$.
\item  求证在$\triangle ABC$中,
$a^2-b^2=ac\cos\beta -bc\cos\alpha$
\item  已知三角形三边的比是$3:5:7$, 求最大角。
\item  如果$\triangle ABC$的边角之间满足下列关系,那么这个三角形
是什么三角形,
\begin{enumerate}
    \item $\sin^2\alpha=\sin^2\beta +\sin^\gamma$
    \item $a\cos A=b\cos B$
\end{enumerate}

\item  等腰梯形$ABCD$的上底$\overline{AD}=18$cm, 下底$\overline{BC}=22$cm, 
$\angle ABC=60^{\circ}$, 求它的面积。
\item 已知梯形$ABCD$, $AD\parallel BC$. $\overline{AB}=13$, $\overline{BC}=18$, 
$\overline{CD}=15$, $\overline{DA}=4$, 求它的面积。
\item 如果已知四边形$ABCD$的$\angle A=90^{\circ}$, $\overline{AB}=32$, $\overline{BC}=
27$, $\overline{CD}=35$, $\overline{DA}=24$, 那么它的面积是多少?
\item 已知一四边形的两条对角线的长分别为$x$、$y$, 夹角为$\theta$, 
面积为$S$, 求证:
\[S=\frac{1}{2}xy\sin\theta\]
\end{enumerate}

\section*{复习题六}
\begin{enumerate}
    \item 用作图法求锐角$x$.
\begin{multicols}{2}
    \begin{enumerate}
        \item $\sin x=\frac{4}{5}$
        \item $\cos x=\frac{1}{3}$
        \item $\tan x=\sqrt{3}+\sqrt{2}$
        \item $\sin x=2\cos x$
    \end{enumerate}
\end{multicols}

\item 已知:$\tan\alpha=m,\; m\ge 0$, 求$\cos\alpha$、$\sin\alpha$.
\item 已知
   $ \sin\alpha+\cos\alpha=1.2$, 求$\sin\alpha$、$\cos\alpha$的值。
    \item 计算:
\begin{enumerate}
    \item $3\sin90^{\circ}+2\cos0^{\circ}-3\sin45^{\circ}$
    \item $\sin^2 30^{\circ}-\frac{1}{2}\cos 90^{\circ}+\cos^2 30^{\circ}$
    \item $\sin^2 45^{\circ}+\tan^2 30^{\circ}$
\end{enumerate}

    \item 已知$\triangle ABC$, $\angle C=90^{\circ}$, 求证
  \[  \tan A+\tan B=\frac{c^2}{ab}\]
    \item 已知在$\triangle ABC$中,$\angle C=90^{\circ}$, $\overline{CE}\overline{CB}$, 延长$CA$到$D$
    使$\overline{AD}=\overline{AB}$, 作$\overline{DB}$, 求$\angle D$的度数,并根据图形求
    $\angle D$的正弦、余弦、正切和余切的值。

\item 在正方形$ABCD$中,已知$E$是$\overline{BC}$的中点,求$\angle AEC$的正
弦和余弦。
\item 在四边形$ABCD$中,$\overline{AB}=2a$, $BC=(\sqrt{3}+1)a$, 
$\overline{CD}=\sqrt{2}a$, $\angle B=60^{\circ}$, $\angle C=75^{\circ}$, 求$\overline{AC}$的长和四边
形$ABCD$的面积。
\item 已知在$\triangle ABC$中,$a=4$、$b=5$、$c=6$, 求$\cos\alpha$、$\sin\gamma$、
$\sin\beta$、$\sin\alpha$, 问$\alpha$、$\beta$、$\gamma$是锐角还是钝角?
\item 一个三角形的三边的长分别为3尺,4尺及$\sqrt{37}$尺,求
此三角形的最大角的度数。
\item 在$\triangle ABC$中,求证:
\[\frac{a}{\sin\alpha}=\frac{b+c}{\sin\beta+\sin\gamma}=\frac{b-c}{\sin\beta-\sin\gamma}\]
\item 设$P$是等边三角形$ABC$外接圆的$\wideparen{BC}$上的一点,求证
\[\overline{PA}^2=\overline{AB}^2+\overline{PB}\cdot \overline{PC}\]
(提示:求$\cos\angle ABP$, $\cos\angle ACP$,利用$\cos\angle ABP=-\cos\angle ACP$化简即得)。
\item 在$\triangle ABC$中,已知$\gamma=60^{\circ}$, $\overline{AC}=4$, 面积为$\sqrt{3}$, 求
$\overline{AB}$及$\overline{BC}$的长。
\item 如图,为求得河对岸某建
筑物的高$\overline{AB}$, 在地面上
引一条基线$\overline{CD}=a$, 测得
$\angle ACB=\alpha$, $\angle BCD=\beta$, 
$\angle BDC=\gamma$, 求$\overline{AB}$.
\item 已知$\odot (A,\sqrt{3})$、
$\odot (B,2-\sqrt{3})$、$\odot (c,1)$。
$\odot A$分别与$\odot B$和$\odot C$相外
切.$\angle BAC=60^{\circ}$, 求$\overline{BC}$
的长和$\angle ACB$的度数。

\item 外国船只除特许者外,不得进入离我海岸$d$海里以内的
区域,设$A$和$B$是我们的两个观测站,$A$与$B$之间的距离为
$S$海里,海岸线是过$A$点及$B$点的直线,一外国船在$P$点,
在$A$站测得$\angle BAP=\alpha$, 同时在$B$站测得$\angle ABP=\beta$, 问$\alpha$
及$\beta$满足什么简单的三角比的不等式,就应当向此未经
特许的外国船发出警告,命令退出我海域。
\item 已知$I$是$\triangle ABC$的内心,$I_a$、$I_b$、$I_c$为$\triangle ABC$的三个旁
心,$r,r',r_a,r_b$分别为$\triangle ABC$的外接圆、内切圆和三个
旁切圆的半径;设$p=\frac{1}{2}(a+b+c)$, $S_{\triangle}$为$\triangle ABC$的面积;求证:
\begin{enumerate}
    \item $S_{\triangle}=r'p=r_a(p-a)=r_b(p-b)=r_c(p-c)$
    \item $4rS_{\triangle}=abc$
    \item $4r=r_a+r_b+r_c-r'$
\end{enumerate}

\begin{figure}[htp]
    \centering
\begin{tikzpicture}
\tkzDefPoints{0/0/B, 3/0/C, 2.4/2/A}
\tkzDrawPolygon[very thick](A,B,C)
\tkzDefTriangleCenter[in](A,B,C) \tkzGetPoint{I}
\tkzDefTriangleCenter[ex](A,B,C) \tkzGetPoint{I_b}
\tkzDefTriangleCenter[ex](B,C,A) \tkzGetPoint{I_c}
\tkzDefTriangleCenter[ex](C,A,B) \tkzGetPoint{I_a}
\tkzDrawLines(I_a,I_b I_a,I_c I_b,I_c)  
\tkzDrawLines[add = .5 and .5](A,B A,C B,C)
\tkzDrawLines[add = 0 and .1](A,I_a B,I_b C,I_c)
\tkzLabelPoints[above left](B)
\tkzLabelPoints[below right](C,I_b,I_a)
\tkzLabelPoints[above](A, I_c,I)
\tkzInterLL(B,I_b)(A,C) \tkzGetPoint{X}
\tkzDrawCircle(I_b,X)
\end{tikzpicture}
    \caption*{第17题}
\end{figure}


\end{enumerate}



\end{document}